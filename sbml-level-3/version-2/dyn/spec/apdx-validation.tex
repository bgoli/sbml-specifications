\newcounter{arraysCtr}
\newcommand{\printValid}{\validRule{arrays-\arabic{arraysCtr}\addtocounter{arraysCtr}{1}}}
\section{Validation Rules}
\label{validation}

This section summarizes all the conditions that must (or in some cases,at least \emph{should}) be true of an SBML Level~3 Version~2 model that uses the Dynamic Structures package. We use the same conventions as are used in the SBML Level~3 Version~2 Core specification document.  In particular, there are different degrees of rule strictness. Formally, the differences are expressed in the statement of a rule: either a rule states that a condition \emph{must} be true, or a rule states that it \emph{should} be true. Rules of the former kind are strict SBML validation rules---a model encoded in SBML must conform to all of them in order to be considered valid.  Rules of the latter kind are consistency rules. To help highlight these differences, we use the following three symbols next to the rule numbers:

\begin{description}

\item[\hspace*{6.5pt}\vSymbol\vsp] A \vSymbolName indicates a
  \emph{requirement} for SBML conformance. If a model does not follow
  this rule, it does not conform to the Dynamic Structures
  specification.  (Mnemonic intention behind the choice of symbol:
  ``This must be checked.'')

\item[\hspace*{6.5pt}\cSymbol\csp] A \cSymbolName indicates a
  \emph{recommendation} for model consistency.  If a model does not
  follow this rule, it is not considered strictly invalid as far as the
  Dynamic Structures specification is concerned; however, it
  indicates that the model contains a physical or conceptual
  inconsistency.  (Mnemonic intention behind the choice of symbol:
  ''This is a cause for warning.'')

\item[\hspace*{6.5pt}\mSymbol\msp] A \mSymbolName indicates a strong
  recommendation for good modeling practice.  This rule is not strictly
  a matter of SBML encoding, but the recommendation comes from logical
  reasoning.  As in the previous case, if a model does not follow this
  rule, it is not considered an invalid SBML encoding.  (Mnemonic
  intention behind the choice of symbol: ''You're a star if you heed
  this.'')
\end{description}

The validation rules listed in the following subsections are all stated
or implied in the rest of this specification document.  They are
enumerated here for convenience. Unless explicitly stated, all
validation rules concern objects and attributes specifically defined in
the Dynamic Structures package.

For \notice convenience and brevity, we use the shorthand
``\token{dyn:x}'' to stand for an attribute or element name \token{x}
in the namespace for the Dynamic Structures package, using
the namespace prefix \token{dyn}.  In reality, the prefix string may be
different from the literal ``\token{dyn}'' used here (and indeed, it
can be any valid XML namespace prefix that the modeler or software
chooses).  We use ``\token{dyn:x}'' because it is shorter than to
write a full explanation everywhere we refer to an attribute or element
in the Dynamic Structures package namespace.

\subsubsection*{General rules about the Dynamic Structures package}
\validRule{dyn-10101}{To conform to Version~1 of the Dynamic Structures package
  specification for SBML Level~3, an SBML document must declare the
  use of the following XML Namespace: \\ \textls[-25]{\uri{http://www.sbml.org/sbml/level3/version2/dyn/version1}}. (Reference: \sbmlthreedynamic, \sec{subsec:xml-namespace}.)}
  
\validRule{dyn-10102}{Wherever they appear in an SBML document,
  elements and attributes from the Dynamic Structures
  package must be declared either implicitly or explicitly to be in the
  XML namespace: \\ 
  \textls[-25]{\uri{http://www.sbml.org/sbml/level3/version2/dyn/version1}}.
  (Reference: \sbmlthreedynamic, \sec{subsec:xml-namespace}.) }

\subsubsection*{Rules for the extended \class{SBML} class} 
\validRule{dyn-10201} {In all SBML documents using the Dynamic Structures package, the \SBML object must include a value for  the attribute \token{dyn:required} attribute.  (Reference: SBML Level~3 Version~2 Core, Section~4.1.3.)}
  
\validRule{dyn-10202} {The value of attribute \token{dyn:required} on the \SBML object must be of the data type \primtype{boolean}. (Reference: SBML Level~3 Version~2 Core, Section~4.1.3.) }
  
\validRule{dyn-10203} {The value of attribute \token{dyn:required} on
  the \SBML object must be set to \val{true} (Reference: \sbmlthreedynamic, \sec{subsec:xml-namespace}.) }


\subsubsection*{General rules for CBO usage} 

\consistencyRule{dyn-10301} {The value of the attribute \token{dyn:cboTerm} on a \Model object should be a full CBO identifier referring to a model qualifier defined under the \textbf{CBO\textunderscore Object} branch. The value must be a term derived from the \textbf{System\textemdash model} and \textbf{SystemQuality\textemdash model} hierarchies. (Reference: \sbmlthreedynamic, \sec{sec:CBO}.)}

\consistencyRule{dyn-10302} {The value of the attribute \token{dyn:cboTerm} on a \FunctionDefinition object should be a full CBO identifier referring to a mathematical expression derived from either the \textbf{CBO\textunderscore Object} or \textbf{CBO\textunderscore Process} branch. (Reference: \sbmlthreedynamic, \sec{sec:CBO}.)}

\consistencyRule{dyn-10303} {The value of the attribute \token{dyn:cboTerm} on a \Parameter object should be a full CBO identifier referring to a quantitative parameter from the \textbf{CBO\textunderscore Object} branch.  (Reference: \sbmlthreedynamic, \sec{sec:CBO}.)}

\consistencyRule{dyn-10304} {The value of the attribute \token{dyn:cboTerm} on an \InitialAssignment object should be a full CBO identifier referring to a mathematical expression derived from either the \textbf{CBO\textunderscore Object} or \textbf{CBO\textunderscore Process} branch. (Reference: \sbmlthreedynamic, \sec{sec:CBO}.)}

\consistencyRule{dyn-10305} {The value of the attribute \token{dyn:cboTerm} on an \AlgebraicRule, \RateRule or \AssignmentRule object should be a full CBO identifier referring to a mathematical expression derived from the \textbf{CBO\textunderscore Process} branch. (Reference: \sbmlthreedynamic, \sec{sec:CBO}.)}

\consistencyRule{dyn-10306} {The value of the attribute \token{dyn:cboTerm} on a \Constraint object should be a full CBO identifier referring to a mathematical expression derived from the \textbf{CBO\textunderscore Process} branch. (Reference: \sbmlthreedynamic, \sec{sec:CBO}.)}

\consistencyRule{dyn-10307} {The value of the attribute \token{dyn:cboTerm} on a \Reaction object should be a full CBO identifier referring to a biological process derived from the \textbf{CBO\textunderscore Process} branch.  (Reference: \sbmlthreedynamic, \sec{sec:CBO}.)}

\consistencyRule{dyn-10308} {The value of the attribute \token{dyn:cboTerm} on a \SpeciesReference or a \ModifierSpeciesReference object should be a full CBO identifier referring to a physical entity (reactant, product or modifier) derived from the \textbf{CBO\textunderscore Object} branch .  (Reference: \sbmlthreedynamic, \sec{sec:CBO}.)}

\consistencyRule{dyn-10309} {The value of the attribute \token{dyn:cboTerm} on a \KineticLaw object should be a full CBO identifier referring to the rate law of a biological process derived from the \textbf{CBO\textunderscore Process} branch. (Reference: \sbmlthreedynamic, \sec{sec:CBO}.)}

\consistencyRule{dyn-10310} {The value of the attribute \token{dyn:cboTerm} on an \Event object should be a full CBO identifier referring to an existential cellular process derived from the \textbf{CBO\textunderscore Process} branch. (Reference: \sbmlthreedynamic, \sec{sec:CBO}.)}

\consistencyRule{dyn-10311} {The value of the attribute \token{dyn:cboTerm} on an \EventAssignment object should be a full CBO identifier referring to a mathematical expression derived from the \textbf{CBO\textunderscore Process} branch. (Reference: \sbmlthreedynamic, \sec{sec:CBO}.)}

\consistencyRule{dyn-10312} {The value of the attribute \token{dyn:cboTerm} on a \Compartment object should be a full CBO identifier referring to a physical modeling entity derived from the \textbf{CBO\textunderscore Object} branch. (Reference: \sbmlthreedynamic, \sec{sec:CBO}.)}

\consistencyRule{dyn-10313} {The value of the attribute \token{dyn:cboTerm} on a \Species object should be a full CBO identifier referring to a biological entity derived from the \textbf{CBO\textunderscore Object} branch.  (Reference: \sbmlthreedynamic, \sec{sec:CBO}.)}

\consistencyRule{dyn-10314} {The value of the attribute \token{dyn:cboTerm} on a \Trigger object should be a full CBO identifier referring to the mathematical expression of a biological process derived from the \textbf{CBO\textunderscore Process} branch. (Reference: \sbmlthreedynamic, \sec{sec:CBO}.)}

\consistencyRule{dyn-10315} {The value of the attribute \token{dyn:cboTerm} on a \Delay object should be a full CBO identifier referring to the mathematical expression of a biological process derived from the \textbf{CBO\textunderscore Process} branch.  (Reference: \sbmlthreedynamic, \sec{sec:CBO}.)}

\consistencyRule{dyn-10316} {The value of the attribute \token{dyn:cboTerm} on a \LocalParameter object should be a full CBO identifier of a quantitative parameter from the \textbf{CBO\textunderscore Object} branch. (Reference: \sbmlthreedynamic, \sec{sec:CBO}.)}

\subsubsection*{Rules for extended \class{SBase} abstract class } 

\validRule{comp-20101}{Any object derived from the extended \SBase class (defined in the Dynamic Structures package) may contain a \token{dyn:cboTerm} attribute. (References: \sbmlthreedynamic, \sec{subsec:extSBase}.) }

\validRule{dyn-20102}{The value of a \token{dyn:cboTerm} attribute on an SBML object must always conform to the syntax of the \primtype{CBOTerm} data type. (References: \sbmlthreedynamic, \sec{attr:cboTerm}.)}

\subsubsection*{Rules for extended \class{Event} objects } 

\validRule{dyn-20201}{An \Event object must have the attribute \token{dyn:applyToAll} because it is required. With the exception of the \token{dyn:cboTerm} attribute from \SBase, no other attributes from the Dynamic Structures namespace are permitted on an \Event object. (References: \sbmlthreedynamic, \sec{subsec:extEvent}.) }

\validRule{dyn-20202}{ The value of an \token{dyn:applyToAll} attribute on a \Event object must be of the data type \primtype{boolean}. (References: \sbmlthreedynamic, \sec{attr:applyToAll}.) }

\validRule{dyn-20203}{There may be at most one instance of a \ListOfElements subobject within an \Event object that uses the Dynamic Structures package. (References: \sbmlthreedynamic, \sec{subsec:extEvent}.) }

\validRule{dyn-20204}{The \ListOfElements subobject within an \Event object is optional, but if present, this container object must not be empty. (References: \sbmlthreedynamic, \sec{subsec:ListOfElements}.) }

\validRule{dyn-20205}{Apart from the general notes and annotation subobjects permitted on all SBML objects, a \ListOfElements container object may only contain \Element objects. (References:\sbmlthreedynamic, \sec{subsec:ListOfElements}.) }

\validRule{dyn-20206}{A \ListOfElements object may have the optional attributes \token{id}, \token{name}, \token{metaid} and \token{sboTerm} defined by SBML Level~3 Core. No other attributes from the SBML Level~3 Core namespace or the Dynamic Structures namespace are permitted on a \ListOfElements object. (References: SBML Level~3 Version~2 Core, Section~4.2.7.) }

\subsubsection*{Rules for \class{Element} objects } 

\validRule{dyn-20301}{An \Element object must have the attribute \token{dyn:element} because it is required.  No other attributes from the Dynamic Structures namespace are permitted on an \Element object. (References: \sbmlthreedynamic, \sec{subsec:Element}.) }

\validRule{dyn-20302}{An \Element object may have the optional SBML Level~3 Core attributes \token{id}, \token{name}, \token{metaid} and \token{sboTerm}. (References: SBML Level~3 Version~2 Core, Section~4.2.)}

\validRule{dyn-20303}{An \Element object may have the optional SBML Level~3 Core subobjects for notes and annotation. No other elements from the SBML Level 3 Core namespace are permitted on an \Element object. (References: SBML Level~3 Version~2 Core, Section~4.2.)}

\validRule{dyn-20304}{The value of an \token{dyn:element} attribute on an \Element object must always conform to the syntax of the SBML data type \primtype{SIdRef}. (References: \sbmlthreedynamic, \sec{attr:element}.)}

\validRule{dyn-20305}{The value of an \token{dyn:element} attribute on an \Element object must be the identifier of an SBML component whose token{id} is of type SId. (References:\sbmlthreedynamic, \sec{attr:element}.)}

\subsubsection*{Rules for extended \class{Compartment} objects } 

\validRule{dyn-20401}{There may be at most one instance of a \ListOfSpatialComponents subobject within a \Compartment object that uses the Dynamic Structures package. (References: \sbmlthreedynamic, \sec{subsec:extCompartment}.) }

\validRule{dyn-20402}{The \ListOfSpatialComponents subobject within a \Compartment object is optional, but if present, this container object must not be empty. (References: \sbmlthreedynamic, \sec{subsec:extCompartment}.) }

\validRule{dyn-20403}{Apart from the general notes and annotation subobjects permitted on all SBML objects, a \ListOfSpatialComponents container object may only contain \SpatialComponent objects. (References:\sbmlthreedynamic, \sec{subsec:listSpatialComp}.)}

\validRule{dyn-20404}{A \ListOfSpatialComponents object may have the optional attributes \token{id}, \token{name}, \token{metaid} and \token{sboTerm} defined by SBML Level~3. No other attributes from the SBML Level~3 Core namespace or the Dynamic Structures namespace are permitted on a \ListOfSpatialComponents object. (References: SBML Level~3 Version~2 Core, Section~4.2.7.) }

\subsubsection*{Rules for \class{SpatialComponent} objects } 

\validRule{dyn-20501}{A \SpatialComponent object must have the attributes \token{dyn:spatialIndex} and \token{dyn:variable} because they are required. No other attributes from the Dynamic Structures namespace are permitted on a \SpatialComponent object. (References: \sbmlthreedynamic, \sec{subsec:spatialComp}.) }

\validRule{dyn-20502}{A \SpatialComponent object may have the optional SBML Level~3 Core attributes \token{id}, \token{name}, \token{metaid} and \token{sboTerm}. (References: SBML Level~3 Version~2 Core, Section~4.2.)}

\validRule{dyn-20503}{A \SpatialComponent object may have the optional SBML Level 3 Core subobjects for notes and annotation. No other elements from the SBML Level 3 Core namespace are permitted on a \SpatialComponent object. (References: SBML Level~3 Version~2 Core, Section~4.2.)}

\validRule{dyn-20504}{The value of a \token{dyn:spatialIndex} attribute on a \SpatialComponent object must always conform to the syntax and allowed values of the newly defined \primtype{SpatialKind}. Permitted values for this attribute include: \val{cartesianX}, \val{cartesianY}, and \val{cartesianZ} for position; \val{alpha}, \val{beta}, and \val{gamma} for rotational angles; and \val{F\textunderscore x}, \val{F\textunderscore y}, and \val{F\textunderscore z} for force in case of movement. (References: \sbmlthreedynamic, \sec{attr:spatialIndex}.)}

\consistencyRule{dyn-20505}{If a given \Compartment extends a \ListOfSpatialComponents, which contains \SpatialComponent subobjects to indicate location, values for their \token{dyn:spatialIndex} attributes must be used in a specific order. For 1-dimensional cases, \val{cartesianX} is used; for 2-dimensional cases, \val{cartesianX} and \val{cartesianY} are used; and for 3-dimensional cases, \val{cartesianX} , \val{cartesianY} , and \val{cartesianZ} are used. (References: \sbmlthreedynamic, \sec{attr:spatialIndex}.)}

\validRule{dyn-20506} {The value of a \token{dyn:variable} attribute on a \SpatialComponent object must always conform to the syntax of the data type \primtype{SIdRef}. (References: \sbmlthreedynamic, \sec{attr:variable}.)}

\validRule{dyn-20507}{The value of a \token{dyn:spatialIndex} attribute on a \SpatialComponent object is bound in a Cartesian coordinate system. (References: \sbmlthreedynamic, \sec{attr:spatialIndex}.)}

\validRule{dyn-20508}{The value of a \token{dyn:variable} attribute on a \SpatialComponent object must be the identifier of an SBML parameter which cannot be constant. (References:\sbmlthreedynamic, \sec{attr:variable}.)}
 