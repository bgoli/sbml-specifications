\newcounter{arraysCtr}
\newcommand{\printValid}{\validRule{arrays-\arabic{arraysCtr}\addtocounter{arraysCtr}{1}}}
\section{Validation Rules}
\label{validation}

This section summarizes all the conditions that must (or in some cases,at least \emph{should}) be true of an SBML Level~3 Version~2 model that uses the Dynamic Structures package. We use the same conventions as are used in the SBML Level~3 Version~2 Core specification document.  In particular, there are different degrees of rule strictness. Formally, the differences are expressed in the statement of a rule: either a rule states that a condition \emph{must} be true, or a rule states that it \emph{should} be true. Rules of the former kind are strict SBML validation rules---a model encoded in SBML must conform to all of them in order to be considered valid.  Rules of the latter kind are consistency rules.  To help highlight these differences, we use the following three symbols next to the rule numbers:

\begin{description}

\item[\hspace*{6.5pt}\vSymbol\vsp] A \vSymbolName indicates a
  \emph{requirement} for SBML conformance. If a model does not follow
  this rule, it does not conform to the Dynamic Structures
  specification.  (Mnemonic intention behind the choice of symbol:
  ``This must be checked.'')

\item[\hspace*{6.5pt}\cSymbol\csp] A \cSymbolName indicates a
  \emph{recommendation} for model consistency.  If a model does not
  follow this rule, it is not considered strictly invalid as far as the
  Dynamic Structures specification is concerned; however, it
  indicates that the model contains a physical or conceptual
  inconsistency.  (Mnemonic intention behind the choice of symbol:
  ``This is a cause for warning.'')

\item[\hspace*{6.5pt}\mSymbol\msp] A \mSymbolName indicates a strong
  recommendation for good modeling practice.  This rule is not strictly
  a matter of SBML encoding, but the recommendation comes from logical
  reasoning.  As in the previous case, if a model does not follow this
  rule, it is not considered an invalid SBML encoding.  (Mnemonic
  intention behind the choice of symbol: ``You're a star if you heed
  this.'')
\end{description}

The validation rules listed in the following subsections are all stated
or implied in the rest of this specification document.  They are
enumerated here for convenience. Unless explicitly stated, all
validation rules concern objects and attributes specifically defined in
the Dynamic Structures package.

For \notice convenience and brevity, we use the shorthand
``\token{dyn:x}'' to stand for an attribute or element name \token{x}
in the namespace for the Dynamic Structures package, using
the namespace prefix \token{dyn}.  In reality, the prefix string may be
different from the literal ``\token{dyn}'' used here (and indeed, it
can be any valid XML namespace prefix that the modeler or software
chooses).  We use ``\token{dyn:x}'' because it is shorter than to
write a full explanation everywhere we refer to an attribute or element
in the Dynamic Structures package namespace.

\subsubsection*{General rules about the Dynamic Structures package}
\validRule{dyn-10101}{To conform to Version 1 of the Dynamic Structures package
  specification for SBML Level~3, an SBML document must declare the
  use of the following XML Namespace: \\ \textls[-25]{\uri{http://www.sbml.org/sbml/level3/version2/dyn/version1}}. (Reference: \sbmlthreedynamic, \sec{subsec:xml-namespace}.)}
  
\validRule{dyn-10102}{Wherever they appear in an SBML document,
  elements and attributes from the Dynamic Structures
  package must be declared either implicitly or explicitly to be in the
  XML namespace: \\ 
  \textls[-25]{\uri{http://www.sbml.org/sbml/level3/version2/dyn/version1}}.
  (Reference: \sbmlthreedynamic, \sec{subsec:xml-namespace}.) }

\subsubsection*{Rules for the extended \class{SBML} class} 
\validRule{dyn-10201} {In all SBML documents using the Dynamic Structures package, the \SBML object must include a value for  the attribute \token{dyn:required} attribute.  (Reference: SBML Level~3 Version~1 Core, Section~4.1.2.)}
  
\validRule{dyn-10202} {The value of attribute \token{dyn:required} on
  the \SBML object must be of the data type \primtype{boolean}.
  (Reference: SBML Level~3 Version~1 Core, Section~4.1.2.) }
  
\validRule{dyn-10203} {The value of attribute \token{dyn:required} on
  the \SBML object must be set to \val{true} (Reference: \sbmlthreedynamic, \sec{subsec:xml-namespace}.) }

 
%\subsection*{General rules about identifiers} 
%This does not need to be included since this package does not include any 

 \subsubsection*{Rules for extended \class{Event} objects } 
  
 \validRule{dyn-20101}{An \Event object must have the attributes \mbox{\token{dyn:cboTerm}} and \token{comp:applyToAll} because they are required.  No other attributes from the Dynamic Structures namespace are permitted on an \Event object. (References: \sbmlthreedynamic, \sec{subsec:extEvent}.) }
 
 \validRule{dyn-20102}{The value of an \token{dyn:cboTerm} attribute on a \Event object must always conform to the 	syntax of the \primtype{CBOTerm} data type. (References: \sbmlthreedynamic, \sec{attr:cboTerm}.)}
 
 \validRule{dyn-20103}{ The value of an \token{dyn:applyToAll} attribute on a \Event object must be of the data type \primtype{boolean}. (References: \sbmlthreedynamic, \sec{attr:applyToAll}.) }
 
 \validRule{dyn-20104}{ The value of an \token{dyn:cboTerm} attribute on a \Event object must be a full CBO term identifier taken from the CBO\textunderscore Process branch defined in CBO and supported in this package. (References: \sbmlthreedynamic, \sec{sec:CBO}.) }
 
 \validRule{dyn-20105}{There may be at most one instance of a \ListOfElements subobject within an \Event object that uses the Dynamic Structures package. (References: \sbmlthreedynamic, \sec{subsec:extEvent}.) }
 
 \validRule{dyn-20106}{The \ListOfElements subobject within an \Event object is optional, but if present, this container object must not be empty. (References: \sbmlthreedynamic, \sec{subsec:ListOfElements}.) }
 
 \validRule{dyn-20107}{Apart from the general notes and annotation subobjects permitted on all SBML objects, a \ListOfElements container object may only contain \Element objects.  (References:\sbmlthreedynamic, \sec{subsec:ListOfElements}.) }
 
 \validRule{dyn-20108}{A \ListOfElements object may have the optional attributes \token{metaid} and \token{sboTerm} defined by SBML Level~3 Core.  No other attributes from the SBML Level~3 Core namespace or the	Dynamic Structures namespace are permitted on a \ListOfElements object.  (References: SBML Level 3 Version 1 Core, Section 3.2.) }
 
 \subsubsection*{Rules for \class{Element} objects } 
 
 \validRule{dyn-20201}{An \Element object must have the attribute \token{dyn:element} because it is required.  No other attributes from the Dynamic Structures namespace are permitted on an \Element object. (References: \sbmlthreedynamic, \sec{subsec:Element}.) }

 \validRule{dyn-20202}{An \Element object may have the optional SBML Level~3 Core attributes \token{metaid} and \token{sboTerm}. (References: SBML Level 3 Version 1 Core, Section 3.2.)}
 
 \validRule{dyn-20203}{An \Element object may have the optional SBML Level 3 Core subobjects for notes and annotation. No other elements from the SBML Level 3 Core namespace are permitted on an \Element object. (References: SBML Level 3 Version 1 Core, Section 3.2.)}
 
 \validRule{dyn-20204}{The value of an \token{dyn:element} attribute on an \Element object must always conform to the syntax of the SBML data type \primtype{SIdRef}. (References: \sbmlthreedynamic, \sec{attr:element}.)}
 
 \validRule{dyn-20205}{The value of an \token{dyn:element} attribute on an \Element object must be the identifier of an SBML component whose token{id} is of type SId. (References:\sbmlthreedynamic, \sec{attr:element}.)}
 
 \subsubsection*{Rules for extended \class{Compartment} objects } 
  
 \validRule{dyn-20301}{There may be at most one instance of a \ListOfCoordinateComponents subobject within a \Compartment object that uses the Dynamic Structures package. (References: \sbmlthreedynamic, \sec{subsec:extCompartment}.) }
 
 \validRule{dyn-20302}{The \ListOfCoordinateComponents subobject within a \Compartment object is optional, but if present, this container object must not be empty. (References: \sbmlthreedynamic, \sec{subsec:extCompartment}.) }
 
 \validRule{dyn-20303}{Apart from the general notes and annotation subobjects permitted on all SBML objects, a \ListOfCoordinateComponents container object may only contain \CoordinateComponent objects. (References:\sbmlthreedynamic, \sec{subsec:listCoordComp}.)}
 
 \validRule{dyn-20304}{A \ListOfCoordinateComponents object may have the optional attributes \token{metaid} and \token{sboTerm} defined by SBML Level~3 Core.  No other attributes from the SBML Level~3 Core namespace or the	Dynamic Structures namespace are permitted on a \ListOfCoordinateComponents object. (References: SBML Level 3 Version 1 Core, Section 3.2.) }
 
 \subsubsection*{Rules for \class{CoordinateComponent} objects } 

\validRule{dyn-20401}{A \CoordinateComponent object must have the attributes \token{dyn:coordinateIndex} and token{dyn:variable} because they are required. No other attributes from the Dynamic Structures namespace are permitted on an \CoordinateComponent object. (References: \sbmlthreedynamic, \sec{subsec:coordComp}.) }

\validRule{dyn-20402}{An \CoordinateComponent object may have the optional SBML Level~3 Core attributes \token{metaid} and \token{sboTerm}. (References: SBML Level 3 Version 1 Core, Section 3.2.)}

\validRule{dyn-20403}{An \CoordinateComponent object may have the optional SBML Level 3 Core subobjects for notes and annotation. No other elements from the SBML Level 3 Core namespace are permitted on an \CoordinateComponent object. (References: SBML Level 3 Version 1 Core, Section 3.2.)}

\validRule{dyn-20404}{The value of a \token{dyn:coordinateIndex} attribute on a \CoordinateComponent object must always conform to the syntax and allowed values of the newly defined \primtype{CoordinateKind}. Permitted values for this attribute include" \val{cartesianX} , \val{cartesianY} , and \val{cartesianZ}. (References: \sbmlthreedynamic, \sec{attr:coordIndex} .)}

\consistencyRule{dyn-20405}{If a given \Compartment extends a \ListOfCoordinateComponents, which contains \CoordinateComponent subobjects to indicate location, values for their \token{coordinateIndex} attributes must be used in a specific order. For 1-dimensional cases, \val{cartesianX} is used; for 2-dimensional cases, \val{cartesianX} and \val{cartesianY} are used; and for 3-dimensional cases, \val{cartesianX} , \val{cartesianY} , and \val{cartesianZ} are used. (References: \sbmlthreedynamic, \sec{attr:coordIndex} .)}

\validRule{dyn-20406} {The value of a \token{dyn:variable} attribute on a \CoordinateComponent object must always conform to the syntax of the data type \primtype{SIdRef}. (References: \sbmlthreedynamic, \sec{attr:variable}.)}

\validRule{dyn-20407}{The value of a \token{dyn:coordinateIndex} attribute on a \CoordinateComponent object is set in a Cartesian coordinate system. (References: \sbmlthreedynamic, \sec{attr:coordIndex}.)}

\validRule{dyn-20408}{The value of a \token{dyn:variable} attribute on an \CoordinateComponent object must be the identifier of an SBML parameter. (References:\sbmlthreedynamic, \sec{attr:variable}.)}
 
{\color{red} Harold: \notice  Change references to proper sections within the L3V2 Core specification}