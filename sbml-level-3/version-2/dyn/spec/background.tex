% -*- TeX-master: "main"; fill-column: 72 -*-

\section{ Background }
\label{sec:background}

\subsection{ Problems with current SBML approaches }
\label{subsec:problems}

Representation of dynamical systems that undergo discrete structural changes during simulation cannot currently be accomplished mainly because SBML's structural constructs are fixed. This means that it is not possible to define the creation or removal of species, compartments, reactions, or other components from within a model. To simulate the creation or destruction of compartments, one has to use tricks. For example, a model could define all the compartments it could ever need and use variables to indicate which compartments are actually "active" at any given time. However, this would only work if the total number of compartments needed is known at the beginning of a simulation. This approach hard-codes the anatomical description of a model, which limits portability and becomes impractical beyond a few compartments. 

Another limitation is the lack of appropriate constructs to represent the spatial location rearrangement that follows dynamic cellular processes. It can be argued that though unavailable in Core, SBML Level~3 Version~1 extensions such as Layout and Spatial provide the necessary constructs to represent the spatial positioning of modeling elements. However, location as implemented in the Layout package only indicates how components are to be positioned only for user visualization, which may be different from where elements are during simulation. Similarly, though the Spatial package enables the spatial mapping of modeling elements, the position of mapped components is not readily accessible, so it can not be updated as cells move, die and proliferate throughout simulation.

In defense of SBML's limitation in this area, it should be pointed out that the advent of software tools supporting dynamic cellular processes was only a handful until a few years ago. It is now clear that a variety modeling problems and platforms would benefit from this capability. 

\subsection{ Past work on this problem or similar topics }
\label{subsec:pastWork}

A previous proposal for this language extension outlined the need to use compound data structures available in the Arrays package to accommodate for dynamic behavior of static SBML components. In this way, cell proliferation could be represented by adding a new cell to an array of cells, or increasing the dimension of an array of compartments. Similarly, cell death could be represented by the removal of a cell from a set. Nonetheless, dynamic creation/destruction of components using arrays proved to be impractical for large dynamic simulations. The current approach involves extending already existing SBML components and introducing new ones to emulate dynamic cellular processes as opposed to using constructs from other SBML extensions.
		

