% -*- TeX-master: "sbml-level-3-version-2-core"; fill-column: 66 -*-
% ----------------------------------------------------------------

\section{Overview of SBML}
\label{sec:overview}

The following is an example of a simple network of
biochemical reactions that can be represented in SBML:
\begin{linenomath}
  \begin{equation*}
    \begin{array}{@{}ccl@{}}
      S_1 & \overset{\underrightarrow{k_1 [S_1] / ([S_1] + k_2)}}{} & S_2 \\ \\[-5pt]
      S_2 & \overset{\underrightarrow{\rule{0.26in}{0pt}k_3 [S_2]\rule{0.26in}{0pt}}}{} & S_3 + S_4
    \end{array}
  \end{equation*}
\end{linenomath}
In this particular set of chemical equations above, the symbols in
square brackets (e.g., ``$[S_1]$'') represent concentrations of molecular
species, the arrows represent reactions, and the formulas above
the arrows represent the rates at which the reactions take place.
(And while this example uses concentrations, it could equally have used
other measures such as molecular counts.)
Broken down into its constituents, this model contains a number of
components: reactant species, product species, reactions,
reaction rates, and parameters in the rate expressions.  To analyze or
simulate this network, additional components must be made
explicit, including compartments for the species, and units on the
various quantities.

SBML allows models of arbitrary complexity to be represented.
Each type of component in a model is described using a specific
type of data object that organizes the relevant information.
The top level of an SBML model definition consists of lists of
these components, with every list being optional:

\vspace*{2ex}
\begin{center}
  \begin{edtable}{tabular}{>{\hspace*{10pt}\slshape}l>{\hspace*{40pt}}l}
    \hspace*{-10pt}beginning of model definition \\
    list of function definitions (optional)	& (Section~\ref{sec:functiondefinition}) \\
    list of unit definitions (optional)	& (Section~\ref{sec:unitdefinitions}) \\
    list of compartments (optional)	 	& (Section~\ref{sec:compartments}) \\
    list of species (optional)		& (Section~\ref{sec:species}) \\
    list of parameters (optional)		& (Section~\ref{sec:parameters}) \\
    list of initial assignments (optional)	& (Section~\ref{sec:initialAssignment}) \\
    list of rules (optional)			& (Section~\ref{sec:rules}) \\
    list of constraints (optional)		& (Section~\ref{sec:constraints}) \\
    list of reactions (optional)		& (Section~\ref{sec:reactions}) \\
    list of events (optional)			& (Section~\ref{sec:events}) \\
    \hspace*{-10pt}end of model definition \\
  \end{edtable}
\end{center}
\vspace*{2ex}

The meaning of each component is as follows:

\begin{description}
  
\item \emph{Function definition}: A named mathematical function
  that may be used throughout the rest of a model.

\item \emph{Unit definition}: A named definition of a new unit of
  measurement.  Named units can be used in the expression of
  quantities in a model.

\item \emph{Compartment}: A well-stirred container of finite size
  where species may be located.  Compartments may or may not
  represent actual physical structures.

\item \emph{Species}: A pool of entities of the same kind located
  in a compartment and participating in reactions (processes).  In
  biochemical network models, common examples of species include
  ions, proteins and other molecules; however, in practice, an
  SBML species can be any kind of entity that makes sense in the
  context of a given model.

\item \emph{Parameter}: A quantity with a symbolic name.  In SBML,
  the term \emph{parameter} is used in a generic sense to refer to
  named quantities regardless of whether they are constants or
  variables in a model.  SBML \thisL provides the ability to
  define parameters that are global to a model as well as
  parameters that are local to a single reaction.
  
\item \emph{Initial Assignment}: A mathematical
  expression used to determine the initial conditions of a
  model.  This type of object can only be used to define how
  the value of a variable can be calculated from other values
  and variables at the start of simulated time.
  
\item \emph{Rule}: A mathematical expression added to the set of
  equations constructed based on the reactions defined in a model.
  Rules can be used to define how a variable's value can be
  calculated from other variables, or used to define the rate of
  change of a variable.  The set of rules in a model can be used
  with the reaction rate equations to determine the behavior of
  the model with respect to time.  Rules constrain the model for
  the entire duration of simulated time.

\item \emph{Constraint}: A means of detecting out-of-bounds
  conditions during a dynamical simulation and optionally issuing
  diagnostic messages.  Constraints are defined by an arbitrary
  mathematical expression computing a true/false value from model
  variables, parameters and constants.  An SBML constraint applies
  at all instants of simulated time; however, the set of
  constraints in model should not be used to \emph{determine} the
  behavior of the model with respect to time.
  
\item \emph{Reaction}: A statement describing some transformation,
  transport or binding process that can change the amount of one
  or more species.  For example, a reaction may describe how
  certain entities (reactants) are transformed into certain other
  entities (products).  Reactions have associated kinetic rate
  expressions describing how quickly they take place.
  
\item \emph{Event}: A statement describing an instantaneous,
  discontinuous change in one or more variables of any type
  (species, compartment, parameter, etc.)\ when a triggering
  condition is satisfied.

\end{description}

A software package can read an SBML model description and
translate it into its own internal format for model analysis.  For
example, a package might provide the ability to simulate the model
by constructing differential equations representing the network
and then perform numerical time integration on the equations to
explore the model's dynamic behavior.  By supporting SBML as an
input and output format, different software tools can all operate
on an identical external representation of a model, removing
opportunities for errors in translation and assuring a common
starting point for analyses and simulations.
