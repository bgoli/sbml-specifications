% -*- TeX-master: "sbml-level-3-version-2-core"; fill-column: 66 -*-
% ----------------------------------------------------------------

\section{A standard format for the \token{annotation} element}
\label{sec:finney-novere}
\label{sec:annotation-standard}

This section describes the recommended non-proprietary format for
the content of \Annotation objects in SBML when (a) referring to
controlled vocabulary terms and database identifiers that define
and describe biological and biochemical entities, and (b)
describing the creator of a model and its modification history.
Such a structured format should facilitate the generation of
models compliant with the MIRIAM guidelines for model
curation~\citep{le_novere:2005}.

The format described in this section is intended to be the form of
one of the top-level elements that could reside in an \Annotation
object attached to an SBML object derived from \SBase.  The
element is named \token{rdf:RDF}.  The format described here is
compliant with the constraints placed on the form of annotation
elements described in Section~\ref{sec:annotation-use}.  We refer
readers to Section~\ref{sec:annotation-use} for important
information on the structure and organization of
application-specific annotations; these are not described here.

The annotations described in this section are optional on a model,
but if present, they must conform to the details specified here in
order to be considered valid annotations in this format.  If they
do not conform to the format described here, it does not render
the overall SBML model invalid, but the annotations are then
considered to be in a proprietary format rather than being
\emph{SBML MIRIAM annotations}.


\subsection{Motivation}

The SBML structures described elsewhere in this document do not
have any biochemical or biological semantics.  This section
provides a scheme for linking SBML structures to external
resources so that those structures can be given semantics.  The
motivation for the introduction of this scheme is similar to that
given for the introduction of \token{sboTerm}; however, the
general annotation scheme here is more flexible.

It is generally not recommended that this format be used to refer
to SBO terms.  In most cases, the SBO terms should be assigned
using the attribute \token{sboTerm} on objects derived from \SBase
(Section~\ref{sec:sboTerm}).  However in certain situations, for
instance to be able to add additional information about the
functional role of a species, it is necessary to add the
information using the annotation format described here.


\subsection{XML namespaces in the standard annotation}

This format uses a restricted form of Dublin
Core~\citep{DCMI:2003} and BioModels.net qualifier elements (see
\url{http://biomodels.net/qualifiers/}) embedded in the XML form
of RDF~\citep{w3c:2004}.  The scheme defined here uses a number of
external XML standards and associated XML namespaces.
Table~\ref{tab:namespaces-for-standard-annotation} lists these
namespaces and relevant documentation on those namespaces.  The
format constrains the order of elements in these namespaces beyond
the constraints defined in the standard definitions for those
namespaces.  For each standard listed, the format only uses a
subset of the possible syntax defined by the given standard.
Thus, it is possible for an \token{annotation} element to include
XML that is compliant with those external standards but is not
compliant with the format described here.

\begin{table}[bh]
  \small
  \centering
  \begin{edtable}{tabular}{lll}
    \toprule
    \textbf{Prefix used in}\\
    \textbf{examples here} & \textbf{Namespace URI} & \textbf{Reference/description} \\
    \midrule
    \token{rdf}     & \uri{http://www.w3.org/1999/02/22-rdf-syntax-ns\#} & \cite{w3c:2004b} \\[3pt]
    \token{\changed{dcterms}} & \uri{http://purl.org/dc/terms/} & \cite{kokkelink:2002},\\
                    &                                 & \cite{DCMIUB:2005} \\[3pt]
    \token{vcard}   & \uri{http://www.w3.org/2001/vcard-rdf/3.0\#} & \cite{iannella:2001} \\[3pt]
    \token{\changed{vcard4}}  & \changed{\uri{http://www.w3.org/2006/vcard/ns\#}} & \changed{\cite{perreault:2011}} \\[3pt]
    \token{bqbiol}  & \uri{http://biomodels.net/biology-qualifiers/} & \url{http://sbml.org/miriam/qualifiers/} \\[3pt]
    \token{bqmodel} & \uri{http://biomodels.net/model-qualifiers/} & \url{http://sbml.org/miriam/qualifiers/} \\[3pt]
    \bottomrule
  \end{edtable}
  \vspace*{-0.95ex}
  \caption{The XML standards used in the SBML standard format for annotations.
    The namespace prefixes are only shown to indicate the prefix
    used in the main text; the prefixes are not required to be the
    specific strings shown here.}
  \label{tab:namespaces-for-standard-annotation}
\end{table}


% The default page break is too awkward here.
\clearpage

%=============================================================================
\subsection{General syntax for the standard annotation}
\label{sec:general-syntax}
%=============================================================================

This standard format for an SBML annotation is placed in a single
\token{rdf:RDF} element contained within the SBML
\token{annotation} element.  It can contain other elements in any
order as described in Section~\ref{sec:annotation-use}.  The
format described in this section only defines the form of the
\token{rdf:RDF} element.  The containing SBML \SBase element must
have a \token{metaid} attribute value (and note that, because it
is of XML type \primtype{ID}, its value must be unique to the
entire SBML document).  An outline of the format's syntax is shown
below.

\newcommand{\sbmlElementNS}    {\colorbox[gray]{.4}{\texttt{\color{white}{SBML\_ELEMENT}}}}
\newcommand{\sbmlElement}      {\sbmlElementNS\xspace}
\newcommand{\otherContentNS}   {\colorbox[rgb]{.82,.9,.82}{\texttt{+++}}}
\newcommand{\otherContent}     {\otherContentNS\xspace}
\newcommand{\otherContentTall} {{\setlength{\fboxsep}{3.5pt}{\scalefont{1.2}\otherContent}}}
\newcommand{\moreOfTheSameNS}  {\colorbox[rgb]{.95,.8,.8}{\texttt{...}}}
\newcommand{\moreOfTheSame}    {\moreOfTheSameNS\xspace}
\newcommand{\relationElementNS}{\texttt{\textsl{RELATION\_ELEMENT}}}
\newcommand{\relationElement}  {\relationElementNS\xspace}
\newcommand{\sbmlMetaNS}       {\underline{\texttt{SBML\_META\_ID}}}
\newcommand{\sbmlMeta}         {\sbmlMetaNS\xspace}
\newcommand{\historyNS}        {\colorbox[gray]{.7}{\texttt{[HISTORY]}}}
\newcommand{\history}          {\historyNS\xspace}
\newcommand{\resourceURINS}    {\colorbox[rgb]{.8,.8,1}{\underline{\texttt{URI}}}}
\newcommand{\resourceURI}      {\resourceURINS\xspace}
\newcommand{\nestedContentNS}  {\colorbox[gray]{.7}{\texttt{[NESTED CONTENT]}}}
\newcommand{\nestedContent}    {\nestedContentNS\xspace}

\vspace*{-0.5ex}
\begin{example}
<\sbmlElement \otherContentTall metaid="\sbmlMetaNS" \otherContentTall>
  \otherContent
  <annotation>
    \otherContent
    <rdf:RDF xmlns:rdf="http://www.w3.org/1999/02/22-rdf-syntax-ns\#"
             \changed{xmlns:dcterms="http://purl.org/dc/terms/"}
             xmlns:vcard="http://www.w3.org/2001/vcard-rdf/3.0\#"
             xmlns:bqbiol="http://biomodels.net/biology-qualifiers/"
             xmlns:bqmodel="http://biomodels.net/model-qualifiers/" >\vspace*{0.5ex}
      <rdf:Description rdf:about="#\sbmlMetaNS">
        \history
        <\relationElementNS>
          <rdf:Bag>
            <rdf:li rdf:resource="\resourceURINS" />
            \moreOfTheSame
            \nestedContent
          </rdf:Bag>
        </\relationElementNS>
        \moreOfTheSame
      </rdf:Description>
      \otherContent
    </rdf:RDF>
    \otherContent
  </annotation>
  \otherContent
</\sbmlElementNS>
\end{example}
\vspace*{-0.5ex}

The outline above shows the expected order of the elements.  The
capitalized identifiers refer to generic strings of specific
types, as follows: \sbmlElement refers to any SBML element name
that can contain an \token{annotation} element; \sbmlMeta is an
XML \primtype{ID} string; \relationElement refers to element names
in either the namespace
{\scalefont{0.98}\uri{http://biomodels.net/biology-qualifiers/}}
or {\scalefont{0.98}\uri{http://biomodels.net/model-qualifiers/}}
(see Section~\ref{sec:qualified-dc-annotation}); \resourceURI is a
URI identifying a resource (see
Section~\ref{sec:uri-in-annotation}); and \history refers to
optional content described in
Section~\ref{sec:model-history-annotation}.
\changed{\nestedContent refers to other optional RDF elements such as \token{bqbiol:isDescribedBy} that describe a clarification or another annotation about the \relationElement in which it appears.  Nested content allows one to, for example, describe protein modifications on species, or to add evidence codes for an annotation.}
The string
`\otherContent' is a placeholder for either no content or valid
XML content that is not defined by the annotation scheme described
here but is consistent with the relevant standards for the
enclosing elements.  Finally, the string `\moreOfTheSame' is a
placeholder for zero or more elements of the same form as the
immediately preceding element.  The precise form of whitespace and
the XML namespace prefix definitions is not constrained; however,
the elements and attributes must be in the namespaces shown. The
rest of this section describes the format formally in English.

The first element of the \token{rdf:RDF} element must be an
\token{rdf:Description} element with an \token{rdf:about}
attribute. The value of the
\token{rdf:about} attribute must be of the form
\texttt{\#<string>} where the string component is equal to the
value of the \token{metaid} attribute of the containing SBML element.
This format doesn't define the form of subsequent
subelements of the \token{rdf:RDF} element. In particular, the unique \token{rdf:RDF} element contained in the annotation can contain other \token{rdf:Description}, which content can be any valid RDF.

The \token{rdf:Description} element can contain only an optional
history section (see Section~\ref{sec:model-history-annotation})
followed by zero or more BioModels.net relation elements.  The
specific relation elements used will depend on the intended
relationship between the SBML component and referenced information
or resource.  Although Section~\ref{sec:qualified-dc-annotation}
describes the detailed semantics of each of the relation element
types, the content of these elements follows the same form shown
in the template above.  A BioModels.net relation element must only
contain a single \token{rdf:Bag} element which in turn must 
contain one or more \token{rdf:li} elements, \changed{and may additionally contain nested content that provides additional annotations about the contents of the \token{rdf:Bag}}.
The \token{rdf:li}
elements must only have a \token{rdf:resource} attribute
containing a URI referring to an information resource (see
Section~\ref{sec:uri-in-annotation}).

\changed{Note that the various namespaces (\token{xmlns:rdf}, \token{xmlns:dcterms}, etc.) may be declared in any order, and that only the namespaces that are actually used need be declared.  If no \token{vcard} terms are used in a particular annotation, for example, the line \token{xmlns:vcard="http://www.w3.org/2001/vcard-rdf/3.0\#"} is optional.}

%Annotations in this format can be located at different depths
%within a model component.

%LS DEBUG:  I have no idea why the following 'clearpage' is required, but without it, compilation hangs.  It's the weirdest thing.
%\clearpage

%=============================================================================
\subsection{Use of URIs}
\label{sec:uri-in-annotation}
%=============================================================================

The SBML MIRIAM annotation format allows the expression of
relationships between SBML elements on the one hand, and resources
referred to by values of \token{rdf:resource} attributes on the
other.  The BioModels.net relation elements (see
Section~\ref{sec:qualified-dc-annotation}) simply define the
nature of the relationship.

The value of a \token{rdf:resource} attribute is a URI that
uniquely identifies both the resource and the data within the
resource.  Since a URI is not a URL, it does not have to map to a
physical Web object; it simply needs to identify, uniquely, a
controlled vocabulary term or database object.  It is essentially
just a label.  For instance, an actual URL for an Internet
resource might be \uri{http://www.uniprot.org/entry/P12999}, and
this might correspond to the URI \changed{\uri{http://identifiers.org/uniprot/P12999}}.

It is important that the portion of a \token{rdf:resource} value
that identifies a data entry is always a perennial identifier.
For example, a \Species object representing a protein could be
annotated with a reference to the database UniProt by the resource
identifier \changed{\uri{http://identifiers.org/uniprot/P12999}}, thereby identifying
exactly the intended protein.  This identifier maps to a unique
entry in UniProt which is never deleted from the database.  In the
case of UniProt, this is known as the ``accession'' portion of the
entry.  When the entry is merged with another one, both
``accession'' entries are conserved.  A UniProt entry also
possesses an ``entry name'' (the Swiss-Prot ``identifier''), a
``protein name'', ``synonyms'', and other parts, but only the
``accession'' is perennial, and that is what should be used.

SBML does not define how to interpret URIs.  There may be several
ways of transforming a URI into a physical URL.  For example,
\changed{\uri{http://identifiers.org/go/GO:0007268}} can be translated into any
of the following:
\begin{itemize}\setlength{\parskip}{-0.7ex}

\item \uri{http://www.ebi.ac.uk/ego/GTerm?id=GO:0007268}
\item \uri{http://www.godatabase.org/cgi-bin/amigo/go.cgi?view=details\&query=GO:0007268}
\item \uri{http://www.informatics.jax.org/searches/GO.cgi?id=GO:0007268}

\end{itemize}
\vspace*{-1ex}

Similarly, the URI \changed{\uri{http://identifiers.org/ec-code/3.5.4.4}} can refer to
any of the following (among many):
\begin{itemize}\setlength{\parskip}{-0.7ex}

\item \uri{http://www.ebi.ac.uk/intenz/query?cmd=SearchEC\&ec=3.5.4.4}
\item \uri{http://www.genome.jp/dbget-bin/www\_bget?ec:3.5.4.4}
\item \uri{http://us.expasy.org/cgi-bin/nicezyme.pl?3.5.4.4}

\end{itemize}\vspace*{-1ex}

To enable interoperability of URIs between software systems, the
community has standardized the URI rules for use within the SBML
MIRIAM annotation format.  These URIs are not part of the SBML
standard per se, and will grow independently from specific SBML
levels and versions.  As the set changes, existing URIs will not
be modified, although the physical resources associated with each
one may change (for example, to use updated URLs).  The form of
the URIs will always have the form \emph{resource:identifier}.  An
up-to-date list and explanations of the URIs are available online
at the address \url{http://biomodels.net/qualifiers}.  Each
entry lists the relation elements in which the given URI can be
appropriately embedded.  The URI rule list will evolve with the
evolution of databases and resources.

Note this means that all \token{rdf:resource} \emph{must} be
MIRIAM URIs and thus cannot refer to, for example, other elements
in the model.  While it would be possible to place such
information in RDF content elsewhere (e.g., after the first
\token{rdf:Description} element), the information will be outside
the scope of the simple annotation scheme described here, and as
such, there is no guarantee that other software could understand
it.

\begin{table}[b]
  \small
  \centering
  \setlength{\tabcolsep}{4pt}
  \begin{edtable}{tabular}{lp{4.84in}}
    \toprule
    \textbf{Qualifier element} & \textbf{Meaning} \\
    \midrule
    \token{bqmodel:is}
    & The modeling object encoded by the SBML
    component is the subject of the referenced resource.  This
    might be used, e.g., to link the model to an entry in a
    model database.
    \\[7pt]
    \token{bqmodel:isDerivedFrom}
    & The modeling object
    represented by the component of the encoded model is derived
    from the modeling object represented by the referenced resource.
    For instance, they can be the fruit of a refinement or their
    adaptation for use in a different context.
    \\[7pt]
    \token{bqmodel:isDescribedBy}
    & The modeling object
    encoded by the SBML component is described by
    the referenced resource. This could link a component (e.g., a
    reaction) to a publication describing it.
    \\[4pt]
    \midrule
    \token{bqbiol:encodes}
    & The biological entity represented by
    the model component encodes, either directly or by virtue of
    transitivity, the subject of the referenced resource.
    \\[7pt]
    \token{bqbiol:hasPart}
    & The biological entity represented by
    the SBML component includes the subject of the referenced
    resource, either physically or logically. This relation might be
    used to link a complex to a description of its components.
    \\[7pt]
    \token{bqbiol:hasProperty}
    & The subject of the referenced
    resource is a property of the biological entity represented by
    the model component.  This relation might be used when a
    biological entity has a given activity or exerts a specific
    function.
    \\[7pt]
    \token{bqbiol:hasVersion}
    & The subject of the referenced
    resource is a version or an instance of the biological entity
    represented by the SBML component.
    \\[7pt]
    \token{bqbiol:is}
    & The biological entity represented by the
    SBML component is the subject of the referenced resource. This
    could serve to link a reaction to its counterpart in (e.g.)\
    the \changed{CHEBI} or Reactome databases.
    \\[7pt]
    \token{bqbiol:isDescribedBy}
    & The biological entity
    represented by the SBML component is described by the referenced
    resource.  This relation could be used, for example, to link a
    species or a parameter to a publication describing the
    quantity of the species or the value of the parameter.
    \\[7pt]
    \token{bqbiol:isEncodedBy}
    & The biological entity represented
    by the model component is encoded, either directly or by virtue
    of transitivity, by the subject of the referenced resource.
    \\[7pt]
    \token{bqbiol:isHomologTo}
    & The biological entity represented
    by the SBML component is homolog, to the subject of the
    referenced resource, i.e., they share a common ancestor.
    \\[7pt]
    \token{bqbiol:isPartOf}
    & The biological entity represented by
    the SBML component is a physical or logical part of the subject
    of the referenced resource. This relation might be used to link
    a component to the description of the complex to which it belongs.
    \\[7pt]
    \token{bqbiol:isPropertyOf}
    & The biological entity
    represented by the SBML component is a property of the
    referenced resource.
    \\[7pt]
    \token{bqbiol:isVersionOf}
    & The biological entity represented
    by the SBML component is a version or an instance of the subject
    of the referenced resource.
    \\[7pt]
    \token{bqbiol:occursIn}
    & The biological entity represented by
    the \changed{SBML} component takes place in the subject of the reference
    resource.
    \\[7pt]
    \token{\changed{bqbiol:hasTaxon}}
    & \changed{The biological entity represented by the SBML component is taxonomically restricted, where the restriction is the subject of the referenced resource (biological entity B). This relation may be used to ascribe a species restriction to a biochemical reaction.}
    \\[3pt]
    \bottomrule
  \end{edtable}
  \caption{BioModels.net qualifiers at the time of this writing, and a
    summary of their meanings.  The complete list of the qualifier
    elements is documented online at
    \url{http://biomodels.net/qualifiers/}.  (The
      definitions given above are slightly modified compared
      to the originals, to reflect the SBML-specific 
      nature of this SBML specification document.)} 
  \label{tab:miriam-qualifiers}
\end{table}


%=============================================================================
\subsection{Relation elements}
\label{sec:qualified-dc-annotation}
%=============================================================================

Different BioModels.net qualifier elements encode different types
of relationships.  As described above, when appearing in an
annotation, each qualifier element encloses a set of
\token{rdf:li} elements.  Its appearance in a relation element
implies a specific relationship between the enclosing SBML object
and the resources referenced by the \token{rdf:li} elements.
When several relation elements with the same name are placed in
the same SBML element's annotation, they represent alternatives.
For example, two \token{bqbiol:hasPart} elements within a \Species
object represent two different sets of references to the parts
making up the biological entity represented by the species.  (The
species is not made up of \emph{all} the entities represented by
all the references combined; they are alternatives.)

Table~\vref{tab:miriam-qualifiers} lists the elements defined at
the time of this writing.  The list is divided into two symbol
namespaces.  One is for model qualifiers, and this one has the URI
\uri{http://biomodels.net/model-qualifiers/} (for which we use the
prefix \token{bqmodel} in examples shown in this section).  The
other namespace is for biological qualifiers; this has the URI
\uri{http://biomodels.net/biology-qualifiers/} (for which we use
the prefix \token{bqbiol}).  The list will only grow; \ie no
element will be removed from the list.

%===============================================================
\subsection{History}
\label{sec:model-history-annotation}
%================================================================

The SBML MIRIAM annotation format described in
Section~\ref{sec:general-syntax} can include additional elements
to describe the history of the \emph{SBML encoding of the model}
or its individual components.  (Note the emphasis on the SBML
encoding---the history of the conceptual model underlying the
encoding is not addressed by this scheme.)  If this history data
is present, it must occur immediately before the first
BioModels.net relation elements of an annotation.  The history
encodes information about the creator(s) of the encoding and a
sequence of dates recording the dates of creation and subsequent
modifications of the SBML model encoding.  The syntax for these
elements is outlined below.

\newcommand{\familyNameNS}  {\colorbox[gray]{.4}{\texttt{\color{white}{FAMILY\_NAME}}}}
\newcommand{\familyName}    {\familyNameNS\xspace}
\newcommand{\givenNameNS}   {\colorbox[gray]{.4}{\texttt{\color{white}{GIVEN\_NAME}}}}
\newcommand{\givenName}     {\givenNameNS\xspace}
\newcommand{\emailAddressNS}{\colorbox[gray]{.4}{\texttt{\color{white}{EMAIL\_ADDRESS}}}}
\newcommand{\emailAddress}  {\emailAddressNS\xspace}
\newcommand{\orgNameNS}     {\colorbox[gray]{.4}{\texttt{\color{white}{ORGANIZATION\_NAME}}}}
\newcommand{\orgName}       {\orgNameNS\xspace}
\newcommand{\historyDateNS} {\colorbox[gray]{.4}{\texttt{\color{white}{DATE}}}}
\newcommand{\historyDate}   {\historyDateNS\xspace}
\newcommand{\fullNameNS}    {\colorbox[gray]{.4}{\texttt{\color{white}{FULL\_NAME}}}}
\newcommand{\fullName}      {\fullNameNS\xspace}

\vspace*{1ex}
\begin{example}
<\changed{dcterms}:creator>
  <rdf:Bag>
    <rdf:li rdf:parseType="Resource">
\colorbox[gray]{.94}{\begin{minipage}{3.75in}\begin{example}
\otherContent
<vCard:N rdf:parseType="Resource">
  <vCard:Family>\familyNameNS</vCard:Family>
  <vCard:Given>\givenNameNS</vCard:Given>
</vCard:N>
\otherContent
[<vCard:EMAIL>\emailAddressNS</vCard:EMAIL>]
\otherContent
[<vCard:ORG rdf:parseType="Resource" >
  <vCard:Orgname>\orgNameNS</vCard:Orgname>
</vCard:ORG>]
\otherContent
\end{example}\end{minipage}}
    </rdf:li>
    \moreOfTheSame
  </rdf:Bag>
</\changed{dcterms}:creator>
<dcterms:created rdf:parseType="Resource">
  <dcterms:W3CDTF>\historyDateNS</dcterms:W3CDTF>
</dcterms:created>
<dcterms:modified rdf:parseType="Resource">
  <dcterms:W3CDTF>\historyDateNS</dcterms:W3CDTF>
</dcterms:modified>
\moreOfTheSame
\end{example}
\vspace*{1ex}

\begin{blockChanged}
Alternatively, the \token{vCard} section above in the light gray box may be replaced by the following, using \token{vCard4}:

\vspace*{1ex}
\begin{example}
\colorbox[gray]{.94}{\begin{minipage}{5.1in}\begin{example}
\otherContent
<vCard4:FN>\fullNameNS</vCard4:FN>
\otherContent
[<vCard4:email>\emailAddressNS</vCard4:email>]
\otherContent
[<vCard4:ORG rdf:parseType="Resource" >
  <vCard4:organization-name>\orgNameNS</vCard4:organization-name>
</vCard4:ORG>]
\otherContent
\end{example}\end{minipage}}
\end{example}
\vspace*{1ex}

\end{blockChanged}


The order of elements must be as shown above, except that elements
of the format contained in the light gray box \changed{(\token{vCard:N}, \token{vCard:EMAIL}, and \token{vCard:ORG} for \token{vCard} version 3, and \token{vCard4:FN}, \token{vCard4:email}, and \token{vCard:ORG} for \token{vCard} version 4)} can occur in any
order.  The elements of the format contained between \texttt{[}
and \texttt{]} \changed{(\token{vCard:EMAIL}, \token{vCard:ORG}, \token{vCard4:email}, and \token{vCard4:ORG})} are optional, \changed{but everything else is required}.  The precise form of the whitespace,
and the specific XML namespace prefixes used \changed{(\val{dcterms}, \val{rdf}, \val{vCard}, and \val{vCard4})} are not constrained.

The \token{\changed{dcterms}:creator} element describes the person(s) who created
the SBML encoding of the model or model component.  It must
contain a single \token{rdf:Bag} element.  The \token{rdf:Bag}
element can contain any number of elements; however, the first
such element must be \token{rdf:li}.  The \token{rdf:li} element
can in turn contain any number of elements in any order.  Among
those elements can be the following: \token{vCard:N},
\token{vCard:EMAIL} and \token{vCard:ORG} (if using \token{vCard} version 3), or \token{vCard4:FN}, \token{vCard4:email}, and \token{vCard4:ORG} (if using \token{vCard} version 4).
The \token{vCard:N}, \changed{\token{vCard:ORG}, \token{vCard4:ORG},}
\token{dcterms:created}, and \token{dcterms:modified} elements
must have the attribute \token{rdf:parseType} set to the literal
value \val{Resource}.

\begin{blockChanged}
Note that \token{dcterms:creator} has been added to the \token{http://purl.org/dc/terms/} namespace relatively recently, but the same term (with the same meaning) once only lived in the \token{http://purl.org/dc/elements/1.1/} namespace.  It is legal to continue using the old namespace (called \val{dc} in previous versions of the SBML specifications), but as all the terms once defined there are now also defined in \token{http://purl.org/dc/terms/}, it is now recommended to just use the latter.
\end{blockChanged}

The content placeholders \familyName and \givenName stand for the
family name (surname) and the first (given) name, respectively, of
a person who created the model; \changed{when using \token{vCard4}, \fullName stands for the full name of that person.}  \emailAddress is the email address
of the same person who created the model; and \orgName is the name
of the organization with which the same person who created the
model is affiliated.  The string \historyDate is a date in W3C
date format~\citep{wolf:1998}, which is a profile of (i.e., a
restricted form of) ISO 8601.  Finally, as in the overall template
shown in Section~\ref{sec:general-syntax}, `\otherContent' is
a placeholder for either no content or valid XML syntax that is
not defined by this scheme but is consistent with the relevant
standards for the enclosing elements, and ellipses
(`\moreOfTheSame') are placeholders for zero or more elements of
the same form as the immediately preceding element.

Section~\ref{sec:annotation-examples} below provides an example of
using these history elements in the SBML MIRIAM annotation format.


%================================================================
\subsection{Examples}
\label{sec:annotation-examples}
%=================================================================

The following shows the annotation of a model with model creation
data and links to external resources:

\begin{example}
<model metaid="_180340" id="GMO" name="Goldbeter1991_MinMitOscil">
    <annotation>
        <rdf:RDF
                xmlns:rdf="http://www.w3.org/1999/02/22-rdf-syntax-ns\#"
                \changed{xmlns:dcterms="http://purl.org/dc/terms/"}
                xmlns:vCard="http://www.w3.org/2001/vcard-rdf/3.0\#"
                xmlns:bqbiol="http://biomodels.net/biology-qualifiers/"
                xmlns:bqmodel="http://biomodels.net/model-qualifiers/"
        >
            <rdf:Description rdf:about="#_180340">
                <\changed{dcterms}:creator>
                    <rdf:Bag>
                        <rdf:li rdf:parseType="Resource">
                            <vCard:N rdf:parseType="Resource">
                                <vCard:Family>Shapiro</vCard:Family>
                                <vCard:Given>Bruce</vCard:Given>
                            </vCard:N>
                            <vCard:EMAIL>bshapiro@jpl.nasa.gov</vCard:EMAIL>
                            <vCard:ORG rdf:parseType="Resource">
                                <vCard:Orgname>NASA Jet Propulsion Laboratory</vCard:Orgname>
                            </vCard:ORG>
                        </rdf:li>
                    </rdf:Bag>
                </\changed{dcterms}:creator>
                <dcterms:created rdf:parseType="Resource">
                    <dcterms:W3CDTF>2005-02-06T23:39:40+00:00</dcterms:W3CDTF>
                </dcterms:created>
                <dcterms:modified rdf:parseType="Resource">
                    <dcterms:W3CDTF>2005-09-13T13:24:56+00:00</dcterms:W3CDTF>
                </dcterms:modified>
                <bqmodel:is>
                    <rdf:Bag>
                        <rdf:li rdf:resource="\changed{http://identifiers.org/biomodels.db/BIOMD0000000003}\!"/>
                    </rdf:Bag>
                </bqmodel:is>
                <bqmodel:isDescribedBy>
                     <rdf:Bag>
                         <rdf:li rdf:resource="\changed{http://identifiers.org/pubmed/1833774}\!"/>
                     </rdf:Bag>
                </bqmodel:isDescribedBy>
                <bqbiol:isVersionOf>
                    <rdf:Bag>
                        <rdf:li rdf:resource="\changed{http://identifiers.org/wikipathways/WP179}\!"/>
                        <rdf:li rdf:resource="\changed{http://identifiers.org/reactome:REACT_152}\!"/>
                    </rdf:Bag>
                </bqbiol:isVersionOf>
            </rdf:Description>
        </rdf:RDF>
    </annotation>
</model>
\end{example}


The following example shows a \Reaction object annotated with a
reference to its exact Reactome counterpart.

\begin{example}
<reaction id="cdc2Phospho" metaid="jb007" reversible="true" fast="false">
  <annotation>
    <rdf:RDF
      xmlns:bqbiol="http://biomodels.net/biology-qualifiers/"
      xmlns:rdf="http://www.w3.org/1999/02/22-rdf-syntax-ns\#"
    >
      <rdf:Description rdf:about="#jb007">
        <bqbiol:is>
          <rdf:Bag>
            <rdf:li rdf:resource="\changed{http://identifiers.org/reactome/REACT_6327}\!"/>
          </rdf:Bag>
        </bqbiol:is>
      </rdf:Description>
    </rdf:RDF>
  </annotation>
  <listOfReactants>
    <speciesReference species="cdc2" stoichiometry="1"/>
  </listOfReactants>
  <listOfProducts>
    <speciesReference species="cdc2-Y15P" stoichiometry="1"/>
  </listOfProducts>
  <listOfModifiers>
    <modifierSpeciesReference species="wee1"/>
  </listOfModifiers>
</reaction>
\end{example}

The following example describes a species that represents a
complex between the protein calmodulin and calcium ions:

\begin{example}
<species id="Ca_calmodulin" metaid="cacam" compartment="C"
         hasOnlySubstanceUnits="false" boundaryCondition="false"
         constant="false">
  <annotation>
    <rdf:RDF
      xmlns:rdf="http://www.w3.org/1999/02/22-rdf-syntax-ns\#"
      xmlns:bqbiol="http://biomodels.net/biology-qualifiers/"
    >
      <rdf:Description rdf:about="\#cacam">
        <bqbiol:hasPart>
          <rdf:Bag>
            <rdf:li rdf:resource="\changed{http://identifiers.org/uniprot/P62158}\!"/>
            <rdf:li rdf:resource="\changed{http://identifiers.org/chebi/CHEBI:29108}\!"/>
          </rdf:Bag>
        </bqbiol:hasPart>
      </rdf:Description>
    </rdf:RDF>
  </annotation>
</species>
\end{example}

The following example describes a species that represents either
``Calcium/calmodulin-dependent protein kinase type II alpha
chain'' or ``Calcium/calmodulin-dependent protein kinase type II
beta chain''. This is the case, for example, in the somatic
cytoplasm of striatal medium-size spiny neurons, where both are
present but they cannot be functionally differentiated.


\begin{example}
<species id="calcium_calmodulin" metaid="cacam" compartment="C"
         hasOnlySubstanceUnits="false" boundaryCondition="false"
         constant="false">
  <annotation>
    <rdf:RDF
      xmlns:rdf="http://www.w3.org/1999/02/22-rdf-syntax-ns\#"
      xmlns:bqbiol="http://biomodels.net/biology-qualifiers/"
    >
      <rdf:Description rdf:about="\#cacam">
        <bqbiol:hasVersion>
          <rdf:Bag>
            <rdf:li rdf:resource="\changed{http://identifiers.org/uniprot/Q9UQM7}\!"/>
            <rdf:li rdf:resource="\changed{http://identifiers.org/uniprot/Q13554}\!"/>
          </rdf:Bag>
        </bqbiol:hasVersion>
      </rdf:Description>
    </rdf:RDF>
  </annotation>
</species>
\end{example}


The above approach should not be used to describe ``any
Calcium/calmodulin-dependent protein kinase type II chain'',
because such an annotation requires referencing the products of
other genes such as gamma or delta. All the known proteins could
be enumerated, but such an approach would almost surely lead to
inaccuracies because biological knowledge continues to evolve.
Instead, the annotation should refer to generic information such
as Ensembl family ENSFM00250000000111 ``CALCIUM/CALMODULIN DEPENDENT
KINASE TYPE II CHAIN'' or PIR superfamily PIRSF000594
``Calcium/calmodulin-dependent protein kinase type~II''.

%While with \texttt{HasVersion}, the described component could
%represent several alternative, with \texttt{isVersionOf} the
%described component is one of the alternative understated by the
%referenced resource.

The following two examples show how to use the qualifier
\token{isVersionOf}. The first example is the relationship
between a reaction and an EC code. An EC code describes an
enzymatic activity and an enzymatic reaction involving a
particular enzyme can be seen as an instance of this activity. For
example, the following reaction represents the phosphorylation of
a glutamate receptor by a complex calcium/calmodulin kinase II.

\begin{example}
<reaction id="NMDAR_phosphorylation" metaid="thx1138"
          reversible="true" fast="false">
  <annotation>
    <rdf:RDF
      xmlns:bqbiol="http://biomodels.net/biology-qualifiers/"
      xmlns:rdf="http://www.w3.org/1999/02/22-rdf-syntax-ns\#"
    >
      <rdf:Description rdf:about="#thx1138">
        <bqbiol:isVersionOf>
          <rdf:Bag>
            <rdf:li rdf:resource="\changed{http://identifiers.org/ec-code/2.7.1.17}\!"/>
          </rdf:Bag>
        </bqbiol:isVersionOf>
      </rdf:Description>
    </rdf:RDF>
  </annotation>
  <listOfReactants>
    <speciesReference species="NMDAR" stoichiometry="1"/>
  </listOfReactants>
  <listOfProducts>
    <speciesReference species="P-NMDAR" stoichiometry="1"/>
  </listOfProducts>
  <listOfModifiers>
    <modifierSpeciesReference species="CaMKII"/>
  </listOfModifiers>
  <kineticLaw>
    <math xmlns="http://www.w3.org/1998/Math/MathML">
      <apply>
        <times/>
        <ci>CaMKII</ci>
        <ci>kcat</ci>
        <apply>
          <divide/>
          <ci>NMDAR</ci>
          <apply> </times> <ci>NMDAR</ci> <ci>Km</ci> </apply>
        </apply>
      </apply>
    </math>
    <listOfLocalParameters>
      <localParameter id="kcat" value="1"/>
      <localParameter id="Km" value="5e-10"/>
    </listOfLocalParameters>
  </kineticLaw>
</reaction>
\end{example}

The second example of the use of \token{isVersionOf} is the
complex between Calcium/calmodulin-dependent protein kinase type
II alpha chain and Calcium/calmodulin, that is only one of the
``calcium- and calmodulin-dependent protein kinase complexes''
described by the Gene Ontology term GO:0005954.  (Note also how
the GO identifier is written---we return to this point below.)

\begin{example}
<species id="CaCaMKII" metaid="C8H10N4O2" compartment="C"
         hasOnlySubstanceUnits="false" boundaryCondition="false"
         constant="false">
  <annotation>
    <rdf:RDF
      xmlns:rdf="http://www.w3.org/1999/02/22-rdf-syntax-ns\#"
      xmlns:bqbiol="http://biomodels.net/biology-qualifiers/"
    >
      <rdf:Description rdf:about="\#C8H10N4O2">
        <bqbiol:isVersionOf>
          <rdf:Bag>
            <rdf:li rdf:resource="\changed{http://identifiers.org/go/GO:0005954}\!"/>
          </rdf:Bag>
        </bqbiol:isVersionOf>
      </rdf:Description>
    </rdf:RDF>
  </annotation>
</species>
\end{example}

In the example above, the URN for the GO term is written as
\token{GO\%3A0005954}, yet in reality, the actual GO identifier is
\token{GO:0005954}.  The reason for this rests in the definition
of RDF/XML and URNs.  The essential point is that the colon
character (\val{:}) is a reserved character representing the
component separator in URNs.  Thus, when an identifier contains a
colon character as part of it (as GO, ChEBI, and certain other
identifiers do), the colon characters must be percent-encoded.
The sequence \val{\%3A} is the percent-encoded form of \val{:}.
Applications must percent-encode \val{:} characters that appear in
entity identifiers (whether from ECO, ChEBI, GO, or other) when
writing them in MIRIAM URIs, and percent-decode the identifiers
when reading the URIs.  More examples of this appear throughout
the rest of this section.

The previous case is different from the following one, although they
could seem similar at first sight. The
``Calcium/calmodulin-dependent protein kinase type II alpha
chain'' is a part of the above mentioned ``calcium- and
calmodulin-dependent protein kinase complex''.

\begin{example}
<species id="CaMKIIalpha" metaid="C10H14N2" compartment="C"
         hasOnlySubstanceUnits="false" boundaryCondition="false"
         constant="false">
  <annotation>
    <rdf:RDF
      xmlns:rdf="http://www.w3.org/1999/02/22-rdf-syntax-ns\#"
      xmlns:bqbiol="http://biomodels.net/biology-qualifiers/"
    >
      <rdf:Description rdf:about="\#C10H14N2">
        <bqbiol:isPartOf>
          <rdf:Bag>
            <rdf:li rdf:resource="\changed{http://identifiers.org/go/GO:0005954}\!"/>
          </rdf:Bag>
        </bqbiol:isPartOf>
      </rdf:Description>
    </rdf:RDF>
  </annotation>
</species>
\end{example}

It is possible describe a component with several alternative sets
of qualified annotations. For example, the following species
represents a pool of  GMP, GDP and GTP. We annotate it with the
three corresponding KEGG compound identifiers but also with the
three corresponding ChEBI identifiers.  The two alternative
annotations are encoded in separate \token{hasVersion} qualifier
elements.

\begin{example}
<species id="GXP" metaid="GXP" compartment="C"
         hasOnlySubstanceUnits="false" boundaryCondition="false"
         constant="false">
  <annotation>
    <rdf:RDF
      xmlns:rdf="http://www.w3.org/1999/02/22-rdf-syntax-ns\#"
      xmlns:bqbiol="http://biomodels.net/biology-qualifiers/"
    >
      <rdf:Description rdf:about="\#GXP">
        <bqbiol:hasVersion>
          <rdf:Bag>
            <rdf:li rdf:resource="\changed{http://identifiers.org/chebi/CHEBI:17345}\!"/>
            <rdf:li rdf:resource="\changed{http://identifiers.org/chebi/CHEBI:17552}\!"/>
            <rdf:li rdf:resource="\changed{http://identifiers.org/chebi/CHEBI:17627}\!"/>
          </rdf:Bag>
        </bqbiol:hasVersion>
        <bqbiol:hasVersion>
          <rdf:Bag>
            <rdf:li rdf:resource="\changed{http://identifiers.org/kegg.compound/C00035}\!"/>
            <rdf:li rdf:resource="\changed{http://identifiers.org/kegg.compound/C00044}\!"/>
            <rdf:li rdf:resource="\changed{http://identifiers.org/kegg.compound/C00144}\!"/>
          </rdf:Bag>
        </bqbiol:hasVersion>
      </rdf:Description>
    </rdf:RDF>
  </annotation>
</species>
\end{example}

The following example presents a reaction being actually the
combination of three different elementary molecular reactions. We
annotate it with the three corresponding KEGG reactions, but also
with the three corresponding enzymatic activities.  Again the two
\token{hasPart} elements represent two alternative annotations.
The process represented by the \Reaction structure is
composed of three parts, and  not six parts.

\begin{example}
<reaction id="adenineProd" metaid="adeprod" reversible="true" fast="false">
  <annotation>
    <rdf:RDF
      xmlns:bqbiol="http://biomodels.net/biology-qualifiers/"
      xmlns:rdf="http://www.w3.org/1999/02/22-rdf-syntax-ns\#"
    >
      <rdf:Description rdf:about="\#adeprod">
        <bqbiol:hasPart>
          <rdf:Bag>
            <rdf:li rdf:resource="\changed{http://identifiers.org/ec-code/2.5.1.22}\!"/>
            <rdf:li rdf:resource="\changed{http://identifiers.org/ec-code/3.2.2.16}\!"/>
            <rdf:li rdf:resource="\changed{http://identifiers.org/ec-code/4.1.1.50}\!"/>
          </rdf:Bag>
        </bqbiol:hasPart>
        <bqbiol:hasPart>
          <rdf:Bag>
            <rdf:li rdf:resource="\changed{http://identifiers.org/kegg.reaction/R00178}\!"/>
            <rdf:li rdf:resource="\changed{http://identifiers.org/kegg.reaction/R01401}\!"/>
            <rdf:li rdf:resource="\changed{http://identifiers.org/kegg.reaction/R02869}\!"/>
          </rdf:Bag>
        </bqbiol:hasPart>
      </rdf:Description>
    </rdf:RDF>
  </annotation>
</reaction>
\end{example}

It is possible to mix different URIs in a given set. The
following example presents two alternative annotations of the human
hemoglobin, the first with ChEBI heme and the second with KEGG
heme.

\begin{example}
<species id="heme" metaid="heme" compartment="C"
         hasOnlySubstanceUnits="false" boundaryCondition="false"
         constant="false">
  <annotation>
    <rdf:RDF
      xmlns:rdf="http://www.w3.org/1999/02/22-rdf-syntax-ns\#"
      xmlns:bqbiol="http://biomodels.net/biology-qualifiers/"
    >
     <rdf:Description rdf:about="\#heme">
       <bqbiol:hasPart>
         <rdf:Bag>
           <rdf:li rdf:resource="\changed{http://identifiers.org/uniprot/P69905}\!"/>
           <rdf:li rdf:resource="\changed{http://identifiers.org/uniprot/P68871}\!"/>
           <rdf:li rdf:resource="\changed{http://identifiers.org/chebi/CHEBI:17627}\!"/>
         </rdf:Bag>
       </bqbiol:hasPart>
       <bqbiol:hasPart>
         <rdf:Bag>
          <rdf:li rdf:resource="\changed{http://identifiers.org/uniprot/P69905}\!"/>
           <rdf:li rdf:resource="\changed{http://identifiers.org/uniprot/P68871}\!"/>
           <rdf:li rdf:resource="\changed{http://identifiers.org/kegg.compound/C00032}\!"/>
         </rdf:Bag>
       </bqbiol:hasPart>
     </rdf:Description>
   </rdf:RDF>
  </annotation>
</species>
\end{example}

As formally defined above it is possible to use different
qualifiers in the same annotation element. The following
phosphorylation is annotated by its exact KEGG counterpart and by
the generic GO term ``phosphorylation''.

\begin{example}
<reaction id="phosphorylation" metaid="phosphorylation"
          reversible="true" fast="false">
  <annotation>
    <rdf:RDF
      xmlns:bqbiol="http://biomodels.net/biology-qualifiers/"
      xmlns:rdf="http://www.w3.org/1999/02/22-rdf-syntax-ns\#"
    >
      <rdf:Description rdf:about="\#phosphorylation">
        <bqbiol:is>
          <rdf:Bag>
            <rdf:li rdf:resource="\changed{http://identifiers.org/kegg.reaction/R03313}\!"/>
          </rdf:Bag>
        </bqbiol:is>
        <bqbiol:isVersionOf>
          <rdf:Bag>
            <rdf:li rdf:resource="\changed{http://identifiers.org/go/GO:0016310}\!"/>
          </rdf:Bag>
        </bqbiol:isVersionOf>
      </rdf:Description>
    </rdf:RDF>
  </annotation>
</reaction>
\end{example}

\begin{blockChanged}
The following example demonstrates the use of nested terms to describe not only that a species is in a particular compartment, but why this is believed to be true:

\begin{example}
<species id="S1" metaid="_000004" compartment="lysosome"
         hasOnlySubstanceUnits="false" boundaryCondition="false"
         constant="false">
  <annotation>
    <rdf:RDF xmlns:rdf="http://www.w3.org/1999/02/22-rdf-syntax-ns#"
             xmlns:bqbiol="http://biomodels.net/biology-qualifiers/"
             xmlns:bqmodel="http://biomodels.net/model-qualifiers/">
      <rdf:Description rdf:about="#_000004">
        <bqbiol:occursIn>
          <rdf:Bag>
            <rdf:li rdf:resource="http://identifiers.org/go/GO:0005764"/>
            <bqbiol:isDescribedBy>
              <rdf:Bag>
                <rdf:li rdf:resource="http://identifiers.org/pubmed/1111111"/>
              </rdf:Bag>
            </bqbiol:isDescribedBy>
            <bqbiol:isDescribedBy>
              <rdf:Bag>
                <rdf:li rdf:resource="http://identifiers.org/eco/ECO:0000004"/>
              </rdf:Bag>
            </bqbiol:isDescribedBy>
          </rdf:Bag>
        </bqbiol:occursIn>
      </rdf:Description>
    </rdf:RDF>
  </annotation>
</species>
\end{example}

In descriptive terms, the SBML species \val{S1} (with metaid \val{\_000004}) occurs in \val{go/GO:0005764} (the lysosome).  This is described by the publication \val{pubmed/1111111}, and is believed to be true because of the evidence \val{eco/ECO:0000004} (cell fractionation evidence).

\end{blockChanged}
