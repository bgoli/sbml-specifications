% Emacs: -*- TeX-master: "sbml-level-3-version-2-core"; fill-column: 66 -*-
% -----------------------------------------------------------------------------
% Description       : Version/level-specific LaTeX definitions & helpers
% Original author(s): Michael Hucka <mhucka@caltech.edu>
% Organization      : California Institute of Technology
% -----------------------------------------------------------------------------

% Which level/version/release is this?

\newcommand{\sbmllevel}    {3}
\newcommand{\sbmlversion}  {2 Core}
\newcommand{\sbmlversionnum}{2}
\newcommand{\sbmlrelease}  {0}
\newcommand{\sbmlreleaseNC}{0}
\newcommand{\sbmldate}     {22 April 2014}

\newcommand{\thisRelease}  {Release~\sbmlrelease{}\xspace}
\newcommand{\thisL}	   {Level~\sbmllevel\xspace}
\newcommand{\thisV}	   {Version~\sbmlversion}
\newcommand{\thisVnum}	   {Version~\sbmlversionnum\xspace}
\newcommand{\thisLV}       {Level~\sbmllevel{} \thisV\xspace}
\newcommand{\thisLVnum}    {Level~\sbmllevel{} \thisVnum}
\newcommand{\thisLVR}      {\thisLV \thisRelease}

\newcommand{\previousLVR}  {Level~3 Version~1 Release~1}

% Special commands for web addresses:

\newcommand{\sbmlSchemasURL}{\url{http://sbml.org/specifications/sbml-level-\sbmllevel/version-\sbmlversionnum/schemas/}}

% Special commands for linking the SBML component classes in this document.
% The following defines a special command of the form \Foo for any SBML
% type that has a UML box in the spec.  Note that SId is not in this
% list -- it doesn't have a UML class definition.

\newcommand{\AlgebraicRule}{\defRef{AlgebraicRule}{sec:algebraicrule}}
\newcommand{\Annotation}{\defRef{Annotation}{sec:sbase}}
\newcommand{\AssignmentRule}{\defRef{AssignmentRule}{sec:assignmentrule}}
\newcommand{\Compartment}{\defRef{Compartment}{sec:compartments}}
\newcommand{\Constraint}{\defRef{Constraint}{sec:constraints}}
\newcommand{\Delay}{\defRef{Delay}{sec:event-delay}}
\newcommand{\EventAssignment}{\defRef{EventAssignment}{sec:eventassignment}}
\newcommand{\EventAssignments}{\defRef{EventAssignments}{sec:eventassignment}}
\newcommand{\Event}{\defRef{Event}{sec:events}}
\newcommand{\Events}{\defRef{Events}{sec:events}}
\newcommand{\FunctionDefinition}{\defRef{FunctionDefinition}{sec:functiondefinition}}
\newcommand{\InitialAssignment}{\defRef{InitialAssignment}{sec:initialAssignment}}
\newcommand{\InitialAssignments}{\defRef{InitialAssignments}{sec:initialAssignment}}
\newcommand{\KineticLaw}{\defRef{KineticLaw}{subsec:kinetic-law}}
\newcommand{\ListOfCompartments}{\defRef{ListOfCompartments}{sec:listofcompartments}}
\newcommand{\ListOfConstraints}{\defRef{ListOfConstraints}{sec:listofconstraints}}
\newcommand{\ListOfEventAssignments}{\defRef{ListOfEventAssignments}{sec:listofeventassignments}}
\newcommand{\ListOfEvents}{\defRef{ListOfEvents}{sec:listofevents}}
\newcommand{\ListOfFunctionDefinitions}{\defRef{ListOfFunctionDefinitions}{sec:listoffunctiondefinitions}}
\newcommand{\ListOfInitialAssignments}{\defRef{ListOfInitialAssignments}{sec:listofinitialassign}}
\newcommand{\ListOfLocalParameters}{\defRef{ListOfLocalParameters}{subsec:listoflocalparameters}}
\newcommand{\ListOfModifierSpeciesReferences}{\defRef{ListOfModifierSpeciesReferences}{sec:reaction-type}}
\newcommand{\ListOfPackages}{\defRef{ListOfPackages}{sec:listofpackages}}
\newcommand{\ListOfParameters}{\defRef{ListOfParameters}{sec:listofparameters}}
\newcommand{\ListOfReactions}{\defRef{ListOfReactions}{sec:listofreactions}}
\newcommand{\ListOfRules}{\defRef{ListOfRules}{sec:listofrules}}
\newcommand{\ListOfSpeciesReferences}{\defRef{ListOfSpeciesReferences}{sec:reaction-type}}
\newcommand{\ListOfSpecies}{\defRef{ListOfSpecies}{sec:listofspecies}}
\newcommand{\ListOfUnitDefinitions}{\defRef{ListOfUnitDefinitions}{sec:listofunitdefinitions}}
\newcommand{\ListOfUnits}{\defRef{ListOfUnits}{sec:listofunits}}
\newcommand{\ListOf}{\class{ListOf\rule{0.3in}{0.5pt}}\xspace}
\newcommand{\LocalParameter}{\defRef{LocalParameter}{subsec:localparameter}}
\newcommand{\LocalParameters}{\defRef{LocalParameters}{subsec:localparameter}}
\newcommand{\Message}{\defRef{Message}{sec:constraints}}
\newcommand{\Model}{\defRef{Model}{sec:model}}
\newcommand{\ModifierSpeciesReference}{\defRef{ModifierSpeciesReference}{sec:simplespeciesreference-sboterm}}
\newcommand{\Notes}{\defRef{Notes}{sec:sbase}}
\newcommand{\Package}{\defRef{Package}{sec:package}}
\newcommand{\Parameter}{\defRef{Parameter}{sec:parameters}}
\newcommand{\Priority}{\defRef{Priority}{sec:event-priority}}
\newcommand{\RateRule}{\defRef{RateRule}{sec:raterule}}
\newcommand{\Reaction}{\defRef{Reaction}{sec:reactions}}
\newcommand{\Reactions}{\defRef{Reactions}{sec:reactions}}
\newcommand{\RuleUpright}{\absDefRefUpright{Rule}{sec:rules}}
\newcommand{\Rule}{\absDefRef{Rule}{sec:rules}}
\newcommand{\SBML}{\defRef{SBML}{sec:sbml}}
\newcommand{\SBaseUpright}{\absDefRefUpright{SBase}{sec:sbase}}
\newcommand{\SBase}{\absDefRef{SBase}{sec:sbase}}
\newcommand{\SimpleSpeciesReferenceUpright}{\absDefRefUpright{SimpleSpeciesReference}{sec:simplespeciesreference-sboterm}}
\newcommand{\SimpleSpeciesReference}{\absDefRef{SimpleSpeciesReference}{sec:simplespeciesreference-sboterm}}
\newcommand{\SpeciesReference}{\defRef{SpeciesReference}{sec:simplespeciesreference-sboterm}}
\newcommand{\Species}{\defRef{Species}{sec:species}}
\newcommand{\StoichiometryMath}{\defRef{StoichiometryMath}{sec:reactions}}
\newcommand{\Trigger}{\defRef{Trigger}{sec:trigger}}
\newcommand{\UnitDefinitions}{\defRef{UnitDefinitions}{sec:unitdefinitions}}
\newcommand{\UnitDefinition}{\defRef{UnitDefinition}{sec:unitdefinitions}}
\newcommand{\Unit}{\defRef{Unit}{sec:unitdefinitions}}

% Classes without a single section, or which are from other specifications:
\newcommand{\Math}        {\class{Math}\xspace}
\newcommand{\Port}        {\class{Port}\xspace}

% Macros for SBO.

\newcommand{\sboref}{\uri{http://biomodels.net/SBO/}\xspace}

\newcommand{\sboparticipantroleID}{\token{SBO:0000003}}
\newcommand{\sboparticipantrole}{\sboparticipantroleID, ``participant role''\xspace}
\newcommand{\sboeventID}{\token{SBO:0000231}}
\newcommand{\sboevent}{\sboeventID, ``event''\xspace}
\newcommand{\sboproductID}{\token{SBO:0000011}}
\newcommand{\sboproduct}{\sboproductID, ``product''\xspace}
\newcommand{\sboreactantID}{\token{SBO:0000010}}
\newcommand{\sboreactant}{\sboreactantID, ``reactant''\xspace}
\newcommand{\sbomodifierID}{\token{SBO:0000019}}
\newcommand{\sbomodifier}{\sbomodifierID, ``modifier''\xspace}
\newcommand{\sboparameterID}{\token{SBO:0000002}}
\newcommand{\sboparameter}{\sboparameterID, ``quantitative parameter''\xspace}
\newcommand{\sboratelawID}{\token{SBO:0000001}}
\newcommand{\sboratelaw}{\sboratelawID, ``rate law''\xspace}
\newcommand{\sbomathformulaID}{\token{SBO:0000064}}
\newcommand{\sbomathformula}{\sbomathformulaID, ``mathematical expression''\xspace}
\newcommand{\sboframeworkID}{\token{SBO:0000004}}
\newcommand{\sboframework}{\sboframeworkID, ``modeling framework''\xspace}
\newcommand{\sbomaterialentityID}{\token{SBO:0000240}}
\newcommand{\sbomaterialentity}{\sbomaterialentityID, ``material entity''\xspace}
\newcommand{\sbointeractionID}{\token{SBO:0000231}}
\newcommand{\sbointeraction}{\sbointeractionID, ``interaction''\xspace}

% Special commands for example models.
% NOTICE THE BLANK LINE.  Make sure to leave it!  Otherwise the
% first line of the inputted file gets indented.

\newcommand{\sbmlexample}[1]{\begin{example}

\vspace*{-0.5ex}
\input{examples/#1}\end{example}}
