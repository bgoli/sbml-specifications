% -*- TeX-master: "sbml-level-3-version-1-core"; fill-column: 66 -*-
% $Id: acknowledgments.tex 10469 2009-12-14 08:08:31Z mhucka $
% $HeadURL: https://svn.code.sourceforge.net/p/sbml/code/trunk/specifications/sbml-level-3/version-1/core/spec/acknowledgments.tex $
% ----------------------------------------------------------------

\section{Acknowledgments}
\label{sec:acknowledgements}
\label{sec:acknowledgments}

The development of SBML was originally funded by the Japan Science
and Technology Agency (JST) under the ERATO Kitano Symbiotic
Systems Project during the years 2000--2003.  From 2003 to the
present, funding for development of SBML and associated software
such as libSBML and the SBML Test Suite has been provided chiefly
by the National Institute of General Medical Sciences (USA) via
grant numbers GM070923 and GM077671.  Additional grant funding has
in the past been provided by National Human Genome Research
Institute (USA); the International Joint Research Program of NEDO
(Japan); the JST ERATO-SORST Program (Japan); the Japanese
Ministry of Agriculture; the Japanese Ministry of Education,
Culture, Sports, Science and Technology; the BBSRC e-Science
Initiative (UK); the DARPA IPTO Bio-Computation Program (USA); the
Army Research Office's Institute for Collaborative Biotechnologies
(USA); and the Air Force Office of Scientific Research (USA).

Additional support has been or continues to be provided by the
following institutions, either directly for activities related to
SBML or indirectly by supporting the work of present and past SBML
Editors: the Beckman Institute at the California Institute of
Technology (USA), EML Research gGmbH (Germany), the University of
Heidelberg (Germany), the European Molecular Biology Laboratory's
European Bioinformatics Institute (UK), the Molecular Sciences
Institute (USA), the University of Hertfordshire (UK), the
University of Newcastle (UK), the Systems Biology Institute
(Japan), and the Virginia Bioinformatics Institute (USA).

The following individuals served as past SBML Editors and authors
of SBML specifications.  Their efforts helped shape what SBML is
today:
\begin{itemize}\setlength{\parskip}{-0.2ex}

\item Hamid Bolouri
\item Andrew M. Finney
\item Nicolas Le~Nov\`{e}re
\item Herbert M. Sauro

\end{itemize}

SBML was first conceived at the JST/ERATO-sponsored \emph{First
  Workshop on Software Platforms for Systems Biology}, held in
April, 2000, at the California Institute of Technology in
Pasadena, California, USA.  The participants collectively decided
to begin developing a common XML-based declarative language for
representing models.  The development and evolution of the Systems
Biology Markup Language has continued ever since.  Many
discussions are archived online in the mailing list/forums area of
\link{http://sbml.org/Forums}{http://sbml.org}; many more
discussions took place during meetings and workshops (a list of
which is also available at
\link{http://sbml.org/Events}{http://sbml.org}).

SBML Level~3 has benefitted from so many contributions, large and
small, by so many people who constitute the international
\emph{SBML Forum}, that we regret it has become infeasible to list
individuals by name.  We thank everyone who has participated in
SBML's development throughout the years, and we hope that this
latest specification before you is a good step forward in SBML's
continued evolution.

