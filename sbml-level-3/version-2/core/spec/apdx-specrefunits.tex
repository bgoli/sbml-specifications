% -*- TeX-master: "sbml-level-3-version-2-core"; fill-column: 66 -*-
% ----------------------------------------------------------------
\begin{blockChanged}
\section{A mathematical technique for maintaining unit consistency in a kinetic law with variable stoichiometry}
\label{apdx:variable-species-reference-units}

\emph{(Appendix contributed by Chris Myers, University of Utah.)}

\sec{subsec:speciesreference} describes how the stoichiometry of a \SpeciesReference can be changed as part of a simulation.  This can be useful in some cases, such as reactions where the stoichiometry depends upon pH.

However, it can be difficult to get the \KineticLaw of such reactions to maintain the correct units as the stoichiometry of the reaction changes.  For example, let us assume that we are modeling the following set of chemical reactions:
      \begin{larray*}
        A + A &\ce{<=>[k\sub{f}][k\sub{r}]}& A_2 \\
        A_2 + A &\ce{<=>[k\sub{f}][k\sub{r}]}& A_3 \\
        &\ldots& \\
        A_{n-1} + A &\ce{<=>[k\sub{f}][k\sub{r}]}& A_n
      \end{larray*}

where we would like to allow $n$ to be a variable.  In order to allow for this, we can approximate the above set of equations with a single equation as shown below using a \emph{quasi-steady state approximation} (namely that the species $A_2$, \ldots $A_{n-1}$ have a short lifetime).
      \begin{larray*}
        nA &\ce{<=>[\ce{k\sub{r} \cdot K\sub{eq}^{n-1}}][k\sub{r}]}& A_n
      \end{larray*}

where $K\sub{eq}$ is equal to $k\sub{f} / k\sub{r}$.  The rate law for the above reaction is:
      \begin{larray*}
        k\sub{r} \cdot {K\sub{eq}}^{n-1} \cdot A^n - k\sub{r} \cdot A_n
      \end{larray*}

Let us assume that all species $A$, $A_2$, \ldots, $A_n$ have units of mole, $k\sub{f}$ is in units of $(\mathrm{mole} \cdot \mathrm{second})^{-1}$, and $k\sub{r}$ is in units of $\mathrm{second}^{-1}$.  Therefore, $K\sub{eq}$ is in units of $\mathrm{mole}^{-1}$ which makes the rate law above have units of $\mathrm{mole}/\mathrm{second}$ as desired, regardless of the value of $n$, the stoichiometry of $A$.

\end{blockChanged}
