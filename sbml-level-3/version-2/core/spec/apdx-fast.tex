% -*- TeX-master: "sbml-level-3-version-1-core"; fill-column: 66 -*-
% $Id: apdx-fast.tex 9429 2009-03-27 09:00:59Z mhucka $
% $HeadURL: https://svn.code.sourceforge.net/p/sbml/code/trunk/specifications/sbml-level-3/version-1/core/spec/apdx-fast.tex $
% ----------------------------------------------------------------

\section{Mathematical consequences of the \token{fast} attribute on
  \class{Reaction}}
\label{apdx:consequences-of-being-fast}

\emph{(Appendix contributed by James C. Schaff, University of
  Connecticut Health Center, Connecticut, U.S.A.)}

Section~\ref{sec:fast} described the \token{fast} flag available
on \Reaction.  In this appendix, we discuss the principles
involved in interpreting this attribute in the context of a simple
biochemical reaction model.  The derivation presented here is not
fully rigorous and this section is not considered normative;
achieving a higher level of rigor would require considerably more
background exposition and a much longer appendix.  Nevertheless,
we hope this section is sufficient to answer unambiguously the
question ``How should a system of reactions be treated if some of
the reactions have \token{fast}=\token{true}?''


\subsection*{Identification of ``fast'' reactions}

First, it is worth noting that the identification of so-called
\emph{fast} reactions is actually a modeling issue, not an SBML
representation issue.  The notion of fast reactions is the
following.  A system may be decomposable into two sets of
reactions, where one set may have characteristic times that are
much faster than the other time scales in the system.  An
approximation that is sometimes useful is to assume that the fast
reactions have kinetics that settle infinitely fast compared to
the other reactions in the system.  In other words, the fast
reactions are assumed to be always in equilibrium.  This is called
a pseudo-steady state approximation (PSSA), and is also known as a
quasi-steady state approximation (QSSA).  Given a case where the
time-scale separation between fast and other reactions in the
system is large, an accurate and efficient approach for computing
the time-course of the system behavior is to treat the fast
reactions as being always in equilibrium.

The key to successful application of a PSSA is that there should
be a significant separation of time scales between these fast
reactions and other reactions in the system.  The determination of
which reactions qualify as fast is up to the creator of the model,
because there is currently no known general algorithm for doing
so.


\subsection*{Simple one-compartment biochemical system model}

\newcommand{\Abold}{\ensuremath{\mathbf{A}}}
\newcommand{\Bbold}{\ensuremath{\mathbf{B}}}
\newcommand{\nbold}{\ensuremath{\mathbf{n}}}
\newcommand{\vbold}{\ensuremath{\mathbf{v}}}
\newcommand{\xbold}{\ensuremath{\mathbf{x}}}

\newcommand{\Astar}{\ensuremath{\Abold^{\!*}}}
\newcommand{\nstar}{\ensuremath{\nbold^*}}
\newcommand{\vstar}{\ensuremath{\vbold^*}}
\newcommand{\xstar}{\ensuremath{\xbold^*}}

\newcommand{\xs}{\xstar\xspace}
\newcommand{\xb}{\xbold\xspace}
\newcommand{\vs}{\vstar\xspace}
\newcommand{\vb}{\vbold\xspace}
\newcommand{\ns}{\nstar\xspace}
\newcommand{\As}{\Astar\xspace}
\newcommand{\Ab}{\Abold\xspace}
\newcommand{\Bb}{\Bbold\xspace}

\newcommand{\boldN}{\ensuremath{\mathbf{N}}}
\newcommand{\bN}{\boldN\xspace}
\newcommand{\boldm}{\ensuremath{\mathbf{m}}}
\newcommand{\bm}{\boldm\xspace}

To explain how to solve a system containing fast reactions, we use
a simple model of a biochemical reaction network located in a
single compartment.  Let \xs represent a vector of all the species
in the system, \vs a vector of the reaction rates, and \As the
stoichiometry matrix, with the vector dimension being \ns.  Then
the system can be described using the following matrix equation:
\begin{linenomath}
\begin{equation*}
  \frac{d\xstar}{dt} = \Astar \, \vstar (\xstar) 
\end{equation*}
\end{linenomath}

This system can be optionally reduced by noting that mass
conservation usually implies there are linear combinations of
species quantities in the system and the value of these
combinations do not change over time.  Identifying these
combinations is the topic of structural analysis and is described
in the literature~\citep{reder_1988,sauro_2003}.  Briefly, let \bN
be defined as the left null space of \As:
\begin{linenomath}
\begin{equation*}
  \boldN \Astar = \mathbf{0}
\end{equation*}
\end{linenomath}
Now, premultiply the previous equation by \bN to get
\begin{linenomath}
\begin{equation*}
    \boldN \dfrac{d\xstar}{dt} = \boldN \Astar \vstar (\xstar) = \mathbf{0}
\end{equation*}
\end{linenomath}
Thus, \bN captures the space of solutions to the equation
\begin{linenomath}
\begin{equation*}
  \boldm^T \left( \frac{d\xstar}{dt} \right) = 0
\end{equation*}
\end{linenomath}
where \bm is a vector representing the coefficients in a mass
conservation relationship, that is, combinations of species that
are time-invariant.  Now, let
\begin{linenomath}
\begin{align*}
  r &= \operatorname{rank}(\Astar)\\
  n &= \operatorname{dim}(\xstar)
\end{align*}
\end{linenomath}
Then the system has $n - r$ mass conservation relationships, each
of which is a linear equation.  We can use these $n - r$ linear
equations to solve for $n - r$ dependent variables in terms of $r$
independent variables and the initial masses of all species.
Doing that allows us to decompose \xs into $n - r$ dependent
variables $\xbold_d$ and $r$ independent variables $\xbold_i$
where $\mathbf{L}$ is an $(n - r) \times r$ matrix that is derived
from \bN and represents $\xbold_d$ in terms of $\xbold_i$,
$\mathbf{I}$ is the $r \times r$ identity matrix, and $\mathbf{T}$
is an $n \times r$ matrix:
\begin{linenomath}
\begin{equation*}
  \xstar \equiv \begin{bmatrix} \xbold_i \\ \xbold_d \end{bmatrix}
  = \begin{bmatrix} \mathbf{I} \\ \mathbf{L} \end{bmatrix} \xbold_i
  = \mathbf{T} \xbold_i
\end{equation*}
\end{linenomath}
Using this equation, we can define a new vector of reaction
velocities \vb in terms of $\xbold_i$ only:
\begin{linenomath}
\begin{equation*}
  \vbold ( \xbold_i ) \equiv \vstar ( \mathbf{T} \xbold_i )
\end{equation*}
\end{linenomath}
With this \vb, we can now write a reduced system by substituting
terms.  First we define $\mathbf{A}$ as the $r$ independent rows
of \Astar corresponding to $\xbold_i$.  Then:
\begin{linenomath}
\begin{equation*}
  \dfrac{d\xbold_i}{dt} = \Abold \vbold ( \xbold_i )
\end{equation*}
\end{linenomath}
This is a set of $r$ independent differential equations in $r$
unknowns (\ie an $r$-dimensional system).  To simplify the
notation slightly, let
\begin{linenomath}
\begin{equation*}
  \xbold \equiv \xbold_i
\end{equation*}
\end{linenomath}
and, thus,
\begin{linenomath}
\begin{equation*}
  \dfrac{d\xbold}{dt} = \Abold \vbold ( \xbold )
\end{equation*}
\end{linenomath}

%\xb, \Ab and \vb are
%defined as the reduced (independent) system of variables, the
%stochiometric coefficients, the substituted rates, and the new


\subsection*{Application of a PSSA to biochemical systems}

Assume that we have eliminated redundant variables and equations
using the mass conservation analysis above.  Further assume that
we have some external means of classifying some reactions in a
given system as being \emph{fast} as discussed earlier.  We now
need to apply this to the system under study.  We begin by
decomposing the vector of reaction velocities \vb according to
fast and slow reactions:
\begin{linenomath}
\begin{equation*}
  \dfrac{d\xbold}{dt} = \Abold_1 \vbold_f(\xbold) + \Abold_2 \vbold_s(\xbold)
\end{equation*}
\end{linenomath}
In the expression above, $\Abold_1$ represents the stoichiometry
of the set of reactions operating on the fast time scale, and
$\Abold_2$ the stoichiometry of the set of reactions operating on
a slower time scale.  We find the left null space of $\Abold_1$
(\ie the space of solutions to $\mathbf{m}^T[d\xbold/dt] = 0$ on a
fast time scale), and call this matrix \Bb:
\begin{linenomath}
\begin{equation*}
  \Bbold \Abold_1 = \mathbf{0}
\end{equation*}
\end{linenomath}
The matrix \Bb represents the linear combination of species that
do not change on a fast time scale, \ie the slow species in the
system.  Now, we premultiply the equation for $d\xbold/dt$ by \Bb:
\begin{linenomath}
\begin{align*}
  \Bbold \dfrac{d\xbold}{dt} &= \Bbold \Abold_1 \vbold_f(\xbold) 
                               + \Bbold \Abold_2 \vbold_s(\xbold) \\
                             &= \Bbold \Abold_2 \vbold_s(\xbold)
\end{align*}
\end{linenomath}
where the second line follows from the fact that $\Bbold \Abold_1
= \mathbf{0}$.  The above is an ordinary differential equation in
terms of only the slow dynamics.  The remaining fast dynamics are
handled by applying the pseudo-steady state approximation, with
fast transients assumed to have settled with respect to the slow
time scale.  This produces a system of nonlinear algebraic
equations:
\begin{linenomath}
\begin{equation*}
  \Abold_1 \vbold_f = \mathbf{0}
\end{equation*}
\end{linenomath}
The last two equations form the system of equations resulting from
the application of the PSSA.  If $r_1 =
\operatorname{rank}(\Abold_1)$ and $r =
\operatorname{rank}(\Abold)$, then there will be $r_1$ degrees of
freedom that will be determined by solving an algebraic system
(the equation $\Abold_1 \vbold_f = \mathbf{0}$ above), and there
will be $r - r_1$ degrees of freedom that will be determined by
ordinary differential equations (the equation for $\Bbold\,
d\xbold/dt$).

