% -*- TeX-master: "main"; fill-column: 72 -*-

\section{Best practices}
\label{best-practices}

In this section, we recommend a number of practices for using and interpreting various constructs in the SBML Level~3 Hierarchical Model Composition package.  These recommendations are non-normative, but we advocate them strongly; ignoring them will not render a model invalid, but may reduce interoperability between software and models.


\subsection{Best practices for using \class{SBaseRef} for references}
\label{best-practices-sbaseref}

As is clear from its description in \sec{sbaseref-class}, there are multiple approaches for using \SBaseRef objects to refer to SBML components.  To help increase interoperability of models that use Hierarchical Model Composition, we recommend the following order of preference in choosing between the approaches:

\begin{enumerate}

\item \emph{By port}.  Using ports, a modeler can make clear their intentions about how other models are intended to interact with a particular component.  The best-designed models intended for reuse and composition will provide port definitions; similarly, models being interfaced to port-containing models should use those ports and not bypass them.

\item \emph{By \fixttspace\primtype{SId} value}.  Most components that are desirable to replace during composition of models have an attribute named \token{id}, with a value type of \primtype{SId}.  If a model does not have ports, the next-most preferable approach to referencing component in the model is using regular identifiers.  Note that the \primtype{SIdRef} namespace is the namespace of the \emph{submodel}, not the parent model, and refers to the component namespace in that submodel (see \sec{namespaces}).

\item \emph{By \fixttspace\primtype{UnitSId} value}.  The identifier of a \UnitDefinition is defined in the core specification to exist in its own namespace.  Therefore, unit definitions in a submodel can be replaced by referencing a \UnitDefinition identifier.  (See \sec{namespaces} on identifier scoping.)  Note that the space of values of \primtype{UnitSId} is defined by the SBML Level~3 Core specification to contain reserved values that may \emph{not} be replaced or deleted; these values are unit names such as ``second'', ``kilogram'', etc., and are defined in Section~4.4 of the Level~3 Core specification.  (These reserved identifiers are \emph{only} reserved for \UnitDefinition values, and not for other SBML model components, so no such restriction exists for the identifiers of other components.)

\item \emph{By meta-identifier}.  Some SBML components have no \primtype{SId}-valued attribute, and for some other components, giving them a value is optional.  Another option for referencing components is to use their meta-identifier if it is defined, since meta identifiers are optional attributes of \SBase (via the \token{metaid} attribute) and therefore all SBML components have the potential to have one.  Of course, a given model may or may not have given a meta identifier value to a given component, but if one is present and the component has no port or regular identifier, model composition may reference the meta identifier instead.

\item \emph{By a component of a submodel}.  The above four options all give access to components in a submodel, but cannot provide access to component in the submodel's submodels.  If the object referred to by one of the above methods is itself a submodel, adding an \SBaseRef child to the \SBaseRef allows you to find components further down inside the hierarchy.  This can, in turn, refer to a deeper submodel, allowing access to any component of any arbitrary depth using this construct.  However, this is considered inelegant design; it is better to promote a component in a submodel to a local element in the referencing model if that component needs to be referenced. Unfortunately, if the submodel is fixed and unmodifiable, no other option may be available.

\end{enumerate}

These approaches do not allow access to components that have none of these options available.  If you do not have control over the model in question (for example, because it is in a third-party database), and you want to reference a component that is not a unit definition, has no port, no regular identifier, no meta identifier, and is not a component in a submodel, then the only option left is to create a copy of the original model and add (for example) a \token{metaid} value to the component that you wish to reference.


\subsection{Best practices for using ports}
\label{best-practices-ports}

%Software developers who wish to include restrictions are encouraged to
%experiment here, and add new attributes in a namespace of their own
%devising.

%From Chris:

Ports provide a mechanism for the definition of interfaces to a model; these interfaces can indicate the elements that are \emph{expected} to be modified when the model is used as a submodel.  As mentioned above, ports are the preferred means of referring to components for replacements and deletions.  If a modeler is able to modify a given model, then it is possible to accomplish all replacements and deletions via ports, without using other kinds of references, and it is possible to avoid using recursive \SBaseRef structures (by defining ports for all components that need them).

The use of ports has notable advantages for model composition.  First, it facilitates modular design of models, both by advertising to other modelers how a model is expected to be used and by allowing different modelers to compose separate submodels in a more regular fashion.  By indicating the interface points that modelers should use, it reduces the chances of unexpected side-effects when a modeler uses modular models designed by other people.  Second, it can simplify software user interfaces, by allowing software to decide what to show users insofar as the locations for potential replacements and deletions during model composition.  Third, it can simplify the maintenance of models with a software tool, since all replacements and deletions on a given submodel will remain valid as long as the ports on the submodel remain unchanged. Also, using ports to (in a sense) forward connectivity to nested submodels, rather than need to use recursive \SBaseRef objects, makes it possible for software to check only the ports on one level of hierarchy.

Most modeling situations involve models that a modeler controls physically (e.g., as files in their local file store).  Thus, using ports does not limit the options for modeling.  If a given model lacks ports on components that, over time, are discovered to be useful targets for replacements or deletions, then a user can usually modify the model physically to define the necessary ports.  Only in exceptional situations would a modeler be unable to make a copy of a model to make suitable modifications.

%From syntax.tex, and which should go in a new section (and which references the 'deletion' attribute of SBaseRef elements

% Original text, some of which should go into the best practices:
%
% The most likely use case for the 'deletion' attribute of an SBaseRef 
% element is in an 'N to M' replacement proposed
% by Andrew Finney ; perhaps an entire pathway is being replaced by a more
% detailed pathway with more reaction steps.  In this case, no one
% reaction step is replacing any one original reaction step, but the path
% as a whole is being conceptually replaced.  The way to implement this is
% to delete the original reaction steps from the submodel, and include the
% new reaction steps in the parent model.  If you wish to annotate those
% deletions, you may list the deletions as being replaced by elements of
% the new pathway.  This has no material effect on the model composition
% or on the math: it is merely a way to conceptually annotate the
% modeler's decision-making process.  As such, a deletion is the only type
% of Subelement that may be listed in more than one ListOfReplacements.
% It is recommended that in the above N to M scenario, all N deleted
% elements be listed under all M replacement elements, to make things
% easier on visualization software that may try to display the results.



\subsection{Best practices for deletions and replacements}
\label{best-practices-deletions}
\label{best-practices-replacements}

If you replace or delete an element that itself has children, those children are considered to be deleted unless replaced.  This can have repercussions on other aspects of a model; for example, if you replace a \KineticLaw object, any annotations that referred to the meta identifiers of its local parameters will become invalid.  One approach to dealing with this, in the case of annotations, is to explicitly delete the no-longer-valid annotations or replace them by new ones.  It is legal to delete explicitly a component that is deleted by implication, if you need to refer to it elsewhere---the resulting model is exactly the same.

% From Chris:

%In an SBaseRef, an element can potentially be referenced in three ways, by its identifier (SId), port identifier (PortId), or meta identifier (metaId).  It is best if an element is referred to in a consistent way.  Therefore, when an element can be referred to using a PortId, this should be preferred.  The second choice should be to refer to its SId.  Only as a last resort should it be referred to by its metaId.  

%In the case that it does not have any of these identifiers, there is no way to refer to the element in a deletion or replacement defined in this specification.  In this case, a copy must be made of the original model such that it can be modified. (Presumably, the original model was readable in the first place, or else composition would have been impossible anyway.) Copying a model and making one?s own version may have additional benefits, such as the ability to control versions explicitly and references.  A second method may be to delete or replace the parent object of the element you wish to replace, assuming that element has an identifier, meta identifier, or port identifier.  When this is performed, the errant element will be deleted implicitly, allowing you to create replacements in the containing model without overlapping functionality. 

% \subsection{Best practices for deletions and replacements}
% \label{best-practices-deletions}
% \label{best-practices-replacements}

% Note that there may be model composition situations in which a model
% contains elements that do not have an identifier, nor a meta identifier,
% nor a port identifier.  In that case, there is no way to refer to it
% using the with the \Deletion or \ReplacedElement objects defined in this specification.  A viable alternative to use in
% that case is to copy the original model and modify it, either to perform
% the desired deletions directly or to add the necessary identifiers so
% that \Deletion objects can be defined and used in a submodel.
% (Presumably, the original model was readable in the first place, or else
% composition would have been impossible anyway.)  Copying a model and
% making one's own version may have additional benefits, such as the
% ability to control versions explicitly and references.  A second method may be to delete or replace the parent object of the element you wish to replace, assuming that element has an identifier, meta identifier, or port identifier.  When this is performed, the errant element will be deleted implicitly, allowing you to create replacements in the containing model without overlapping functionality.

% .... However, don't go overboard with this capability: it is legal in
% this scheme to replace an Event with a Species, but it is probably never
% wise.  We expect that tools written to produce hierarchical SBML will
% have their own restrictions that make sense in context.  This relaxation
% of the official validation allows freer intercompatibility with other
% package extensions-it may be that a Species could be validly replaced by
% a multi-component species, or it may not, but we will rely here on the
% normal validation rules that package supplies to dictate the results.

