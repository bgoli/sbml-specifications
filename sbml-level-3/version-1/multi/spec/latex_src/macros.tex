
%\usepackage{pstricks} % not required
\usepackage{dashbox}
\usepackage{datetime}
\usepackage{verbatim}


\newcommand{\sbmlpkg}{\texorpdfstring{%
    \textls[-25]{\textsc{SBMLPkgSpec}}}{%
    \textsc{SBMLPkgSpec}}\xspace}
\newcommand{\sbmlpkghead}{\texorpdfstring{%
    \textls[-50]{\textsc{SBMLPkgSpec}}}{%
    \textsc{SBMLPkgSpec}}\xspace}
\newcommand{\sbmlpkgfile}{\literalFont{sbmlpkgspec.cls}\xspace}
\newcommand{\latex}{\LaTeX{}\xspace}
\newcommand{\tex}{\TeX{}\xspace}
\newcommand{\distURL}{http://sourceforge.net/projects/sbml/files/specifications/tex}
\newcommand{\srcURL}{https://sbml.svn.sourceforge.net/svnroot/sbml/trunk/project/tex/sbmlpkgspec}
\newcommand{\webURL}{http://sbml.org/Documents/Specifications/The_SBMLPkgSpec_LaTeX_class}
\newcommand{\cmd}[1]{\literalFont{\textbackslash #1}}

% Custom latex listing style, for use with the listings package.  The default
% highlights far too many things, IMHO.  This keeps it simple and only adjusts
% the appearance of comments within listings.

\lstdefinelanguage{mylatex}{%
  morekeywords={},%
  sensitive,%
  alsoother={0123456789$_},%$
  morecomment=[l]\%%
}[keywords,tex,comments]

\lstdefinestyle{latex}{language=mylatex}

\newcommand{\fixttspace}{\hspace*{1pt}}


% -- my new commands --

% -- alias
\newcommand{\attFont}{\token}
%\newcommand{\objFont}{\textit}
\newcommand{\objFont}{\token}
\newcommand{\primtypeFont}{\fixttspace\primtypeNC}
\newcommand{\noticeFont}{\textit}

%-- attributes
\newcommand{\isTypeAtt}{\attFont{isType}}
\newcommand{\idAtt}{\attFont{id}}
\newcommand{\nameAtt}{\attFont{name}}
\newcommand{\isBindingSiteAtt}{\attFont{is\-Binding\-Site}}

\newcommand{\compartmentAtt}{\attFont{compartment}}
\newcommand{\compartmentReferenceAtt}{\attFont{compartment\-Reference}}
\newcommand{\compartmentTypeAtt}{\attFont{compartment\-Type}}
\newcommand{\identifyingParentAtt}{\attFont{identifying\-Parent}}
\newcommand{\occurAtt}{\attFont{occur}}
\newcommand{\speciesTypeAtt}{\attFont{species\-Type}}
\newcommand{\bindingStatusAtt}{\attFont{binding\-Status}}
\newcommand{\componentAtt}{\attFont{component}}
\newcommand{\speciesTypeComponentIndexAtt}{\attFont{species\-Type\-Component\-Index}}
\newcommand{\relationAtt}{\attFont{relation}}
%\newcommand{\isIntraSpeciesReactionAtt}{\attFont{isIntraSpeciesReaction}}
\newcommand{\speciesFeatureTypeAtt}{\attFont{species\-Feature\-Type}}
\newcommand{\reactantAtt}{\attFont{reactant}}
\newcommand{\reactantComponentAtt}{\attFont{reactant\-Component}}
\newcommand{\productComponentAtt}{\attFont{product\-Component}}
\newcommand{\reactantSpeciesFeatureAtt}{\attFont{reactant\-Species\-Feature}}
\newcommand{\productSpeciesFeatureAtt}{\attFont{product\-Species\-Feature}}
\newcommand{\numericValueAtt}{\attFont{numeric\-Value}}
\newcommand{\initialAmountAtt}{\attFont{initial\-Amount}}
\newcommand{\initialConcentrationAtt}{\attFont{initial\-Concentration}}
\newcommand{\valueAtt}{\attFont{value}}
\newcommand{\bindingSiteOneAtt}{\attFont{binding\-Site1}}
\newcommand{\bindingSiteTwoAtt}{\attFont{binding\-Site2}}
\newcommand{\speciesReferenceAtt}{\attFont{species\-Reference}}
\newcommand{\representationTypeAtt}{\attFont{representation\-Type}}

\newcommand{\mIsTypeAtt}{\attFont{multi:\-isType}}
\newcommand{\mIdAtt}{\attFont{multi:\-id}}
\newcommand{\mNameAtt}{\attFont{multi:\-name}}
\newcommand{\mIsBindingSiteAtt}{\attFont{multi:\-is\-Binding\-Site}}
\newcommand{\mCompartmentAtt}{\attFont{multi:\-compartment}}
\newcommand{\mCompartmentReferenceAtt}{\attFont{multi:\-compartment\-Reference}}
\newcommand{\mCompartmentTypeAtt}{\attFont{multi:\-compartment\-Type}}
\newcommand{\mIdentifyingParentAtt}{\attFont{multi:\-identi\-fy\-ing\-Parent}}
\newcommand{\mCccurAtt}{\attFont{multi:\-occur}}
\newcommand{\mSpeciesTypeAtt}{\attFont{multi:\-species\-Type}}
\newcommand{\mBindingStatusAtt}{\attFont{multi:\-binding\-Status}}
\newcommand{\mComponentAtt}{\attFont{multi:\-component}}
\newcommand{\mSpeciesTypeComponentIndexAtt}{\attFont{multi:\-species\-Type\-Component\-Index}}
\newcommand{\mRelationAtt}{\attFont{multi:\-relation}}
\newcommand{\mSpeciesFeatureTypeAtt}{\attFont{multi:\-species\-Feature\-Type}}
\newcommand{\mReactantAtt}{\attFont{multi:\-reactant}}
\newcommand{\mReactantComponentAtt}{\attFont{multi:\-reactant\-Component}}
\newcommand{\mProductComponentAtt}{\attFont{multi:\-product\-Component}}
\newcommand{\mReactantSpeciesFeatureAtt}{\attFont{multi:\-reactant\-Species\-Feature}}
\newcommand{\mProductSpeciesFeatureAtt}{\attFont{multi:\-product\-Species\-Feature}}
\newcommand{\mNumericValueAtt}{\attFont{multi:\-numeric\-Value}}
\newcommand{\mValueAtt}{\attFont{multi:\-value}}
\newcommand{\mBindingSiteOneAtt}{\attFont{multi:\-binding\-Site1}}
\newcommand{\mBindingSiteTwoAtt}{\attFont{multi:\-binding\-Site2}}
\newcommand{\mSpeciesReferenceAtt}{\attFont{multi:\-species\-Reference}}
\newcommand{\mRepresentationTypeAtt}{\attFont{multi:\-representation\-Type}}
\newcommand{\mOccurAtt}{\attFont{multi:\-occur}}

% -- classes and objects
\newcommand{\notes}{\objFont{notes}}
\newcommand{\annotation}{\objFont{anno\-ta\-tion}}

\newcommand{\ExModel}{\defRef{Model}{def:Model}}
\newcommand{\model}{\objFont{model}}
%\newcommand{\models}{\objFont{models}} % this command has been defined somewhere, can not redefine it here
\newcommand{\smodels}{\objFont{model}s}

\newcommand{\parameter}{\objFont{para\-me\-ter}}

\newcommand{\ListOfSpeciesTypes}{\defRef{List\-Of\-Species\-Types}{def:ListOfSpeciesTypes}}
\newcommand{\listOfSpeciesTypes}{\objFont{list\-Of\-Species\-Types}}

\newcommand{\ExCompartment}{\defRef{Compart\-ment}{def:Compartment}}
\newcommand{\compartment}{\objFont{compart\-ment}}
\newcommand{\compartments}{\objFont{compart\-ment}s}

\newcommand{\ListOfCompartmentReferences}{\defRef{List\-Of\-Compart\-ment\-References}{def:ListOfCompartmentReferences}}
\newcommand{\listOfCompartmentReferences}{\objFont{list\-Of\-Compart\-ment\-References}}

\newcommand{\CompartmentReference}{\defRef{Compart\-ment\-Reference}{def:CompartmentReference}}
\newcommand{\compartmentReference}{\objFont{compart\-ment\-Reference}}
\newcommand{\compartmentReferences}{\objFont{compart\-ment\-Reference}s}

\newcommand{\SpeciesType}{\defRef{Species\-Type}{def:SpeciesType}}
\newcommand{\speciesType}{\objFont{species\-Type}}
\newcommand{\speciesTypeCap}{\objFont{Species\-Type}}
\newcommand{\speciesTypes}{\objFont{species\-Type}s}

\newcommand{\BindingSiteSpeciesType}{\defRef{Binding\-Site\-Species\-Type}{def:BindingSiteSpeciesType}}
\newcommand{\bindingSiteSpeciesType}{\objFont{binding\-Site\-Species\-Type}}
\newcommand{\bindingSiteSpeciesTypeCap}{\objFont{Binding\-Site\-Species\-Type}}
\newcommand{\bindingSiteSpeciesTypes}{\objFont{binding\-Site\-Species\-Types}}

%\newcommand{\bindingSite}{\defRef{\objFont{bindingSite}}{def:BindingSiteSpeciesType}}
%\newcommand{\bindingSites}{\defRef{\objFont{bindingSites}}{def:BindingSiteSpeciesType}}
\newcommand{\bindingSite}{\objFont{binding site}}
\newcommand{\bindingSites}{\objFont{binding sites}}

\newcommand{\ListOfSpeciesTypeComponentIndexes}{\defRef{List\-Of\-Species\-Type\-Component\-Indexes}{def:ListOfSpeciesTypeComponentIndexes}}
\newcommand{\listOfSpeciesTypeComponentIndexes}{\objFont{list\-Of\-Species\-Type\-Component\-Indexes}}

\newcommand{\ListOfSpeciesFeatureTypes}{\defRef{List\-Of\-Species\-Feature\-Types}{def:ListOfSpeciesFeatureTypes}}
\newcommand{\listOfSpeciesFeatureTypes}{\objFont{list\-Of\-Species\-Feature\-Types}}

\newcommand{\ListOfPossibleSpeciesFeatureValues}{\defRef{List\-Of\-Possible\-Species\-Feature\-Values}{def:ListOfPossibleSpeciesFeatureValues}}
\newcommand{\listOfPossibleSpeciesFeatureValues}{\objFont{list\-Of\-Possible\-Species\-Feature\-Values}}

\newcommand{\ListOfSpeciesTypeInstances}{\defRef{List\-Of\-Species\-Type\-Instances}{def:ListOfSpeciesTypeInstances}}
\newcommand{\listOfSpeciesTypeInstances}{\objFont{list\-Of\-Species\-Type\-Instances}}

\newcommand{\ListOfInSpeciesTypeBonds}{\defRef{List\-Of\-In\-Species\-Type\-Bonds}{def:ListOfInSpeciesTypeBonds}}
\newcommand{\listOfInSpeciesTypeBonds}{\objFont{list\-Of\-In\-Species\-TypeBonds}}

\newcommand{\SpeciesFeatureType}{\defRef{Species\-Feature\-Type}{def:SpeciesFeatureType}}
\newcommand{\speciesFeatureType}{\objFont{species\-Feature\-Type}}
\newcommand{\speciesFeatureTypes}{\objFont{species\-Feature\-Type}s}

\newcommand{\PossibleSpeciesFeatureValue}{\defRef{Possible\-Species\-Feature\-Value}{def:PossibleSpeciesFeatureValue}}
\newcommand{\possibleSpeciesFeatureValue}{\objFont{possible\-Species\-Feature\-Value}}
\newcommand{\possibleSpeciesFeatureValues}{\objFont{possible\-Species\-Feature\-Value}s}

\newcommand{\SpeciesTypeInstance}{\defRef{Species\-Type\-Instance}{def:SpeciesTypeInstance}}
\newcommand{\speciesTypeInstance}{\objFont{species\-Type\-Instance}}
\newcommand{\speciesTypeInstanceCap}{\objFont{Species\-Type\-Instance}}
\newcommand{\speciesTypeInstances}{\objFont{species\-Type\-Instance}s}

\newcommand{\InSpeciesTypeBond}{\defRef{In\-Species\-Type\-Bond}{def:InSpeciesTypeBond}}
\newcommand{\inSpeciesTypeBond}{\objFont{in\-Species\-Type\-Bond}}
\newcommand{\inSpeciesTypeBonds}{\objFont{in\-Species\-Type\-Bond}s}

\newcommand{\SpeciesTypeComponentIndex}{\defRef{Species\-Type\-Component\-Index}{def:SpeciesTypeComponentIndex}}
\newcommand{\speciesTypeComponentIndex}{\objFont{species\-Type\-Component\-Index}}
\newcommand{\speciesTypeComponentIndexes}{\objFont{species\-Type\-Component\-Index}es}

\newcommand{\componentWR}{\defRef{\objFont{component}}{def:SpeciesType:component}} % with reference
\newcommand{\component}{\objFont{component}}
%\newcommand{\components}{\defRef{\objFont{components}}{def:SpeciesType:component}}
\newcommand{\components}{\objFont{component}s}

\newcommand{\ExSpecies}{\defRef{Species}{def:ExSpecies}}
\newcommand{\species}{\objFont{species}}

\newcommand{\ListOfOutwardBindingSites}{\defRef{List\-Of\-Outward\-Binding\-Sites}{def:ListOfOutwardBindingSites}}
\newcommand{\listOfOutwardBindingSites}{\objFont{list\-Of\-Outward\-Binding\-Sites}}

\newcommand{\ListOfSpeciesFeatures}{\defRef{List\-Of\-Species\-Features}{def:ListOfSpeciesFeatures}}
\newcommand{\listOfSpeciesFeatures}{\objFont{list\-Of\-Species\-Features}}

\newcommand{\SubListOfSpeciesFeatures}{\defRef{Sub\-List\-Of\-Species\-Features}{def:SubListOfSpeciesFeatures}}
\newcommand{\subListOfSpeciesFeatures}{\objFont{sub\-List\-Of\-Species\-Features}}

\newcommand{\OutwardBindingSite}{\defRef{Outward\-Binding\-Site}{def:OutwardBindingSite}}
\newcommand{\outwardBindingSite}{\objFont{outward\-Binding\-Site}}
\newcommand{\outwardBindingSites}{\objFont{outward\-Binding\-Sites}}

\newcommand{\SpeciesFeature}{\defRef{Species\-Feature}{def:SpeciesFeature}}
\newcommand{\speciesFeature}{\objFont{species\-Feature}}
\newcommand{\speciesFeatures}{\objFont{species\-Feature}s}

\newcommand{\ListOfSpeciesFeatureValues}{\defRef{List\-Of\-Species\-Feature\-Values}{def:ListOfSpeciesFeatureValues}}
\newcommand{\listOfSpeciesFeatureValues}{\objFont{list\-Of\-SpeciesvFeature\-Values}}

\newcommand{\SpeciesFeatureValue}{\defRef{Species\-Feature\-Value}{def:SpeciesFeatureValue}}
\newcommand{\speciesFeatureValue}{\objFont{species\-Feature\-Value}}
\newcommand{\speciesFeatureValues}{\objFont{species\-Feature\-Value}s}

\newcommand{\ExReaction}{\defRef{Reaction}{def:ExReaction}}
\newcommand{\reaction}{\objFont{reaction}}
\newcommand{\reactions}{\objFont{reaction}s}

\newcommand{\IntraSpeciesReaction}{\defRef{Intra\-Species\-Reaction}{def:IntraSpeciesReaction}}
\newcommand{\intraSpeciesReaction}{\objFont{intra\-Species\-Reaction}}
\newcommand{\intraSpeciesReactions}{\objFont{intra\-Species\-Reactions}}

\newcommand{\ExSimpleSpeciesReference}{\defRef{Simple\-Species\-Reference}{def:ExSimpleSpeciesReference}}
\newcommand{\simpleSpeciesReference}{\objFont{simple\-Species\-Reference}}
\newcommand{\simpleSpeciesReferences}{\objFont{simple\-Species\-Reference}s}

\newcommand{\modifierSpeciesReference}{\objFont{modifier\-Species\-Reference}}
\newcommand{\modifierSpeciesReferences}{\objFont{modifier\-Species\-Reference}s}
\newcommand{\listOfModifierSpeciesReferences}{\objFont{list\-Of\-Modifier\-SpeciesReferences}}

\newcommand{\ExSpeciesReference}{\defRef{Species\-Reference}{def:ExSpeciesReference}}
\newcommand{\speciesReference}{\objFont{species\-Reference}}
\newcommand{\speciesReferences}{\objFont{species\-Reference}s}

\newcommand{\reactant}{\objFont{reactant}}
\newcommand{\reactants}{\objFont{reactant}s}
\newcommand{\product}{\objFont{product}}
\newcommand{\products}{\objFont{product}s}

\newcommand{\ListOfSpeciesTypeComponentMapsInProduct}{\defRef{List\-Of\-Species\-Type\-Component\-Maps\-In\-Product}{def:ListOfSpeciesTypeComponentMapsInProduct}}
\newcommand{\listOfSpeciesTypeComponentMapsInProduct}{\objFont{list\-Of\-Species\-Type\-Component\-Maps\-In\-Product}}

\newcommand{\SpeciesTypeComponentMapInProduct}{\defRef{Species\-Type\-Component\-Map\-In\-Product}{def:SpeciesTypeComponentMapInProduct}}
\newcommand{\speciesTypeComponentMapInProduct}{\objFont{species\-Type\-Component\-Map\-In\-Product}}
\newcommand{\speciesTypeComponentMapsInProduct}{\objFont{species\-Type\-Component\-Map\-In\-Product}s}

\newcommand{\ExMath}{\defRef{Math}{def:Reaction:Math:ci}}
\newcommand{\ci}{\objFont{ci}}

% -- abbreviations
\newcommand{\SbmlLevelThree}{SBML Level 3}
\newcommand{\SbmlLevelThreeWC}{\SbmlLevelThree\ [\cite{ref:sbmll3v1}]}
\newcommand{\SbmlLevelThreeVersionOne}{SBML Level 3 Version 1}
\newcommand{\SbmlLevelThreeVersionOneWC}{\SbmlLevelThreeVersionOne\ [\cite{ref:sbmll3v1}]}
\newcommand{\SbmlLevelThreeVersionOneCore}{SBML Level 3 Version 1 Core}
\newcommand{\SbmlLevelThreeCore}{SBML Level 3 Core}
\newcommand{\SbmlLevelThreeCoreWC}{\SbmlLevelThreeCore\ [\cite{ref:sbmll3v1}]}

\newcommand{\BioNetGen}{\textit{BioNetGen}}
\newcommand{\Simmune}{\textit{Simmune}}
\newcommand{\Kappa}{\textit{Kappa}}

\newcommand{\notatypecompartmentWC}{\val{not-a-type} \compartment\ (see \sec{def:Compartment:isType:false})}
\newcommand{\notatypecompartment}{\val{not-a-type} \compartment}
\newcommand{\notatypecompartmentsWC}{\val{not-a-type} \compartments\ (see \sec{def:Compartment:isType:false})}
\newcommand{\notatypecompartments}{\val{not-a-type} \compartments}

\newcommand{\fullydefined}{\emph{fully defined}}
\newcommand{\Fullydefined}{\emph{Fully defined}}
\newcommand{\fullydefinedspecies}{\emph{fully defined} \objFont{species}}
\newcommand{\Fullydefinedspecies}{\emph{Fully defined} \objFont{species}}
\newcommand{\fullydefinedspeciesWC}{\emph{fully defined} \objFont{species} (see \sec{def:Species:FullyDefined})}
\newcommand{\FullydefinedspeciesWC}{\emph{Fully defined} \objFont{species} (see \sec{def:Species:FullyDefined})}


\newcommand{\multi}{Multi}   % caption for the first letter
\newcommand{\multiLC}{\textit{multi}} % lower case
\newcommand{\multiOneProposal}{the previous \multi\ proposal (2010)}
\newcommand{\multiOneProposalCap}{The previous \multi\ proposal (2010)}
\newcommand{\multiOneProposalWC}{the previous \multi\ proposal [\cite{ref:multi1}]}
\newcommand{\multiOneProposalCapWC}{The previous \multi\ proposal [\cite{ref:multi1}]}
\newcommand{\multiTwoProposalVerOne}{the new \multi\ proposal (Aug 2012)}
\newcommand{\multiTwoProposalVerOneWC}{the new \multi\ proposal [\cite{ref:revisedMulti}]}
\newcommand{\multiTwoProposalVerTwo}{the new \multi\ proposal [Rev 221, \cite{ref:multiproposal221}]}
%\newcommand{\multiTwoProposalVerTwoWC}{}
\newcommand{\multiTwoProposalVerThreeOne}{the new \multi\ proposal (April 2013, Rev 269)}
\newcommand{\multiTwoProposalVerThreeTwo}{the new \multi\ proposal (May 2013, Rev 280)}
\newcommand{\multiTwoProposalVerThreeTwoCap}{The new \multi\ proposal (May 2013, Rev 280)}
\newcommand{\multiTwoProposalVerThreeTwoWC}{the new \multi\ proposal [Rev 280, \cite{ref:multiproposal280}]}
\newcommand{\multiTwoProposalVerThreeTwoCapWC}{The new \multi\ proposal [Rev 280, \cite{ref:multiproposal280}]}

% -- primitive data types
\newcommand{\positiveIntegerPT}{\primtypeFont{positiveInteger}}
\newcommand{\stringPT}{\primtypeFont{string}}
\newcommand{\booleanPT}{\primtypeFont{boolean}}
\newcommand{\intPT}{\primtypeFont{int}}
\newcommand{\SIdPT}{\primtypeFont{SId}}
\newcommand{\SIdRefPT}{\primtypeFont{SIdRef}}
\newcommand{\BindingStatusPT}{\primtypeFont{BindingStatus}}
\newcommand{\RelationPT}{\primtypeFont{Relation}}
\newcommand{\RepresentationTypePT}{\primtypeFont{RepresentationType}}

\newcommand{\RelationPTWC}{\defRef{\primtypeFont{Relation}}{def:Primtype:Relation}}
\newcommand{\BindingStatusPTWC}{\defRef{\primtypeFont{BindingStatus}}{def:Primtype:BindingStatus}}
\newcommand{\RepresentationTypePTWC}{\defRef{\primtypeFont{RepresentationType}}{def:Primtype:RepresentationType}}

% -- example file fragment
\newcommand{\exampleFileLinerange}[3][style=XML, showstringspaces=false]{%
  \lstset{#1}\examplespacing\lstinputlisting[linerange={#3}]{#2}\regularspacing}

% conditional blockChange
\newcommand{\mBlockChangedBegin}[1]{%
\ifnum\pdfstrcmp{#1}{\thisVersion}=0
   \begin{blockChanged}
\fi}

\newcommand{\mBlockChangedEnd}[1]{%
\ifnum\pdfstrcmp{#1}{\thisVersion}=0
   \end{blockChanged}
\fi}