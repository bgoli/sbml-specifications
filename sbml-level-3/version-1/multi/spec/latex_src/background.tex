% -*- TeX-master: "multi" -*-

\section{Background and context}
\label{def:Background}

Rule-based, domain-detailed modeling has been extremely valuable in systems biology related studies [\cite{ref:nathan2015} and \cite{ref:jamesFader2013}]. Rule-based, domain-detailed modeling approaches (\BioNetGen\ [\cite{ref:bionetgen2009}], \Kappa\ [\cite{ref:kappa2004}], and \Simmune\ [\cite{ref:simmune2012, ref:simmune2006}]) define rules for \mBlockChangedBegin{\revTwentyTwentyMarch}transformations of molecular domains and \mBlockChangedEnd{\revTwentyTwentyMarch}interactions between pairs of molecule domains, specifying how the \mBlockChangedBegin{\revTwentyTwentyMarch}transformations/\mBlockChangedEnd{\revTwentyTwentyMarch}interactions depend on particular states of the molecules (pattern) and their locations in specific compartments. In order to generate networks of biochemical reactions these rules are applied to the molecular components of the systems to be modeled, either at the beginning of the modeling (simulation) process or ``on the fly'' (as molecule complexes emerge from the interaction rules). Expressing such rule-based, domain-detailed reaction networks using the concepts of \Species and \Compartment in SBML (L3 core and L2) can be difficult for rules and molecule sets that lead to large numbers of resulting molecular complexes. It would therefore be desirable to have an SBML standard for encoding rule-based, domain-detailed \smodels\ using their ``native'' concepts for describing reactions instead of having to apply the rules and unfold the networks prior to encoding in an SBML format.

We proposed a revised proposal of the \multi\ package: ``Multistate, Multicomponent and Multicompartment Species Package for SBML Level 3''  (abbreviated as Multi) [\cite{ref:revisedMulti} and \cite{ref:multiproposal280}] which takes the scopes and some data structures developed in \multiOneProposalWC\ and addresses main issues arising from a rule-based, domain-detailed modeling point of view with the data structures consistent with that used in the available rule-based, domain-detailed modeling tools. 

\label{def:OtherRuleBasedModels}
\noticeFont{Note: \notice\\
This specification was developed with the main goal of taking into account \mBlockChangedBegin{\revTwentyTwentyMarch}molecular transformations and \mBlockChangedEnd{\revTwentyTwentyMarch}bi-molecular interactions mediated through specific binding domains (or sites). Models without such detailed description of the molecular interactions can be encoded as well if the other features in this specification such as \SpeciesFeatureType, \SpeciesFeature, and extended \ExCompartment satisfy the model requirements.
}


%-------------------------------------------------------
\subsection{Past work on this problem or similar topics}
\label{def:Past_work}

\begin{itemize}
 \item 
  Nicolas Le Nov\`ere and Anika Oellrich proposed the previous version of the Multi proposal [\cite{ref:multi1}]. However, it was realized that a more detailed treatment of molecular binding sites and their state-dependent interactions would be desirable.
 
 \item In August 2012, Fengkai Zhang from the \Simmune\ group presented `` Draft for discussion SBML Proposals for Revised Multi, Simple Spatial and Multi-Spatial Extensions'' at COMBINE 2012 [\cite{ref:revisedMulti}]. The three proposals cover the goals and scope of \multiOneProposal, revise it and add some new features that improve usage of the proposal for rule-based approaches.
 
 \item Based on the discussions and suggestions received during COMBINE 2012 as well as on feedback from the SBML discussion forum, \multiTwoProposalVerTwo\ was released to the SBML-Multi community, which integrates and covers most of the features in the three previous proposals of August 2012. 
 
 \item In May 2013, a new revision (rev 280) of the Multi proposal [\cite{ref:multiproposal280}] was released before the meeting of HARMONY 2013. The extended \ExCompartment class and its related classes have been reorganized. All optional boolean attributes have been removed/replaced. A new optional Multi attribute, \val{whichValue}, was added to the \token{ci} elements in \class{KineticLaw} to identify  the sources of \species. (Lucian Smith gave many comments/suggestions about this proposal and William Hlavacek gave thoughtful feedback about the \BioNetGen\ example in this proposal). This revision (rev 280) was presented at HARMONY 2013  [\cite{ref:harmony2013}] with new features to configure multiple occurrences of \SpeciesFeatureType. Several new or revised features were discussed during and after HARMONY 2013, including multiple occurrences of \SpeciesFeatureType, multiple copies of \SpeciesTypeInstance, the \numericValueAtt\ attribute for \PossibleSpeciesFeatureValue and concentration summation of pattern \species. These features are covered or updated in the specifications from v1.0.1.
  
\end{itemize}

%-------------------------------------------------------
\subsection{Revision history}
\label{def:revision_history}

The versioning convention used in this document: \\
\objFont{x.y.z (status)}

\objFont{x}: version of SBML Level 3 core. \\
\objFont{y}: version of the Multi package. \\
\objFont{z}: release of the Multi package at its version \objFont{y}. \\ 
\objFont{status}: \val{draft}, \val{release candidate}, or \val{release}. Absence of status means \val{release}.

For example, the current version is \val{\thisVersion} \\
\objFont{x} = \val{\thisCoreVersion} \\
\objFont{y} = \val{\thisMultiVersion} \\
\objFont{z} = \val{\thisMultiRelease} \\

The followings are the revision history of the Multi package:

\newcommand{\mVersionTileFont}[1]{\textbf{#1}}


\subsubsection{Release(s)}

\begin{itemize}
%  \mBlockChangedBegin{1.1.2}
%    \item \mVersionTileFont{Version: 1.1.2, this version} 
%    \label{def:v1_1_2}
%  \mBlockChangedEnd{1.1.2} 
  \mBlockChangedBegin{\revTwentyTwentyMarch}
    \item \mVersionTileFont{Version: \revTwentyTwentyMarch}, this version
    \label{def:v1_2}
    \begin{itemize}
     \item Fix issues raised by the community after ``Version 1 release 1''.
     \item Revise parts of the text.
     \item Add a paragraph to clarify ``XML Namespace use''(see \sec{def:xml_namespace_use}).
     \item Validate the complete models and add ``demonstration only'' notes to those ``incomplete'' models in the example section (see \sec{def:Examples}).
    \end{itemize}

  \mBlockChangedEnd{\revTwentyTwentyMarch} 

  \mBlockChangedBegin{1.1.1}
    \item \mVersionTileFont{Version: 1.1.1, March 2017} 
    \label{def:v1_1_1}
  \mBlockChangedEnd{1.1.1} 
\end{itemize}


\subsubsection{Release Candidates}

\begin{itemize}
  \mBlockChangedBegin{1.1.rc5}
  \item \mVersionTileFont{Version: 1.1.rc5 (release candidate), March 2017} 
  \label{def:v1_1_rc5}
    \begin{itemize}
      \item Add two validation rules \defRef{multi-21213}{validation:species:21213} and \defRef{multi-21214}{validation:species:21214} to check the \speciesTypeAtt\ attribute of a \species\ with \listOfOutwardBindingSites\ and/or \listOfSpeciesFeatures\. (See \sec{def:ExSpecies}.)
      \item Add a constraint to the \relationAtt\ attribute of a \subListOfSpeciesFeatures\ having a \speciesFeature\ child referencing a \speciesFeatureType\ with \val{occur > 1}. (See \sec{def:SubListOfSpeciesFeatures:relation} and \sec{validation:species:21215}.)
      \item Enforce the \SubListOfSpeciesFeatures class to have at least two \speciesFeature s and set \relationAtt\ as a required attribute. (See \sec{def:SubListOfSpeciesFeatures}.)
    \end{itemize}
\mBlockChangedEnd{1.1.rc5} 

  \item \mVersionTileFont{Version: 1.1.rc4 (release candidate), March 2017} \\
  \label{def:v1_1_rc4}
  More updates on validation rule numbers, line breaks, and the example about \SubListOfSpeciesFeatures. 

  \item \mVersionTileFont{Version: 1.1.rc3 (release candidate), Febrary 2017} \\
  \label{def:v1_1_rc3}
  Modify the numbers of several rules to be consistent with the general SBML validation rule conventions. 

  \item \mVersionTileFont{Version: 1.1.rc2 (release candidate), January 2017} \\
  \label{def:v1_1_rc2}
  Add a new validation rule \defRef{multi-22006}{def:validationRule:22006} to prevent circular referencing among the extended \ExCompartment objects.
 
  Revise the specification text with minor changes towards a version of the official release candidate. 

  \item \mVersionTileFont{Version: 1.1.rc1 (release candidate), November 2016} \\
  \label{def:v1_1_rc1}
  Revise the specification text with minor changes towards a version of the official release candidate. 

\end{itemize}

\subsubsection{Drafts}

\begin{itemize}
 \item \mVersionTileFont{Version: 1.0.7 (draft), August 2016} \\
 \label{def:v1_0_7}
 Remove the \class{SpeciesFeatureChange} and \class{ListOfSpeciesFeatureChanges} classes under \SpeciesTypeComponentMapInProduct. The relations expressed in \class{SpeciesFeatureChange} can be inferred from the \speciesTypeComponentMapInProduct\ and the \species\ of the mapped \reactant\ and \product.
 
 Add a new validation rule 21306, ``an \outwardBindingSite\ cannot be a binding site in a bond of the species'' (see \sec{def:OutwardBindingSite:component} and \sec{def:validation:21306})

 \item \mVersionTileFont{Version: 1.0.6 (draft), March 2016} \\
 \label{def:v1_0_6}
 Remove recursively referencing relationship in the \ListOfSpeciesFeatures class and add a \SubListOfSpeciesFeatures class. See the details in \ExSpecies.
 
 Version 1.0.6.1 with minor document update is released in April 2016.

 \item \mVersionTileFont{Version 1.0.5 (draft), November 2015} \\
 \label{def:v1_0_5}
 This version has been developed from the previous release v1.0.4 with the following modifications based on the discussion during and after COMBINE 2015 [\cite{ref:combine2015}]: 
 \begin{itemize}
  \item Drop the \occurAtt\ attribute in the class of \SpeciesTypeInstance.
  \item Drop the \occurAtt\ attribute in the class of \SpeciesTypeComponentIndex.
  \item Drop the class of \class{DenotedSpeciesTypeComponentIndex}.
  \item Revise the scope of \PossibleSpeciesFeatureValue ids to be global.
 \end{itemize}

Version 1.0.5.1 with minor document update is released in Dec 2015.

 \item \mVersionTileFont{Version 1.0.4 (draft), June 2015} \\
 \label{def:v1_0_4}
 This version has been developed from the previous release v1.0.3 with minor document update and complete validation rules.

 \item \mVersionTileFont{Version 1.0.3 (draft), April 2015} \\
 \label{def:v1_0_3}
 This version has been developed from the previous release v1.0.2 mainly based on the discussion in COMBINE 2014 with focus on how to facilitate tools to export and import \smodels\ encoded in the Multi format [\cite{ref:combine2014}] 

 \item \mVersionTileFont{Version 1.0.2 (draft), November 2014} \\
 \label{def:v1_0_2}
 This version has been developed from the previous release v1.0.1 with the following modifications: 

 \begin{itemize}
  \item A new \BindingSiteSpeciesType\ sub-class inheriting the \SpeciesType class for \objFont{binding sites}. Accordingly, the \isBindingSiteAtt\ attribute has been dropped from \SpeciesType.
  \item Restriction on \objFont{binding sites} which have to be atomic.
  \item Restriction on \SpeciesType that a \speciesType\ cannot have a \listOfSpeciesFeatureTypes\ if it has a \listOfInSpeciesTypeBonds. 
  \item A new \IntraSpeciesReaction sub-class inheriting the \Reaction class for the reactions happening within a \ExSpecies object. Accordingly, the \token{isIntraSpeciesReaction} attribute has been dropped from \Reaction.
  \item Validation rules. 
 \end{itemize}

 \item \mVersionTileFont{Version 1.0.1 (draft), September 2013} \\
 \label{def:v1_0_1}
 This was released and presented in COMBINE 2013 [\cite{ref:combine2013}], mainly addressing the scenario of multiple occurrences of identical components and/or identical features.

\end{itemize}


\subsubsection{Revision history before draft version 1.0.1}
\label{def:v1_before}

See the past work (\sec{def:Past_work}).


\clearpage

