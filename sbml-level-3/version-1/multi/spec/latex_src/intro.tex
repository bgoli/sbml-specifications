% -*- TeX-master: "multi" -*-

%%%%%%%%%%%%%%
% introduction
%%%%%%%%%%%%%%
\section{Introduction}
\label{def:Introduction}

This Multistate, Multicomponent and Multicompartment Species (Multi) package provides an extension of \SbmlLevelThreeWC\ that supports encoding \smodels\ with molecular complexes that have multiple components and can exist in multiple states and in multiple \compartments. One of its goals also is to provide a platform for sharing \smodels\ based on the specifications of bi-molecular interactions and the rules governing such interactions [\cite{ref:simmune2012, ref:scienceSignaling2006, ref:FeretPnas2009, ref:modeler2013}]. This specification covers the goals and features described in \multiOneProposalWC\ for extending SBML to carry the information for \textit{multistate multicomponent} \species\ with revised data structure. In addition, this specification includes the feature for \textit{multicompartment} \species\ as described in the releases of the Multi proposal [\cite{ref:multiproposal280}, \cite{ref:revisedMulti}].

\subsection{Proposal and specifications}
\label{def:Proposal}

The proposal corresponding to this package specification is available at:

\hspace{3ex}\url{http://sbml.org/Community/Wiki/SBML_Level_3_Proposals/Multistate_and_Multicomponent_Species_Proposal}

The specifications (v1.0.1 to current) are located at:

\hspace{3ex}\url{https://sourceforge.net/p/sbml/code/HEAD/tree/trunk/specifications/sbml-level-3/version-1/multi/spec/}

\subsection{Package dependencies}
\label{def:Package_dependencies}

The Multi package has no dependencies on other \SbmlLevelThree\ packages.

\subsection{Document conventions}
\label{def:Document_conventions}

UML 1.0 notation is used in this document to define the constructs provided by this package. Colors 
in the diagrams carry the following additional information for the benefit of those viewing the 
document on media that can display color:

\begin{itemize}
 \item {\color{black}\framebox{\textit{Black}}} Items colored black are components taken unchanged 
      from their definitions in the \SbmlLevelThreeCore\ specification document.
 \item {\color{mediumgreen}\dbox{\textit{Green}}} Items colored green are components that exist in 
      \SbmlLevelThreeCore, but are extended by this package. Class boxes are also drawn with with 
      dashed lines to further distinguish them.
 \item {\color{sbmlblue}\framebox{Blue}} Items colored blue are new components introduced in this 
      package specification. They have no equivalent in  the \SbmlLevelThreeCore\ specification. 
\end{itemize}
 
For other matters involving the use of UML, XML and typographical conventions, this document follows the conventions
used in the \SbmlLevelThreeCore\ specification document [\cite{ref:sbmll3v1}].

For simplicity, \val{...} in all example code refers to some unspecified code content, that is not important for the purpose of illustrating the issue at hand.
