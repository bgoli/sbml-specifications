% -*- TeX-master: "main"; fill-column: 72 -*-
\section{The Systems Biology Ontology and the \token{sboTerm} attribute}
\label{sec:fbcsboTerm}
\label{sec:fbcsbo}

%\begin{verbatim}
%
%SBML::Document allows
%
%    SBO:0000624
%
%SBML::Parameter allows
%
%    SBO:0000625
%    SBO:0000626
%
%\end{verbatim}

The following text on the usage of SBO has been extracted from the \sbmlthreecore specification and is provided here for your convenience. Please consult the official documentation for more details. \ref{tab:fbcsboterm-availability} shows the additional SBO terms specified by the \FBCPackage.
%
\begin{table}[h]
  \centering
  \begin{tabular}{lll}
    \toprule
    \textbf{SBML Element}& \textbf{SBO Name} & \textbf{SBO term} \\
    \midrule
    \textbf{Document}   & flux balance framework 		& SBO:0000624 \\
    \Parameter          & flux bound    	            & SBO:0000625 \\
    \Parameter          & default flux bound          	& SBO:0000626 \\
    \bottomrule
  \end{tabular}
  \caption{SBML components and the additional types of SBO terms specified in the \FBCPackage that may be assigned to them. Note that the
    important aspect here is the set of specific SBO identifiers,
    not the SBO term names, because the names may change as SBO
    continues to evolve. See the text and SBO online resources for detailed descriptions on the meaning and usage of these terms.}
  \label{tab:fbcsboterm-availability}
\end{table}

The values of \token{id} attributes on SBML components allow the components to be cross-referenced within a model. The values of \token{name} attributes on SBML components provide the opportunity to assign them meaningful labels suitable for display to humans.  The specific identifiers and labels used in a model necessarily must be unrestricted by SBML, so that software and users are free to pick whatever they need. However, this freedom makes it more difficult for software tools to determine, without additional human intervention, the semantics of models more precisely than the semantics provided by the SBML object classes defined in other sections of this document.

An approach to solving this problem is to associate model components with terms from carefully curated controlled vocabularies (CVs).  This is the purpose of the optional \token{sboTerm} attribute provided on the SBML class \SBase.  The \token{sboTerm} attribute always refers to terms belonging to the Systems Biology Ontology\footnote{\sboref}. In this section, we discuss the \token{sboTerm} attribute, SBO, the motivations and theory behind their introduction, and guidelines for their use.

SBO is not part of SBML; it is being developed separately, to allow the modeling community to evolve the ontology independently of SBML.  However, the terms in the ontology are being designed keeping SBML components in mind, and are classified into subsets that can be directly related with SBML components such as reaction rate expressions, parameters, and a few others, see below.  The use of \token{sboTerm} attributes is optional, and the presence of \token{sboTerm} on an element does not change the way the model is \emph{interpreted}.  Annotating SBML elements with SBO terms adds additional semantic information that may be used to \emph{convert} the model into another model, or another format. Although SBO support provides an important source of information to understand the meaning of a model, software does not need to support \token{sboTerm} to be considered SBML-compliant.

Although the use of SBO can be beneficial, it is critical to keep in mind that the presence of an \token{sboTerm} value on an object \emph{must not change the fundamental mathematical meaning} of the model.  An SBML model must be defined such that it stands on its own and does not depend on additional information added by SBO terms for a correct mathematical interpretation.  SBO term definitions will not imply any alternative mathematical semantics for any SBML object labeled with that term.  Two important reasons motivate this principle.  First, it would be too limiting to require all software tools to be able to understand the SBO vocabularies in addition to understanding SBML. Supporting SBO is not only additional work for the software developer; for some kinds of applications, it may not make sense. If SBO terms on a model are optional, it follows that the SBML model \emph{must} remain unambiguous and fully interpretable without them, because an application reading the model may ignore the terms.  Second, we believe allowing the use of \token{sboTerm} to alter the mathematical meaning of a model would allow too much leeway to shoehorn inconsistent concepts into SBML objects, ultimately reducing the interoperability of the models.

%\subsection{Using SBO and \token{sboTerm}}
%
%The \token{sboTerm} attribute data type is always \primtype{SBOTerm}. When present in a given model object instance, the attribute's value must be an identifier taken from the Systems Biology Ontology.  This identifier must refer to a single SBO term that best defines the entity encoded by the SBML object in question.  An example of the type of relationship intended is: \emph{the KineticLaw in reaction R1 is a  first-order irreversible mass action rate law}.
%
%Note the careful use of the words ``defines'' and ``entity encoded by the SBML object'' in the paragraph above.  As mentioned, the relationship between the SBML object and the URI is:
%
%\begin{quote}
%  The ``thing'' encoded by this SBML object has a characteristic
%  that is an instance of the ``thing'' represented by the
%  referenced SBO term.
%\end{quote}

