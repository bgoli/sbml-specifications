% -*- TeX-master: "main"; fill-column: 72 -*-

\section{Introduction}
\label{intro}

Constraint based modelling is a widely used methodology used to analyse and
study biological networks on both a small and whole organism (genome)
scale. Typically these models are underdetermined and constraint based
methods (e.g., linear, quadratic optimization) are used to optimise
specific model properties. This is assumed to occur under a defined set of
constraints (e.g., stoichiometric, metabolic) and bounds (e.g.,
thermodynamic, experimental and environmental) on the values that the
solution fluxes can obtain.

Perhaps the most well known (and widely used) analysis method is Flux
Balance Analysis~\citep[FBA; ][]{orth_2010}, where for a Genome Scale
Reconstruction (GSR) model~\citep{oberhardt_2009} a target flux is
maximised (typically a flux to biomass) where other input/output fluxes
have been bounded to simulate a selected growth environment.

As constraint based models are generally underdetermined, i.e., few or none
of the kinetic rate equations and related parameters are known, it is
crucial that a model definition includes the ability to define optimization
parameters such as objective functions, flux bounds and constraints
... currently this is not possible in SBML.

The question of how to encode constraint based (or historically FBA) models
in SBML is not new. However, with the advances in the methods used to
construct GSR scale models and the wider adoption of constraint based
modelling in biotechnological/medical applications has led to a rapid
increase in both the number of models being constructed and the tools used
to analyse them.

Faced with such rapidly growing diversity, the need for a standardised
description for the definition, exchange and annotation of constraint
based models is vital. As the core model components (e.g., species,
reactions, stoichiometry) can already be effectively described in SBML
(with its significant community, software and tool support) it seems
prudent to use it as a basis for such a description. In addition, the
modularised extension mechanism now available in SBML Level~3 provides
an ideal platform for an efficient implementation.
