% -*- TeX-master: "main"; fill-column: 72 -*-

\section{ Introduction and motivation } \label{intro}

Constraint-based modeling is a widely accepted methodology used to
analyze and study biological networks on both a small and whole organism
(genome) scale. Due to their large size these models are generally
underdetermined and constraint-based optimization methods (such as linear or
mixed integer convex optimization) are used to analyze them. Optimization is
assumed to occur within a defined set of constraints (e.g. stoichiometric,
metabolic) and bounds (e.g. thermodynamic, experimental and environmental) on
the values that the solution fluxes can obtain.

Perhaps the most well known (and widely used) analysis method is Flux
Balance Analysis (FBA) which is performed on Genome Scale Metabolic
Reconstructions~(GSR's; \citealt{oberhardt_2009}). Using FBA a
target flux is optimized (e.g. maximizing a flux to biomass or
minimizing ATP production) while other fluxes can be bounded to simulate
a selected growth environment or specific metabolic state.
%~\citep[FBA; ][]{orth_2010}

As constraint-based models are generally underdetermined and few or
none of the kinetic rate equations, flux capacity constraints and related
parameters are known it is crucial that a model definition includes the
ability to define optimization parameters such as objective functions, flux
bounds and constraints. Currently this is not possible in the Systems Biology
Markup Language (SBML) Level 2 or Level 3 core specification \citep{sbml,
sbml3core}.

The question of how to encode constraint-based (also referred to as
steady state or FBA) models in SBML is not new. However, advances in the
methods used to construct genome scale constraint-based models and the
wider adoption of constraint-based modeling in biotechnological/medical
applications have led to a rapid increase in both the number of models
being constructed and the tools used to analyze them.

Faced with such growth, both in number and diversity, the need for a
standardized data format for the definition, exchange and annotation of
constraint-based models has become critical. As the core model
components (e.g. species, reactions, stoichiometry) can already be
efficiently described in SBML (with its associated active community,
software and tool support) the \FBCPackage aims to extend SBML Level~3
core by adding the elements necessary to encode current and future
constraint-based models.

\subsection{Proposal corresponding to this package specification}

\newtxt{This specification for Flux Balance Constraints in SBML Level~3
Version~1 is based on the proposal (Olivier and Bergmann) archived at the URL:}

%\begin{center}
  \vspace*{1ex}\small
  \url{https://web.archive.org/web/20151006163154/http://sbml.org/Community/Wiki/SBML_Level_3_Proposals/Flux_Balance_Constraints_Proposal_(2012)}
  \vspace*{1ex}
%\end{center}

\changed{Issues with this and other SBML packages can be filed at \small\url{https://github.com/sbmlteam/sbml-specifications/issues}.  Issues pertaining to this package in particular are labeled with the 'L3 Package: fbc' tag.}
The version of the proposal used as the starting point for this specification is the version of March 2012. Previous versions of the current proposal are:

\begin{description}
  \item[Proposal version 3 (March 2012)]
  %\item [] {\small\url{http://sbml.org/Community/Wiki/SBML_Level_3_Proposals/Flux_Balance_Constraints_Proposal_(2012)}}
  \item [] {\small\url{https://web.archive.org/web/20151006163154/http://sbml.org/Community/Wiki/SBML_Level_3_Proposals/Flux_Balance_Constraints_Proposal_(2012)}}
  \item[Proposal version 2 (March 2011)]
  %\item [] {\small\url{http://sbml.org/Community/Wiki/SBML_Level_3_Proposals/Flux_Constraints_Proposal}}
  \item [] {\small\url{https://web.archive.org/web/20120824234050/http://sbml.org/Community/Wiki/SBML_Level_3_Proposals/Flux_Constraints_Proposal}}
  \item[Proposal version 1 (February 2010)]
  %\item [] {\small\url{http://precedings.nature.com/documents/4236/version/1}}
  \item [] {\small\url{https://doi.org/10.1038/npre.2010.4236.1}}
\end{description}

Details of related, earlier, independent proposals are provided in \ref{background}.

\subsection{Package dependencies}

The \FBCPackage adds additional attributes and classes to \sbmlthreecore and has no dependency on any other SBML Level~3 package.

%It is also designed to work seamlessly with other SBML Level~3 packages.

\subsection{Document conventions} \label{conventions}

Following the precedent set by the SBML Level~3 Core specification
document, we use UML~1.0 (Unified Modeling Language;
\citealt{eriksson:1998,oestereich:1999}) class diagram notation to
define the constructs provided by this package. We also use color in the
diagrams to carry additional information for the benefit of those
viewing the document on media that can display color. The following are
the colors we use and what they represent:

\begin{itemize}

\item[\raisebox{2.75pt}{\colorbox{black}{\rule{0.8pt}{0.8pt}}}]
\emph{Black}: Items colored black in the UML diagrams are components
taken unchanged from their definition in the SBML Level~3 Core
specification document.

\item[\raisebox{2.75pt}{\colorbox{mediumgreen}{\rule{0.8pt}{0.8pt}}}]
\emph{\textcolor{mediumgreen}{Green}}: Items colored green are
components that exist in SBML Level~3 Core, but are extended by this
package. Class boxes are also drawn with dashed lines to further
distinguish them.

\item[\raisebox{2.75pt}{\colorbox{darkblue}{\rule{0.8pt}{0.8pt}}}]
\emph{\textcolor{darkblue}{Blue}}: Items colored blue are new components
introduced in this package specification. They have no equivalent in the
SBML Level~3 Core specification.

\item[\raisebox{2.75pt}{\colorbox{red}{\rule{0.8pt}{0.8pt}}}]
\emph{\textcolor{red}{Red}}: Text and items colored red represent changes that have been made since the last time the proposal was officially distributed.

% \item[\raisebox{2.75pt}{\colorbox{ashgrey}{\rule{0.8pt}{0.8pt}}}]
% \emph{\textcolor{ashgrey}{Grey}}: Items colored ashgrey are components
% deprecated in this package specification. They have no equivalent in the
% SBML Level~3 Core specification.

\end{itemize}

We also use the following typographical conventions to distinguish the
names of objects and data types from other entities; these conventions
are identical to the conventions used in the SBML Level~3 Core
specification document:

\begin{description}

\item \abstractclass{AbstractClass}: Abstract classes are classes that
are never instantiated directly, but rather serve as parents of other
object classes. Their names begin with a capital letter and they are
printed in a slanted, bold, sans-serif typeface. In electronic document
formats, the class names defined within this document are also
hyperlinked to their definitions; clicking on these items will, given
appropriate software, switch the view to the section in this document
containing the definition of that class. (However, for classes that are
unchanged from their definitions in SBML Level~3 Core, the class names
are not hyperlinked because they are not defined within this document.)

\item \class{Class}: Names of ordinary (concrete) classes begin with a
capital letter and are printed in an upright, bold, sans-serif typeface.
In electronic document formats, the class names are also hyperlinked to
their definitions in this specification document. (However, as in the
previous case, class names are not hyperlinked if they are for classes
that are unchanged from their definitions in the SBML Level~3 Core
specification.)

\item \token{SomeThing}, \token{otherThing}: Attributes of classes, data
type names, literal XML, and generally all tokens \emph{other} than SBML
UML class names, are printed in an upright typewriter typeface.
Primitive types defined by SBML begin with a capital letter; SBML also
makes use of primitive types defined by XML
Schema~1.0~\citep{biron:2000,fallside:2000,thompson:2000}, but
unfortunately, XML~Schema does not follow any capitalization convention
and primitive types drawn from the XML~Schema language may or may not
start with a capital letter.

\end{description}

For other matters involving the use of UML and XML, we follow the
conventions used in the SBML Level~3 Core specification document.

