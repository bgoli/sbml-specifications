% -*- TeX-master: "main"; fill-column: 72 -*-

\section{Preamble}
\label{preamble}

\subsection{ Details of `fbc' package proposal }
This specification is based on a set of proposals proposed and refined by the authors. Details of these proposals are presented here while details of previous proposals are provided in Section~\ref{background}.

\subsubsection{ Tracking number }
As initially listed in the SBML issue tracking system. \url{http://sourceforge.net/tracker/?func=detail&aid=3154219&group_id=71971&atid=894711
3154219}

\subsubsection{ Version history }
\begin{description}
  \item[Version 3 (March 2012)]
  \item [] \url{http://sbml.org/Community/Wiki/SBML_Level_3_Proposals/Flux_Balance_Constraints_Proposal_(2012)}
  \item[Version 2 (March 2011)]
  \item [] \url{http://sbml.org/Community/Wiki/SBML_Level_3_Proposals/Flux_Constraints_Proposal}
  \item[Version 1 (February 2010)]
  \item [] \url{http://precedings.nature.com/documents/4236/version/1}
\end{description}


\section{ Introduction and motivation }
\label{intro}

Constraint based modelling is a widely used methodology used to analyse and study biological networks on both a small and whole organism (genome) scale. Typically these models are underdetermined and constraint based methods (e.g. linear, quadratic optimization) are used to optimise specific model properties. This is assumed to occur under a defined set of constraints (e.g. stoichiometric, metabolic) and bounds (e.g. thermodynamic, experimental and environmental) on the values that the solution fluxes can obtain.

Perhaps the most well known (and widely used) analysis method is Flux Balance Analysis~\citep[FBA; ][]{orth_2010} which is performed on Genome Scale Reconstructions~\citep[GSR's; ][]{oberhardt_2009}. Using FBA a target flux is optimized (e.g. maximising a flux to biomass or minimising ATP production) while other fluxes can be bounded to simulate a selected growth environment or specific metabolic state.

As constraint based models are generally underdetermined, i.e. few or none of the kinetic rate equations and related parameters are known, it is crucial that a model definition includes the ability to define optimisation parameters such as objective functions, flux bounds and constraints ... currently this is not possible in the Systems Biology Markup Language (\SBML) Level 2 or Level 3 core specification \citep{sbml, sbml3core}.

The question of how to encode constraint based (a.k.a. `FBA') models in \SBML is not new. However, advances in the methods used to construct GSR scale models and the wider adoption of constraint based modelling in biotechnological/medical applications have led to a rapid increase in both the number of models being constructed and the tools used to analyse them.

Faced with such growth, both in number and diversity, the need for a standardised data format for the definition, exchange and annotation of constraint based models has become critical. As the core model components (e.g. species, reactions, stoichiometry) can already be efficiently described in \SBML (with its significant community, software and tool support) the `FBC' package aims to extend \SBML core by adding the elements necessary to describe current and future constraint based models.
