% -*- TeX-master: "main"; fill-column: 72 -*-

\section{Examples}
\label{examples}

This section contains a variety of examples of SBML Level~3 Version~1
documents employing the Flux Balance Constraints package.

\subsection{\FBC syntax examples}

\paragraph{Encoding the \FluxBound}
As described in \ref{fluxbound-class} the flux bound represents a mathematical (in)equality of the form <\token{reaction}> <\token{operator}> <\token{value}>. In SBML Level~3 Version~1 with \FBC this is encoded as:
%
\exampleFile{examples/ex_fb_fbc.txt}
%
This example illustrates two things: the encoding of $\infty$ and that care should be used when selecting inequalities such as \val{less} or \val{greater}. While mathematically there is a difference, this difference is only practically relevant when working with rational arithmetic (solvers).

\paragraph{Encoding the \Objective}
The \FBC allows for the definition of multiple `objective functions' with one being designated as active (see \ref{objective-class}) the following example illustrates this:
%
\exampleFile{examples/ex_objf_fbc.txt}
%
Note how both \Objective instances differ in \token{type} and each contains different set of \class{FluxObjectives}.

\paragraph{The extended \Species}
The \FBC package extend the \SBML \Species of \sbmlthreecore by providing attributes for storing \token{charge} and \token{chemicalFormula}
%
\exampleFile{examples/ex_spec_l3.txt}