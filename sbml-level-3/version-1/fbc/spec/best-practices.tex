% -*- TeX-master: "main"; fill-column: 72 -*-

\section{Best practices}
\label{best-practices}

In this section, we recommend a number of practices for using and
interpreting various constructs in the \FBCPackage.
These recommendations are non-normative, but we advocate them strongly;
ignoring them will not render a model invalid, but may reduce
inter-operability between software and models.

\subsection{Examples contrasting the current \SBML L2 encoding with L3 and \FBC}
These examples contrast some elements of an existing model, iJR904 from the \textsf{BiGG} Database encoded in the \textsf{COBRA} format \cite{ijr904, bigg, cobra} that have been translated into SBML Level~3 Version~1 using the \textsf{CBMPy} implementation of the \FBC package \cite{pysces, cbmpy} and \textsf{libSBML} experimental ver.~5.6.0 \cite{libsbml}.

\subsubsection*{Objective function definition}
\paragraph{\SBML Level 2 objective function}
\exampleFile{examples/ex_objf_bigg.txt}

\paragraph{The \SBML Level 3 objective function}
\protect\exampleFile{examples/ex_objf_l3.txt}

\subsubsection*{Species definition}
It is particularly useful to contrast the differences in the \Species definition as it is used in genome scale models.

\paragraph{\SBML Level 2 \Species annotation version 1}
To begin with we examine the \SBML Level~2 Version~1 species definition used by the BiGG database and \textsf{COBRA} \cite{bigg, cobra}. Note how the \token{name} attribute is overloaded with the chemical formula.
%
\exampleFile{examples/ex_spec_bigg.txt}

\paragraph{\SBML Level 2 \Species annotation version 2}
A newer variation of the above, probably necessitated by the discontinuation of the \token{charge} attribute in \SBML and \textsf{libSBML}
%
\exampleFile{examples/ex_spec_cobra.txt}

\paragraph{The \SBML Level 3 \Species attributes}
Hopefully, with the adoption of \SBML \FBC these species properties can be unified into a common format.
%
\exampleFile{examples/ex_spec_l3.txt}

\subsubsection*{Reaction definition and flux bounds}

\paragraph{\SBML Level 2 \Reaction}
%
\exampleFile{examples/ex_reaction_bigg.txt}

\paragraph{The \SBML Level 3 \Reaction and \FluxBound}

As an example and where (unambiguously) possible the \SBML Level 2 annotation has been converted into MIRIAM compliant RDF, in this case the \textit{EC number}. In addition this example highlights an open issue, namely, how to deal with the information currently encoded in the \Notes element. Currently this information is not completely encodable using the current \SBML annotation mechanism, however, that issue is currently discussed on the package working group.
%for an example of an alternative mechanism, implemented as a tool specific annotation please see \ref{future}.
%
\exampleFile{examples/v2harmony-ex_reaction_l3_nokv.txt}
%
\exampleFile{examples/v2harmony-ex_fb_l3.txt}
