% -*- TeX-master: "main"; fill-column: 72 -*-

\section{Best practices}
\label{best-practices}
\begin{newsection}

In this section, we illustrate a number of practices for using and
interpreting various constructs in the \FBCPackage.
These recommendations are non-normative, ignoring them will not render a model invalid, but may reduce inter-operability.

\subsection{Examples contrasting the current \SBML L2 encoding with L3 and \FBC}
These examples contrast some elements of an existing model, iJR904 from the \textsf{BiGG} Database encoded in the \textsf{COBRA} SBML Level~2 Version~1 format \citep{ijr904, bigg, cobra} that have been translated into SBML Level~3 Version~1 \FBC Version 2.
%using the \textsf{CBMPy} implementation of the \FBC package \citep{pysces, cbmpy} and \textsf{libSBML} experimental ver.~5.6.0 \citep{libsbml}.

\subsubsection*{Objective function definition}
\paragraph{\SBML Level 2 objective function}
\exampleFile{examples/ex_objf_bigg.txt}

\paragraph{The \SBML Level 3 objective function}
\protect\exampleFile{examples/ex_objf_l3.txt}

\subsubsection*{Species definition}
\paragraph{\SBML Level 2 \Species annotation version 1}
Examine the \SBML Level~2 Version~1 \Species definition. Note how the \token{name} attribute is overloaded with the chemical formula in a tool specific way.
%\textsf{COBRA} \citep{bigg, cobra}
%
\exampleFile{examples/ex_spec_bigg.txt}

\paragraph{\SBML Level 2 \Species annotation version 2}
A variation of the previous syntax that appeared in later models.
%
\exampleFile{examples/ex_spec_cobra.txt}

\paragraph{The \SBML Level 3 \FBC \Species attributes}
With the adoption of \SBML \FBC these \Species properties can now be unified into a common format.
%
\exampleFile{examples/ex_spec_l3.txt}

\subsubsection*{Reaction definition and flux bounds}
%\textbf{\color{Mahogany}To be updated when V2 syntax is stable.}
\paragraph{\SBML Level 2 \Reaction}
%
\exampleFile{examples/ex_reaction_bigg.txt}

\paragraph{The \SBML Level 3 \FBC \Reaction}

As an example of a good annotation practice the \textsf{EC number} stored in the \Notes element has been converted into MIRIAM compliant RDF. The \FBCPackage also facilitates the structured definition and use of gene protein associations and flux capacity constraints.

\exampleFile{examples/v2harmony-ex_reaction_l3_nokv.txt}
%
%\exampleFile{examples/v2harmony-ex_fb_l3.txt}

\end{newsection} 