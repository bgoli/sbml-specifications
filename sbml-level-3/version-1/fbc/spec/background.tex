% -*- TeX-master: "main"; fill-column: 72 -*-

\section{Background}
\label{background}

\subsection{Problems with current SBML approaches}

While there is currently no official way of encoding constraint based models
in SBML L2 there have been pragmatic approaches used by a variety of groups
and applications. Arguably the most comprehensive and widely used format is
that used by the \textsf{COBRA toolbox}~\citep{cobra} where the metabolic
reaction network is well defined using SBML's \Reaction and \Species
classes. However, other FBA specific model components such as flux bounds
and the reactions that take part in the objective function are less well
defined. In this case \LocalParameter elements are used which (implicitly)
rely on e.g.~all tools knowing and using the same naming convention for the
parameter ID's. Furthermore, reaction annotations are generally stored as
tool specific HTML key-value pairs in a \Notes element which has routinely
led to different research groups and software using in-house and/or tool
specific ways to describe the same information.
%
\newtxt{A common example of such an annotation is the so called `gene association' 
used to assess the effect of gene deletions on a reaction network.}
%
While a step in the right direction this format is not suitable for 
implementation in SBML~Level~3.

It is worth noting that while SBML~Level~2 does have a construct known as
\Constraint its function is limited to measuring and reporting the model
variables behavior in time. In contrast the \FluxBound enforces the bounds
on a steady-state flux and they can therefore be considered to be
independent of one another.
%For example while a \Constraint reports behavior on a \Species during a
%dynamic time simulation and a \FluxBound determining the behavior of the
%fluxes at steady state where, by definition, the variable \Species are
%constant.
%
In addition, certain attributes that were widely used by the constraint based 
modeling community such as the \Species attribute \token{charge} were
removed from SBML. This has had the effect that a significant number of
tools and models are still limited to using SBML Level~2 Version~1 \cite{oberhardt_2009}.

\subsection{Past work on this problem or similar topics}
The problem of describing and annotating FBA models in SBML has been raised
at various times in the past few years. In this regard there are two known
putative proposals one by Karthik Raman and the other by the Church
Laboratory. As far as we are aware these proposals never developed beyond
their initial presentation at SBML forums/hackathons. In 2009 the discussion
was reopened at the SBML Forum held in Stanford and has subsequently
developed into the current active package proposal and this document (see
\ref{intro}). In reverse chronological order these are:
%
\begin{description}
  \item[Brett Olivier (2009)] SBML Level 3 FBA package discussion
  \item[]\url{http://sbml.org/images/4/4a/Olivier_sbml_forum_2009_09_04.pdf}
  \item[Karthik Raman (2005)] Flux annotations in SBML
  \item[]\url{http://sbml.org/images/d/d9/Raman-flux-annotations.pdf}
  \item[Church laboratory (pre 2005)] Metabolic flux model annotations
  \item[]\url{http://sbml.org/Community/Wiki/Old_known_SBML_annotations_list}
\end{description}


%\subsection{ Design goals of the current \FBC Package }
%\label{design-goals}
