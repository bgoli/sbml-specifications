% -*- TeX-master: "main"; fill-column: 72 -*-

\section{Package syntax and semantics}
\label{syntax}

In this section, we define the syntax and semantics of the Flux Balance
Constraints package for SBML Level~3 Version~1.  We expound on the various
data types and constructs defined in this package, then in \sect{examples},
we provide complete examples of using the constructs in example SBML
models.

% -----------------------------------------------------------------------------
\subsection{Namespace URI and other declarations necessary for using this package}
\label{xml-namespace}

Every SBML Level~3 package is identified uniquely by an XML namespace URI.
For an SBML document to be able to use a given SBML Level~3 package, it
must declare the use of that package by referencing its URI.  The following
is the namespace URI for this version of the Flux Balance Constraints
package for SBML Level~3 Version~1:
\begin{center}
\uri{http://www.sbml.org/sbml/level3/version1/fbc/version1}
\end{center}

In addition, SBML documents using a given package must indicate whether
understanding the package is required for complete mathematical
interpretation of a model, or whether the package is optional.  This is
done using the attribute \token{required} on the \token{<sbml>} element in
the SBML document.  For the Flux Balance Constraints package, the value of
this attribute must be set to \val{true}.

The following fragment illustrates the beginning of a typical SBML model
using SBML Level~3 Version~1 and this version of the Flux Balance
Constraints package:

\begin{example}
<?xml version="1.0" encoding="UTF-8"?>
<sbml xmlns="http://www.sbml.org/sbml/level3/version1/core" level="3" version="1"
      xmlns:fbc="http://www.sbml.org/sbml/level3/version1/fbc/version1" fbc:required="true">
\end{example}
    
