\documentclass{jib}
\newlength{\platz}
\setlength{\platz}{15pt}
\RequirePackage{listings}

\usepackage{changepage} %test, TODO remove

\lstset{%
  basicstyle=\ttfamily,
  fontadjust,
  flexiblecolumns=true,
  frame=L,
  xleftmargin=15pt,
  framesep=5pt,
  emphstyle=\rmfamily\itshape}


%%%%%%%%%%%%%%%%%%%%%%%%%%%%%%%%%%%%%%%%%%%%%%%%%%%%%%%%%%
% JIB Header/Footer
%%%%%%%%%%%%%%%%%%%%%%%%%%%%%%%%%%%%%%%%%%%%%%%%%%%%%%%%%%
%\jibvolume{XX} % insert volume
%\jibissue{X}   % insert issue
%\jibpages{XXX} % insert article ID
%\jibyear{XXXX} % insert year
%\makeHeaderFooter{} % leave as is
%%%%%%%%%%%%%%%%%%%%%%%%%%%%%%%%%%%%%%%%%%%%%%%%%%%%%%%%%%

\begin{document}

%%%%%%%%%%%%%%%%%%%%%%%%%%%%%%%%%%%%%%%%%%%%%%%%%%%%%%%%%%
%
% Title Page
%
%%%%%%%%%%%%%%%%%%%%%%%%%%%%%%%%%%%%%%%%%%%%%%%%%%%%%%%%%%

\begin{jibtitlepage}

\jibtitle{SBML Level 3 Package: Flux Balance Constraints version 2}


%We did not provide author(s) nor author footnote(s), please complete as applicable.
% Please make sure to use unique footnote characters for each author
\jibauthor{Brett G.\ Olivier\iref[,]{vu}\iref[,]{hdbg}\iref{cal} and
           Frank T.\ Bergmann\iref[,]{hdbg}\iref{cal}}

\addjibinstitution{vu}{Systems Bioinformatics, AIMMS, Vrije Universiteit Amsterdam, The Netherlands}
\addjibinstitution{hdbg}{Modelling of Biol. Processes,  BioQUANT/COS, Heidelberg University, Germany}
\addjibinstitution{cal}{Department of Computing and Mathematical Sciences, California Institute of Technology,Pasadena, CA, USA}

\end{jibtitlepage}

\begin{adjustwidth}{}{1cm}

\abstract{Constraint-based modeling is a well established modeling methodology used to analyze and study biological networks on both a medium and genome scale. Due to their large size and complexity such steady-state flux models are, typically, analyzed using constraint-based optimization techniques, for example, Flux Balance Analysis (FBA).

The Flux Balance Constraints (FBC) Package extends SBML Level 3 and provides a standardized format for the encoding, exchange and annotation of constraint-based models. It includes support for modeling concepts such as objective functions, flux bounds and model component annotation that facilitates reaction balancing. Version two expands on the original release \cite{fbcv1} by adding official support for encoding gene-protein associations and their associated elements. In addition to providing the elements necessary to unambiguously encode existing constraint-based models, the FBC Package provides an open platform facilitating the continued, cross-community development of an interoperable, constraint-based model encoding format.
}

\end{adjustwidth} % please sdo not change

\textbf{Keywords:} SBML, constraint-based modeling, standards

\addcontentsline{toc}{section}{References}
\bibliographystyle{vancouver}
\bibliography{fbc_references}
\nocite{*}


\end{document}
