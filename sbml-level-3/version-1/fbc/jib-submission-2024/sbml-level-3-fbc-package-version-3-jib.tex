\documentclass{jib}
\newlength{\platz}
\setlength{\platz}{15pt}
\RequirePackage{listings}

\usepackage{changepage} %test, TODO remove

\lstset{%
  basicstyle=\ttfamily,
  fontadjust,
  flexiblecolumns=true,
  frame=L,
  xleftmargin=15pt,
  framesep=5pt,
  emphstyle=\rmfamily\itshape}


%%%%%%%%%%%%%%%%%%%%%%%%%%%%%%%%%%%%%%%%%%%%%%%%%%%%%%%%%%
% JIB Header/Footer
%%%%%%%%%%%%%%%%%%%%%%%%%%%%%%%%%%%%%%%%%%%%%%%%%%%%%%%%%%
%\jibvolume{XX} % insert volume
%\jibissue{X}   % insert issue
%\jibpages{XXX} % insert article ID
%\jibyear{XXXX} % insert year
%\makeHeaderFooter{} % leave as is
%%%%%%%%%%%%%%%%%%%%%%%%%%%%%%%%%%%%%%%%%%%%%%%%%%%%%%%%%%

\begin{document}

%%%%%%%%%%%%%%%%%%%%%%%%%%%%%%%%%%%%%%%%%%%%%%%%%%%%%%%%%%
%
% Title Page
%
%%%%%%%%%%%%%%%%%%%%%%%%%%%%%%%%%%%%%%%%%%%%%%%%%%%%%%%%%%

\begin{jibtitlepage}

\jibtitle{SBML Level 3 Package: Flux Balance Constraints version 3}


%We did not provide author(s) nor author footnote(s), please complete as applicable.
% Please make sure to use unique footnote characters for each author
\jibauthor{Brett G.\ Olivier\iref{vu} and Frank T.\ Bergmann\iref{hdbg} and Sarah Keating\iref{ukl} and Matthias König\iref{hu}}
%  and Matthias K\ouml;nig\iref{hu}}

\addjibinstitution{vu}{Systems Biology, A-LIFE, Vrije Universiteit Amsterdam, The Netherlands}
\addjibinstitution{hdbg}{Modelling of Biol. Processes,  BioQUANT/COS, Heidelberg University, Germany}
\addjibinstitution{ukl}{Advanced Research Computing Centre, University College London, London, UK}
\addjibinstitution{hu}{Institute for Biology, Institute for Theoretical Biology, Humboldt-Universität zu Berlin, Berlin, Germany}


\end{jibtitlepage}

\begin{adjustwidth}{}{1cm}

\abstract{Constraint-based modeling is a well-established modeling methodology used to analyze and study biological networks at both the medium-scale and genome-scale. Due to their large size and complexity, such steady-state flux models are typically analyzed using constraint-based optimization techniques, such as Flux Balance Analysis (FBA).

The Flux Balance Constraints (FBC) Package extends SBML Level 3 to provide a standardized format for encoding, exchanging, and annotating constraint-based models. It includes support for modeling concepts such as objective functions, flux bounds, and annotation of model components that facilitate reaction balancing. Version two extended the original release by adding official support for encoding gene-protein associations and their associated elements. Version three extends version two by adding 
additional constraints, improves the syntax for storing chemical formulas, and adds a key-value pair for storing additional information in the context of constraint-based modeling.

In addition to providing the elements necessary to uniquely encode existing constraint-based models, the FBC package provides an open platform that facilitates the continued, cross-community development of an interoperable constraint-based model encoding format.
}

\end{adjustwidth} % please do not change

%\textbf{Keywords:} SBML, constraint-based modeling, standards
%\addcontentsline{toc}{section}{References}
%\bibliographystyle{vancouver}
%\bibliography{fbc_references}
\nocite{*}


\end{document}
