% -*- TeX-master: "main" -*-

\section{Background and context}
\label{background}

\subsection{Problems with current SBML approaches}
There is no standard way of specifying spatial models in SBML short of introducing an explicit spatial discretization in the form of a large number of compartments with duplicate species and reactions and additional reactions for coupling due to transport.  This approach hard-codes the numerical methods which destroys portability and is not practical beyond a few compartments.  Tools have been forced to resort to proprietary extensions (e.g. MesoRD custom annotations) to encode geometry. 

\subsection{Past work on this problem or similar topics}
There are many standards for the exchange of geometric information of engineered parts in Computer Aided Design and Manufacturing.  These formats are designed for geometric shapes which are directly specified by a designer rather than the data driven, freeform biological structures encountered in cell biology.  

There also exist standards for the representation of unstructured computational meshes that can encode these freeform biological structures more readily.  However, it is important to note that while a computational mesh necessarily encodes an approximation to the shapes of geometric objects, the particular form will be algorithm dependent.  

To ensure model interoperability, we must encode the geometric shapes in a way that is independent of the numerical methods and even the mathematical framework.  The representation of a spatial model within SBML should be largely invariant of the particular encoding of the geometry definition within that model.  For example, a spatial model represented in SBML that encodes geometry as a set of geometric primitives (e.g. spheres, cylinders) should be easily portable to a tool that only supports polygonal surface tessellation.  It is expected that a geometry translation library will be very useful for interoperability the same way that libSBML greatly improved model interchange by solving similar implementation problems in a standard way.

\subsection{Prior work}

The first version of the Spatial proposal was written [fill in history]

{\color{red} Lucian: \notice We probably do need something here.}
