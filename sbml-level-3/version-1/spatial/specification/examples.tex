% -*- TeX-master: "main" -*-

\section{Examples}
\label{examples}

This section will hopefully contain examples employing the Spatial package for SBML Level~3.


%\exampleFile{figs/speciestype-example.xml}
\exampleFile{examples/analytic_3d.xml}
\exampleFile{examples/csgOnly.xml}
\exampleFile{examples/parametric_1dom.xml}
\exampleFile{examples/parametric_2dom.xml}
\exampleFile{examples/sampledfield_3d.xml}



%\subsection{Simple species typing via annotations}
%\label{examples-speciestype}

%The following is a simple example of using this proposed grouping facility to do something similar to the \SpeciesType example shown in Section~4.6.3 of the SBML Level~2 Version~4 specification (p.~43).

%\exampleFile{figs/speciestype-example.xml}

%In this example, both species \val{ATPc} and \val{ATPm} are intended to be pools of ATP, but located in different compartments.  To indicate that they are both conceptually the same kind of molecular entity, the model includes a group definition of the \val{classification} variety.  The two species \val{ATPc} and \val{ATPm} are both listed as members of the same group.  The \ListOfMembers is given the \token{sboTerm} \val{SBO:0000248} to indicate that both species are small molecules.  Four restrictions are set in place by the \ListOfMemberConstraints element:  both species must be the same type, must be in different \token{compartments}, and must share the same \token{initialConcentration} and \token{constant} values.

%The group definition could be enhanced further by including an annotation on the \ListOfMembers that references the ChEBI database entry for ATP; we omit that detail here in order to concentrate on the Spatial constructs.


%\subsection{Example using meta identifiers}

%In the next example, a group is used to annotate two rules to annotate the fact that the two rules influence a model for some particular reason.

%\clearpage

%\exampleFile{figs/rule-example.xml}

%The key point of this example is the use of meta identifiers for SBML entities (in particular, rules) that do not have regular identifiers (i.e., \token{id} attributes).
