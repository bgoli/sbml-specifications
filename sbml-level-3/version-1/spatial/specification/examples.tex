% -*- TeX-master: "main" -*-

\section{Examples}
\label{examples}

Several models have been created to demonstrate the basic capabilities of the 'spatial' package.  Most of the smaller examples in this specification are snippets from these models.

\begin{itemize}
  
\item \url{https://github.com/sbmlteam/sbml-specifications/tree/release/sbml-level-3/version-1/spatial/specification/examples/analytic_3d.xml}: An example that uses the \AnalyticGeometry construct to define spatial domains by the application of mathematical formulas.

\item \url{https://github.com/sbmlteam/sbml-specifications/tree/release/sbml-level-3/version-1/spatial/specification/examples/csgOnly.xml}: An example that uses the \CSGeometry construct to define spatial domains through building elements based on primitive shapes.

\item \url{https://github.com/sbmlteam/sbml-specifications/tree/release/sbml-level-3/version-1/spatial/specification/examples/parametric_1dom.xml}: An example that uses the \ParametricGeometry construct to define spatial domains through building a mesh that defines a surface.

\item \url{https://github.com/sbmlteam/sbml-specifications/tree/release/sbml-level-3/version-1/spatial/specification/examples/parametric_2dom.xml}: A slightly more complicated \ParametricGeometry example that defines three nested three-dimensional domains, with two-dimensional domains at the boundaries.

\item \url{https://github.com/sbmlteam/sbml-specifications/tree/release/sbml-level-3/version-1/spatial/specification/examples/sampledfield_3d.xml}: An example that uses the \SampledFieldGeometry construct to define spatial domains through assigning different values in a field (such as those produced in an image) to different domains.

\item \url{https://github.com/sbmlteam/sbml-specifications/tree/release/sbml-level-3/version-1/spatial/specification/examples/sampledfield_asnt.xml}: The same \SampledFieldGeometry example extended to use the \SampledField values in an initial assignment for a species.

\end{itemize}
