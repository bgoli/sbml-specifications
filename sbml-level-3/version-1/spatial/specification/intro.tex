% -*- TeX-master: "main" -*-

\section{Introduction}
\label{intro}

A set of biochemical process in cellular physiology may be modeled using different choices of spatial and temporal scales depending on the questions to be addressed.  SBML Level 3 Core has explicit support for multi-compartmental modeling where cellular organization is approximated by a set of compartments (e.g. membrane-bound organelles) containing well-stirred populations of molecules.  However, the coupling between localized biochemical reactions and diffusive molecular transport within the constraints of cellular geometry often results in important nonuniform molecular distributions.  While it is often possible to approximate the influence of spatial organization and localization within a compartmental model using altered parameters and additional species, it is sometimes simpler and always more mechanistic to directly model these spatial processes.  

An increasing number of modeling and simulation tools include direct support for modeling with explicitly defined cellular geometry.  These models generally include heterogeneous molecular distributions, diffusive transport, and spatially localized reactions.  These spatial models generally belong to two different mathematical frameworks, stochastic (where each molecule is tracked in space and time) and deterministic (where time varying species concentration fields are described by partial differential equations).  

There are a sufficient number of spatial modeling tools and spatial models to justify the effort of creating a spatial modeling extension of SBML.  All such models must describe the cellular geometry, map molecular species to spatial locations, map reactions to spatial locations, and specify molecular transport within geometric compartments and at boundaries of these compartments.  

It is the purpose of this SBML Level 3 extension to define a common representation for cellular geometry, spatial mappings of species and reactions, and explicit species transport.  


\subsection{Proposal corresponding to this package specification}

This specification for Spatial in SBML Level~3 Version~1 is based on the proposal located at the following URL:

\begin{center}
  \vspace*{1ex}\small
  \url{https://sbml.svn.sf.net/svnroot/sbml/trunk/specifications/sbml-level-3/version-1/spatial/proposal}
  \vspace*{1ex}
\end{center}

The tracking number in the SBML issue tracking system~\citep{tracker} for Spatial package activities is 188 (\url{http://sourceforge.net/p/sbml/sbml-specifications/188/}).


\subsection{Package dependencies}

The Spatial package has no dependencies on other SBML Level~3 packages.


\subsection{Document conventions}
\label{conventions}

UML~1.0 (Unified Modeling Language; \citealt{eriksson:1998, oestereich:1999}) notation is used in this document to define the constructs provided by this package.  Colors in the diagrams carry the following additional information for the benefit of those viewing the document on media that can display color:

\begin{itemize}

\item[\raisebox{2.75pt}{\colorbox{black}{\rule{0.8pt}{0.8pt}}}]
  \emph{Black}: Items colored black in the UML diagrams are components
  taken unchanged from their definition in the SBML Level~3 Core
  specification document.

\item[\raisebox{2.75pt}{\colorbox{mediumgreen}{\rule{0.8pt}{0.8pt}}}]
  \emph{\textcolor{mediumgreen}{Green}}: Items colored green are
  components that exist in SBML Level~3 Core, but are extended by this
  package.  Class boxes are also drawn with dashed lines to further
  distinguish them.

\item[\raisebox{2.75pt}{\colorbox{darkblue}{\rule{0.8pt}{0.8pt}}}]
  \emph{\textcolor{darkblue}{Blue}}: Items colored blue are new
  components introduced in this package specification.  They have no
  equivalent in the SBML Level~3 Core specification.

\end{itemize}

The following typographical conventions distinguish the names of objects and data types from other entities; these conventions are identical to the conventions used in the SBML Level~3 Core specification document:

\begin{description}
  
\item \abstractclass{AbstractClass}: Abstract classes are never instantiated directly, but rather serve as parents of other classes.  Their names begin with a capital letter and they are printed in a slanted, bold, sans-serif typeface.  In electronic document formats, the class names defined within this document are also hyperlinked to their definitions; clicking on these items will, given appropriate software, switch the view to the section in this document containing the definition of that class.  (However, for classes that are unchanged from their definitions in SBML Level~3 Core, the class names are not hyperlinked because they are not defined within this document.)
  
\item \class{Class}: Names of ordinary (concrete) classes begin with a capital letter and are printed in an upright, bold, sans-serif typeface.  In electronic document formats, the class names are also hyperlinked to their definitions in this specification document.  (However, as in the previous case, class names are not hyperlinked if they are for classes that are unchanged from their definitions in the SBML Level~3 Core specification.)

\item \emph{\token{SomeThing}}, \token{otherThing}: Attributes of classes, data type names, literal XML, and tokens \emph{other} than SBML class names, are printed in an upright typewriter typeface.

\item \token{[elementName]}:  In some cases, an element may contain a child of any class inheriting from an abstract base class.  In this case, the name of the element is indicated by giving the abstract base class name in brackets, meaning that the actual name of the element is the de-capitalized form of whichever subclass is used.

\end{description}

For other matters involving the use of UML and XML, this document follows the conventions used in the SBML Level~3 Core specification document.
