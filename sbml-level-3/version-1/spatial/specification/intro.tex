% -*- TeX-master: "main" -*-

\section{Introduction}
\label{intro}

There is no better test of our understanding of a biological system than to create a mechanistic model of that system and show that it possesses all known properties of the system -- i.e., that it is consistent with all measurements or observations of the system in hand. However, the creation of such models is even more useful when a system is not already understood, because the model may suggest mechanisms by which known behaviors are accomplished, or predict behaviors not previously expected. Extensive work has been done on the creation of methods for simulating biochemical systems \citep{hoops:2006, tomita:1999, loew:2001, stiles:1998, blinov:2004, andrews:2004}, including the development of languages, such as the Systems Biology Markup Language (SBML), for saving and exchanging the definition of the system being simulated and the simulation parameters used \citep{keating:2020, hucka:2003}. Historically, most tools for biochemical simulations either did not consider possible spatial relationships or compartmentalization of system components, or used a compartmental modeling approach \citep{jacquez:1985} in which spatial organization is approximated by a set of compartments (e.g. membrane-bound organelles in a cell) containing well-mixed populations of molecules. However, a growing number of modeling and simulation tools do permit specification of the size, shape and positions of compartments and the initial spatial distributions of components (typically referred to as a 'geometry' definition). While SBML Level 3 Core has explicit support for multi-compartmental modeling, it does not have a standardized mechanism to store or exchange these geometries. We address this deficit here through the definition of a package of extensions to SBML termed SBML spatial.

Saving geometries in a standardized way is useful to ensure that the definition is complete (which is important when, for example, provided as part of supplementary information for published studies), and to permit comparison of simulations of the same spatial system using different simulation tools, to enable simulations of a new biochemical model using the same geometry as a previous model (or vice versa). Historically, the creation of a geometry has been done (often through painstaking manual work) by the same investigators who defined a biochemical model and performed a simulation. The advent of SBML has permitted a partial division of this labor, by enabling databases of biochemical models, such as the BioModels database \citep{lenovere:2006, li:2010}, to be created. Building on this theme, having a mechanism to exchange geometries will enable \changed{an} open access 'marketplace' in which some investigators focus on the creation of well-characterized geometries (and distribute them through databases or repositories) enabling others to focus on biochemical model creation and/or performance of simulations. These geometries, however, typically need to be more than raw images. To be most broadly useful across different modeling approaches, they need, for example, to define explicit watertight boundaries for cells and intracellular compartments and contain statistically accurate estimates of concentrations of components at various locations. This can be accomplished for individual images by hand segmentation \citep{loew:2001} or automated segmentation \citep{perez:2014}. An alternative is to produce synthetic geometries that are drawn from generative models of cell shape and organization learned from many images \citep{zhao:2007}. In both cases, the availability of well-characterized geometries for examples of different cell types (and multiple cells of each type) can enhance the use of simulation methods.

Examples of simulations using biochemical models and geometries from different sources have already occurred using a preliminary version of the SBML spatial standard \citep{sullivan:2015, donovan:2016}. In these, cell geometries created using CellOrganizer and saved using the SBML spatial extension were combined with biochemical models created using BioNetGen and saved in SBML.

Additional considerations when considering spatially-realistic simulations is that properties of molecules or reactions may be spatially varying, and that rates of transport between locations need to be specified.

In summary, the purpose of this SBML Level 3 extension is to meet the needs identified above by defining standardized ways of representing geometries, spatial mappings of species and reactions, and explicit species transport.



%A set of biochemical process in cellular physiology may be modeled using different choices of spatial and temporal scales depending on the questions to be addressed.  SBML Level 3 Core has explicit support for multi-compartmental modeling where cellular organization is approximated by a set of compartments (e.g. membrane-bound organelles) containing well-stirred populations of molecules.  However, the coupling between localized biochemical reactions and diffusive molecular transport within the constraints of cellular geometry often results in important nonuniform molecular distributions.  While it is often possible to approximate the influence of spatial organization and localization within a compartmental model using altered parameters and additional species, it is sometimes simpler and always more mechanistic to directly model these spatial processes.  

%An increasing number of modeling and simulation tools include direct support for modeling with explicitly defined cellular geometry.  These models generally include heterogeneous molecular distributions, diffusive transport, and spatially localized reactions.  These spatial models generally belong to two different mathematical frameworks, stochastic (where each molecule is tracked in space and time) and deterministic (where time varying species concentration fields are described by partial differential equations).  

%There are a sufficient number of spatial modeling tools and spatial models to justify the effort of creating a spatial modeling extension of SBML.  All such models must describe the cellular geometry, map molecular species to spatial locations, map reactions to spatial locations, and specify molecular transport within geometric compartments and at boundaries of these compartments.  

%It is the purpose of this SBML Level 3 extension to define a common representation for cellular geometry, spatial mappings of species and reactions, and explicit species transport.  

\pagebreak

\subsection{Proposal corresponding to this package specification}

This specification for Spatial in SBML Level~3 Version~1 is based on the proposal located at the following URL:

\begin{center}
  \vspace*{1ex}\small
  \url{https://sbml.svn.sf.net/svnroot/sbml/trunk/specifications/sbml-level-3/version-1/spatial/proposal}
  \vspace*{1ex}
\end{center}

The Package Working Group for the Spatial package is coordinated on the mailing list \url{https://lists.sourceforge.net/lists/listinfo/sbml-spatial}.  More information about the spatial package is available at \url{http://sbml.org/Documents/Specifications/SBML_Level_3/Packages/spatial}.

\subsection{Package dependencies}

The Spatial package has no dependencies on other SBML Level~3 packages.


\subsection{Document conventions}
\label{conventions}

UML~1.0 (Unified Modeling Language; \citealt{eriksson:1998, oestereich:1999}) notation is used in this document to define the constructs provided by this package.  Colors in the diagrams carry the following additional information for the benefit of those viewing the document on media that can display color:

\begin{itemize}

\item[\raisebox{2.75pt}{\colorbox{black}{\rule{0.8pt}{0.8pt}}}]
  \emph{Black}: Items colored black in the UML diagrams are components
  taken unchanged from their definition in the SBML Level~3 Core
  specification document.

\item[\raisebox{2.75pt}{\colorbox{mediumgreen}{\rule{0.8pt}{0.8pt}}}]
  \emph{\textcolor{mediumgreen}{Green}}: Items colored green are
  components that exist in SBML Level~3 Core, but are extended by this
  package.  Class boxes are also drawn with dashed lines to further
  distinguish them.

\item[\raisebox{2.75pt}{\colorbox{darkblue}{\rule{0.8pt}{0.8pt}}}]
  \emph{\textcolor{darkblue}{Blue}}: Items colored blue are new
  components introduced in this package specification.  They have no
  equivalent in the SBML Level~3 Core specification.

\item[\raisebox{2.75pt}{\colorbox{red}{\rule{0.8pt}{0.8pt}}}]
  \emph{\textcolor{red}{Red lines}}: Classes with red lines in the corner are fully defined in a different figure.

\end{itemize}

The following typographical conventions distinguish the names of objects and data types from other entities; these conventions are identical to the conventions used in the SBML Level~3 Core specification document:

\begin{description}
  
\item \abstractclass{AbstractClass}: Abstract classes are never instantiated directly, but rather serve as parents of other classes.  Their names begin with a capital letter and they are printed in a slanted, bold, sans-serif typeface.  In electronic document formats, the class names defined within this document are also hyperlinked to their definitions; clicking on these items will, given appropriate software, switch the view to the section in this document containing the definition of that class.  (However, for classes that are unchanged from their definitions in SBML Level~3 Core, the class names are not hyperlinked because they are not defined within this document.)
  
\item \class{Class}: Names of ordinary (concrete) classes begin with a capital letter and are printed in an upright, bold, sans-serif typeface.  In electronic document formats, the class names are also hyperlinked to their definitions in this specification document.  (However, as in the previous case, class names are not hyperlinked if they are for classes that are unchanged from their definitions in the SBML Level~3 Core specification.)

\item \emph{\token{SomeThing}}, \token{otherThing}: Attributes of classes, data type names, literal XML, and tokens \emph{other} than SBML class names, are printed in an upright typewriter typeface.

\item \token{[elementName]}:  In some cases, an element may contain a child of any class inheriting from an abstract base class.  In this case, the name of the element is indicated by giving the abstract base class name in brackets, meaning that the actual name of the element is the de-capitalized form of whichever subclass is used.

\end{description}

For other matters involving the use of UML and XML, this document follows the conventions used in the SBML Level~3 Core specification document.
