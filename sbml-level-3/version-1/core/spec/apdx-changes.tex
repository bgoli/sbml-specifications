% -*- TeX-master: "sbml-level-3-version-1-core"; fill-column: 66 -*-
% $Id$
% $HeadURL$
% ----------------------------------------------------------------

\section{Major changes between versions of SBML Level~2 and implications for backward compatibility}
\label{apdx:changes}

In this section, we list the cumulative changes introduced in SBML
Level~2 since Version~1.

\subsection{Between Version 2 and Version 1}

The following features were removed between \sbmltwoone and
Version~2:
\begin{itemize}
  
\item The \token{offset} attribute on \UnitDefinition.  (See
  Section~\ref{sec:unit-structure}.)  The definition of offsets in
  \sbmltwoone was in fact incorrect; moreover, a proper
  implementation would have required a complete change in the SBML
  unit scheme.  Few models appeared to use offsets on unit
  definitions, so the impact of this change on models is expected
  to be small.
  
\item The \val{Celsius} predefined unit.  (See
  Section~\ref{sec:unit-structure}.)  The removal of offsets on
  unit definitions meant an inconsistency existed if the Celsius
  predefined unit was left in the system.  Removing Celsius
  removes the inconsistency.  Alternative ways of using Celsius
  are discussed in Section~\ref{sec:unit-structure}.
  
\item The \token{substanceUnit} and \token{timeUnits} attributes on
  \KineticLaw.  (See Section~\ref{subsec:kinetic-law}.)  The
  ability to redefine the substance units on each reaction
  separately, coupled with other features in \sbmltwoone, created
  the opportunity for defining a valid system of reactions which
  potentially could not be combined into a consistent system of
  equations without external knowledge.

\end{itemize}

The following features were deprecated between \sbmltwoone and
Version~2:
\begin{itemize}
  
\item The \token{charge} attribute on \Species.  (See
  Section~\ref{sec:charge}.)  This attribute does not appear to be
  supported by any existing software, and moreover, since its
  value cannot be accessed from mathematical formulas in SBML, the
  impact of this change is expected to be small.

\end{itemize}

The following additional changes were made between \sbmltwoone and
Version~3:
\begin{itemize}

\item \sbmltwoone did not clearly specify the value space of
  integer and floating-point numbers permitted in the MathML
  expressions in SBML; moreover, it used the XML Schema type
  \val{integer} instead of \sbmltwotwo's \val{int}.  Although
  extremely unlikely, some previously valid \sbmltwoone documents
  \emph{may} not be valid in Version~2 and Version~3 as a result
  of these changes.  See Sections~\ref{sec:integer},
  \ref{sec:double} and~\ref{sec:cn-token} for more information.

\item \sbmltwoone did not define a default value for the attribute
  \token{fast} on \Reaction.  \sbmltwotwo introduced a default
  value, and the value is \val{false}.  Further, SBML now requires
  that software tools \emph{must} respect the value or indicate to
  the user that they do not have the capacity to do so.  See
  Section~\ref{sec:fast}.
  
\item As of \sbmltwotwo, SBML is somewhat stricter about how the
  content of \token{annotation} elements must be organized and
  written.  Previously valid \sbmltwoone documents \emph{may}
  need changes to their \token{annotation} elements to comply with
  the specification beginning with Version~2 and Version~3.  See
  Section~\ref{sec:annotation-use} for more details.
  
\item As of \sbmltwotwo, SBML is slightly stricter about how the
  content of \token{notes} elements must be organized.  Previously
  valid \sbmltwoone documents \emph{may} need changes to their
  \token{notes} elements to comply with the specification
  beginning with Version~2 and Version~3.  See
  Section~\ref{sec:notes} for more details.
  
\item \sbmltwotwo corrected numerous errata and ambiguities
  discovered in \sbmltwoone.  These errata are listed on the
  project web site at \url{http://sbml.org}.  As a result of
  changes to \sbmltwo implied by these errata, some existing
  \sbmltwoone models, even when modified as explained above, may
  still not be compliant with Version~2 or Version~3.  The
  ultimate impact of the changes depends on the specific features
  used by a given model and the assumptions under which the model
  was created.

\end{itemize}




\subsection{Between Version 3 and Version 2}

The following features were removed between \sbmltwotwo and
Version~3:
\begin{itemize}

\item The \token{spatialSizeUnits} attribute on \Species.  (See
  Section~\ref{sec:species-units}.)  This attribute introduced an
  implicit unit conversion between the spatial units used in
  defining the quantity of a species and the size of the
  compartment in which the species was located.  Moreover, the
  semantics of \token{spatialSizeUnits} were confusing and
  required complicated unit conversions to be written explicitly
  into reaction rate expressions by either the modeler or the
  software.  Although the conversions could be worked out
  unambiguously, the potential for error was judged to exceed by
  far the utility of this feature.

\item The \token{timeUnits} attribute \Event.  (See
  Section~\ref{sec:events}.)  This attribute was judged to add
  needless complexity and inconsistency.  For instance, the
  ability to change the time units of the delay of an \Event to be
  different from the units of time for the whole model meant that
  computing an \Event's time of triggering and its delay might
  have to be done using two different sets of units.  The ability
  to redefine the units of time for the delay of an \Event was
  also inconsistent with the lack of such an attribute on other SBML
  components involving an element of time; for example, \RateRule,
  and now \KineticLaw, have no such attributes.

\end{itemize}

The following additional changes were made between \sbmltwotwo and
Version~3:
\begin{itemize}

\item The definition of the XML type \primtype{ID} was incorrectly
  given in the \sbmltwotwo specification.  This type is used as
  the type of the attribute \token{metaid} on \SBase.  The error
  in the definition of \primtype{ID} was such that the type did
  not include the colon (\token{:}) character and all Unicode
  characters actually permitted in XML \primtype{ID}.  This change
  is therefore entirely backward compatible: all models with valid
  \token{metaid} values valid prior to \sbmltwothree are still
  valid under the new definition.

\item The SBML specifications prior to \sbmltwothree did not
  indicate what units are assumed for literal numbers appearing in
  MathML expressions (\ie numbers inside MathML's \token{cn}
  elements).  \sbmltwothree stipulates that there are no units
  associated with numbers (not even \val{dimensionless}), and
  provides suggestions for how to associate units with numbers
  (Section~\ref{sec:units-of-mathml}).

\item The SBML specifications prior to \sbmltwothree did not make
  clear what units are required by the arguments to various MathML
  operators and other constructs.  \sbmltwothree clarifies this
  (Section~\ref{sec:operator-arg-types}).

\item The \primtype{UnitKind} enumeration previously defined in
  the context of \Unit and \UnitDefinition has been eliminated in
  favor of simply defining the symbols as reserved words in the
  value space of \primtype{UnitSId}.  This has no effect on
  written models and is completely backward compatible.  It was
  done to resolve the problem that, previously, the values in
  \primtype{UnitKind} were technically inaccessible from attributes
  whose data type was \primtype{UnitSId}.

\item The SBML specification did not point out that the value
  space of the data type \primtype{boolean} is different in
  \mathmltwo and \xmlschemaone.  This means that the permitted
  values of attributes on SBML objects is different from the
  values permitted in MathML formulas.  (Section~\ref{sec:boolean}
  explains the difference.)

\item The SBML specifications were inconsistent about the
  permitted number of items inside
  \token{listOf\rule{0.5in}{0.5pt}} lists: the text portion of the
  specifications claimed the lists could have zero length, but the
  XML Schema definition for SBML required one or more items.  As
  of \sbmltwothree, the specification is consistent on requiring
  one or more items inside these lists.

\item The SBO term hierarchy (Section~\ref{sec:sbo}) has grown in
  the time intervening between \sbmltwotwo and Version~3, and the
  mapping of terms between SBO and SBML components was revised as
  a result of community discussions during the 2006 SBML Forum
  meeting.

\item The \token{sboTerm} attribute, introduced on many components in
  \sbmltwotwo, has been moved to \SBase as an optional attribute
  and removed from the individual components.  The result is that
  all model components may now have SBO terms associated with
  them.  This change is completely backward compatible.

\item A number of validation rules in
  Appendix~\ref{apdx:validation-rules} have been introduced; some
  were missing from previous specifications, and some were added
  to cover changes introduced in \sbmltwothree (for example, for
  validation of SBO terms assigned to various SBML model
  components).

\item The SBML specifications prior to \sbmltwothree did not
  adequately explain the assumptions regarding XML namespace
  declarations within the \token{annotation} element on SBML
  components.  \sbmltwothree makes these assumptions more
  explicit, including the assumption that applications may not
  preserve another applications annotations unless those
  annotations are self-contained with the XML Namespace
  declaration.  See Section~\ref{sec:annotation-use} for more
  details.

\end{itemize}




% Contortion to make the section number be colored for the rest of
% this subsection.
\renewcommand{\thesubsection}{\Alph{section}.\arabic{subsection}}

\subsection{Between Version 4 and Version 3}

The following significant changes exist between \sbmltwothree and
Version~4:
\begin{itemize}

\item The result of an SBML community vote taken late in 2007
  indicated that users and developers preferred SBML \emph{not} to
  \emph{require} consistency of units of measurement on quantities
  and mathematical expressions in a model.  Consequently, the
  language and validation rules involving units in \sbmltwofour
  have been changed to use the wording ``should'' instead of
  ``must''---units \emph{should} be consistent, but models that do
  not exhibit strict unit consistency are not invalid.

\item Discussions held on the
  \link{http://sbml.org/Forums}{sbml-discuss@caltech.edu} mailing
  list in 2007--2008 as well as the SBML Forum Meeting in 2008
  (G\"{o}teborg, Sweden) made clear two points about \Event: (1)
  many developers misunderstood the specification with regards to
  the time at which the mathematical formula in an
  \EventAssignment was to be evaluated, and implemented the
  computation such that it was performed at the time the event was
  \emph{executed} instead of (as the specification actually
  stipulated) the time at which the event was \emph{fired}; and
  (2) once informed, most developers found the actual definition
  in the SBML specification counterintuitive.  We believed it
  would have been too confusing and error-prone to change the
  sense of the assignments between versions of an SBML Level, so
  in order to address popular demand, Version~4 includes a new
  attribute (\token{useValuesFromTriggerTime}) on \Event allowing
  a model to indicate which sense is intended.  The default value
  of the \token{useValuesFromTriggerTime} attribute results in the
  same interpretation of an \Event's \EventAssignment{}s that
  \sbmltwothree specifies.  There is also a new validation rule
  (21206) related to the new attribute, and the SBML schema
  presented in Appendix~A includes the new
  \token{useValuesFromTriggerTime} attribute in the definition of
  \Event.

\item \sbmltwothree had small syntactic errors in the RDF
  described in Section~\ref{sec:annotation-standard}.  Version~4
  includes corrected RDF.

\item Version~3 never made explicit the requirement that the
  number of arguments in a call to a user-defined function must
  match the number of arguments defined in the instance of the
  \FunctionDefinition.  \sbmltwofour contains a new subsection
  (\ref{sec:functiondefinition-calling}) and a new validation rule
  (10219) to capture this requirement.

\item The description of \Compartment's \token{outside} attribute
  mistakenly stated that the compartment referenced by the
  attribute value must be one that is already defined in the
  model, meaning that compartments had to be defined before being
  referenced.  There was no validation rule to that effect, and
  moreover, this requirement was at odds with the general trend in
  later versions of \sbmltwo to remove requirements for
  element ordering.  In Version~4,
  Section~\ref{sec:compartment-outside} no longer implies a
  requirement for ordering.

\item Prior to \sbmltwofour, it was never made clear what, if any,
  relationship existed between the sizes of compartments when
  compartments defined inside/outside relationships using the
  \token{outside} attribute.  A new paragraph in
  Section~\ref{sec:size} of the Version~4 specification makes
  clear there is no implication on compartment sizes---the size of
  the outside compartment does not include the size(s) of the
  inside compartment(s).  Moreover, additional clarifications have
  been added to explain that \token{outside} does not necessarily
  imply a containment relationship between compartments.

\item Since the time that \sbmltwothree Release~2 was issued,
  three new MIRIAM qualifiers have been defined by the MIRIAM
  project: \token{bqbiol:isEncodedBy}, \token{bqbiol:encodes}, and
  \token{bqbiol:occursIn}.  Section~\ref{sec:annotation-standard}
  now includes these qualifiers.

% FTB: The following sections really should go, for now i comment them 
%      out, as the CompartmentType and SpeciesType macros are no longer there.
% \item Since the time that \sbmltwothree Release~2 was issued, the
%  Systems Biology Ontology (SBO) underwent reorganization and
%  improvement.  This required numerous changes to
%  Section~\ref{sec:sbo} to be consistent with SBO now.  Chief
%  among the changes that impact SBML models are that (1) the SBO
%  branches for \Model, \CompartmentType, \SpeciesType,
%  \Compartment and \Species have all changed slightly, and (2) the
%  \sbmltwofour specification does not \emph{require} specific
%  choices for \token{sboTerm} attribute values, only
%  \emph{recommends} them.

%\item There were some cut-and-paste errors in the text of the
%  descriptions of the \token{sboTerm} attribute on several SBML
%  components such as \Species, \SpeciesType, and others.  The text
%  of the Version~4 specification includes corrections for this.

\item Prior Versions of \sbmltwo used three validation rules,
  20406, 20407, and~20408, to encode requirements about the
  predefined unit \token{volume}.  This was inconsistent and not
  parallel with the way validation rules for the other predefined
  units such as \token{substance} were defined.  In Version~4, the
  rules 20407 and 20408 are no longer defined and their content
  has been moved into rule 20406.

% \item Reaction identifiers may be used in formulas in a model to
%   refer to the rate of the corresponding reaction.  However, this
%   only makes sense if the reaction in question has a \KineticLaw
%   element; otherwise, the value of the identifier is
%   indeterminate, leading to an incomplete model.  A new validation
%   rule (10220) encodes this consistency check.

\item In Section~\ref{sec:sbml}, SBML Level 3 specifications
  stated that ``well-formed XML documents must begin with an XML
  declaration''.  In fact, this is false; the requirement is
  stipulated in XML~1.1, not in XML~1.0---the version of XML that
  SBML actually uses.  However, as a concession to helping greater
  software portability, SBML Level~4 nonetheless requires the XML
  declaration to be present.

\item Since elsewhere in SBML, the requirements on ordering of
  elements have been eliminated, the SBML Editors decided there
  was no point in maintaining the ordering requirement on function
  definitions.  Therefore, in SBML Level~2 Version~4, forward
  references to other user-defined functions are permitted.
  (However, recursive or mutually recursive functions are still
  prohibited.)  This also causes the removal of validation rule
  number 20302.

\item Validation rules 10211, 10212, and 10708 used vague language
  about requirements; these have been corrected to be specific.

\end{itemize}





% Restore the section command definition done at the beginning of
% this file.
\renewcommand{\thesubsection}{\Alph{section}.\arabic{subsection}}
