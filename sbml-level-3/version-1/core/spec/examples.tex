% -*- TeX-master: "sbml-level-3-version-1-core"; fill-column: 66 -*-
% $Id$
% $HeadURL$
% ----------------------------------------------------------------

\section{Example models expressed in XML using SBML}
\label{sec:xml-rep}
\label{sec:examples}

In this section, we present several examples of complete models
encoded in XML using SBML Level~2.

%Our approach to translating
%the UML-based structure definitions presented in the previous
%sections is described elsewhere~\citep{hucka:2000b}.
%Appendix~\ref{apdx:schemas} gives the full listing of an XML
%Schema corresponding to SBML Level~2.


%-----------------------------------------------------------------------------
\subsection{A simple example application of SBML}
\label{sec:modeleg}
%-----------------------------------------------------------------------------

Consider the following representation of an enzymatic reaction:
\begin{linenomath}\vspace*{-0.5ex}
\begin{reaction}\centering\notag
  E + S \eqbm^{\cit k\sub{on}}_{\cit k\sub{off}} ES \yields^{\cit k\sub{cat}} E + P
\end{reaction}
\end{linenomath}\vspace*{-1ex}%
The following is the minimal SBML document encoding the model shown above:

\sbmlexample{enzymekinetics.xml}

In this example, the model has the identifier
\val{EnzymaticReaction}.  The model contains one compartment (with
identifier \val{cytosol}), four species (with identifiers
\val{ES}, \val{P}, \val{S}, and \val{E}), and two reactions
(\val{veq} and \val{vcat}).  The elements in the
\token{listOfReactants} and \token{listOfProducts} in each
reaction refer to the names of elements listed in the
\token{listOfSpecies}.  The correspondences between the various
elements is explicitly stated by the \token{speciesReference}
elements.

The model also features local parameter definitions in each
reaction.  In this case, the three parameters (\val{kon},
\val{koff}, \val{kcat}) all have unique identifiers and they could
also have just as easily been declared global parameters in the
model.  Local parameters frequently become more useful in larger
models, where it may become tedious to assign unique identifiers
for all the different parameters.


%-----------------------------------------------------------------------------
\subsection{Example involving units}
\label{apdx:units-eg}
%-----------------------------------------------------------------------------

The following model uses the units features of SBML Level~2. In
this model, the default value of \token{substance} is changed to
be mole units with a scale factor of $-3$, or millimoles.  This
sets the default substance units in the model.  The \token{volume}
and \token{time} built-in units are left to their defaults, meaning
volume is in litres and time is in seconds.  The result is that, in
this model, kinetic law formulas define rates in millimoles per
second and the species identifiers in them represent concentration
values in millimoles per litres.  All the \token{species} elements
set the initial amount of every given species to 1 millimole.  The
parameters \val{vm} and \val{km} are defined to be in millimoles
per litres per second, and millimoles per litres, respectively.

\sbmlexample{units.xml}


%-----------------------------------------------------------------------------
\subsection{Example of a discrete version of a simple dimerization reaction}
\label{sec:discrete-eg}
%-----------------------------------------------------------------------------

\emph{(Model contributed by Darren J. Wilkinson, Newcastle
  University, Newcastle upon Tyne, UK.)}

This example illustrates subtle differences between models
formulated for use in a continuous simulation framework (\eg using
differential equations) and those intended for a discrete
simulation framework.  The model shown here is suitable for use
with a discrete stochastic simulation algorithm of the sort
developed by \cite{gillespie:1977}.  In such an approach, species
are described in terms of molecular counts and simulation
proceeds by computing the probability of the time and identity of
the next reaction, then updating the species amounts
appropriately.

The model involves a simple dimerization reaction for a protein
named \val{P}:
\begin{linenomath}
\begin{equation*}
    2 P  \leftrightarrow  P_2
\end{equation*}
\end{linenomath}
The SBML representation is shown below.  There are several
important points to note.  First, the species \val{P} and \val{P2}
declare they are always in discrete amounts by using the flag
\token{hasOnlySubstanceUnits}=\val{true}.  This indicates that
when the species identifiers appear in mathematical formulas, the
units are \quantity{substance}, not the default of
\quantity{substance}/\quantity{size}.  A second point is that, as
a result, the corresponding ``kinetic law'' formulas do not need
volume corrections.  In Gillespie's approach, the constants in the
rate expressions (here, \val{c1} and \val{c2}) contain a
contribution from the kinetic constants of the reaction and the
size of the compartment in which the reactions take
place. This is a convention commonly adopted by stochastic
modellers, but is in no way essential --- it is perfectly
reasonable to factor volume out of the rate constants, and in
certain situations it may be desirable to do so (eg. for models having
time-varying compartment volume), but due to the use of substance
units, it must be done differently to the deterministic case.
Thirdly, although the reaction is reversible, it is encoded as two
separate irreversible reactions, one each for the forward and
reverse directions, as averaging over the reactions will affect
the stochasticity.
Finally, it is worth noting the rate expression for the forward
reaction is a second-order mass-action reaction, but it is the
\emph{discrete} formulation of such a reaction
rate~\citep{gillespie:1977}.

\sbmlexample{dimerization.xml}

This example also illustrates the need to provide additional
information in a model so that software tools using different
mathematical frameworks can properly interpret it.  In this case,
a simulation tool designed for continuous ODE-based simulation
would likely misinterpret the model (in particular the reaction
rate formulas), unless it deduced that a discrete stochastic
simulation was intended.  One of the purposes of SBO annotations
(Section~\ref{sec:sboTerm}) is to enable such interpretation
without the need for deduction. However, the interpretation of the
model is essentially the same irrespective of whether the model is
to be simulated in a deterministic or stochastic manner, and a
properly SBML-compliant deterministic simulator will in most cases
correctly simulate from the continuous deterministic approximation
to the stochastic model even if it has no stochastic simulation
capability.

The interpretation of rate laws for stochastic models is similar
to, yet different from that of deterministic models. Taking the
first reaction as an example, the rate law is $c_1P(P-1)/2$ reaction
events per second. In the continuous deterministic case, the
interpretation of this is that the extent of the reaction in time
$dt$ is $[c_1P(P-1)/2]dt$ (and this leads naturally to the usual ODE
formulation of the model). In the stochastic case, the
interpretation is that the \emph{propensity} (or \emph{rate}, or
\emph{hazard}) of the reaction is $c_1P(P-1)/2$. That is, the
\emph{probability} of a single reaction event occurring in time
$dt$ is $[c_1P(P-1)/2]dt$ (and note that the \emph{expected} extent of
the reaction will be $[c_1P(P-1)/2]dt$). This interpretation leads to a Markov
jump process for the system dynamics, where the inter-event times
are exponentially distributed. Such dynamics can be simulated
using a discrete event simulation algorithm such as the
\emph{Gillespie algorithm}. In this case, the algorithm for
simulating the model can be described as follows:
\begin{enumerate}
\item Initialise $t:=0,\ c_1:=0.00166,\ c_2:=0.2,\ P:=301,\ P_2:=0$
\item Compute $h_1:=c_1P(P-1)/2,\ h_2:=c_2P_2$
\item Compute $h_0=h_1+h_2$
\item Simulate $t'\sim \operatorname{Exp}(h_0)$ and set $t:=t+t'$
\item With probability $h_1/h_0$ set $P:=P-2,\ P_2:=P_2+1$,
otherwise set $P:=P+2,\ P_2:=P_2-1$.
\item Output $t,\ P,\ P_2$
\item If $t<T_{max}$, return to step 2, otherwise stop.
\end{enumerate}
Although this is a simulation algorithm is a very practical way of
describing how to construct exact realisations of the Markov jump
process corresponding to the discrete stochastic kinetic model, it
is not a concise mathematical description. Such a description can
be provided by writing the model as a time change of a pair of
independent unit Poisson processes. Let $N_1(t)$ and $N_2(t)$ be
the counting functions of these processes, so that for each
$i=1,2$, $t>0$, $N_i(t)\sim \operatorname{Poisson}(t)$. Then,
writing $P(t)$ and $P_2(t)$ for the numbers of molecules of $P$
and $P_2$ at time $t$, respectively, we have that the stochastic process
$\{P(t),P_2(t)|t>0\}$ satisfies the stochastic integral equation
\begin{align*}
P_2(t) &= N_1\left(\int_0^t
c_1\frac{P(\tau)[P(\tau)-1]}{2}d\tau\right) - N_2\left(\int_0^t
c_2 P_2(\tau)d\tau\right) \\
P(t) &= 301 - 2P_2(t).
\end{align*}
The above representation is arguably the most useful for
mathematical analysis of the stochastic model; see \cite{ball:2006} for
details. Another popular representation is the so-called chemical
Master equation (CME) for the probability distribution of the possible
states at all times \citep{gillespie:1992}. In this case, since there are
151 possible states of the system (corresponding to the 151
possible values of $P_2$), the CME consists of 151 coupled
ODEs, \emph{viz}
\[
\frac{d}{dt}p(P,P_2,t) =
\left\{
\begin{array}{ll}
\displaystyle-\frac{c_1}{2}\times 301\times 299p(301,0,t)+c_2p(299,1,t),
&P=301,\ P_2=0,\\
\displaystyle\frac{c_1}{2}(P+2)(P+1)p(P+2,P_2-1,t)-\frac{c_1}{2}P(P-1)p(P,P_2,t)&P=301-x,\ P_2=x,\\
\qquad+c_2(P_2+1)p(P-2,P_2+1,t)-c_2P_2p(P,P_2,t),
&\quad x=1,2,\ldots,149,\\
\displaystyle\frac{c_1}{2}\times 2\times 3p(3,149,t)-c_2\times 150p(1,150,t),
&P=1,\ P_2=150,
\end{array}
\right.
\]
where $p(P,P_2,t)$ denotes the probability that there are $P$
molecules of $P$ and $P_2$ molecules of $P_2$ at time $t$, and the
ODEs are subject to the initial conditions
\[
p(301,0,0)=1,\ p(301-2x,x,0)=0,\ x=1,2,\ldots,150.
\]
See \cite{evans:2008} for further examples of discrete stochastic
kinetic models encoded in SBML and \cite{wilkinson_2006} for an
introduction to discrete stochastic modelling using SBML.



%-----------------------------------------------------------------------------
\subsection{Example involving assignment rules}
\label{apdx:rules-eg}
%-----------------------------------------------------------------------------

This section contains a model that simulates a system containing a
fast reaction.  This model uses rules to express the mathematics
of the fast reaction explicitly rather than using the \token{fast}
attribute on a reaction element.  The system modeled is
\begin{linenomath}
\begin{equation*}
  \begin{array}{@{}ccc@{}}
    X_0 & \overset{\underrightarrow{k_1 X_0}}{}           & S_1 \\[6pt]
    S_1 & \overset{\underrightarrow{k_f S_1 - k_r S_2}}{} & S_2 \\[6pt]
    S_2 & \overset{\underrightarrow{k_2 S_2}}{}           & X_1 \\[6pt]
  \end{array}
\end{equation*}
\begin{equation*}
    k_1 = 0.1, \quad k_2 = 0.15, \quad k_f = K_{eq} 10000, \quad k_r = 10000, \quad K_{eq} = 2.5.
\end{equation*}
\end{linenomath}
where $X_0$, $S_1$, $S_1$, and $S_2$ are species in concentration units,
and $k_1$, $k_2$, $k_f$, $k_r$, and $K_{eq}$ are parameters.  This
system of reactions can be approximated with the following new
system:
\begin{linenomath}
\begin{equation*}
  \begin{array}{@{}ccc@{}}
    X_0 & \overset{\underrightarrow{k_1 X_0}}{} & T \\[6pt]
    T & \overset{\underrightarrow{k_2 S_2}}{} & X_1\\[6pt]
  \end{array}
\end{equation*}
\begin{equation*}
\begin{aligned}
    S_1 &= \dfrac{T}{1 + K_{eq}} \\[6pt]
    S_2 &= K_{eq} S_1
  \end{aligned}
\end{equation*}
\end{linenomath}

where $T$ is a new species.  The following example SBML model
encodes the second system.

\sbmlexample{assignmentrules.xml}


%-----------------------------------------------------------------------------
\subsection{Example involving algebraic rules}
\label{sec:algeraiceg}
%-----------------------------------------------------------------------------

This section contains an example model that contains an
\AlgebraicRule object.  The model contains a different
formulation of the fast reaction described in
Section~\ref{apdx:rules-eg}.  The system described in
Section~\ref{apdx:rules-eg} can be approximated with the following
system:
\begin{linenomath}
\begin{equation*}
  \begin{array}{@{}ccc@{}}
    X_0 & \overset{\underrightarrow{k_1 X_0}}{} & T \\[6pt]
    T & \overset{\underrightarrow{k_2 S_1}}{} & X_1
  \end{array}
\end{equation*}
\begin{equation*}
    S_2 = K_{eq} S_1\\[10pt]
\end{equation*}
\end{linenomath}
with the constraint:
\begin{linenomath}
\begin{equation*}
    S_1 + S_2 - T = 0
\end{equation*}
\end{linenomath}

The following example SBML model encodes this approximate form.

\sbmlexample{algebraicrules.xml}


%-----------------------------------------------------------------------------
\subsection{Example with combinations of
  \token{boundaryCondition} and \token{constant} values on \class{Species}
  with \class{RateRule} objects}
\label{sec:constantspecieseg}
%-----------------------------------------------------------------------------

In this section, we discuss a model that includes four species,
each with a different combination of values for their
\token{boundaryCondition} and \token{constant} attributes.  The model
represents a hypothetical system containing one reaction,
\begin{linenomath}
\begin{equation*}
  \begin{array}{@{}ccc@{}}
    S_1 + S_2 & \overset{\underrightarrow{k_1 S_1 S_2 S_3}}{} & S_4 \\ \\[-4pt]
  \end{array}
\end{equation*}
\end{linenomath}
where $S_3$ is a species that catalyzes the conversion of species
$S_1$ and $S_2$ into $S_4$.  $S_1$ and $S_2$ are on the boundary
of the system (\ie $S_1$ and $S_2$ are reactants but their values
are not determined by a kinetic law).  The value of $S_1$ in the
system is determined over time by the rate rule:
\begin{linenomath}
\begin{equation*}
  \dfrac{d S_1}{d t} = k_2
\end{equation*}
\end{linenomath}
The values of constant parameters in the system are:
\begin{linenomath}
\begin{equation*}
    S_2 = 1, \quad S_3 = 2, \quad k_1 = 0.5, \quad k_2 = 0.1
  \end{equation*}
\end{linenomath}
and the initial values of species are:
\begin{linenomath}
\begin{equation*}
    S_1 = 0, \quad S_4 = 0
\end{equation*}
\end{linenomath}

The value of $S_1$ varies over time so in SBML $S_1$ has a
\token{constant} attribute with a default value of \val{false}.  The
values of $S_2$ and $S_3$ are fixed so in SBML they have a
\token{constant} attribute values of \val{true}.  $S_3$ only occurs as
a modifier so the value of its \token{boundaryCondition} attribute can
default to \val{false}.  $S_4$ is a product whose value is
determined by a kinetic law and therefore in the SBML
representation has \val{false} (the default) for both its
\token{boundaryCondition} and \token{constant} attributes.

The following is the SBML rendition of the model shown above:

\sbmlexample{boundarycondition.xml}


%-----------------------------------------------------------------------------
\subsection{Example of translation from a multi-compartmental model to ODEs}
\label{sec:odeeg}
%-----------------------------------------------------------------------------

This section contains a model with 2 compartments and 4 reactions.
The model is derived from Lotka-Volterra, with the addition of a
reversible transport step.  When observed in a time-course
simulation, three of this model's species display damped
oscillations.

\begin{figure}[htb]
  \vspace*{5pt}
  \centering
  \begin{picture}(260,60)
    \put(0,10){\framebox(255,50)[tl]{ cytosol}}
    \put(10,19){\framebox(105,29)[tl]{ nucleus}}
    \put(24,26){$
        X + Y_{1n} \yields^{\cit k\sub{1}} 2\,Y_{1n}
        \eqbm^{\cit K\sub{T}} 2\,Y_{1c} + 2\,Y_2
        \yields^{\cit k\sub{2}} 4\,Y_2 \yields^{\cit k\sub{3}} \emptyset
        $}
  \end{picture}
  \vspace*{-8pt}
  \caption{A example multi-compartmental model.}
  \label{fig:multicomp}
\end{figure}

Figure~\ref{fig:multicomp} illustrates the arrangement of
compartments and reactions in the model
\token{LotkaVolterra\_tranport}.  The text of the SBML
representation of the model is shown below, and it is followed by
its complete translation into ordinary differential equations.  In
this SBML model, the reaction equations are in substance per time
units.  The reactions have also been simplified to reduce common
stoichiometric factors.  The species variables are in
concentration units; their initial quantities are declared using
the attribute \token{initialAmount} on the \token{species}
definitions, but since the attribute \token{hasOnlySubstanceUnits}
is \emph{not} set to true, the identifiers of the species
represent their concentrations when those identifiers appear in
mathematical expressions elsewhere in the model.  Note that the
species whose identifier is \val{X} is a boundary condition, as
indicated by the attribute \token{boundaryCondition}=\val{true} in
its definition.  

\sbmlexample{multicomp.xml}

The ODE translation of this model is as follows.  First, we give
the values of the constant parameters:
\begin{linenomath}
\begin{equation*}
  k_1 = 2500, \quad k_2 = 2500, \quad K_3 = 25000, \quad K_T = 2500
\end{equation*}
\end{linenomath}
Now on to the initial conditions of the variables.  In the
following, the symbols representing species ($X$, $Y_{1n}$,
$Y_{1c}$, and $Y_2$) have values in terms of concentrations.
(Readers may wonder why, when their values in the SBML
model are given as initial \emph{amounts}.  The reason goes back
to the \Species defaults and the meaning of the
\token{hasOnlySubstanceUnits} attribute: if the attribute is not
set and the compartment in which the species is located has more
than 0 spatial dimensions, a species' symbol in a model is
interpreted as a concentration or density regardless of whether
its initial value is set using \token{initialAmount} or
\token{initialConcentration}.)  We use $V_n$ to represent the
size of compartment \val{nucleus} and $V_c$ the
size of compartment \val{cytoplasm}:
\begin{linenomath}
\begin{equation*}
  V_n = 1, \quad V_c = 5, \quad X = 1, \quad Y_{1n} = 1, \quad Y_{1c} = 0, \quad Y_2 = 1/5
\end{equation*}
\end{linenomath}
And finally, here are the differential equations:
\begin{linenomath}
\begin{align*}
  \dfrac{d X}{d t}    &= 0 \\[6pt]
  V_n \dfrac{d Y_{1n}}{d t} &= k_1 \cdot X \cdot Y_{1n} \cdot V_n - K_T \cdot (Y_{1n} - Y_{1c}) \cdot V_c
    && \text{reactions production and transport} \\[6pt]
  V_c \dfrac{d Y_{1c}}{d t} &= K_T \cdot (Y_{1n} - Y_{1c}) \cdot V_c - k_2 \cdot Y_{1c} \cdot Y_2 \cdot V_c
    && \text{reactions transport and transformation} \\[6pt]
  V_c \dfrac{d Y_2}{d t}   &= k_2 \cdot Y_{1c} \cdot Y_2 \cdot V_c - k_3 \cdot Y_2 \cdot V_c
    && \text{reactions transformation and degradation}
\end{align*}
\end{linenomath}

As formulated here, this example assumes constant volumes.  If the
sizes of the compartments \val{cytoplasm} or \val{nucleus} could
change during simulation, then it would be preferable to use a
different approach to constructing the differential equations.  In
this alternative approach, the ODEs would compute substance change
rather than concentration change, and the concentration values
would be computed using separate equations.  This approach is used
in Section~\ref{sec:about-kinetic-laws}.


%-----------------------------------------------------------------------------
\subsection{Example involving function definitions}
\label{sec:functioneg}
%-----------------------------------------------------------------------------

This section contains a model that uses the function definition
feature of SBML.  Consider the following hypothetical system:
\begin{linenomath}
\begin{equation*}
  \begin{array}{@{}ccc@{}}
    S_1 & \overset{\underrightarrow{f(S_1)}}{} & S_2 \\ \\[-4pt]
  \end{array}
\end{equation*}
\end{linenomath}
where
\begin{linenomath}
\begin{equation*}
    f(x) = 2 \times x
\end{equation*}
\end{linenomath}

The following is the XML document that encodes the model shown
above:

\sbmlexample{functiondef.xml}


%-----------------------------------------------------------------------------
\subsection{Example involving \emph{delay} functions}
\label{sec:delayeg}
%-----------------------------------------------------------------------------

The following is a simple model illustrating the use of $delay$ to
represent a gene that suppresses its own expression.  The model
can be expressed in a single rule:
\begin{linenomath}
\begin{equation*}
  \frac{d P}{d t} = \dfrac{ \dfrac{1}{1 + m (P_{delayed})^q} - P }{ \tau }
\end{equation*}
\end{linenomath}
where
\begin{linenomath}
\begin{equation*}
\begin{array}{rl}
P_{delayed} & \mbox{is } delay(P, \Delta_t) \mbox{ or P at } t - \Delta_t\\
P & \mbox{is protein concentration}\\
\tau & \mbox{is the response time}\\
m & \mbox{is a multiplier or equilibrium constant}\\
q & \mbox{is the Hill coefficient}\\
\end{array}
\end{equation*}
\end{linenomath}
and the species quantities are in concentration units.
The text of an SBML encoding of this model is given below:

\sbmlexample{delay.xml}


%-----------------------------------------------------------------------------
\subsection{Example involving events}
\label{sec:eventeg}
%-----------------------------------------------------------------------------

This section presents a simple model system that demonstrates the
use of events in SBML.  Consider a system with two genes,
$G_1$ and $G_2$.  $G_1$ is initially
on and $G_2$ is initially off.  When turned on, the two
genes lead to the production of two products, $P_1$ and $P_2$,
respectively, at a fixed rate.  When $P_1$ reaches a given
concentration, $G_2$ switches on.  This system can be
represented mathematically as follows:
\begin{linenomath}
\begin{eqnarray*}
  \dfrac{d P_1}{d t} & = & k_1 (G_1 - P_1)\\[3pt]
  \dfrac{d P_2}{d t} & = & k_2 (G_2 - P_2)\\
  G_2 & = &
    \begin{cases}
      0 & \text{when $P_1 \leq \tau$},\\
      1 & \text{when $P_1 > \tau$}.
    \end{cases}
\end{eqnarray*}
\end{linenomath}

The initial values are:
\begin{linenomath}
\begin{equation*}
  G_1 = 1, \quad G_2 = 0, \quad \tau = 0.25, \quad P_1 = 0, \quad P_2 = 0, \quad k_1 = k_2 = 1.
\end{equation*}
\end{linenomath}

The SBML Level 2 representation of this as follows:

\sbmlexample{events.xml}


%-----------------------------------------------------------------------------
\subsection{Example involving two-dimensional compartments}
\label{sec:two-dimensional-eg}
%-----------------------------------------------------------------------------

The following example is a model that uses a two-dimensional
compartment.  It is a fragment of a larger model of calcium
regulation across the plasma membrane of a cell.  The model
includes a calcium influx channel, \val{Ca\_channel}, and a
calcium-extruding PMCA pump, \val{Ca\_Pump}.  It also includes two
cytosolic proteins that buffer calcium via the
\val{CalciumCalbindin\_gt\_BoundCytosol} and
\val{CalciumBuffer\_gt\_BoundCytosol} reactions.  Finally, the
rate expressions in this model do not include explicit factors of
the compartment volumes; instead, the various rate constants are
assume to include any necessary corrections for volume.

\sbmlexample{twodimensional.xml}



