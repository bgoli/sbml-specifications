% -*- TeX-master: "sbml-level-3-version-1-core"; fill-column: 66 -*-
% $Id$
% $HeadURL$
% ----------------------------------------------------------------

\section{Example models expressed in XML using SBML}
\label{sec:xml-rep}
\label{sec:examples}

In this section, we present several examples of complete models
encoded in XML using SBML Level~3.

%Our approach to translating
%the UML-based structure definitions presented in the previous
%sections is described elsewhere~\citep{hucka:2000b}.
%Appendix~\ref{apdx:schemas} gives the full listing of an XML
%Schema corresponding to SBML Level~2.


%-----------------------------------------------------------------------------
\subsection{A simple example application of SBML}
\label{sec:modeleg}
%-----------------------------------------------------------------------------

Consider the following representation of an enzymatic reaction:
\begin{center}
  \ce{E + S <=>[k_{\text{on}}][k_{\text{off}}] ES ->[k_{\text{cat}}] E + P}
\end{center}
Suppose we have the following initial species concentrations and
parameter values,
\begin{linenomath}
  \begin{equation*}
    \begin{array}{@{}lll@{}}
      \textrm{[E]}   & = 5 \cdot 10^{-21}\; \emph{mole} \; \emph{litre}^{\,-1}\\
      \textrm{[S]}   & = 10^{-20}\; \emph{mole} \; \emph{litre}^{\,-1}\\
      \textrm{[P]}   & = 0\; \emph{mole} \; \emph{litre}^{\,-1}\\
      \textrm{[ES]}  & = 0\; \emph{mole} \; \emph{litre}^{\,-1}\\
      k_{\text{on}}  & = 1\:000\:000 \;\emph{litre} \;\emph{mole}^{\,-1} \emph{second}^{\,-1}\\
      k_{\text{off}} & = 0.2 \;\emph{second}^{\,-1}\\
      k_{\text{cat}} & = 0.1 \;\emph{second}^{\,-1}
    \end{array}
  \end{equation*}
\end{linenomath}
and we place everything in a single compartment called
\token{comp} whose volume is $10^{-14}$ liters.  The following
is a minimal but complete SBML document encoding this model:

\sbmlexample{enzymekinetics.xml}

In this example, the model contains one compartment (with
identifier \val{comp}), four species (with identifiers
\val{ES}, \val{P}, \val{S}, and \val{E}), and two reactions
(\val{veq} and \val{vcat}).  The elements in the
\token{listOfReactants} and \token{listOfProducts} in each
reaction refer to elements listed in the \token{listOfSpecies}.
The correspondences between the various elements are explicitly
stated by the \token{speciesReference} elements.

The model also features local parameter definitions in each
reaction.  In this case, the three parameters (\val{kon},
\val{koff}, \val{kcat}) all have unique identifiers and they could
also have just as easily been declared global parameters in the
model.  Local parameters frequently become more useful in larger
models, where it may become tedious to assign unique identifiers
for all the different parameters.

Finally, the example above also demonstrates the use of unit
specifications throughout the model.  The \token{model} components
define the units of kinetic laws as being
\emph{mole}/\emph{second} by virtue of the values of the
attributes \token{extentUnits} and \token{timeUnits}.  Elsewhere
in the model, species, parameters and compartments are defined
with appropriate units so that the mathematical formulas inside
the \token{kineticLaw} elements work out to be
\emph{mole}/\emph{second}.   


%-----------------------------------------------------------------------------
\subsection{Example of a discrete version of a simple dimerization reaction}
\label{sec:discrete-eg}
%-----------------------------------------------------------------------------

\emph{(SBO annotations for this model contributed by Lukas Endler,
  EMBL-EBI, Cambridge, UK.)}

This example illustrates subtle differences between models
formulated for use in a continuous simulation framework (\eg using
differential equations) and those intended for a discrete
simulation framework.  The model shown here is suitable for use
with a discrete stochastic simulation algorithm of the sort
developed by \cite{gillespie:1977}.  In such an approach, species
are described in terms of molecular counts and simulation
proceeds by computing the probability of the time and identity of
the next reaction, then updating the species amounts
appropriately.

The model involves a simple dimerization reaction for a protein
named \val{P}:
\begin{linenomath}
\begin{equation*}
    2 P  \leftrightarrow  P_2
\end{equation*}
\end{linenomath}
The SBML representation is shown below.  There are several
important points to note.  First, the species \val{P} and \val{P2}
declare they are always in discrete amounts by using the flag
\token{hasOnlySubstanceUnits}=\val{true}.  This indicates that
when the species identifiers appear in mathematical formulas, the
units are \quantity{substance}, not 
\quantity{substance}/\quantity{size}.  A second point is that, as
a result, the corresponding ``kinetic law'' formulas do not need
volume corrections.  In Gillespie's approach, the constants in the
rate expressions (here, \val{c1} and \val{c2}) contain a
contribution from the kinetic constants of the reaction and the
size of the compartment in which the reactions take
place. This is a convention commonly adopted by stochastic
modellers, but is in no way essential---it is perfectly
reasonable to factor volume out of the rate constants, and in
certain situations it may be desirable to do so (\eg for models having
time-varying compartment volume), but due to the use of substance
units, it must be done differently compared to the deterministic case.
Thirdly, although the reaction is reversible, it is encoded as two
separate irreversible reactions, one each for the forward and
reverse directions, as averaging over the reactions will affect
the stochasticity.
Finally, it is worth noting the rate expression for the forward
reaction is a second-order mass-action reaction, but it is the
\emph{discrete} formulation of such a reaction
rate~\citep{gillespie:1977}.

\sbmlexample{dimerization.xml}

This example also illustrates the need to provide additional
information in a model so that software tools using different
mathematical frameworks can properly interpret it.  In this case,
a simulation tool designed for continuous ODE-based simulation
would likely misinterpret the model (in particular the reaction
rate formulas), unless it deduced that a discrete stochastic
simulation was intended.  One of the purposes of SBO annotations
(Section~\ref{sec:sboTerm}) is to enable such interpretation
without the need for deduction. However, the interpretation of the
model is essentially the same irrespective of whether the model is
to be simulated in a deterministic or stochastic manner, and a
properly SBML-compliant deterministic simulator will in most cases
correctly simulate the continuous deterministic approximation
of the stochastic model even if it has no stochastic simulation
capability.

\clearpage

The interpretation of rate laws for stochastic models is similar
to, yet different from, that of deterministic models. Taking the
first reaction as an example, the rate law is $c_1P(P-1)/2$ reaction
events per second. In the continuous deterministic case, the
interpretation of this is that the extent of the reaction in time
$dt$ is $[c_1P(P-1)/2]dt$ (and this leads naturally to the usual ODE
formulation of the model). In the stochastic case, the
interpretation is that the \emph{propensity} (or \emph{rate}, or
\emph{hazard}) of the reaction is $c_1P(P-1)/2$. That is, the
\emph{probability} of a single reaction event occurring in time
$dt$ is $[c_1P(P-1)/2]dt$ (and note that the \emph{expected} extent of
the reaction will be $[c_1P(P-1)/2]dt$). This interpretation leads to a Markov
jump process for the system dynamics, where the inter-event times
are exponentially distributed. Such dynamics can be simulated
using a discrete event simulation algorithm such as the
\emph{Gillespie algorithm}. In this case, the algorithm for
simulating the model can be described as follows:
\begin{enumerate}
\item Initialise $t:=0,\ c_1:=0.00166,\ c_2:=0.2,\ P:=301,\ P_2:=0$
\item Compute $h_1:=c_1P(P-1)/2,\ h_2:=c_2P_2$
\item Compute $h_0=h_1+h_2$
\item Simulate $t'\sim \operatorname{Exp}(h_0)$ and set $t:=t+t'$
\item With probability $h_1/h_0$ set $P:=P-2,\ P_2:=P_2+1$,
otherwise set $P:=P+2,\ P_2:=P_2-1$.
\item Output $t,\ P,\ P_2$
\item If $t<T_{max}$, return to step 2, otherwise stop.
\end{enumerate}
Although this is a simulation algorithm is a very practical way of
describing how to construct exact realisations of the Markov jump
process corresponding to the discrete stochastic kinetic model, it
is not a concise mathematical description. Such a description can
be provided by writing the model as a time change of a pair of
independent unit Poisson processes. Let $N_1(t)$ and $N_2(t)$ be
the counting functions of these processes, so that for each
$i=1,2$, $t>0$, $N_i(t)\sim \operatorname{Poisson}(t)$. Then,
writing $P(t)$ and $P_2(t)$ for the numbers of molecules of $P$
and $P_2$ at time $t$, respectively, we have that the stochastic process
$\{P(t),P_2(t)\,|\,t>0\}$ satisfies the stochastic integral equation
\begin{align*}
P_2(t) &= N_1\left(\int_0^t
c_1\frac{P(\tau)[P(\tau)-1]}{2}d\tau\right) - N_2\left(\int_0^t
c_2 P_2(\tau)d\tau\right) \\
P(t) &= 301 - 2P_2(t).
\end{align*}
The above representation is arguably the most useful for
mathematical analysis of the stochastic model; see \cite{ball:2006} for
details. Another popular representation is the so-called chemical
Master equation (CME) for the probability distribution of the possible
states at all times \citep{gillespie:1992}. In this case, since there are
151 possible states of the system (corresponding to the 151
possible values of $P_2$), the CME consists of 151 coupled
ODEs,
\[
\frac{d}{dt}p(P,P_2,t) =
\left\{
\begin{array}{ll}
\displaystyle-\frac{c_1}{2}\times 301\times 299p(301,0,t)+c_2p(299,1,t),
&P=301,\ P_2=0,\\
\displaystyle\frac{c_1}{2}(P+2)(P+1)p(P+2,P_2-1,t)-\frac{c_1}{2}P(P-1)p(P,P_2,t)&P=301-x,\ P_2=x,\\
\qquad+c_2(P_2+1)p(P-2,P_2+1,t)-c_2P_2p(P,P_2,t),
&\quad x=1,2,\ldots,149,\\
\displaystyle\frac{c_1}{2}\times 2\times 3p(3,149,t)-c_2\times 150p(1,150,t),
&P=1,\ P_2=150,
\end{array}
\right.
\]
where $p(P,P_2,t)$ denotes the probability that there are $P$
molecules of $P$ and $P_2$ molecules of $P_2$ at time $t$, and the
ODEs are subject to the initial conditions
\[
p(301,0,0)=1,\ p(301-2x,x,0)=0,\ x=1,2,\ldots,150.
\]
See \cite{evans:2008} for further examples of discrete stochastic
kinetic models encoded in SBML and \cite{wilkinson_2006} for an
introduction to discrete stochastic modelling using SBML.



%-----------------------------------------------------------------------------
\subsection{Example involving assignment rules}
\label{apdx:rules-eg}
%-----------------------------------------------------------------------------

This section contains a model that simulates a system containing a
fast reaction.  This model uses rules to express the mathematics
of the fast reaction explicitly rather than using the \token{fast}
attribute on a reaction element.  The system modeled is
\begin{linenomath}
\begin{equation*}
  \begin{array}{@{}ccc@{}}
    X_0 & \overset{\underrightarrow{k_1 [X_0]}}{}           & S_1 \\[6pt]
    S_1 & \overset{\underrightarrow{k_f [S_1] - k_r [S_2]}}{} & S_2 \\[6pt]
    S_2 & \overset{\underrightarrow{k_2 [S_2]}}{}           & X_1 \\[6pt]
  \end{array}
\end{equation*}
\begin{equation*}
    k_1 = 0.1, \quad k_2 = 0.15, \quad k_f = K_{eq} 10000, \quad k_r = 10000, \quad K_{eq} = 2.5.
\end{equation*}
\end{linenomath}
where $[X_0]$, $[S_1]$, $[S_1]$, and $[S_2]$ are species in concentration units,
and $k_1$, $k_2$, $k_f$, $k_r$, and $K_{eq}$ are parameters.  This
system of reactions can be approximated with the following new
system:
\begin{linenomath}
  \begin{equation*}
    \begin{array}{@{}ccc@{}}
      X_0 & \overset{\underrightarrow{k_1 [X_0]}}{} & T \\[6pt]
      T & \overset{\underrightarrow{k_2 [S_2]}}{} & X_1
    \end{array}
  \end{equation*}
  \begin{equation*}
    \begin{aligned}\\[-3ex]
      [S_1] & = \dfrac{[T]}{1 + K_{eq}} \\[6pt]
      [S_2] & = K_{eq} [S_1]
    \end{aligned}
  \end{equation*}
\end{linenomath}

where $T$ is a new species.  The following example SBML model
encodes the second system.

\sbmlexample{assignmentrules.xml}


%-----------------------------------------------------------------------------
\subsection{Example involving algebraic rules}
\label{sec:algeraiceg}
%-----------------------------------------------------------------------------

This section contains an example model that contains two
\AlgebraicRule objects that are necessary to determine the values
of two variables within the model.  In this particular case, the
rules cannot be rewritten in terms of \AssignmentRule.  This
example illustrates a more rigorous analysis of the enzymatic
reaction given in the example of Section~\ref{sec:modeleg}.
\begin{center}
  \ce{E + S <=>[k1_{\text{on}}][k1_{\text{off}}] ES ->[k2] E + P}
\end{center}
In this example, we describe a quasi-steady-state approximation of
the enzymatic reaction equation shown above.  It is based on two
assumptions.  First, the rate at which the concentration of the
substrate bound enzyme ($[ES]$) changes is assumed to be slow
compared to the rate of change of concentration of both the
substrate ($[S]$) and product ($[P]$).  Second, the total
concentration of the enzyme is assumed to stay constant over time.
This means we can assume the concentration of $[ES]$ and $[E]$ are
not governed by the reactions, and so some other equations must be
used to determine the values of these concentrations in order to
be able to simulate the model.

Applying the first assumption means that the rate of change of
$[ES]$ should be set to zero:
\begin{linenomath}
\begin{equation*}
  \dfrac{d [ES]}{d t} = k1\sub{on} \cdot [E] \cdot [S] - (k1\sub{off} + k2) \cdot [ES] = 0
\end{equation*}
\end{linenomath}

The second assumption can be written as
\begin{linenomath}
\begin{equation*}
  [E\sub{total}] = [E] + [ES]\\[6pt]
\end{equation*}
\end{linenomath}
which, after rearranging, becomes
\begin{linenomath}
\begin{equation*}
  [E\sub{total}] - ([E] + [ES]) = 0
\end{equation*}
\end{linenomath}

Thus, we have two algebraic rules that must be applied to
determine the the values of $[E]$ and $[ES]$.  The SBML encoding
of this model is given below.  Note that the species $E$ and $ES$
have their \token{boundaryCondition} attribute set to \val{true}.
This means that a simulation tool should not construct equations
for them based on the reactions in the system.  Their values are
instead set using the rules in the model.  Also, the model uses a
dummy species $E\sub{total}$ with its \token{constant} attribute
set to \val{true}; its role is to assign the total concentration
of the enzyme in the model.  This could just as easily have been
done using a parameter instead of a constant dummy species, but we
use the latter approach as an illustration.

\sbmlexample{twoalgebraicrules.xml}


%-----------------------------------------------------------------------------
\subsection{Example with combinations of
  \token{boundaryCondition} and \token{constant} values on \class{Species}
  with \class{RateRule} objects}
\label{sec:constantspecieseg}
%-----------------------------------------------------------------------------

In this section, we discuss a model that includes four species,
each with a different combination of values for their
\token{boundaryCondition} and \token{constant} attributes.  The
model represents a hypothetical system containing one reaction,
\begin{linenomath}
\begin{equation*}
  \begin{array}{@{}ccc@{}}
    S_1 + S_2 & \overset{\underrightarrow{k_1 [S_1] [S_2] [S_3]}}{} & S_4 \\ \\[-4pt]
  \end{array}
\end{equation*}
\end{linenomath}
where $S_3$ is a species that catalyzes the conversion of species
$S_1$ and $S_2$ into $S_4$.  Species $S_1$ and $S_2$ are on the
boundary of the system (\ie $S_1$ and $S_2$ are reactants but
their values are not determined by kinetic laws).  The value of
$S_1$ in the system is determined over time by the rate rule:
\begin{linenomath}
\begin{equation*}
  \dfrac{d [S_1]}{d t} = k_2
\end{equation*}
\end{linenomath}
The species $S_2$ and $S_3$ are not affected by the either the
reaction or the rate rule, and have the following initial
concentration values:
\begin{linenomath}
\begin{equation*}
    [S_2] = 1, \quad [S_3] = 2
  \end{equation*}
\end{linenomath}
The values of constant parameters in the system are:
\begin{linenomath}
\begin{equation*}
    k_1 = 0.5, \quad k_2 = 0.1
  \end{equation*}
\end{linenomath}
and the initial values of varying species are:
\begin{linenomath}
\begin{equation*}
    [S_1] = 0, \quad [S_4] = 0
\end{equation*}
\end{linenomath}

The value of $[S_1]$ varies over time and it is on the boundary,
so in the SBML representation, \token{S1} has a \token{constant}
attribute with a value of \val{false} and a
\token{boundaryCondition} attribute with a value of \val{true}.
The value of $[S_2]$ is fixed and it is also on the boundary, so
\token{S2} has a \token{constant} attribute value of \val{false}
and a \token{boundaryCondition} attribute value of \val{true}.
$[S_3]$ is fixed but not on the boundary, so the \token{constant}
attribute is \val{true} and the \token{boundaryCondition}
attribute is \val{false}.  Finally, $[S_4]$ is a product whose
value is determined by a kinetic law and therefore in the SBML
representation has \val{false} for both its
\token{boundaryCondition} and \token{constant} attributes.

The following is the SBML rendition of the model shown above:

\sbmlexample{boundarycondition.xml}


%-----------------------------------------------------------------------------
\subsection{Example of translation from a multi-compartmental model to ODEs}
\label{sec:odeeg}
%-----------------------------------------------------------------------------

This section contains a model with two compartments and four
reactions.  The model is derived from Lotka-Volterra, with the
addition of a reversible transport step.  When observed in a
time-course simulation, three of this model's species display
damped oscillations.

\begin{figure}[htb]
  \vspace*{5pt}
  \centering
  \begin{picture}(260,60)
    \put(0,10){\framebox(255,50)[tl]{ cytosol}}
    \put(10,19){\framebox(105,29)[tl]{ nucleus}}
    \put(24,26){$
        X + Y_{1n} \yields^{\cit k\sub{1}} 2\,Y_{1n}
        \eqbm^{\cit K\sub{T}} 2\,Y_{1c} + 2\,Y_2
        \yields^{\cit k\sub{2}} 4\,Y_2 \yields^{\cit k\sub{3}} \emptyset
        $}
  \end{picture}
  \vspace*{-8pt}
  \caption{A example multi-compartmental model.}
  \label{fig:multicomp}
\end{figure}

Figure~\ref{fig:multicomp} illustrates the arrangement of
compartments and reactions in the model
\token{LotkaVolterra\_transport}.  The reaction between the
compartments called \token{cytosol} and \token{nucleus} is a
transport reaction whose mechanisms are not modeled here; in
particular, the reaction does not take place on the membrane
between the compartments, and is modeled here simply as a process
that spans the two three-dimensional compartments.

The text of the SBML representation of the model is shown below,
and it is followed by its complete translation into ordinary
differential equations.  As usual, in this SBML model, the
reaction rate equations in the kinetic laws are in substance per
time units.  The reactions have also been simplified to reduce
common stoichiometric factors in the original system depicted in
Figure~\vref{fig:multicomp}.  The species variables in this SBML
representation are in concentration units; their initial
quantities are declared using the attribute \token{initialAmount}
on the \token{species} definitions, but since the attribute
\token{hasOnlySubstanceUnits} is \emph{not} set to true, the
identifiers of the species represent their concentrations when
those identifiers appear in mathematical expressions elsewhere in
the model.  Note that the species whose identifier is \val{X} is a
boundary condition, as indicated by the attribute
\token{boundaryCondition}=\val{true} in its definition.

\sbmlexample{multicomp.xml}

The ODE translation of this model is as follows.  First, we give
the values of the constant parameters:
\begin{linenomath}
  \begin{equation*}
    \begin{array}{@{}lll@{}}
      k_1   & = 2500\; \emph{litre} \; \emph{mole}^{\,-1}\; \emph{second}^{\,-1}\\
      k_2   & = 2500\; \emph{litre} \; \emph{mole}^{\,-1}\; \emph{second}^{\,-1}\\
      K_3   & = 25000\; \emph{second}^{\,-1}\\
      K_T   & = 25000\; \emph{second}^{\,-1}\\
    \end{array}
  \end{equation*}
\end{linenomath}
Now on to the initial conditions of the variables.  In the
following, the terms $[X]$, $[Y_{1n}]$, $[Y_{1c}]$, and $[Y_2]$
refer to the species' concentrations.  Note that the corresponding
species identifiers \token{X}, \token{Y\_{1n}}, \token{Y\_{1c}}
and \token{Y\_2} in the model are in concentration units, even
though all the values in the model are initialized in terms of
amounts.  (The reason the species identifiers in the model are
still in concentration units goes back to the meaning of the
\token{hasOnlySubstanceUnits} attribute on a \Species; if the
attribute is set to a value of \val{false}, a species' symbol in a
model is interpreted as a concentration or density regardless of
whether its initial value is set using \token{initialAmount} or
\token{initialConcentration}.)  We use $V_n$ to represent the size
of compartment \val{nucleus} and $V_c$ the size of compartment
\val{cytoplasm}:
\begin{linenomath}
  \begin{equation*}
    \begin{array}{@{}lll@{}}
      V_n    & = 1\; \emph{litre}\\
      V_c    & = 5\; \emph{litre}\\
      X      & = 1\; \emph{mole}\\
      Y_{1n} & = 1\; \emph{mole}\\
      Y_{1c} & = 0\; \emph{mole}\\
      Y_2    & = 1/5\; \emph{mole}\\
    \end{array}
  \end{equation*}
\end{linenomath}
And finally, here are the differential equations:
\begin{linenomath}
\begin{align*}
  \dfrac{d [X]}{d t}    &= 0 \\[6pt]
  V_n \dfrac{d [Y_{1n}]}{d t} &= k_1 [X] [Y_{1n}] [V_n] - K_T \big([Y_{1n}] - [Y_{1c}]\big) V_c
    && \text{reactions production and transport} \\[6pt]
  V_c \dfrac{d [Y_{1c}]}{d t} &= K_T \big([Y_{1n}] - [Y_{1c}]\big) V_c - k_2 [Y_{1c}] [Y_2] V_c
    && \text{reactions transport and transformation} \\[6pt]
  V_c \dfrac{d [Y_2]}{d t}    &= k_2 [Y_{1c}] [Y_2] V_c - k_3 [Y_2] V_c
    && \text{reactions transformation and degradation}
\end{align*}
\end{linenomath}

As formulated here, this example assumes constant volumes.  If the
sizes of the compartments \val{cytoplasm} or \val{nucleus} could
change during simulation, then it would be preferable to use a
different approach to constructing the differential equations.  In
this alternative approach, the ODEs would compute substance change
rather than concentration change, and the concentration values
would be computed using separate equations.  This approach is used
in Section~\ref{sec:about-kinetic-laws}.


%-----------------------------------------------------------------------------
\subsection{Example involving function definitions}
\label{sec:functioneg}
%-----------------------------------------------------------------------------

This section contains a model that uses the function definition
feature of SBML.  Consider the following hypothetical system:
\begin{linenomath}
\begin{equation*}
  \begin{array}{@{}ccc@{}}
    S_1 & \overset{\underrightarrow{f([S_1])}}{} & S_2 \\ \\[-4pt]
  \end{array}
\end{equation*}
\end{linenomath}
where
\begin{linenomath}
\begin{equation*}
    f(x) = 2 x
\end{equation*}
\end{linenomath}

The following is the XML document that encodes the model shown
above:

\sbmlexample{functiondef.xml}


%-----------------------------------------------------------------------------
\subsection{Example involving \emph{delay} functions}
\label{sec:delayeg}
%-----------------------------------------------------------------------------

The following is a simple model illustrating the use of $delay$ to
represent a gene that suppresses its own expression.  The model
can be expressed in a single rule:
\begin{linenomath}
\begin{equation*}
  \frac{d [P]}{d t} = \dfrac{ \dfrac{1}{1 + m [P_{delayed}]^q} - [P] }{ \tau }\\
\end{equation*}
\end{linenomath}
where
\begin{linenomath}
  \begin{equation*}
    \begin{array}{rll}
      [P_\text{delayed}] & \text{is } delay([P], \Delta_t) \text{ or [P] at } t - \Delta_t\\[0pt]
      [P]           & \text{is protein concentration}\\
      \tau          & \text{is the response time}\\
      m             & \text{is a multiplier or equilibrium constant}\\
      q             & \text{is the Hill coefficient}\\
    \end{array}
  \end{equation*}
\end{linenomath}
and the species quantities are in concentration units.
The text of an SBML encoding of this model is given below:

\sbmlexample{delay.xml}


%-----------------------------------------------------------------------------
\subsection{Example involving events}
\label{sec:eventeg}
%-----------------------------------------------------------------------------

This section presents a simple model system that demonstrates the
use of events in SBML.  Consider a system with two genes,
$G_1$ and $G_2$.  $G_1$ is initially
on and $G_2$ is initially off.  When turned on, the two
genes lead to the production of two products, $P_1$ and $P_2$,
respectively, at a fixed rate.  When $P_1$ reaches a given
concentration, $G_2$ switches on.  This system can be
represented mathematically as follows:
\begin{linenomath}
\begin{eqnarray*}
  \dfrac{d [P_1]}{d t} & = & k_1 \big([G_1] - [P_1]\big)\\[3pt]
  \dfrac{d [P_2]}{d t} & = & k_2 \big([G_2] - [P_2]\big)\\[0pt]
  [G_2] & = &
    \begin{cases}
      0 & \text{when $[P_1] \leq \tau$},\\
      1 & \text{when $[P_1] > \tau$}.
    \end{cases}
\end{eqnarray*}
\end{linenomath}

The initial values are:
\begin{linenomath}
\begin{equation*}
  [G_1] = 1, \quad [G_2] = 0, \quad \tau = 0.25, \quad P_1 = 0, \quad P_2 = 0, \quad k_1 = k_2 = 1.
\end{equation*}
\end{linenomath}

The SBML Level 3 representation of this is as follows:

\sbmlexample{events.xml}


%-----------------------------------------------------------------------------
\subsection{Example involving two-dimensional compartments}
\label{sec:two-dimensional-eg}
%-----------------------------------------------------------------------------

The following example is a model that uses a two-dimensional
compartment.  It is a fragment of a larger model of calcium
regulation across the plasma membrane of a cell.  The model
includes a calcium influx channel, \val{Ca\_channel}, and a
calcium-extruding PMCA pump, \val{Ca\_Pump}.  It also includes two
cytosolic proteins that buffer calcium via the
\val{CalciumCalbindin\_gt\_BoundCytosol} and
\val{CalciumBuffer\_gt\_BoundCytosol} reactions.  Finally, the
rate expressions in this model do not include explicit factors of
the compartment volumes; instead, the various rate constants are
assumed to include any necessary corrections for volume.

\sbmlexample{twodimensional.xml}


%-----------------------------------------------------------------------------
\subsection{Example of a reaction located at a membrane}
\label{sec:eg:membrane-reaction}
%-----------------------------------------------------------------------------

\label{sec:reaction-membrane-eg}This section describes a model containing
one single enzymatic reaction where substrate and product are located
in the same compartment but the enzyme is localized at the membrane
surrounding the compartment.

\begin{center}
$R$: \ce{S ->[\ensuremath{[E]}] P}
\par\end{center}

The model contains two compartments, a three-dimensional one
called {}``cytosol'' and a two-dimensional one called
{}``membrane'' that is assumed to be the boundary of the cell. The
reaction $R$ has a substrate $S$ and a product $P$ that are both
located in the cytosol. The enzyme $E$ that catalyzes the
reactions is located at the membrane. The kinetic law of reaction
$R$ is\[ v=A\cdot\frac{k_{cat}\cdot[E]\cdot[S]}{K_{M}+[S]}\] where
$A$ is the area of the membrane (measured in $\mu m^{2}$), $[E]$
is the \emph{density }of the enzyme on the membrane (in
$\mu\mathrm{mol}~\mu m^{-2}$), $[S]$ is the \emph{concentration
}of the substrate (in $\mu mol~l^{\,-1}$), $K_{M}$ the
Michaelis-Menten constant (also in $\mu mol~l^{\,-1}$), and
$k_{cat}$ the rate constant (in $min^{-1}$). The units of the
result of the kinetic law are in $\mu mol~min^{-1}$. Since the
units for the amounts of all species ($S$, $P$, and $E$) and for
the reaction extent are the same ($\mu mol$) the model does not
require unit conversion factors.

The kinetic law as it is given here scales correctly for changes in
cytosol volume, membrane area, or enzyme density. This means that
if one of these values is changed (even if it is variable during a
simulation) the rate law is still valid.

The following is the text of the model's SBML representation.

\sbmlexample{membrane.xml}

