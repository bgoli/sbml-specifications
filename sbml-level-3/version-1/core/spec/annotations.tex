% -*- TeX-master: "sbml-level-3-version-1-core"; fill-column: 66 -*-
% $Id$
% $HeadURL$
% ----------------------------------------------------------------

\section{A standard format for the \token{annotation} element}
\label{sec:finney-novere}
\label{sec:annotation-standard}

This section describes the recommended non-proprietary
format for the content of \token{annotation} elements
in SBML when (a) referring to controlled vocabulary
terms and database identifiers which define and describe
biological and biochemical entities, and (b) describing the
creator of a model and its modification history. Such a structured
format should facilitate the generation of models compliant with
the MIRIAM guidelines for model curation~\citep{le_novere:2005}.

The format described in this section is intended to be the form of
one of the top-level elements that could reside in an
\token{annotation} element attached to an SBML object derived from
\SBase.  The element is named \token{rdf:RDF}.  The format
described here is compliant with the constraints placed on the
form of annotation elements described in
Section~\ref{sec:annotation-use}.  We refer readers to
Section~\ref{sec:annotation-use} for important information on the
structure and organization of application-specific annotations;
these are not described here.

The annotations described in this section are optional on a model,
but if present, they must conform to the details specified here in
order to be considered valid annotations in this format.  If they
do not conform to the format described here, it does not render
the overall SBML model invalid, but the annotations are then
considered to be in a proprietary format rather than being
\emph{SBML MIRIAM annotations}.


\subsection{Motivation}

The SBML structures described elsewhere in this document do not
have any biochemical or biological semantics.  This section
provides a scheme for linking SBML structures to external
resources so that those structures can be given semantics.  The
motivation for the introduction of this scheme is similar to that
given for the introduction of \token{sboTerm}; however, the
general annotation scheme here is more flexible.

It is generally not recommended that this format be used to refer
to SBO terms.  In most cases, the SBO terms should be referred to
using the attribute \token{sboTerm} part of \token{SBase}
(Section~\ref{sec:sboTerm}). However in certain situations, for
instance to be able to add further information about the
functional role of a species, it is necessary to add this
additional information using the annotation format described here.

\subsection{XML namespaces in the standard annotation}

This format uses a restricted form of Dublin
Core~\citep{DCMI:2003} and BioModels qualifier elements (see
\url{http://sbml.org/miriam/qualifiers/}) embedded in the XML form
of RDF~\citep{w3c:2004}.  The scheme defined here uses a number of
external XML standards and associated XML namespaces.
Table~\ref{tab:namespaces-for-standard-annotation} lists these
namespaces and relevant documentation on those namespaces.  The
format constrains the order of elements in these namespaces beyond
the constraints defined in the standard definitions for those
namespaces.  For each standard listed, the format only uses a
subset of the possible syntax defined by the given standard.
Thus, it is possible for an \token{annotation} element to include
XML that is compliant with those external standards but is not
compliant with the format described here.

\begin{table}[bh]
  \small
  \centering
  \begin{edtable}{tabular}{lll}
    \toprule
    \textbf{Prefix used in}\\
    \textbf{examples here} & \textbf{Namespace URI} & \textbf{Reference/description} \\
    \midrule
    \token{dc}      & \uri{http://purl.org/dc/elements/1.1/} & \cite{powell:2003}\\
    \token{rdf}     & \uri{http://www.w3.org/1999/02/22-rdf-syntax-ns\#} & \cite{w3c:2004b} \\
    \token{dcterms} & \uri{http://purl.org/dc/terms/} & \cite{kokkelink:2002},\\
                    &                                 & \cite{DCMIUB:2005} \\
    \token{vcard}   & \uri{http://www.w3.org/2001/vcard-rdf/3.0\#} & \cite{iannella:2001} \\
    \token{bqbiol}  & \uri{http://biomodels.net/biology-qualifiers/} & \url{http://sbml.org/miriam/qualifiers/} \\
    \token{bqmodel} & \uri{http://biomodels.net/model-qualifiers/} & \url{http://sbml.org/miriam/qualifiers/} \\
    \bottomrule
  \end{edtable}
  \vspace*{-0.95ex}
  \caption{The XML standards used in the SBML standard format for annotation.
  The namespace prefix are shown to indicate only the prefix used in the main text.}
  \label{tab:namespaces-for-standard-annotation}
\end{table}


% The default page break is too awkward here.
\clearpage

%=============================================================================
\subsection{General syntax for the standard annotation}
%=============================================================================

An outline of the format syntax is shown below.

\vspace*{0.1ex}

\begin{example}
<SBML_ELEMENT +++ metaid="SBML_META_ID" +++ >
  +++
  <annotation>
    +++
    <rdf:RDF xmlns:rdf="http://www.w3.org/1999/02/22-rdf-syntax-ns\#"
             xmlns:dc="http://purl.org/dc/elements/1.1/"
             xmlns:dcterm="http://purl.org/dc/terms/"
             xmlns:vcard="http://www.w3.org/2001/vcard-rdf/3.0\#"
             xmlns:bqbiol="http://biomodels.net/biology-qualifiers/"
             xmlns:bqmodel="http://biomodels.net/model-qualifiers/"
    >
      <rdf:Description rdf:about="#SBML_META_ID">
        [HISTORY]
        <RELATION_ELEMENT>
          <rdf:Bag>
            <rdf:li rdf:resource="URI" />
            ...
          </rdf:Bag>
        </RELATION_ELEMENT>
        ...
      </rdf:Description>
      +++
    </rdf:RDF>
    +++
  </annotation>
  +++
</SBML_ELEMENT>
\end{example}

The above outline shows the order of the elements. The capitalized
identifiers refer to generic strings of a particular type:
\texttt{SBML\_ELEMENT} refers to any SBML element name that can
contain an \token{annotation} element; \texttt{SBML\_META\_ID} is
a XML \primtype{ID} string; \texttt{RELATION\_ELEMENT} refers to
element names in either the namespace
\uri{http://biomodels.net/biology-qualifiers/} or
\uri{http://biomodels.net/model-qualifiers/}; and \texttt{URI} is
a URI (See Section~\ref{sec:uri-in-annotation}).
\texttt{[HISTORY]} refers to an optional section described
in Section~\ref{sec:model-history-annotation}. `\texttt{+++}' is a
placeholder for either no content or valid XML syntax that is not
defined by the standard annotation scheme but is consistent with
the relevant standards for the enclosing elements. `\texttt{...}'
is a placeholder for zero or more elements of the same form as the
immediately preceding element.  The precise form of whitespace and
the XML namespace prefix definitions is not constrained; however,
the elements and attributes must be in the namespaces shown. The
rest of this section describes the format formally in English.

In this format, the annotation of an element is located in a single
\token{rdf:RDF} element contained within an SBML
\token{annotation} element. The annotation element can contain
other elements in any order as described in 
Section~\ref{sec:annotation-use}.  The format described in this
section only defines the form of the \token{rdf:RDF} element. The
containing SBML \SBase element must have a \token{metaid} attribute
value. (As this attribute is of the type \primtype{ID} its value
must be unique to the entire SBML document.)

The first element of the \token{rdf:RDF} element must be an
\token{rdf:Description} element with an \token{rdf:about}
attribute. The value of the
\token{rdf:about} attribute must be of the form
\texttt{\#<string>} where the string component is equal to the
value of the \token{metaid} attribute of the containing SBML element.
This format doesn't define the form of subsequent
subelements of the \token{rdf:RDF} element. In particular, the unique \token{rdf:RDF} element contained in the annotation can contain other \token{rdf:Description}, which content can be any valid RDF.

The \token{rdf:Description} element can contain only an optional
history section (see
Section~\ref{sec:model-history-annotation}) followed by a sequence
of zero or more BioModels relation elements.  
The specific type of the relation elements will vary depending on
the relationship between the SBML component and referenced
information or resource.

Although Section~\ref{sec:qualified-dc-annotation} describes the
detailed semantics of each of the relation element types, the
content of these elements follows the same form.  The BioModels
qualifiers relation elements must only contain a single
\token{rdf:Bag} element which in turn must only contain one or
more \token{rdf:li} elements.  The \token{rdf:li} elements must
only have a \token{rdf:resource} attribute containing a URI
referring to an information resource (See
Section~\ref{sec:uri-in-annotation}).

%Annotations in this format can be located at different depths
%within a model component.

%=============================================================================
\subsection{Use of URIs}
\label{sec:uri-in-annotation}
%=============================================================================

The format represents a set of relationships between SBML elements
and resources referred to by \token{rdf:resource} attribute
values.  The BioModels relation elements simply define the 
relationship.

For example, a \Species element representing a protein could be
annotated with a reference to the database UniProt by the
\uri{urn:miriam:uniprot:P12999} resource identifier,
identifying exactly the protein described by the \Species element.
This identifier maps to a unique entry in UniProt which is never
deleted from the database. In the case of UniProt, this is the
``accession'' of the entry. When the entry is merged with another
one, both ``accession'' are conserved. Similarly in a controlled
vocabulary resource, each term is associated with a perennial
identifier. The UniProt entry also possess an ``entry name'' (the
Swiss-Prot ``identifier''), a ``protein name'', ``synonyms'' etc.
Only the ``accession'' is perennial and should be used.

The value of a \token{rdf:resource} attribute is a URI that both
uniquely identifies the resource, and the data in the resource. 
The value of the \token{rdf:resource} attribute is a URI, not a
URL; as such, a URI does not have to reference a physical web
object but simply identifies a controlled vocabulary term or
database object (a URI is a label). For instance, a true URL for an Internet
resource such as \uri{http://www.uniprot.org/entry/P12999} might
correspond to the URI \uri{urn:miriam:uniprot:P12999}.

SBML does not specify how a parser is to interpret a URI. In the
case of a transformation into a physical URL, there could be
several solutions. For example, the URI
\uri{urn:miriam:obo.go:GO\%3A0007268} can be translated into:

\noindent \uri{http://www.ebi.ac.uk/ego/DisplayGoTerm?selected=GO:0007268}\\
\noindent \uri{http://www.godatabase.org/cgi-bin/amigo/go.cgi?view=details\&query=GO:0007268}\\
\noindent \uri{http://www.informatics.jax.org/searches/GO.cgi?id=GO:0007268}

Similarly the URI \uri{urn:miriam:ec-code:3.5.4.4} can refer to:

\noindent \uri{http://www.ebi.ac.uk/intenz/query?cmd=SearchEC\&ec=3.5.4.4}\\
\noindent \uri{http://www.expasy.org/cgi-bin/nicezyme.pl?3.5.4.4}\\
\noindent \uri{http://www.chem.qmul.ac.uk/iubmb/enzyme/EC3/5/4/4.html}\\
\noindent \uri{http://www.genome.jp/dbget-bin/www\_bget?ec:3.5.4.4}\\[1ex]
\noindent etc.

To enable interoperability, the community has agreed on an initial
set of standardized valid URI syntax rules which may be used
within the standard annotation format.  This set of rules is not
part of the SBML standard but will grow independently from
specific SBML Levels and Versions.  As the set changes, a given
URI syntax rule will not be modified, although the physical
resources associated with the rule may change.  These URIs will
always be composed as \token{resource:id}.  An up-to-date list and
explanation of the URIs is always available online at
\url{http://sbml.org/miriam/qualifiers}.  Each entry contains a
list of SBML and relation elements in which the given URI can be
appropriately embedded.  To enable consistent and thus useful
links to external resources, the URI syntax rule set must have a
consistent view of the concepts represented by the different SBML
elements for the purposes of this format.  For example, as the
rule set is designed to link SBML biological and biochemical
resources the rule set assumes that a \Species element represents
the concept of a biochemical entity type rather than mathematical
symbol.  The URI rule list will evolve with the evolution of
databases and resources.

Note that this means that all \token{rdf:resource} \emph{must} be
MIRIAM URIs and thus cannot refer to for example other elements in
the model.  While it would be possible to place such information
in other RDF content elsewhere (for example after the first
\token{rdf:Description} element), this would place this
information outside the simple annotation scheme described here,
and as such there would be no guarantee that other software
programs could read this information.


%=============================================================================
\subsection{Relation elements}
\label{sec:qualified-dc-annotation}
%=============================================================================

Different BioModels qualifier elements encode different types of
relationships.  When appearing in an annotation, each qualifier
element encloses a set of \token{rdf:li} elements.  Its appearance
in a relation element implies a specific relationship between the
enclosing SBML object and the resources referenced by the
\token{rdf:li} elements.  The detailed semantics of each qualifier
is defined online at \url{http://sbml.org/miriam/qualifiers/}.
Generally, each qualifier is defined with the assumption that the
biological entity represented by a given SBML element is the
concept linked to the reference resource.

Several relation elements with a given tag, enclosed in the same
SBML element, each represent an alternative annotation to the SBML
element.  For example two \token{bqbiol:hasPart} elements within a
\Species SBML element represent two different sets of references
to the parts making up the the chemical entity represented by the
species.  (The species is not made up of all the entities
represented by all the references combined).

The complete list of the qualifier elements in the BioModels
qualifier namespaces is documented online at
\url{http://sbml.org/miriam/qualifiers/}. The list is divided into
two symbol namespaces, one for model qualifiers; this has the URI
\uri{http://biomodels.net/model-qualifiers/} (and we use the
prefix \token{bqmodel} in examples shown in this section).  The
other is for biological qualifiers; this has the URI
\uri{http://biomodels.net/biology-qualifiers/} (and we use the
prefix \token{bqbiol}).  The list will only grow; \ie no element
will be removed from the list.  The following is the list of
elements at the time of writing:

%\begin{figure}[htb]
%  \vspace*{2pt}
%  \centering
%\[
%%\begin{CD}
%\textrm{SBML Element}  @>{qualifier}>> \textrm{Referenced Resource} \\
%@VV{represents}V                @VV{represents}V \\
%\textrm{Biological Entity A}   @>{relationship}>>
%\textrm{Biological Entity B}
%%\end{CD}
%\]
%  \caption{How the relationship between biological
%  entities are defined using relations between resources}
%  \label{fig:lenovere-annotation}
%\end{figure}

\begin{itemize}

\item \token{bqmodel:is} The modeling object encoded by the SBML
  component is the subject of the referenced resource. For
  instance, this qualifier might be used to link the model to a
  model database.

\item \token{bqmodel:isDescribedBy} The modeling object
  encoded by the SBML component is described by
  the referenced resource. This relation might be used to link
  SBML components to the literature that describes this model or
  this kinetic law.

\item \token{bqbiol:is} The biological entity represented by the
  SBML component is the subject of the referenced resource. This
  relation might be used to link a reaction to its exact
  counterpart in (e.g.) KEGG or Reactome.

\item \token{bqbiol:hasPart} The biological entity represented by
  the SBML component includes the subject of the referenced
  resource, either physically or logically. This relation might be
  used to link a complex to the description of its components.

\item \token{bqbiol:isPartOf} The biological entity represented by
  the SBML component is a physical or logical part of the subject
  of the referenced resource. This relation might be used to link
  a component to the description of the complex it belongs to.

\item \token{bqbiol:isVersionOf} The biological entity represented
  by the SBML component is a version or an instance of the subject
  of the referenced resource.

\item \token{bqbiol:hasVersion} The subject of the referenced
  resource is a version or an instance of the biological entity
  represented by the SBML component.

\item \token{bqbiol:isHomologTo} The biological entity represented
  by the SBML component is homolog, to the subject of the
  referenced resource, i.e. they share a common ancestor.

\item \token{bqbiol:isDescribedBy} The biological entity
  represented by the SBML component is described by the referenced
  resource. This relation should be used, for example, to link a
  species or a parameter to the literature that describes the
  quantity of the species or the value of the parameter.


\item \token{bqbiol:isEncodedBy} The biological entity represented
  by the model component is encoded, either directly or by virtue
  of transitivity, by the subject of the referenced resource.

\item \token{bqbiol:encodes} The biological entity represented by
  the model component encodes, either directly or by virtue of
  transitivity, the subject of the referenced resource.

\item \token{bqbiol:occursIn}  The biological entity represented by the model component takes place in the subject of the reference resource.


\end{itemize}

%In all cases using the biology qualifiers, the `object' of the
%relation is the biological or biochemical object represented by
%the enclosing SBML element. In the cases of the model qualifiers,
%the `object' of the relation is the model component encoded by the
%enclosing SBML element. The resources referenced by the
%\class{rdf:li} elements contained within the relation element are
%the `object' of the relation.


%===============================================================
\subsection{History}
\label{sec:model-history-annotation}
%================================================================

The format described in previous sections can include additional
elements to describe the history of the model or a portion of the
model.  If this history data is present, it must occur immediately
before the first BioModels relation elements.  These additional
elements encode information on the model creator and a sequence of
dates recording changes to the model.  The syntax for this section
is outlined below.

\begin{example}
<dc:creator>
  <rdf:Bag>
    <rdf:li rdf:parseType="Resource">
      [[
      +++
      <vCard:N rdf:parseType="Resource">
        <vCard:Family>FAMILY_NAME</vCard:Family>
        <vCard:Given>GIVEN_NAME</vCard:Given>
      </vCard:N>
      +++
      [<vCard:EMAIL>EMAIL_ADDRESS</vCard:EMAIL>]
      +++
      [<vCard:ORG rdf:parseType="Resource" >
        <vCard:Orgname>ORGANIZATION_NAME</vCard:Orgname>
      </vCard:ORG>]
      +++
      ]]
    </rdf:li>
    ...
  </rdf:Bag>
</dc:creator>
<dcterms:created rdf:parseType="Resource">
  <dcterms:W3CDTF>DATE</dcterms:W3CDTF>
</dcterms:created>
<dcterms:modified rdf:parseType="Resource">
  <dcterms:W3CDTF>DATE</dcterms:W3CDTF>
</dcterms:modified>
...
\end{example}

The order of elements is as shown above, except that elements of
the format contained between \texttt{[[} and \texttt{]]} can occur
in any order.  The capitalized identifiers refer to generic
strings of a particular type: \texttt{FAMILY\_NAME} is the family
name of a person who created the model; \texttt{GIVEN\_NAME} is
the first name of the same person who created the model;
\texttt{EMAIL\_ADDRESS} is the email address of the same person
who created the model; and \texttt{ORGANIZATION\_NAME} is the name
of the organization with which the same person who created the
model is affiliated \texttt{DATE} is a date in W3C date
format~\citep{wolf:1998}. \texttt{W3CDTF}, \texttt{N},
\texttt{ORG} and \texttt{EMAIL} are literal strings. The elements
of the format contained between \texttt{[} and \texttt{]} are
optional. `\texttt{+++}' is a placeholder for either no content or
valid XML syntax that is not defined by this scheme but is
consistent with the relevant standards for the enclosing elements.
Ellipses (`\texttt{...}') are placeholders for zero or more
elements of the same form as the immediately preceding element.
The precise form of whitespace and the XML namespace prefix
definitions is not constrained.  The remaining text in this
section describes the syntax formally in English.

The additional elements of the history sub-format consist in
sequence of a \token{dc:creator} element, a
\token{dcterms:created} element and zero or more
\token{dcterms:modified} elements.  The last two elements must have
the attribute \token{rdf:parseType} set to \texttt{Resource}.

The \token{dc:creator} element describes the person who created
the SBML encoding of the model and contains a single
\token{rdf:Bag} element.  The \token{rdf:Bag} element can contain
any number of elements; however, the first element must be a
\token{rdf:li} element.  The \token{rdf:li} element can contain
any number of elements in any order.  The set of elements
contained with the \token{rdf:li} element can include the
following informative elements: \token{vCard:N},
\token{vCard:EMAIL} and \token{vCard:ORG}.  The \token{vCard:N}
contains the name of the creator and must consist of a sequence of
two elements: \token{vCard:Family} and the \token{vCard:Given}
whose content is the family (surname) and given (first) names of
the creator respectively.  The \token{vCard:N} must have the
attribute \token{rdf:parseType} set to \texttt{Resource}.  The
content of the \token{vCard:EMAIL} element must be the email
address of the creator.  The content of the \token{vCard:ORG}
element must contain a single \token{vCard:Orgname} element.  The
\token{vCard:Orgname} element must contain the name of an
organization to which the creator is affiliated.

The \token{dcterms:created} and \token{dcterms:modified} elements
must each contain a single \token{dcterms:W3CDTF} element whose
content is a date in W3C date format~\citep{wolf:1998} which is a
a profile of (restricted form of) ISO 8601.


%================================================================
\subsection{Examples}
%=================================================================

The following shows the annotation of a model with model creation
data and links to external resources:

\begin{example}
<model metaid="_180340" id="GMO" name="Goldbeter1991_MinMitOscil">
    <annotation>
        <rdf:RDF
                xmlns:rdf="http://www.w3.org/1999/02/22-rdf-syntax-ns\#"
                xmlns:dc="http://purl.org/dc/elements/1.1/"
                xmlns:dcterms="http://purl.org/dc/terms/"
                xmlns:vCard="http://www.w3.org/2001/vcard-rdf/3.0\#"
                xmlns:bqbiol="http://biomodels.net/biology-qualifiers/"
                xmlns:bqmodel="http://biomodels.net/model-qualifiers/"
        >
            <rdf:Description rdf:about="#_180340">
                <dc:creator>
                    <rdf:Bag>
                        <rdf:li rdf:parseType="Resource">
                            <vCard:N rdf:parseType="Resource">
                                <vCard:Family>Shapiro</vCard:Family>
                                <vCard:Given>Bruce</vCard:Given>
                            </vCard:N>
                            <vCard:EMAIL>bshapiro@jpl.nasa.gov</vCard:EMAIL>
                            <vCard:ORG rdf:parseType="Resource">
                                <vCard:Orgname>NASA Jet Propulsion Laboratory</vCard:Orgname>
                            </vCard:ORG>
                        </rdf:li>
                    </rdf:Bag>
                </dc:creator>
                <dcterms:created rdf:parseType="Resource">
                    <dcterms:W3CDTF>2005-02-06T23:39:40+00:00</dcterms:W3CDTF>
                </dcterms:created>
                <dcterms:modified rdf:parseType="Resource">
                    <dcterms:W3CDTF>2005-09-13T13:24:56+00:00</dcterms:W3CDTF>
                </dcterms:modified>
                <bqmodel:is>
                    <rdf:Bag>
                        <rdf:li rdf:resource="urn:miriam:biomodels.db:BIOMD0000000003"/>
                    </rdf:Bag>
                </bqmodel:is>
                <bqmodel:isDescribedBy>
                     <rdf:Bag>
                         <rdf:li rdf:resource="urn:miriam:pubmed:1833774"/>
                     </rdf:Bag>
                </bqmodel:isDescribedBy>
                <bqbiol:isVersionOf>
                    <rdf:Bag>
                        <rdf:li rdf:resource="urn:miriam:kegg.pathway:hsa04110"/>
                        <rdf:li rdf:resource="urn:miriam:reactome:REACT_152"/>
                    </rdf:Bag>
                </bqbiol:isVersionOf>
        </rdf:Description>
    </rdf:RDF>
</annotation>
\end{example}


The following example shows a \Reaction structure annotated
with a reference to its exact Reactome counterpart.

\begin{example}
<reaction id="cdc2Phospho" metaid="jb007" reversible="true" fast="false">
  <annotation>
    <rdf:RDF
      xmlns:bqbiol="http://biomodels.net/biology-qualifiers/"
      xmlns:rdf="http://www.w3.org/1999/02/22-rdf-syntax-ns\#"
    >
      <rdf:Description rdf:about="#jb007">
        <bqbiol:is>
          <rdf:Bag>
            <rdf:li rdf:resource="urn:miriam:reactome:REACT_6327"/>
          </rdf:Bag>
        </bqbiol:is>
      </rdf:Description>
    </rdf:RDF>
  </annotation>
  <listOfReactants>
    <speciesReference species="cdc2" stoichiometry="1"/>
  </listOfReactants>
  <listOfProducts>
    <speciesReference species="cdc2-Y15P" stoichiometry="1"/>
  </listOfProducts>
  <listOfModifiers>
    <modifierSpeciesReference species="wee1"/>
  </listOfModifiers>
</reaction>
\end{example}

The following example describes a species that represents a
complex between the protein calmodulin and calcium ions:

\begin{example}
<species id="Ca_calmodulin" metaid="cacam" compartment="C"
         hasOnlySubstanceUnits="false" boundaryCondition="false"
         constant="false">
  <annotation>
    <rdf:RDF
      xmlns:rdf="http://www.w3.org/1999/02/22-rdf-syntax-ns\#"
      xmlns:bqbiol="http://biomodels.net/biology-qualifiers/"
    >
      <rdf:Description rdf:about="\#cacam">
        <bqbiol:hasPart>
          <rdf:Bag>
            <rdf:li rdf:resource="urn:miriam:uniprot:P62158"/>
            <rdf:li rdf:resource="urn:miriam:kegg.compound:C00076"/>
          </rdf:Bag>
        </bqbiol:hasPart>
      </rdf:Description>
    </rdf:RDF>
  </annotation>
</species>
\end{example}

The following example describes a species that represents either
``Calcium/calmodulin-dependent protein kinase type II alpha
chain'' or ``Calcium/calmodulin-dependent protein kinase type II
beta chain''. This is the case, for example, in the somatic
cytoplasm of striatal medium-size spiny neurons, where both are
present but they cannot be functionally differentiated.


\begin{example}
<species id="calcium_calmodulin" metaid="cacam" compartment="C"
         hasOnlySubstanceUnits="false" boundaryCondition="false"
         constant="false">
  <annotation>
    <rdf:RDF
      xmlns:rdf="http://www.w3.org/1999/02/22-rdf-syntax-ns\#"
      xmlns:bqbiol="http://biomodels.net/biology-qualifiers/"
    >
      <rdf:Description rdf:about="\#cacam">
        <bqbiol:hasVersion>
          <rdf:Bag>
            <rdf:li rdf:resource="urn:miriam:uniprot:Q9UQM7"/>
            <rdf:li rdf:resource="urn:miriam:uniprot:Q13554"/>
          </rdf:Bag>
        </bqbiol:hasVersion>
      </rdf:Description>
    </rdf:RDF>
  </annotation>
</species>
\end{example}


The above approach should not be used to describe ``any
Calcium/calmodulin-dependent protein kinase type II chain'',
because such an annotation requires referencing the products of
other genes such as gamma or delta. All the known proteins could
be enumerated, but such an approach would almost surely lead to
inaccuracies because biological knowledge continues to evolve.
Instead, the annotation should refer to generic information such
as Ensembl family ENSF00000000194 ``CALCIUM/CALMODULIN DEPENDENT
KINASE TYPE II CHAIN'' or PIR superfamily PIRSF000594
``Calcium/calmodulin-dependent protein kinase type~II''.

%While with \texttt{HasVersion}, the described component could
%represent several alternative, with \texttt{isVersionOf} the
%described component is one of the alternative understated by the
%referenced resource.

The following two examples show how to use the qualifier
\token{isVersionOf}. The first example is the relationship
between a reaction and an EC code. An EC code describes an
enzymatic activity and an enzymatic reaction involving a
particular enzyme can be seen as an instance of this activity. For
example, the following reaction represents the phosphorylation of
a glutamate receptor by a complex calcium/calmodulin kinase II.

\begin{example}
<reaction id="NMDAR_phosphorylation" metaid="thx1138"
          reversible="true" fast="false">
  <annotation>
    <rdf:RDF
      xmlns:bqbiol="http://biomodels.net/biology-qualifiers/"
      xmlns:rdf="http://www.w3.org/1999/02/22-rdf-syntax-ns\#"
    >
      <rdf:Description rdf:about="#thx1138">
        <bqbiol:isVersionOf>
          <rdf:Bag>
            <rdf:li rdf:resource="urn:miriam:ec-code:2.7.1.17"/>
          </rdf:Bag>
        </bqbiol:isVersionOf>
      </rdf:Description>
    </rdf:RDF>
  </annotation>
  <listOfReactants>
    <speciesReference species="NMDAR" stoichiometry="1"/>
  </listOfReactants>
  <listOfProducts>
    <speciesReference species="P-NMDAR" stoichiometry="1"/>
  </listOfProducts>
  <listOfModifiers>
    <modifierSpeciesReference species="CaMKII"/>
  </listOfModifiers>
  <kineticLaw>
    <math xmlns="http://www.w3.org/1998/Math/MathML">
      <apply>
        <times/>
        <ci>CaMKII</ci>
        <ci>kcat</ci>
        <apply>
          <divide/>
          <ci>NMDAR</ci>
          <apply> </times> <ci>NMDAR</ci> <ci>Km</ci> </apply>
        </apply>
      </apply>
    </math>
    <listOfLocalParameters>
      <localParameter id="kcat" value="1"/>
      <localParameter id="Km" value="5e-10"/>
    </listOfLocalParameters>
  </kineticLaw>
</reaction>
\end{example}

The second example of the use of \token{isVersionOf} is the
complex between Calcium/calmodulin-dependent protein kinase type
II alpha chain and Calcium/calmodulin, that is only one of the
``calcium- and calmodulin-dependent protein kinase complexes''
described by the Gene Ontology term GO:0005954.  (Note also how
the GO identifier is written---we return to this point below.)

\begin{example}
<species id="CaCaMKII" metaid="C8H10N4O2" compartment="C"
         hasOnlySubstanceUnits="false" boundaryCondition="false"
         constant="false">>
  <annotation>
    <rdf:RDF
      xmlns:rdf="http://www.w3.org/1999/02/22-rdf-syntax-ns\#"
      xmlns:bqbiol="http://biomodels.net/biology-qualifiers/"
    >
      <rdf:Description rdf:about="\#C8H10N4O2">
        <bqbiol:isVersionOf>
          <rdf:Bag>
            <rdf:li rdf:resource="urn:miriam:obo.go:GO\%3A0005954"/>
          </rdf:Bag>
        </bqbiol:isVersionOf>
      </rdf:Description>
    </rdf:RDF>
  </annotation>
</species>
\end{example}

In the example above, the URN for the GO term is written as
\token{GO\%3A0005954}, yet in reality, the actual GO identifier is
\token{GO:0005954}.  The reason for this rests in the definition
of RDF/XML and URNs.  The essential point is that the colon
character (\val{:}) is a reserved character representing the
component separator in URNs.  Thus, when an identifier contains a
colon character as part of it (as GO, ChEBI, and certain other
identifiers do), the colon characters must be percent-encoded.
The sequence \val{\%3A} is the percent-encoded form of \val{:}.
Applications must percent-encode \val{:} characters that appear in
entity identifiers (whether from ECO, ChEBI, GO, or other) when
writing them in MIRIAM URIs, and percent-decode the identifiers
when reading the URIs.  More examples of this appear throughout
the rest of this section.

The previous case is different from the following one, although they
could seem similar at first sight. The
``Calcium/calmodulin-dependent protein kinase type II alpha
chain'' is a part of the above mentioned ``calcium- and
calmodulin-dependent protein kinase complex''.

\begin{example}
<species id="CaMKIIalpha" metaid="C10H14N2" compartment="C"
         hasOnlySubstanceUnits="false" boundaryCondition="false"
         constant="false">>
  <annotation>
    <rdf:RDF
      xmlns:rdf="http://www.w3.org/1999/02/22-rdf-syntax-ns\#"
      xmlns:bqbiol="http://biomodels.net/biology-qualifiers/"
    >
      <rdf:Description rdf:about="\#C10H14N2">
        <bqbiol:isPartOf>
          <rdf:Bag>
            <rdf:li rdf:resource="urn:miriam:obo.go:GO\%3A0005954"/>
          </rdf:Bag>
        </bqbiol:isPartOf>
      </rdf:Description>
    </rdf:RDF>
  </annotation>
</species>
\end{example}

It is possible describe a component with several alternative sets
of qualified annotations. For example, the following species
represents a pool of  GMP, GDP and GTP. We annotate it with the
three corresponding KEGG compound identifiers but also with the
three corresponding ChEBI identifiers.  The two alternative
annotations are encoded in separate \token{hasVersion} qualifier
elements.

\begin{example}
<species id="GXP" metaid="GXP" compartment="C"
         hasOnlySubstanceUnits="false" boundaryCondition="false"
         constant="false">>
  <annotation>
    <rdf:RDF
      xmlns:rdf="http://www.w3.org/1999/02/22-rdf-syntax-ns\#"
      xmlns:bqbiol="http://biomodels.net/biology-qualifiers/"
    >
      <rdf:Description rdf:about="\#GXP">
        <bqbiol:hasVersion>
          <rdf:Bag>
            <rdf:li rdf:resource="urn:miriam:obo.chebi:CHEBI\%3A17345"/>
            <rdf:li rdf:resource="urn:miriam:obo.chebi:CHEBI\%3A17552"/>
            <rdf:li rdf:resource="urn:miriam:obo.chebi:CHEBI\%3A17627"/>
          </rdf:Bag>
        </bqbiol:hasVersion>
        <bqbiol:hasVersion>
          <rdf:Bag>
            <rdf:li rdf:resource="urn:miriam:kegg.compound:C00035"/>
            <rdf:li rdf:resource="urn:miriam:kegg.compound:C00044"/>
            <rdf:li rdf:resource="urn:miriam:kegg.compound:C00144"/>
          </rdf:Bag>
        </bqbiol:hasVersion>
      </rdf:Description>
    </rdf:RDF>
  </annotation>
</species>
\end{example}

The following example presents a reaction being actually the
combination of three different elementary molecular reactions. We
annotate it with the three corresponding KEGG reactions, but also
with the three corresponding enzymatic activities.  Again the two
\token{hasPart} elements represent two alternative annotations.
The process represented by the \Reaction structure is
composed of three parts, and  not six parts.

\begin{example}
<reaction id="adenineProd" metaid="adeprod" reversible="true" fast="false">
  <annotation>
    <rdf:RDF
      xmlns:bqbiol="http://biomodels.net/biology-qualifiers/"
      xmlns:rdf="http://www.w3.org/1999/02/22-rdf-syntax-ns\#"
    >
      <rdf:Description rdf:about="\#adeprod">
        <bqbiol:hasPart>
          <rdf:Bag>
            <rdf:li rdf:resource="urn:miriam:ec-code:2.5.1.22"/>
            <rdf:li rdf:resource="urn:miriam:ec-code:3.2.2.16"/>
            <rdf:li rdf:resource="urn:miriam:ec-code:4.1.1.50"/>
          </rdf:Bag>
        </bqbiol:hasPart>
        <bqbiol:hasPart>
          <rdf:Bag>
            <rdf:li rdf:resource="urn:miriam:kegg.reaction:R00178"/>
            <rdf:li rdf:resource="urn:miriam:kegg.reaction:R01401"/>
            <rdf:li rdf:resource="urn:miriam:kegg.reaction:R02869"/>
          </rdf:Bag>
        </bqbiol:hasPart>
      </rdf:Description>
    </rdf:RDF>
  </annotation>
</reaction>
\end{example}

It is possible to mix different URIs in a given set. The
following example presents two alternative annotations of the human
hemoglobin, the first with ChEBI heme and the second with KEGG
heme.

\begin{example}
<species id="heme" metaid="heme" compartment="C"
         hasOnlySubstanceUnits="false" boundaryCondition="false"
         constant="false">
  <annotation>
    <rdf:RDF
      xmlns:rdf="http://www.w3.org/1999/02/22-rdf-syntax-ns\#"
      xmlns:bqbiol="http://biomodels.net/biology-qualifiers/"
    >
     <rdf:Description rdf:about="\#heme">
       <bqbiol:hasPart>
         <rdf:Bag>
           <rdf:li rdf:resource="urn:miriam:uniprot:P69905"/>
           <rdf:li rdf:resource="urn:miriam:uniprot:P68871"/>
           <rdf:li rdf:resource="urn:miriam:obo.chebi:CHEBI\%3A17627">
         </rdf:Bag>
       </bqbiol:hasPart>
       <bqbiol:hasPart>
         <rdf:Bag>
          <rdf:li rdf:resource="urn:miriam:uniprot:P69905"/>
           <rdf:li rdf:resource="urn:miriam:uniprot:P68871"/>
           <rdf:li rdf:resource="urn:miriam:kegg.compound:C00032"/>
         </rdf:Bag>
       </bqbiol:hasPart>
     </rdf:Description>
   </rdf:RDF>
  </annotation>
</species>
\end{example}

As formally defined above it is possible to use different
qualifiers in the same annotation element. The following
phosphorylation is annotated by its exact KEGG counterpart and by
the generic GO term ``phosphorylation''.

\begin{example}
<reaction id="phosphorylation" metaid="phosphorylation"
          reversible="true" fast="false">
  <annotation>
    <rdf:RDF
      xmlns:bqbiol="http://biomodels.net/biology-qualifiers/"
      xmlns:rdf="http://www.w3.org/1999/02/22-rdf-syntax-ns\#"
    >
      <rdf:Description rdf:about="\#phosphorylation">
        <bqbiol:is>
          <rdf:Bag>
            <rdf:li rdf:resource="urn:miriam:kegg.reaction:R03313" />
          </rdf:Bag>
        </bqbiol:is>
        <bqbiol:isVersionOf>
          <rdf:Bag>
            <rdf:li rdf:resource="urn:miriam:obo.go:GO\%3A0016310" />
          </rdf:Bag>
        </bqbiol:isVersionOf>
      </rdf:Description>
    </rdf:RDF>
  </annotation>
</reaction>
\end{example}
