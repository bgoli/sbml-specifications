% -*- TeX-master: "sbml-level-3-version-1-core"; fill-column: 66 -*-
% $Id$
% $HeadURL: https://sbml.svn.sourceforge.net/svnroot/sbml/trunk/specifications/sbml-level-3/version-1/core/spec/best-practice.tex $
% ----------------------------------------------------------------

\section{Recommended practices}
\label{sec:best-practices}

In this section, we recommend a number of practices for using and
interpreting various SBML constructs.  These recommendations are
non-normative, but we advocate them strongly; ignoring them will
not render a model invalid, but may hinder interoperability
between different software systems exchanging SBML content.


\subsection{Recommended practices concerning common SBML
  attributes and objects}
\label{sec:bp:common}

Many SBML components share some or all of the following attributes
and objects.  We describe recommendations concerning them here,
separately from discussing the specific SBML components.  In
Section~\ref{sec:bp:specifics}, we turn to the specific SBML
components, but the recommendations described here also apply to
them.


\subsubsection{Identifiers and names}
\label{sec:bp:names}

The \token{id} attribute is available on most (but not all)
objects in SBML, and all objects that have \token{id} attributes
also have an optional \token{name} attribute.  How should models
treat identifiers and names?

The following is the recommended practice for handling
\token{name}.  If a software tool has the capability to
display the content of \token{name} attributes, it should
display this content to the user as a component's label instead of
the component's \token{id}.  If the user interface does not have
this capability (e.g., because it cannot display or use special
characters in symbol names), or if the \token{name} attribute is
missing on a given component, then the user interface should
display the value of the \token{id} attribute instead.  (Script
language interpreters are especially likely to display \token{id}
instead of \token{name}.)

As a consequence of the above, authors of software systems that
automatically generate values for \token{id} attributes should be
aware some other systems may display the \token{id}'s to the user.
Authors therefore may wish to take some care to have their
software create \token{id} values that are: (a) reasonably easy
for humans to type and read, and (b) likely to be meaningful (\eg
by making the \token{id} attribute is an abbreviated form of the
\token{name} attribute value).

\subsubsection{Initial Values}
\label{sec:bp:initialvalues}

SBML allows for the creation of \Compartment, \Species,
\Parameter, \LocalParameter and \SpeciesReference objects without
declaring their initial values directly on the object instances.
That is, a \Compartment object can be created without defining a
value for its \token{size} attribute; a \Species object can be
created without defining a value for either its
\token{initialConcentration} or \token{initialAmount} attribute;
\Parameter and \LocalParameter objects can be created without
giving a value to their \token{value} attributes; and a
\SpeciesReference object can be created without assigning a value
to its \token{stoichiometry} attribute.  A missing value in the
case of \Compartment, \Species, \Parameter, and \SpeciesReference
objects implies that the value is either set via an
\InitialAssignment object elsewhere in the model, or is meant to
be obtained from an external source (\eg by querying the user of a
software system), or is unknown.  In the case of \LocalParameter
objects, a missing value implies that the value is either unknown
or meant to be obtained from an external source.

Where initial values are available and are decimal numbers that
\emph{can} be set using the appropriate attribute on an object,
the best practice recommendation is to do that in preference to
using an \InitialAssignment construct if there is no particular
reason to use \InitialAssignment.  Setting the relevant attribute
directly on the \Compartment, \Species, and \Parameter and
\SpeciesReference object is simpler and may be more interoperable
with different software systems.  This is especially true of
\token{stoichiometry} on \SpeciesReference, which in the vast
majority of models, is never more than a scalar constant value
anyway.

\question{MH: new paragraph}
An additional point is worth noting in passing.  Although the
value attributes of various SBML components are of type
\primtype{double} (\eg \Parameter's attribute \token{value}), this
does not mean that component values are limited only to decimal
numbers.  As noted above, other constructs such as
\InitialAssignment can be used to set the value of an object, and
since those constructs offer the power of MathML, the results may
be rational numbers such as fractions.  Software developers should
be aware of this possibility when planning the type of storage
variables used to hold SBML objects' values.


\subsubsection{The \token{constant} flag}
\label{sec:bp:constant}

There is a mandatory boolean attribute called \token{constant} on
the \Compartment, \Species, \SpeciesReference and
\Parameter components.  A value of \val{true} means that the SBML
object in question will not be changed by other constructs in SBML
(except possibly an \InitialAssignment).  A value of \val{false}
indicates an intention to change the element's value by an
\AssignmentRule, \RateRule, \AlgebraicRule, \Reaction or \Event in
the model.

A \token{constant} attribute value of \val{false} does not
\emph{require} that the object in question is changed; strictly
speaking, an SBML model is valid even if it sets all
\token{constant} attributes to \val{false} but never actually
modify any of the values.  However, the best practice
recommendation is to communicate intentions by setting
\token{constant} to \val{true} unless an entity in a model really
is intended to be changed.  The exception to this is \Species,
which are usually part of the reaction system and thus usually
need to have \token{constant}=\val{false}.


\subsubsection{Annotations}
\label{sec:bp:annotations}

\paragraph{Appropriate uses of annotations}

In the description of the \Annotation object available on every
component derived from \SBase (Section~\ref{sec:annotations}), we
already made the point that it is critical not to put data
essential to understanding a model into annotations.  This raises
a question: what kind of data may be appropriately put into
annotations?

Here are examples of the kinds of data that may be appropriately
stored in \Annotation objects:
\begin{itemize}

\item Identification information for cross-referencing components
  in a model with items in a data resource such as a database.
  This is the purpose of the annotation scheme described in
  Section~\ref{sec:annotation-standard}.

\item Application-specific processing instructions that do not
  change the essential meaning of a model, but help a particular
  application with tasks such as managing the model, maintaining
  state data across editing sessions, etc.

\item Evidence codes for annotating a model with controlled
  vocabulary terms that indicate (e.g.)\ the quality of biological
  evidence supporting the inclusion of each component in the
  model.  The annotation scheme of
  Section~\ref{sec:annotation-standard} can be used in this
  capacity.

\item Information about the model that cannot be readily encoded
  in existing SBML elements, but that does not alter the
  mathematical meaning of the model.

\end{itemize}


\paragraph{Specificity of annotations}

The annotation data (Section~\ref{sec:annotations}) attached to a
specific SBML object in a model is assumed by other applications
to be directly associated with that particular SBML object.
Therefore, it is important to decompose and locate annotation data
appropriately in an SBML document.  Applications are advised to
avoid encoding all their annotations in a single top-level
attribute on (e.g.) the \Model object.  The data associated with,
for example, an individual \Species object in a model should be
encoded in the \token{<annotation>} element enclosed within the
SBML \token{<species>} element representing that species in the
SBML file.


\paragraph{Syntax of annotations}

The annotation scheme described in
Section~\ref{sec:annotation-standard} is useful for many, but not
all, situations.  It is tempting to develop new annotation
syntaxes for situations that fall outside the scope of the SBML
MIRIAM annotation scheme.  However, a proliferation of
proprietary annotation schemes will hinder software
interoperability in the long run.

We recommend the following approach when faced with a need to use
alternate annotation syntaxes:
\begin{enumerate}

\item The modular nature of SBML \thisLV means that data that in
  SBML Level~2 could only be stored in annotations may now be
  supported using a full SBML Level~3 package.  Therefore,
  software developers and modelers should first check if there
  already exists a package that may serve their needs.  A list of
  SBML Level~3 packages is always maintained at the SBML website,
  \url{http://sbml.org}.
  
\item If no package exists, developers and modelers may wish to
  check if someone else has already developed a similar annotation
  syntax for use with another software system.  A list of known
  SBML annotation schemes is maintained online at 
  \url{http://sbml.org/Community/Wiki/Known_SBML_annotations}.

\item If none of the above alternatives provide a satisfactory
  result, developers and modelers should query the SBML discussion
  list (\link{http://sbml.org/Forums}{sbml-discuss@caltech.edu})
  to see if anyone else has been faced with similar problems.
  Other SBML users may have insights or even partial solutions
  already available.

\end{enumerate}


\subsection{Recommended practices concerning specific SBML components}
\label{sec:bp:specifics}

In this section, we describe expectations and recommendations
concerning specific SBML components.  We do not reiterate the
recommendations presented in Section~\ref{sec:bp:common}, but they
apply to many of the SBML components discussed here and should be
kept in mind.  The order of the components discussed here follows
the order of their presentation in Section~\ref{sec:elements}, but
we only include here those components for which we have specific
recommendations.


\subsubsection{Unit definitions}
\label{sec:bp:unitdefinitions}

We advise modelers and software tools to declare the units of all
quantities in a model, insofar as this is possible, using the
various mechanisms provided for this in SBML.  Fully declared
units can allow software tools to perform dimensional analysis on
the units of mathematical expressions, and such analysis can be
valuable in helping modelers produce correct models.  In addition,
it can allow model-wide operations such as conversion or rescaling
of units.


\paragraph{Recommendations for choices of units}
\label{sec:bp:unitdefinitions:recommendedunits}

Table~\ref{tab:recommended-units} lists the units recommended for
different SBML components.

\begin{table}[thb]
  \vspace*{2ex}
  \small
  \centering
  \begin{tabular}{p{1.55in}p{4.5in}}
    \toprule
    \textbf{Component or attribute} & \textbf{Unit recommendations} \\
    \midrule
    \Model \token{substanceUnits}
    &
    \token{mole}, \token{item}, \token{avogadro},
    \token{dimensionless}, \token{kilogram}, \token{gram}, or units
    derived from these
    \\[10pt]
    \Model \token{timeUnits}
    & 
    \token{second}, \token{dimensionless}, or units derived from
    these
    \\[10pt]
    \Model \token{volumeUnits}
    &
    \token{litre}, \token{metre}$^3$, \token{dimensionless}, or
    units derived from these
    \\[10pt]
    \Model \token{areaUnits}
    &
    \token{metre}$^2$, \token{dimensionless}, or units derived from
    these
    \\[10pt]
    \Model \token{lengthUnits}
    &
    \token{metre}, \token{dimensionless}, or units derived from these
    \\[10pt]
    \Model \token{extentUnits}
    &
    \token{mole}, \token{item}, \token{avogadro},
    \token{dimensionless}, or units derived from these
    \\[10pt]
    \Compartment \token{units}
    &
    \begin{minipage}{3.22in}
      \small
      \begin{tabular}{@{}cp{3.19in}@{}}
        \toprule
        \textbf{Value of attribute} \\
        \token{spatialDimensions}   & \textbf{Recommended units}\\
        \midrule
        \val{3}
        & 
        \token{litre}, \token{metre}$^3$,
        \token{dimensionless}, or units derived from these
        \\[5pt]
        \val{2}
        &
        \token{metre}$^2$, \token{dimensionless}, or units
        derived from these
        \\[5pt]
        \val{1}
        & \token{metre}, \token{dimensionless}, or units derived
        from these
        \\[5pt] 
        other
        &
        \emph{no specific recommendations}\\
        \bottomrule
      \end{tabular}
    \end{minipage}
    \\
    \\
    \Species \token{substanceUnits}
    &
    \token{mole}, \token{item}, \token{avogadro},
    \token{dimensionless}, \token{kilogram}, \token{gram}, or units
    derived from these
    \\[10pt]
    \Parameter \token{units}
    &
    \emph{no specific recommendations}
    \\
    \bottomrule
  \end{tabular}
  \caption{Units recommended for use on different SBML model components.}
  \label{tab:recommended-units}
\end{table}


\paragraph{Handling units requiring the use of offsets}
\label{sec:bp:unitdefinitions:offset}

Unit definitions and conversions requiring offsets cannot be done
using the simple approach described in
Section~\ref{sec:unit-simple-approach}.  In fact, SBML does not
predefine a unit for Celsius because it would require the use of
an offset.  Definitions involving Celsius, Fahrenheight or other
units with offsets require a different approach.

We discuss approaches to handling units with offsets, starting
with the case of degrees Celsius:
\begin{itemize}

\item \emph{Handling Celsius}.  A model in which certain
  quantities are temperatures measured in degrees Celsius can be
  converted straightforwardly to a model in which those
  temperatures are in kelvin.  A software tool could do this by
  performing a substitution using the following relationship:
  \begin{linenomath}
    \begin{equation*}
      T_\emph{kelvin} = T_\emph{Celsius} + 273.15
    \end{equation*}
  \end{linenomath}
  In every mathematical formula of the model where a quantity
  (call it $x$) in degrees Celsius appears, replace $x$ with $x_k
  + 273.15$ where $x_k$ is now in kelvin.  An alternative approach
  would be to use a \FunctionDefinition to define a function
  encapsulating this relationship above and then using that in the
  rest of the model as needed.  Since Celsius is a commonly-used
  unit, software tools could help users by providing users with
  the ability to express temperatures in Celsius in the tools'
  interfaces, and making substitutions automatically when writing
  out SBML.

\item \emph{Handling other units requiring offsets}.  The only
  other units requiring offsets in SBML's domain of common
  applications are other temperature units such as Fahrenheit.
  Few modern scientists employ Fahrenheit degrees; therefore, this
  is an unusual situation.  The complication inherent in
  converting between degrees Fahrenheit and kelvin is that both a
  multiplier and an offset are required:
  \begin{linenomath}
    \begin{equation*}
      T_\emph{kelvin} = \frac{T_\emph{F} + 459.67}{1.8}
      \label{eq:fah-kelvin}
    \end{equation*}
  \end{linenomath}

  One approach to handling this is to use a \FunctionDefinition to
  define a function encapsulating the relationship above, then to
  substitute a call to this function wherever the original
  temperature in Fahrenheit appears in the model's mathematical
  formulas.  Here is a candidate definition as an example:
  \begin{example}
<functionDefinition id="Fahrenheit_to_kelvin">
    <math xmlns="http://www.w3.org/1998/Math/MathML"
          xmlns:sbml="http://www.sbml.org/sbml/level3/version1/core">
        <lambda>
            <bvar><ci> temp_in_fahrenheit </ci></bvar>
            <apply>
                <divide/>
                <apply>
                    <plus/>
                    <ci> temp_in_fahrenheit </ci>
                    <cn> 459.67 </cn>
                </apply>
                <cn> 1.8 </cn>
            </apply>
        </lambda>
    </math>
</functionDefinition>
  \end{example}\vspace*{-2ex}

  An alternative approach not requiring the use of function
  definitions is to use an \AssignmentRule for each variable in
  Fahrenheit units.  The \AssignmentRule could compute the
  conversion from Fahrenheit to (say) kelvin, assign its value to
  a variable (with units declared to be \val{kelvin}), and then
  that variable could be used elsewhere in the model.  Still
  another approach is to rewrite the mathematical formulas of a
  model to directly incorporate the conversion above wherever the
  quantity appears.

\end{itemize}


All of these approaches provide general solutions to the problem
of supporting any units requiring offsets in the unit system of
SBML Level~3.  It can be used for other temperature units
requiring an offset (\eg degrees Rankine or degrees R\'{e}aumur),
although the likelihood of a real-life model requiring such other
temperature units seems exceedingly small.

In summary, the fact that SBML units do not support specifying an
offset does not impede the creation of models using alternative
units.  If conversions are needed, then converting between
temperature in degrees Celsius and thermodynamic temperature can
be handled rather easily by the simple substitution described
above.  For the rare case of Fahrenheit and other units requiring
combinations of multipliers and offsets, users are encouraged to
employ the power of \FunctionDefinition, \AssignmentRule, or other
constructs in SBML.



\subsubsection{Compartments}
\label{sec:bp:compartment}


\paragraph{Setting the \token{size} attribute on a \token{compartment}}
\label{sec:bp:size}

As mentioned in Section~\ref{sec:size}, we highly recommend that
every \Compartment object in a model has its \token{size} set.
There are three major technical reasons for this.  First, if the
model contains any species whose initial amounts are given in
terms of concentrations, and there is at least one reaction in the
model referencing such a species, then the model will be
numerically incomplete if it lacks a value for the size of the
compartment in which the species is located.  The reason is that
SBML reactions are expected to be in terms of intensive properties
such as \emph{amount}/\emph{time} (or more generally, \emph{extent
  units}/\emph{time units}; see
Section~\ref{sec:constructing-equations}), and converting from
concentration to amount requires knowing the compartment size.
Second, models ideally should be instantiable in a variety of
simulation frameworks.  A commonly-used one is the discrete
stochastic framework~\citep{gillespie:1977,wilkinson_2006} in
which species are represented as item counts (\eg molecule
counts).  If species' initial quantities are given in terms of
concentrations or densities, it is impossible to convert the
values to item counts without knowing compartment sizes.  Third,
if a model contains multiple compartments whose sizes are not all
identical to each other, it is impossible to quantify the reaction
rate expressions without knowing the compartment volumes.  The
reason for the latter is again that reaction rates in SBML are
defined in terms \emph{extent}/\emph{time}, and when species
quantities are given in terms of concentrations or densities, the
compartment sizes usually become factors in the reaction rate
expressions.


\paragraph{Indicating a default compartment}

Some types of models do not use compartments, for example because
they factor out volumes completely.  Since SBML requires at least
one compartment to be defined if any species exists in a model, the
representation of models where no compartments are needed
sometimes leaves model creators wishing they could indicate that a
compartment is only a ``default'' in some sense.  The recommended
approach to handling this situation is to annotate the
\Compartment object by setting its \token{sboTerm} attribute to an
appropriate SBO term, specifically \val{SBO:0000410}.


\subsubsection{Rules}
\label{sec:bp:rules}

Section~\ref{sec:ruleconstraints} establishes the fact that when
\AlgebraicRule objects are used, it is possible to produce a model
that is overdetermined.  When a model includes both \Event and
\Reaction objects, it is necessary to analyze the set of equations
produced from the rules and reactions and the set of equations
produces from rules and the event assignments of each event.  Each
set of equations must not be overdetermined.  In addition, each
set of equations must be fully determined if accurate simulation
is to be performed.

The following example illustrates a case where the set of
equations is fully determined.  First, we present the SBML
expression of the model:

\sbmlexample{fullydeterminedevent.xml}

There are three species in the model above whose values may vary.
The first set of equations to consider is the set produced by the
\Reaction and the \AlgebraicRule objects:
\begin{linenomath}
\begin{align*}
  \frac{d [S_1]}{d t} &= - C \cdot k_1 \cdot [S_1] \\[2pt]
  \frac{d [S_2]}{d t} &= C \cdot k_1 \cdot [S_1]   \\[2pt]
  [S_1] - [S_3]       &= 0
\end{align*}
\end{linenomath}

This set of equations is fully determined, \ie each of the three
variables $S_1$, $S_2$ and $S_3$ are derived from one equation.
The second set of equations to consider is the set produced by the
\Event and the \AlgebraicRule objects:
\begin{linenomath}
\begin{align*}
  [S_1]         &= 1   \\[2pt]
  [S_2]         &= 1.5 \\[2pt]
  [S_1] - [S_3] &= 0
\end{align*}
\end{linenomath}

Again the set of equations is fully determined, but had the event
assignment for species $S_1$ been absent, the algebraic rule would
produce an ambiguity regarding which variable should be adjusted.

In this example, as is often the case when an \AlgebraicRule has
been used, the \AlgebraicRule could be replaced by an
\AssignmentRule:

\begin{example}
<assignmentRule variable="S3">
    <math xmlns="http://www.w3.org/1998/Math/MathML">
        <apply>
            <ci> S1 </ci>
        </apply>
    </math>
</assignmentRule>
\end{example}

Replacing \AlgebraicRule objects with \AssignmentRule objects,
particularly in models that use events, reduces the possibilities
for creating either overdetermined or ambiguous models and
produces models that can be exchanged with greater ease.


\subsubsection{Reactions}
\label{sec:bp:reactions}

% \todo{MH}{this needs work}

Consider a very simple model consisting of a single enzymatic reaction
$R$ that converts $S_{1}$ to $S_{2}$ for which a traditional kinetic
law $v_{R}$ is given:
\begin{center}
\ce{S_1 ->[v_R] S_2}
\end{center}
where
\begin{linenomath}
  \begin{equation*}
    v_{R} = \frac{v_\text{max} \cdot [S_{1}]}{K_{M} + [S_{1}]}
  \end{equation*}
\end{linenomath}

with $v_{R}$ and $v_\text{max}$ given in units of concentration
per time.

As mentioned above, when a rate law is presented in the
traditional way, it usually embodies (implicitly or explicitly)
several assumptions: that all species are located in the same
compartment, that the compartment size does not change, and that
the reaction takes place uniformly throughout the volume of the
compartment, i.e. the enzyme is not localized in any special way.
Under these circumstances it is possible to construct rate
equations for the \emph{concentration} of the species:
\begin{linenomath}
  \begin{equation*}
    \frac{d[S_{2}]}{dt} = -\frac{d[S_{1}]}{dt} = v_{R}
  \end{equation*}
\end{linenomath}
In SBML, however, the rate equations are constructed for the rate
of change of the \emph{amount} of the species:
\begin{linenomath}
  \begin{equation*}
    \frac{dn_{S_{2}}}{dt} = -\frac{dn_{S_{1}}}{dt} = \hat{v}_{R} = V \cdot v_{R}
  \end{equation*}
\end{linenomath}
where $\hat{v}_{R}$ is the modified SBML kinetic law and $V$ is
the volume of the compartment.  Since the traditional kinetic law
$v_{R}$ describes how fast the amount of the species changes
\emph{per volume}, the SBML kinetic law $\hat{v}_{R}$ simply
equals the product of $v_{R}$ and the compartment volume $V$. This
means that the actual rate of change of the amounts of the species
is proportional to the compartment size, which will only be true
if the reaction takes place uniformly throughout the compartment.
(See Section~\ref{sec:reaction-membrane-eg} for an example of a
reaction that is located at the boundary of a compartment.)  The
concentrations of the species (that are needed in the definition
of $v_{R}$) can easily be recovered through the relation
$[S_{i}]=n_{S_{i}}/V$.

An important property of the amount rate equation is that it is still
valid if the volume $V$ changes during a simulation. This is not
true for the concentration rate equations. 

