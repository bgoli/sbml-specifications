% -*- TeX-master: "sbml-level-3-version-1-core"; fill-column: 66 -*-
% $Id: best-practice.tex $
% $HeadURL: https://sbml.svn.sourceforge.net/svnroot/sbml/trunk/specifications/sbml-level-3/version-1/core/spec/best-practice.tex $
% ----------------------------------------------------------------

\section{Best practice}
\label{sec:best-practice}

This section gives further information regarding intended use and
best practice for application of the specification to systems
biological modelling. The information provided here is somewhat
distinct from the formal technical specification, but nevertheless
strongly advocated.


\subsection{Setting the \token{size} attribute on a \token{compartment}}
\label{sec:bp:size}

As mentioned in Section~\ref{sec:size}, it is highly recommended
that a \token{size} is specified for each \token{compartment}
declared in a model. There are three major technical reasons for
this.  First, if the model contains any species whose initial
amounts are given in terms of concentrations, and there is at
least one reaction in the model referencing such a species, then
the model is numerically incomplete if it lacks a value for the
size of the compartment in which the species is located.  The
reason is simply that SBML
\Reaction{}s are defined in units of
\quantity{substance}/\quantity{time} (see
Section~\ref{subsec:kinetic-law}), not concentration per time, and
thus the compartment size must at some point be used to convert
from species concentration to substance units.  Second, models
ideally should be instantiable in a variety of simulation
frameworks.  A commonly-used one is the discrete stochastic
framework~\citep{gillespie:1977,wilkinson_2006} in which species
are represented as item counts (\eg molecule counts).  If species'
initial quantities are given in terms of concentrations or
densities, it is impossible to convert the values to item counts
without knowing compartment sizes.  Third,
if a model contains multiple compartments whose sizes are not all
identical to each other, it is impossible to quantify the reaction
rate expressions without knowing the compartment volumes.  The
reason for the latter is again that reaction rates in SBML are defined
in terms of \quantity{substance}/\quantity{time}, and when species quantities are
given in terms of concentrations or densities, the compartment
sizes become factors in the reaction rate expressions.
