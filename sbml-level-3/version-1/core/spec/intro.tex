% -*- TeX-master: "sbml-level-3-version-1-core"; fill-column: 66 -*-
% $Id$
% $HeadURL$
% ----------------------------------------------------------------

\section{Introduction}
\label{sec:introduction}

This document defines \thisVnum of the \textbf{S}ystems
\textbf{B}iology \textbf{M}arkup \textbf{L}anguage (SBML) Level~3
Core, an electronic model representation format for systems
biology.  SBML is oriented towards describing biological processes
of the sort common in research on a number of topics, including
metabolic pathways, cell signaling pathways, and many others.
SBML is defined neutrally with respect to programming languages
and software encoding; however, it is oriented primarily towards
allowing models to be encoded using XML, the eXtensible Markup
Language~\citep{bray:2004}.  This document contains many examples
of SBML models written in XML.  Formal schemas describing the
syntax of SBML, as well as other materials and software, are
available for downloading from the SBML project web site,
\url{http://sbml.org/}.

The SBML project is not an attempt to define a universal language
for representing quantitative models.  The rapidly evolving views
of biological function, coupled with the vigorous rates at which
new computational techniques and individual tools are being
developed today, are incompatible with a one-size-fits-all idea of
a universal language. A more realistic alternative is to
acknowledge the diversity of approaches and methods being explored
by different software tool developers, and seek a common
intermediate format---a \emph{lingua franca}---enabling
communication of the most essential aspects of the models.

The definition of the model description language presented here
does not specify \emph{how} programs should communicate or
read/write SBML.  We assume that for a simulation program to
communicate a model encoded in SBML, the program will have to
translate its internal data structures to and from SBML, use a
suitable transmission medium and protocol, etc., but these issues
are outside of the scope of this document.


%-----------------------------------------------------------------------------
\subsection{Developments, discussions, and notifications of updates}
%-----------------------------------------------------------------------------

SBML has been, and continues to be, developed in collaboration
with an international community of researchers and software
developers.  As in many projects, the primary mode of interaction
between members is electronic mail.  Discussions about SBML take
place on the mailing list
\link{http://sbml.org/Forums}{sbml-discuss@caltech.edu}.  The
mailing list archives and a web browser-based interface to the
list are available at \url{http://sbml.org/Forums/}.

A low-volume, broadcast-only mailing list is available 
where notifications of
revisions to the SBML speci\-fication, notices of votes on SBML
technical issues, and other critical matters are announced.  This
list is \link{http://sbml.org/Forums}{sbml-announce@caltech.edu}
and anyone may subscribe to it freely.  This list will never be
used for advertising and its membership list will never be
disclosed.  \emph{It is vitally important that all users of SBML
  stay informed about new releases and other developments by
  subscribing to this list}, even if they do not wish to
participate in discussions on the
\link{http://sbml.org/Forums}{sbml-discuss@caltech.edu} list.
Please visit \url{http://sbml.org/},
for information about how to subscribe to
\link{http://sbml.org/Forums}{sbml-announce@caltech.edu} as well
as for access to the list archives.

In the \nameref{sec:acknowledgements} section, we attempt to acknowledge
as many contributors to SBML's development as we can, but as SBML
evolves, it becomes increasingly difficult to detail the
individual contributions on a project that has truly become an
international community effort.


%-----------------------------------------------------------------------------
\subsection{SBML Levels, Versions, and Releases}
\label{sec:levels-versions-releases}
%-----------------------------------------------------------------------------

Major editions of SBML are termed \emph{levels} and represent
substantial changes to the composition and structure of the
language.  The edition of SBML defined in this document,
\sbmlthree, represents an evolution of the language resulting from
the practical experiences of users and developers working with
SBML since since its introduction in the year
2001~\citep{hucka:2001,hucka:2003}.  All of the constructs of
Level~1 can be mapped to Level~2; likewise, all of the constructs
from Level~2 can be mapped to Level~3 (when Level~3 is considered
in terms of the Core and Level~3 packages; see next section).  In
addition, a subset of Level~3 constructs can be mapped to Level~2,
and a subset of Level~2 constructs can be mapped to Level~1.
However, the levels remain distinct; a valid SBML Level~1 document
is not a valid SBML Level~2 document, and so on.

In practice, once a new Level of SBML is defined, no further
development is undertaken on lower Levels.  An exception is made
for the correction of problems and other issues that may be
identified in the specifications of lower Levels; such corrections
are handled as described below.

Minor revisions of SBML are termed \emph{versions} and constitute
changes within a Level to correct, adjust, and refine language
features.  The present document defines \thisLV.  A separate
document provides information about the changes between \sbmlthree
and \sbmltwo.

% (When we start using color within L3Core, use this version:)
% In Section~\ref{sec:notation-color} below explains how color is used
% in this document to indicate changes; a separate document provides
% detailed listing of the changes between versions of \sbmltwo as
% well as between SBML Level~2 Version~\sbmlversion and SBML Level~2
% Version~3.

Specification documents inevitably require minor editorial changes
as its users discover errors and ambiguities.  As a practical
reality, these discoveries occur over time.  In the context of
SBML, such problems are formally announced publicly as
\emph{errata} in a given specification document.  Borrowing
concepts from the World Wide Web Consortium~\citep{jacobs:2004},
we define SBML errata as changes of the following types: (a)
formatting changes that do not result in changes to textual
content; (b) corrections that do not affect conformance of
software implementing support for a given combination of SBML
Level and Version; and (c) corrections that \emph{may} affect such
software conformance, but add no new language features.  A change
that affects conformance is one that either turns conforming data,
processors, or other conforming software into non-conforming
software, or turns non-conforming software into conforming
software, or clears up an ambiguity or insufficiently documented
part of the specification in such a way that software whose
conformance was once unclear now becomes clearly conforming or
non-conforming~\citep{jacobs:2004}.  In short, errata do not
change the fundamental semantics or syntax of SBML; they clarify
and disambiguate the specification and correct errors.  (New
syntax and semantics are only introduced in SBML Versions and
Levels.)  An electronic tracking system for reporting and
monitoring such issues is available at
\url{http://sbml.org/issue-tracker}.

SBML errata result in new \emph{Releases} of the specification.
Each release is numbered, with the first release of the
specification being number~1.  Subsequent releases of an SBML
specification document contain a section for the accumulated
issues corrected since the first release.  The errata acknowledged
in SBML \thisLV since the publication of Release~1 are also
publicly listed at
\link{http://sbml.org/specifications/sbml-level-\sbmllevel/version-\sbmlversionnum/core/errata/}{http://sbml.org/}
\link{http://sbml.org/specifications/sbml-level-\sbmllevel/version-\sbmlversionnum/core/errata/}{specifications/sbml-level-\sbmllevel/version-\sbmlversionnum/core/errata/}.
Announcements of errata, updates to the SBML specification and
other major changes are made on the
\link{http://sbml.org/Forums}{sbml-announce@caltech.edu} mailing
list.


%-----------------------------------------------------------------------------
\subsection{SBML Level 3 Packages}
\label{sec:packages}
%-----------------------------------------------------------------------------

SBML Level~3 is being developed as a modular language, with a
central core comprising a self-sufficient model definition
language, and extension packages layered on top of this core to
provide additional, optional sets of features.  This document
defines SBML Level~3 Core.  The definition is based largely on
SBML Level~2, with some modifications to address sources of
problems found by experience with Level~2, and some
simplifications to remove Level~2 constructs that are expected to
be supported more thoroughly through SBML Level~3 packages.
Section~\ref{sec:sbml-packages} describes the mechanism by which
models defined in SBML Level~3 can declare which packages they
use.  

The specifications for packages available for SBML Level~3 is
maintained separately on the SBML website at
\url{http://sbml.org/Documents/Specifications}.  A list of
packages is not provided in \emph{this} specification document
(\ie for Level~3 Core) because the development of packages for
Level~3 proceeds independently, and new ones may be introduced
over time after Level~3 Core is published.  The SBML website
provides information ongoing activities in this area, as well as
about the process whereby individuals and groups may propose new
packages.


% vvvvvvvvvvvvvvvvvvvvvvvvvvvv  note  vvvvvvvvvvvvvvvvvvvvvvvvvvvv
% The following does not apply for R1 of V1.  Reinstate when we issue
% a release 2 or a version 2.

% %-----------------------------------------------------------------------------
% \subsection{Language features and backward compatibility}
% \label{sec:deprecated-features}
% %-----------------------------------------------------------------------------

% Some language features of previous SBML Levels and Versions have
% been either deprecated or removed entirely in SBML \thisLV.  For
% the purposes of SBML specifications, the following are the
% definitions of \emph{deprecated feature} and \emph{removed
%   feature}:
% \begin{description}
  
% \item \emph{Removed language feature}: A syntactic construct that
%   was present in previous SBML Levels and/or Versions within a
%   Level, and has been removed beginning with a specific SBML Level
%   and Version.  Models containing such constructs do not conform
%   to the specification of that SBML Level and Version.
  
% \item \emph{Deprecated language feature}: A syntactic construct
%   that was present in previous SBML Levels and/or Versions within
%   a Level, and while still present in the language definition, has
%   been identified as non-essential and planned for future removal.
%   Beginning with the Level and Version in which a given feature is
%   deprecated, software tools should not generate SBML models
%   containing the deprecated feature; however, for backward
%   compatibility, software tools reading SBML should support the
%   feature until it is actually removed.

% \end{description}

% As a matter of SBML design philosophy, the preferred approach to
% removing features is by deprecating them if possible.  Immediate
% removal of SBML features is not done unless serious problems have
% been discovered involving those features, and keeping them would
% create logical inconsistencies or extremely difficult-to-resolve
% problems.  The deprecation or outright removal of features in a
% language, whether SBML or other, can have significant impact on
% backwards compatibility.  Such changes are also inevitable over
% the course of a language's evolution.  SBML must by necessity
% continue evolving through the experiences of its users and
% implementors.  Eventually, some features will be deemed unhelpful
% despite the best intentions of the language editors to design a
% timeless language.

% Throughout the SBML specification, removed and deprecated features
% are discussed in the text of the sections where the features
% previously appeared.  Appendix~\ref{apdx:changes}
% lists the changes and describes their motivations in more detail.


% ^^^^^^^^^^^^^^^^^^^^^^^^^^^^  note ^^^^^^^^^^^^^^^^^^^^^^^^^^^^^


%-----------------------------------------------------------------------------
\subsection{Document conventions}
\label{sec:notation}
%-----------------------------------------------------------------------------

In this section, we describe conventions used in this document to
communicate information more effectively.


\subsubsection{Color conventions}
\label{sec:notation-color}

Through this document, we use blue text to indicate a hyperlink
from one point in this document to another.  Clicking your
computer's pointing device on blue-colored text will cause a jump
to the section, figure, table or page to which the link refers.
(Of course, this capability is only available when using
electronic media that support hyperlinking, such as PDF and HTML.)


% vvvvvvvvvvvvvvvvvvvvvvvvvvvv  note  vvvvvvvvvvvvvvvvvvvvvvvvvvvv
% The following does not apply for R1 of V1.  Reinstate when we issue
% a release 2 or a version 2.

% Throughout this document, we use coloring to carry additional
% information for the benefit of those viewing the document on media
% that can display color:

% \begin{itemize}

% \item We use red color in text and figures to indicate changes
%   between this version of the specification (SBML \thisLVR) and the
%   \emph{most recent previous version} of the specification (which,
%   for the present case, is SBML \previousLVR).  The changes
%   may be either additions or deletions of text; in the case of
%   deletions, entire sentences, paragraphs or sections are colored
%   to indicate a change has occurred inside them.

% \item We use blue color in text to indicate a hyperlink from one
%   point in this document to another.  Clicking your computer's
%   pointing device on blue-colored text will cause a jump to the
%   section, figure, table or page to which the link refers.  (Of
%   course, this capability is only available when using electronic
%   formats that support hyperlinking, such as PDF and
%   HTML.)

% \end{itemize}
% ^^^^^^^^^^^^^^^^^^^^^^^^^^^^  note ^^^^^^^^^^^^^^^^^^^^^^^^^^^^^


\subsubsection{Typographical conventions for names}
\label{sec:notation-typographical}

We use the following typographical conventions to distinguish
objects and data types from other entities:

\begin{description}
  
\item \abstractclass{AbstractClass}: Abstract classes are classes
  that are never instantiated directly, but rather serve as
  parents of other classes.  Their names begin with a capital
  letter and they are printed in a slanted, bold,
  sans-serif typeface.  In electronic document formats, the class
  names are also hyperlinked to their definitions in the
  specification.  For example, in the PDF and HTML versions of
  this document, clicking on the word \SBase will send the reader
  to the section containing the definition of this class.
  
\item \class{Class}: Names of ordinary (concrete) classes begin
  with a capital letter and are printed in an upright,
  bold, sans-serif typeface.  In electronic document
  formats, the class names are also hyperlinked to their
  definitions in the specification.  For example, in the PDF and
  HTML versions of this document, clicking on the word \Species
  will send the reader to the section containing the definition of
  this class.

\item \token{SomeThing}, \token{otherThing}: Attributes of
  classes, data type names, literal XML, and generally all tokens
  \emph{other} than SBML UML class names, are printed in an
  upright typewriter typeface.  Primitive types defined by SBML
  begin with a capital letter; SBML also makes use of primitive
  types defined by \xmlschemaone, but unfortunately, XML~Schema
  does not follow any capitalization convention and primitive
  types drawn from the XML~Schema language may or may not start
  with a capital letter.

\end{description}


\subsubsection{UML notation}
\label{sec:notation-uml}

% fixme: use primary citations for UML below

Previous specifications of SBML used a notation that was at one
time (in the days of \sbmlone) fairly close to UML, the Unified
Modeling Language~\citep{eriksson:1998,oestereich:1999}, though
many details were omitted from the UML diagrams themselves.  Over
the years, the notation used in successive specifications of SBML
grew increasingly less UML-like.  Beginning with \sbmltwothree, we
have completely overhauled the specification's use of UML and once
again define the XML syntax of SBML using, as much as possible,
proper and complete UML~1.0.  We then systematically map this UML
notation to XML.  In the rest of this section, we summarize the
UML notation used in this document and explain the few
embellishments needed to support transformation to XML form.

We see three main advantages to using UML as a basis for defining
SBML data objects.  First, compared to using other notations or
a programming language, the UML visual representations are
generally easier to grasp by readers who are not computer
scientists.  Second, the notation is implementation-neutral: the
objects can be encoded in any concrete implementation
language---not just XML, but C, Java and other languages as well.
Third, UML is a de facto industry standard that is documented in
many resources.  Readers are therefore more likely to be familiar
with it than other notations.


\paragraph{Object class definitions}

Object classes in UML diagrams are drawn as simple tripartite
boxes, as shown in Figure~\ref{fig:simple-class-eg} (left).  UML
allows for operations as well as data attributes to be defined,
but SBML only uses data attributes, so all SBML class diagrams use
only the top two portions of a UML class box
(Figure~\ref{fig:simple-class-eg}, right).

\begin{figure}[htb]
  \centering
  \small
  \begin{classbox}{Class Name}
    attributes\\
    \hline
    operators\\
  \end{classbox}
  \quad  \quad  \quad  \quad
  \begin{classbox}{ExampleClass}
    attribute: int \\
    anotherAttribute: double\\
  \end{classbox}
  \caption{(Left) The general form of a UML class
      diagram.  (Right) Example of a class diagram of the sort
      seen in SBML.  SBML classes never use operators, so SBML
      class diagrams only show the top two parts.}
  \label{fig:simple-class-eg}
\end{figure}

As mentioned above, the names of ordinary (concrete) classes begin
with a capital letter and are printed in an upright,
bold, sans-serif typeface.  The names of attributes
begin with a lower-case letter and generally use a mixed case
(sometimes called ``camel case'') style when the name consists of
multiple words.  Attributes and their data types appear in the
part below the class name, with one attribute defined per line.
The colon character on each line separates the name of the
attribute (on the left) from the type of data that it stores (on
the right).  The subset of data types permitted for SBML
attributes is given in Section~\ref{sec:primitive-types}.

In the right-hand diagram of Figure~\ref{fig:simple-class-eg}, the
symbols \token{attribute} and \token{anotherAttribute} represent
attributes of the object class \class{ExampleClass}.  The data
type of \token{attribute} is \primtype{int}, and the data type of
\token{anotherAttribute} is \primtype{double}.  In the scheme used
by SBML for translating UML to XML, object attributes map directly
to XML attributes.  Thus, in XML, \class{ExampleClass} would yield
an element of the form \token{<\emph{element} attribute="42"
  anotherAttribute="10.0">}.

Notice that the element name is not \token{<ExampleClass ...>}.
Somewhat paradoxically, the name of the element is \emph{not} the
name of the UML class defining its structure.  The reason for this
may be subtle at first, but quickly becomes obvious: object
classes define the form of an object's \emph{content}, but a class
definition by itself does not define the \emph{label} or symbol
referring to an instance of that content.  It is this label that
becomes the name of the XML element.  In XML, this symbol is most
naturally equated with an element name.  This point will hopefully
become more clear with additional examples below.


\paragraph{Subelements}

We use UML composite aggregation to indicate a class object can
have other class objects as parts.  Such containment hierarchies
map directly to element-subelement relationships in XML.
Figure~\ref{fig:subelement-eg} gives an example.

\begin{figure}[htb]
  \centering
  \small
  \begin{tikzpicture}[level distance=0.7in]
    \node[left=0.3in] (a) {
      \begin{classbox}{Whole}
        A: int \\
        B: string \\
      \end{classbox}
    };
    \node[right=0.3in] (b) {
      \begin{classbox}{Part}
        C: double \\
      \end{classbox}
    };
    \draw[diamond-,shorten >=-6pt] (a) -- (b)
      node[left=0.8in,above=2pt] {\textsf{inside}};
  \end{tikzpicture}
  \caption{Example illustrating composite
      aggregation: the definition of one class of objects
      employing another class of objects in a part-whole
      relationship.  In this particular example, an instance of a
      \class{Whole} class object must contain exactly one instance
      of a \class{Part} class object, and the symbol referring to
      the \class{Part} class object is \token{inside}.  In XML,
      this symbol becomes the name of a subelement and the content
      of the subelement follows the definition of \class{Part}.}
  \label{fig:subelement-eg}
\end{figure}

The line with the black diamond indicates composite aggregation,
with the diamond located on the ``container'' side and the other
end located at the object class being contained.  The label on the
line is the symbol used to refer to instances of the contained
object, which in XML, maps directly to the name of an XML element.
The class pointed to by the aggregation relationship (\class{Part}
in Figure~\ref{fig:subelement-eg}) defines the \emph{contents} of
that element.  Thus, if we are told that some element named
\token{barney} is of class \token{Whole}, the following is an
example XML fragment consistent with the class definition of
Figure~\ref{fig:subelement-eg}:

\begin{example}
<barney A="110" B="some string">
    <inside C="444.4">
</barney>
\end{example}

Sometimes numbers are placed above the line near the ``contained''
side of an aggregation to indicate how many instances can be
contained.  The common cases in SBML are the following:
\token{[0..*]} to signify a list containing zero or more;
\token{[1..*]} to signify a list containing at least one; and
\token{[0..1]} to signify exactly zero or one.  The absence of a
numerical label means ``exactly 1''.  This notation appears
throughout this specification document.


\paragraph{Inheritance}

\begin{wrapfigure}[10]{r}{1.9in}
  \centering
  \small
  \vspace*{-7ex}
  \begin{tikzpicture}[level distance=0.75in]
    \node { 
      \begin{classbox}{Parent}
        A: int           \\
        B: boolean \\
      \end{classbox}
    }
    [open triangle 60-,edge from parent fork down,sibling distance=2.15in]
    child {node (a) {
        \begin{classbox}{Child}
          C: int \\
          D: string \\
        \end{classbox}
      }}
    ;
  \end{tikzpicture}
  \caption{Inheritance.}
  \label{fig:inheritance-eg}
\end{wrapfigure}
Classes can inherit properties from other classes.  Since SBML
only uses data attributes and not operations, inheritance in SBML
simply involves data attributes from a parent class being
inherited by child classes.  Inheritance is indicated by a line
between two classes, with an open triangle next to the parent
class; Figure~\ref{fig:inheritance-eg} illustrates this.  In this
example, the instances of object class \class{Child} would have
not only attributes \token{C} and \token{D}, but also attributes
\token{A} and \token{B}.  All of these attributes would be
required (not optional) on instances of class \class{Child}
because they are mandatory on both \class{Parent} and
\class{Child}.



\paragraph{Additional notations for XML purposes}

Not everything is easily expressed in plain UML.  For example, it
is often necessary to indicate some constraints placed on the
values of an attribute.  In computer programming uses of UML, such
constraints are often expressed using Object Constraint Language
(OCL), but since we are most interested in the XML rendition of
SBML, in this specification we use \xmlschemaone (when possible)
as the language for expressing value constraints.  Constraints on
the values of attributes are written as expressions surrounded by
braces (\{ \}) after the data type declaration, as in the example
of Figure~\ref{fig:unit-eg}.

\begin{figure}[t]
  \centering
  \small
  \begin{tikzpicture}[level distance=1.05in]
    \node { \emptyClassbox{\textsl{SBase}} }
      [open triangle 60-,edge from parent fork down,sibling distance=2.75in]
      child {node (a) {
          \begin{classbox}{Sbml}
            xmlns: string \{ "http://www.sbml.org/sbml/level\sbmllevel/version\sbmlversionnum/core" \} \\
            level: positiveInteger \{ use="required" fixed="\sbmllevel" \}   \\
            version: positiveInteger \{ use="required" fixed="\sbmlversionnum" \} \\
            \{ \emph{Additional attributes permitted} \} \\
          \end{classbox}
        }}
      child {node (b) {
          \emptyClassbox{Model}
        }}
     ;
     \draw[diamond-] (a) -- (b) 
       node[above=6pt,left=0.65in] {\textsf{model}};
  \end{tikzpicture}
  \caption{A more complex example definition drawing on the
    concepts introduced so far in this section.  Both \class{Sbml}
    and \class{Model} are derived from \abstractclass{SBase};
    further, \class{Sbml} contains a single \class{Model} object
    named \token{model}.  Note the constraints on the values of
    the attributes in \class{Sbml}; they are enclosed in braces
    and written in XML Schema language.  The particular
    constraints here state that both the \token{level} and
    \token{version} attributes must be present, and that the
    values are fixed as indicated.  In addition, other attributes
    are permitted (for example, such as those added by \thisL
    packages).}
  \label{fig:unit-eg}
\end{figure}

In other situations, when something cannot be concisely expressed
using a few words of XML Schema, we write constraints using
English language descriptions surrounded by braces (\{ \}).  To
help distinguish these from literal XML Schema, we set the English
text in a slanted typeface.  The text accompanying all SBML
component definitions provides explanations of the constraints and
any other conditions applicable to the use of the components.


% Unless something dramatic changes, this will not be true for L3:
%
% \paragraph{Compatibility issues and warnings}
%
% One important and confusing point that goes against the grain of
% XML must be highlighted: the order in which subelements appear
% within SBML elements \emph{is} significant and \emph{must} follow
% the order given in the corresponding object definition.  This
% ordering is also difficult to express in plain UML, so we resort
% to using the approach of stating ordering requirements as
% constraints written in English and (again) enclosed in
% braces (\{ \}).  Figure~\vref{fig:sbase} gives an example of this.
%
% The ordering restriction also holds true when a subclass inherits
% attributes and elements from a base class: the base class
% attributes and elements must occur before those introduced by the
% subclass.
%
% This ordering constraint stems from aspects of XML Schema beyond
% our control (specifically, the need to use XML Schema's
% \token{sequence} construct to define the object classes).  It is
% an occasional source of software compatibility problems, because
% validating XML parsers will generate errors if the ordering
% within an XML element does not correspond to the SBML object class
% definition.
