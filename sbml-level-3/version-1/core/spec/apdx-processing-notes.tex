% -*- TeX-master: "sbml-level-3-version-1-core"; fill-column: 66 -*-
% $Id$
% $HeadURL$
% ----------------------------------------------------------------

\section{Processing and validating \class{SBase}
 \token{notes} and \class{Constraint} \token{message} content}
\label{apdx:processing-notes}

In Section~\ref{sec:notes} and
Section~\ref{sec:constraint-message}, we discussed the
\token{notes} element on \SBase and the \token{message} element on
\Constraint, respectively.  These elements can contain a number of
possible forms of XHTML content.
In this appendix, we describe a general procedure for how
application software can process such content.  We concentrate on
the common case of an SBML-reading application that needs to take
the contents and pass it to an XHTML display and/or editing
function obtained from a third-party API library.  The content of
the \token{notes} and \token{message} may not be a
complete XHTML document, so the
application will have to perform some processing before handing it
to the XHTML editor or validator.  How should this be done?

Based on the three forms of \SBase \token{notes} content described in
Section~\ref{sec:notes} and the identical forms for
\Constraint \token{message} described in Section~\ref{sec:constraint-message},
there are only three cases possible.
Here we give an example approach for handling them, although the
actual implementation details will differ depending on various
factors such as the requirements of the software libraries being
used.  This example approach would be performed for each
\token{notes} and \token{message} to be viewed or edited:
\begin{description}
  
\item \emph{Step 1}. If the XHTML viewing/editing function
  requires a fully compliant XML document, the SBML application
  could create a temporary data object containing an appropriate
  XML declaration and a DOCTYPE declaration; otherwise, the XML
  data object can be initially blank.
  
\item \emph{Step 2}. The application should look at the first
  element inside the \token{notes} or \token{message}
  (or rather, the first element
  that is not an XML comment), and take action based on the
  following three possibilities:
  \begin{itemize}\setlength{\parskip}{1.5ex}
    
  \item If the first element begins with
    \token{<html xhtml="http://www.w3.org/1999/xhtml">}, the
    application could assume that the content is a complete XHTML
    document and insert this into the temporary data object.
    
  \item Else, if the first element is \token{<body>}, the
    application should insert the following into the temporary
    data object, 

    \begin{example}
<html xhtml="http://www.w3.org/1999/xhtml">
    <head><title></title></head>\end{example}
    then insert the content of the \token{notes} 
    or \token{message}, and finally
    insert a closing \token{</html>}.
    
  \item Else, if the content begins with neither of
    the above elements, the application should insert the
    following into the temporary data object,

    \begin{example}
<html xhtml="http://www.w3.org/1999/xhtml">
    <head><title></title></head>
    <body>\end{example}
    then insert the content of the \token{notes} or \token{message}, and finally
    insert \token{</body></html>} to close the XHTML document.
  \end{itemize}

\end{description}
The result of the above would be a temporary XML data object that
the application could then pass to the XHTML viewing/editing API
function.

% [2006-09-05 MH] was thinking of adding the following too, but
% this seems like a too-short description, and I don't want to
% get into all the possible issues in making sure the inversion
% is done correctly.  Still, something to think about for the
% future:
%
%Since the application performing the processing above could
%remember which of the three cases it encountered for a given
%\token{notes} content, upon receiving the results back from the
%XHTML viewing/editing function, it could invert the process and
%insert the results back into the \token{notes} object in the SBML
%model.


