% -*- TeX-master: "main"; fill-column: 72 -*- 

\section{Introduction and motivation} \label{intro} 


\subsection{Proposal corresponding to this package specification} 

This specification for Rendering in SBML Level~3 
Version~1 is based on the proposal, by this documents authors, located 
at the following URL: 
\begin{center} 
  \vspace*{1ex}\small 
  \url{http://sbml.org/Community/Wiki/SBML_Level_3_Proposals/Rendering} 
  \vspace*{1ex} 
\end{center} 

The tracking number in the SBML issue tracking system~\citep{tracker} 
for \RenderPackage activities is 234. The version of the proposal used 
as the starting point for this specification is the version of May 
2011. Previous versions of the current proposal are: 

\begin{description} 
  \item[Version 5 (May 2011)] 
  \item [] \small{\url{http://otto.bioquant.uni-heidelberg.de/sbml/level2/20110525/sbml-render-specification-20110525.pdf}} 
  \item[Version 4 (October 2009)] 
  \item [] \small{\url{http://otto.bioquant.uni-heidelberg.de/sbml/level2/20091029/SBMLRenderExtension-20091029.pdf}} 
  \item[Version 3 (January 2008)] 
  \item [] \small{\url{http://otto.bioquant.uni-heidelberg.de/sbml/level2/20080130/RenderExtensionDraft-20080130.pdf}} 
	\item[Version 2 (March 2010)] 
  \item [] \small{\url{http://otto.bioquant.uni-heidelberg.de/sbml/level2/20070309/RenderExtensionDraft-20070309.pdf}} 
	\item[Version 1 (October 2006)] 
  \item [] \small{\url{http://otto.bioquant.uni-heidelberg.de/sbml/level2/20061012/RenderExtensionDraft-20061012.pdf}} 
	\item[Version 0.2 (October 2003)] 
  \item [] \small{\url{http://otto.bioquant.uni-heidelberg.de/sbml/level2/20031028/SBMLRenderExtension-20031028.pdf}} 
	\item[Version 0.1 (September 2003)] 
  \item [] \small{\url{http://otto.bioquant.uni-heidelberg.de/sbml/level2/20030911/sbml-render-extension-20030918.pdf}} 
\end{description} 

Details of earlier independent proposals are provided 
in \ref{background}. 

\subsection{Tracking number} 
As initially listed in the SBML issue tracking system under:\\ 
\url{http://sourceforge.net/p/sbml/sbml-specifications/234/}. 

\subsection{Package dependencies} 

The \RenderPackage adds additional classes to \sbmlthreecore and extends the 
SBML Level~3 Layout package. 

\subsection{Document conventions} \label{conventions} 

Following the precedent set by the SBML Level~3 Core specification 
document, we use UML~1.0 (Unified Modeling Language; 
\citealt{eriksson:1998,oestereich:1999}) class diagram notation to 
define the constructs provided by this package. We also use color in the 
diagrams to carry additional information for the benefit of those 
viewing the document on media that can display color. The following are 
the colors we use and what they represent: 

\begin{itemize} 

\item[\raisebox{2.75pt}{\colorbox{black}{\rule{0.8pt}{0.8pt}}}] 
\emph{Black}: Items colored black in the UML diagrams are components 
taken unchanged from their definition in the SBML Level~3 Core 
specification document. 

\item[\raisebox{2.75pt}{\colorbox{mediumgreen}{\rule{0.8pt}{0.8pt}}}] 
\emph{\textcolor{mediumgreen}{Green}}: Items colored green are 
components that exist in SBML Level~3 Core, but are extended by this 
package. Class boxes are also drawn with dashed lines to further 
distinguish them. 

\item[\raisebox{2.75pt}{\colorbox{darkblue}{\rule{0.8pt}{0.8pt}}}] 
\emph{\textcolor{darkblue}{Blue}}: Items colored blue are new components 
introduced in this package specification. They have no equivalent in the 
SBML Level~3 Core specification. 

\end{itemize} 

We also use the following typographical conventions to distinguish the 
names of objects and data types from other entities; these conventions 
are identical to the conventions used in the SBML Level~3 Core 
specification document: 

\begin{description} 

\item \abstractclass{AbstractClass}: Abstract classes are classes that 
are never instantiated directly, but rather serve as parents of other 
object classes. Their names begin with a capital letter and they are 
printed in a slanted, bold, sans-serif typeface. In electronic document 
formats, the class names defined within this document are also 
hyperlinked to their definitions; clicking on these items will, given 
appropriate software, switch the view to the section in this document 
containing the definition of that class. (However, for classes that are 
unchanged from their definitions in SBML Level~3 Core, the class names 
are not hyperlinked because they are not defined within this document.) 

\item \class{Class}: Names of ordinary (concrete) classes begin with a 
capital letter and are printed in an upright, bold, sans-serif typeface. 
In electronic document formats, the class names are also hyperlinked to 
their definitions in this specification document. (However, as in the 
previous case, class names are not hyperlinked if they are for classes 
that are unchanged from their definitions in the SBML Level~3 Core 
specification.) 

\item \token{SomeThing}, \token{otherThing}: Attributes of classes, data 
type names, literal XML, and generally all tokens \emph{other} than SBML 
UML class names, are printed in an upright typewriter typeface. 
Primitive types defined by SBML begin with a capital letter; SBML also 
makes use of primitive types defined by XML 
Schema~1.0~\citep{biron:2000,fallside:2000,thompson:2000}, but 
unfortunately, XML~Schema does not follow any capitalization convention 
and primitive types drawn from the XML~Schema language may or may not 
start with a capital letter. 

\end{description} 

For other matters involving the use of UML and XML, we follow the 
conventions used in the SBML Level~3 Core specification document. 

