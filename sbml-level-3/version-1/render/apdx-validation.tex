% -*- TeX-master: "main"; fill-column: 72 -*-
\section{Validation of SBML documents} \label{apdx-validation}

\subsection{Validation and consistency rules} \label{validation-rules}

This section summarizes all the conditions that must (or in some cases,
at least \emph{should}) be true of an SBML Level~3 Version~1 model that
uses the \RenderPackage. We use the same conventions as are used in the
SBML Level~3 Version~1 Core specification document. In particular, there
are different degrees of rule strictness. Formally, the differences are
expressed in the statement of a rule: either a rule states that a
condition \emph{must} be true, or a rule states that it \emph{should} be
true. Rules of the former kind are strict SBML validation rules---a
model encoded in SBML must conform to all of them in order to be
considered valid. Rules of the latter kind are consistency rules. To
help highlight these differences, we use the following three symbols
next to the rule numbers:

\begin{description}

\item[\hspace*{6.5pt}\vSymbol\vsp] A \vSymbolName indicates a
\emph{requirement} for SBML conformance. If a model does not follow this
rule, it does not conform to the Renderingspecification.
(Mnemonic intention behind the choice of symbol: ``This must be
checked.'')

\item[\hspace*{6.5pt}\cSymbol\csp] A \cSymbolName indicates a
\emph{recommendation} for model consistency. If a model does not follow
this rule, it is not considered strictly invalid as far as the Flux
Balance Constraints specification is concerned; however, it indicates
that the model contains a physical or conceptual inconsistency.
(Mnemonic intention behind the choice of symbol: ``This is a cause for
warning.'')

\item[\hspace*{6.5pt}\mSymbol\msp] A \mSymbolName indicates a strong
recommendation for good modeling practice. This rule is not strictly a
matter of SBML encoding, but the recommendation comes from logical
reasoning. As in the previous case, if a model does not follow this
rule, it is not strictly considered an invalid SBML encoding. (Mnemonic
intention behind the choice of symbol: ``You're a star if you heed
this.'')

\end{description}

The validation rules listed in the following subsections are all stated
or implied in the rest of this specification document. They are
enumerated here for convenience. Unless explicitly stated, all
validation rules concern objects and attributes specifically defined in
the Renderingpackage.

For \notice convenience and brevity, we use the shorthand
``\token{render:\-x}'' to stand for an attribute or element name \token{x}
in the namespace for the \RenderPackage, using the namespace prefix
\token{render}. In reality, the prefix string may be different from the
literal ``\token{render}'' used here (and indeed, it can be any valid XML
namespace prefix that the modeler or software chooses). We use
``\token{render:\-x}'' because it is shorter than to write a full
explanation everywhere we refer to an attribute or element in the
\RenderPackage namespace.

\subsubsection*{General rules about this package}

\validRule{render-10101}{To conform to the \RenderPackage specification for
SBML Level~3 Version~1, an SBML document must declare the use of the
following XML Namespace:\\
\textsl[-25]{\uri{http://www.sbml.org/sbml/level3/version1/render/version1}
}. (References: SBML Level~3 Package Specification for Rendering, Version~1, \sec{xml-namespace}.)}

\validRule{render-10102}{Wherever they appear in an SBML document, elements
and attributes from the \RenderPackage must be declared either implicitly
or explicitly to be in the XML namespace
\uri{http://www.sbml.org/sbml/level3/version1/render/version1}.
(References: SBML Level~3 Package Specification for Rendering, Version~1, \sec{xml-namespace}.) }

\subsubsection*{General rules about identifiers}

\validRule{render-10301}{(Extends validation rule \#10301 in the SBML
Level~3 Version~1 Core specification.) Within a \Model the values of the
attributes \token{id} and \token{render:\-id} on every instance of the
following classes of objects must be unique across the set of all
\token{id} and \token{render:\-id} attribute values of all such objects in
a model: the \Model itself, plus all contained \FunctionDefinition,
\Compartment, \Species, \Reaction, \SpeciesReference,
\ModifierSpeciesReference, \Event, and \Parameter objects, plus the
objects defined by the
\RenderPackage. (References: SBML Level~3 Package Specification for Rendering, Version~1, \sec{primtypes}.) }

\validRule{render-10302} {
The value of a \token{render:\-id} attribute must always conform to the
syntax of the SBML data type \primtype{SId}. (References: SBML Level~3
Package Specification for Rendering, Version~1,
\sec{primtypes}.)
}

\subsubsection*{Rules for the extended \class{SBML} class}


% -*- TeX-master: "main"; fill-column: 72 -*-

\section{Validation of SBML documents}
\label{apdx-validation}

\subsection{Validation and consistency rules}
\label{validation-rules}

This section summarizes all the conditions that must (or in some cases,
at least \emph{should}) be true of an SBML Level~3 Version~1 model that
uses the Render package. We use the same conventions as are used in the
SBML Level~3 Version~1 Core specification document. In particular, there
are different degrees of rule strictness. Formally, the differences are
expressed in the statement of a rule: either a rule states that a
condition \emph{must} be true, or a rule states that it \emph{should} be
true. Rules of the former kind are strict SBML validation rules---a
model encoded in SBML must conform to all of them in order to be
considered valid. Rules of the latter kind are consistency rules. To
help highlight these differences, we use the following three symbols
next to the rule numbers:

\begin{description}

\item[\hspace*{6.5pt}\vSymbol\vsp] A \vSymbolName indicates a
\emph{requirement} for SBML conformance. If a model does not follow this
rule, it does not conform to the Render specification. (Mnemonic
intention behind the choice of symbol: ``This must be checked.'')

\item[\hspace*{6.5pt}\cSymbol\csp] A \cSymbolName indicates a
\emph{recommendation} for model consistency. If a model does not follow
this rule, it is not considered strictly invalid as far as the Render
specification is concerned; however, it indicates that the model
contains a physical or conceptual inconsistency. (Mnemonic intention
behind the choice of symbol: ``This is a cause for warning.'')

\item[\hspace*{6.5pt}\mSymbol\msp] A \mSymbolName indicates a strong
recommendation for good modeling practice. This rule is not strictly a
matter of SBML encoding, but the recommendation comes from logical
reasoning. As in the previous case, if a model does not follow this
rule, it is not strictly considered an invalid SBML encoding. (Mnemonic
intention behind the choice of symbol: ``You're a star if you heed
this.'')

\end{description}

The validation rules listed in the following subsections are all stated
or implied in the rest of this specification document. They are
enumerated here for convenience. Unless explicitly stated, all
validation rules concern objects and attributes specifically defined in
the Render package.

For \notice convenience and brevity, we use the shorthand
``\token{render:\-x}'' to stand for an attribute or element name
\token{x} in the namespace for the Render package, using the namespace
prefix \token{fbc}. In reality, the prefix string may be different from
the literal ``\token{render}'' used here (and indeed, it can be any
valid XML namespace prefix that the modeler or software chooses). We use
``\token{render:\-x}'' because it is shorter than to write a full
explanation everywhere we refer to an attribute or element in the Render
namespace.

\subsubsection*{General rules about this package}

\validRule{render-10101}{To conform to the Render package specification
for SBML Level~3 Version~1, an SBML document must declare the use of the
following XML Namespace:
\uri{http://www.sbml.org/sbml/level3/version1/render/version1}.
(Reference: SBML Level~3 Package specification for Render, Version~1
\sec{xml-namespace}.)}

\validRule{render-10102}{Wherever they appear in an SBML document,
elements and attributes from the Render package must be declared either
implicitly or explicitly to be in the XML namespace
\uri{http://www.sbml.org/sbml/level3/version1/render/version1}.
(Reference: SBML Level~3 Package specification for Render, Version~1
\sec{xml-namespace}.)}

\TODO{validation of identifiers}

\TODO{validation of extended elements}



\subsubsection*{Rules for \class{ColorDefinition} object}

\validRule{render-20301}{A \ColorDefinition object may have the optional
SBML Level~3 Core attributes \token{metaid} and \token{sboTerm}. No
other attributes from the SBML Level 3 Core namespaces are permitted on
a \ColorDefinition. (Reference: SBML Level~3 Version~1 Core,
Section~3.2.)}

\validRule{render-20302}{A \ColorDefinition object may have the optional
SBML Level~3 Core subobjects for notes and annotations. No other
elements from the SBML Level 3 Core namespaces are permitted on a
\ColorDefinition. (Reference: SBML Level~3 Version~1 Core,
Section~3.2.)}

\validRule{render-20303}{A \ColorDefinition object may have the optional
attributes \token{render:\-id} and \token{render:\-value}. No other
attributes from the SBML Level 3 Render namespaces are permitted on a
\ColorDefinition object. (Reference: SBML Level~3 Specification for
Render Version~1, \sec{colordefinition-class}.)}

\validRule{render-20304}{The attribute \token{render:\-value} on a
\ColorDefinition must have a value of data type \token{string}.
(Reference: SBML Level~3 Specification for Render Version~1,
\sec{colordefinition-class}.)}


\subsubsection*{Rules for \class{Ellipse} object}

\validRule{render-20401}{An \RenderEllipse object may have the optional
SBML Level~3 Core attributes \token{metaid} and \token{sboTerm}. No
other attributes from the SBML Level 3 Core namespaces are permitted on
an \RenderEllipse. (Reference: SBML Level~3 Version~1 Core,
Section~3.2.)}

\validRule{render-20402}{An \RenderEllipse object may have the optional
SBML Level~3 Core subobjects for notes and annotations. No other
elements from the SBML Level 3 Core namespaces are permitted on an
\RenderEllipse. (Reference: SBML Level~3 Version~1 Core, Section~3.2.)}

\validRule{render-20403}{An \RenderEllipse object must contain one and
only one instance of each of the RelAbsVector, RelAbsVector and
\RelAbsVector elements, and may contain one and only one instance of
each of the RelAbsVector and RelAbsVector elements. No other elements
from the SBML Level 3 Render namespaces are permitted on an
\RenderEllipse object. (Reference: SBML Level~3 Specification for Render
Version~1, \sec{renderellipse-class}.)}


\subsubsection*{Rules for \class{GlobalRenderInformation} object}

\validRule{render-20501}{A \GlobalRenderInformation object may have the
optional SBML Level~3 Core attributes \token{metaid} and
\token{sboTerm}. No other attributes from the SBML Level 3 Core
namespaces are permitted on a \GlobalRenderInformation. (Reference: SBML
Level~3 Version~1 Core, Section~3.2.)}

\validRule{render-20502}{A \GlobalRenderInformation object may have the
optional SBML Level~3 Core subobjects for notes and annotations. No
other elements from the SBML Level 3 Core namespaces are permitted on a
\GlobalRenderInformation. (Reference: SBML Level~3 Version~1 Core,
Section~3.2.)}

\validRule{render-20503}{A \GlobalRenderInformation object may contain
one and only one instance of the \ListOfGlobalStyles element. No other
elements from the SBML Level 3 Render namespaces are permitted on a
\GlobalRenderInformation object. (Reference: SBML Level~3 Specification
for Render Version~1, \sec{globalrenderinformation-class}.)}

\validRule{render-20504}{The \ListOfGlobalStyles subobject on a
\GlobalRenderInformation object is optional, but if present, this
container object must not be empty. (Reference: SBML Level~3
Specification for Render Version~1,
\sec{globalrenderinformation-class}.)}

\validRule{render-20505}{Apart from the general notes and annotations
subobjects permitted on all SBML objects, a \ListOfGlobalStyles
container object may only contain \GlobalStyle objects. (Reference: SBML
Level~3 Specification for Render Version~1,
\sec{globalrenderinformation-class}.)}

\validRule{render-20506}{A \ListOfGlobalStyles object may have the
optional SBML Level~3 Core attributes \token{metaid} and
\token{sboTerm}. No other attributes from the SBML Level 3 Core
namespaces are permitted on a \ListOfGlobalStyles object. (Reference:
SBML Level~3 Specification for Render Version~1,
\sec{globalrenderinformation-class}.)}


\subsubsection*{Rules for \class{GlobalStyle} object}

\validRule{render-20601}{A \GlobalStyle object may have the optional
SBML Level~3 Core attributes \token{metaid} and \token{sboTerm}. No
other attributes from the SBML Level 3 Core namespaces are permitted on
a \GlobalStyle. (Reference: SBML Level~3 Version~1 Core, Section~3.2.)}

\validRule{render-20602}{A \GlobalStyle object may have the optional
SBML Level~3 Core subobjects for notes and annotations. No other
elements from the SBML Level 3 Core namespaces are permitted on a
\GlobalStyle. (Reference: SBML Level~3 Version~1 Core, Section~3.2.)}


\subsubsection*{Rules for \class{GradientBase} object}

\validRule{render-20701}{A \GradientBase object may have the optional
SBML Level~3 Core attributes \token{metaid} and \token{sboTerm}. No
other attributes from the SBML Level 3 Core namespaces are permitted on
a \GradientBase. (Reference: SBML Level~3 Version~1 Core, Section~3.2.)}

\validRule{render-20702}{A \GradientBase object may have the optional
SBML Level~3 Core subobjects for notes and annotations. No other
elements from the SBML Level 3 Core namespaces are permitted on a
\GradientBase. (Reference: SBML Level~3 Version~1 Core, Section~3.2.)}

\validRule{render-20703}{A \GradientBase object must have the required
attribute \token{render:\-id}, and may have the optional attribute
\token{render:\-spreadMethod}. No other attributes from the SBML Level 3
Render namespaces are permitted on a \GradientBase object. (Reference:
SBML Level~3 Specification for Render Version~1,
\sec{gradientbase-class}.)}

\validRule{render-20704}{A \GradientBase object may contain one and only
one instance of the \ListOfGradientStops element. No other elements from
the SBML Level 3 Render namespaces are permitted on a \GradientBase
object. (Reference: SBML Level~3 Specification for Render Version~1,
\sec{gradientbase-class}.)}

\validRule{render-20705}{The value of the attribute
\token{render:\-spreadMethod} of a \GradientBase object must conform to
the syntax of SBML data type \primtype{gradientSpreadMethod} and may
only take on the allowed values of \primtype{gradientSpreadMethod}
defined in SBML; that is the value must be one of the following "pad",
"reflect" or "repeat". (Reference: SBML Level~3 Specification for Render
Version~1, \sec{gradientbase-class}.)}

\validRule{render-20706}{The \ListOfGradientStops subobject on a
\GradientBase object is optional, but if present, this container object
must not be empty. (Reference: SBML Level~3 Specification for Render
Version~1, \sec{gradientbase-class}.)}

\validRule{render-20707}{Apart from the general notes and annotations
subobjects permitted on all SBML objects, a \ListOfGradientStops
container object may only contain \GradientStop objects. (Reference:
SBML Level~3 Specification for Render Version~1,
\sec{gradientbase-class}.)}

\validRule{render-20708}{A \ListOfGradientStops object may have the
optional SBML Level~3 Core attributes \token{metaid} and
\token{sboTerm}. No other attributes from the SBML Level 3 Core
namespaces are permitted on a \ListOfGradientStops object. (Reference:
SBML Level~3 Specification for Render Version~1,
\sec{gradientbase-class}.)}


\subsubsection*{Rules for \class{GradientStop} object}

\validRule{render-20801}{A \GradientStop object may have the optional
SBML Level~3 Core attributes \token{metaid} and \token{sboTerm}. No
other attributes from the SBML Level 3 Core namespaces are permitted on
a \GradientStop. (Reference: SBML Level~3 Version~1 Core, Section~3.2.)}

\validRule{render-20802}{A \GradientStop object may have the optional
SBML Level~3 Core subobjects for notes and annotations. No other
elements from the SBML Level 3 Core namespaces are permitted on a
\GradientStop. (Reference: SBML Level~3 Version~1 Core, Section~3.2.)}

\validRule{render-20803}{A \GradientStop object must have the required
attribute \token{render:\-stop-color}. No other attributes from the SBML
Level 3 Render namespaces are permitted on a \GradientStop object.
(Reference: SBML Level~3 Specification for Render Version~1,
\sec{gradientstop-class}.)}

\validRule{render-20804}{A \GradientStop object must contain one and
only one instance of the RelAbsVector element. No other elements from
the SBML Level 3 Render namespaces are permitted on a \GradientStop
object. (Reference: SBML Level~3 Specification for Render Version~1,
\sec{gradientstop-class}.)}

\validRule{render-20805}{The attribute \token{render:\-stop-color} on a
\GradientStop must have a value of data type \token{string}. (Reference:
SBML Level~3 Specification for Render Version~1,
\sec{gradientstop-class}.)}


\subsubsection*{Rules for \class{RenderGroup} object}

\validRule{render-20901}{A \RenderGroup object may have the optional
SBML Level~3 Core attributes \token{metaid} and \token{sboTerm}. No
other attributes from the SBML Level 3 Core namespaces are permitted on
a \RenderGroup. (Reference: SBML Level~3 Version~1 Core, Section~3.2.)}

\validRule{render-20902}{A \RenderGroup object may have the optional
SBML Level~3 Core subobjects for notes and annotations. No other
elements from the SBML Level 3 Core namespaces are permitted on a
\RenderGroup. (Reference: SBML Level~3 Version~1 Core, Section~3.2.)}

\validRule{render-20903}{A \RenderGroup object may have the optional
attributes \token{render:\-startHead}, \token{render:\-endHead},
\token{render:\-font-family}, \token{render:\-font-weight},
\token{render:\-font-style}, \token{render:\-text-anchor} and
\token{render:\-vtext-anchor}. No other attributes from the SBML Level 3
Render namespaces are permitted on a \RenderGroup object. (Reference:
SBML Level~3 Specification for Render Version~1,
\sec{rendergroup-class}.)}

\validRule{render-20904}{A \RenderGroup object may contain one and only
one instance of each of the RelAbsVector and \ListOfElements elements.
No other elements from the SBML Level 3 Render namespaces are permitted
on a \RenderGroup object. (Reference: SBML Level~3 Specification for
Render Version~1, \sec{rendergroup-class}.)}

\validRule{render-20905}{The value of the attribute
\token{render:\-startHead} of a \RenderGroup object must be the
identifier of an existing \LineEnding object defined in the enclosing
\Model object. (Reference: SBML Level~3 Specification for Render
Version~1, \sec{rendergroup-class}.)}

\validRule{render-20906}{The value of the attribute
\token{render:\-endHead} of a \RenderGroup object must be the identifier
of an existing \LineEnding object defined in the enclosing \Model
object. (Reference: SBML Level~3 Specification for Render Version~1,
\sec{rendergroup-class}.)}

\validRule{render-20907}{The attribute \token{render:\-font-family} on a
\RenderGroup must have a value of data type \token{string}. (Reference:
SBML Level~3 Specification for Render Version~1,
\sec{rendergroup-class}.)}

\validRule{render-20908}{The value of the attribute
\token{render:\-font-weight} of a \RenderGroup object must conform to
the syntax of SBML data type \primtype{fontWeight} and may only take on
the allowed values of \primtype{fontWeight} defined in SBML; that is the
value must be one of the following "normal" or "bold". (Reference: SBML
Level~3 Specification for Render Version~1, \sec{rendergroup-class}.)}

\validRule{render-20909}{The value of the attribute
\token{render:\-font-style} of a \RenderGroup object must conform to the
syntax of SBML data type \primtype{fontStyle} and may only take on the
allowed values of \primtype{fontStyle} defined in SBML; that is the
value must be one of the following "normal" or "italic". (Reference:
SBML Level~3 Specification for Render Version~1,
\sec{rendergroup-class}.)}

\validRule{render-20910}{The value of the attribute
\token{render:\-text-anchor} of a \RenderGroup object must conform to
the syntax of SBML data type \primtype{hTextAnchor} and may only take on
the allowed values of \primtype{hTextAnchor} defined in SBML; that is
the value must be one of the following "start", "middle" or "end".
(Reference: SBML Level~3 Specification for Render Version~1,
\sec{rendergroup-class}.)}

\validRule{render-20911}{The value of the attribute
\token{render:\-vtext-anchor} of a \RenderGroup object must conform to
the syntax of SBML data type \primtype{vTextAnchor} and may only take on
the allowed values of \primtype{vTextAnchor} defined in SBML; that is
the value must be one of the following "top", "middle", "bottom" or
"baseline". (Reference: SBML Level~3 Specification for Render Version~1,
\sec{rendergroup-class}.)}

\validRule{render-20912}{The \ListOfElements subobject on a \RenderGroup
object is optional, but if present, this container object must not be
empty. (Reference: SBML Level~3 Specification for Render Version~1,
\sec{rendergroup-class}.)}

\validRule{render-20913}{Apart from the general notes and annotations
subobjects permitted on all SBML objects, a \ListOfElements container
object may only contain \Transformation2D objects. (Reference: SBML
Level~3 Specification for Render Version~1, \sec{rendergroup-class}.)}

\validRule{render-20914}{A \ListOfElements object may have the optional
SBML Level~3 Core attributes \token{metaid} and \token{sboTerm}. No
other attributes from the SBML Level 3 Core namespaces are permitted on
a \ListOfElements object. (Reference: SBML Level~3 Specification for
Render Version~1, \sec{rendergroup-class}.)}


\subsubsection*{Rules for \class{Image} object}

\validRule{render-21001}{An \Image object may have the optional SBML
Level~3 Core attributes \token{metaid} and \token{sboTerm}. No other
attributes from the SBML Level 3 Core namespaces are permitted on an
\Image. (Reference: SBML Level~3 Version~1 Core, Section~3.2.)}

\validRule{render-21002}{An \Image object may have the optional SBML
Level~3 Core subobjects for notes and annotations. No other elements
from the SBML Level 3 Core namespaces are permitted on an \Image.
(Reference: SBML Level~3 Version~1 Core, Section~3.2.)}

\validRule{render-21003}{An \Image object must have the required
attribute \token{render:\-href}, and may have the optional attribute
\token{render:\-id}. No other attributes from the SBML Level 3 Render
namespaces are permitted on an \Image object. (Reference: SBML Level~3
Specification for Render Version~1, \sec{image-class}.)}

\validRule{render-21004}{An \Image object must contain one and only one
instance of each of the RelAbsVector, RelAbsVector, RelAbsVector and
\RelAbsVector elements, and may contain one and only one instance of the
RelAbsVector element. No other elements from the SBML Level 3 Render
namespaces are permitted on an \Image object. (Reference: SBML Level~3
Specification for Render Version~1, \sec{image-class}.)}

\validRule{render-21005}{The attribute \token{render:\-href} on an
\Image must have a value of data type \token{string}. (Reference: SBML
Level~3 Specification for Render Version~1, \sec{image-class}.)}


\subsubsection*{Rules for \class{LineEnding} object}

\validRule{render-21101}{A \LineEnding object may have the optional SBML
Level~3 Core attributes \token{metaid} and \token{sboTerm}. No other
attributes from the SBML Level 3 Core namespaces are permitted on a
\LineEnding. (Reference: SBML Level~3 Version~1 Core, Section~3.2.)}

\validRule{render-21102}{A \LineEnding object may have the optional SBML
Level~3 Core subobjects for notes and annotations. No other elements
from the SBML Level 3 Core namespaces are permitted on a \LineEnding.
(Reference: SBML Level~3 Version~1 Core, Section~3.2.)}

\validRule{render-21103}{A \LineEnding object must have the required
attribute \token{render:\-id}, and may have the optional attribute
\token{render:\-enableRotationalMapping}. No other attributes from the
SBML Level 3 Render namespaces are permitted on a \LineEnding object.
(Reference: SBML Level~3 Specification for Render Version~1,
\sec{lineending-class}.)}

\validRule{render-21104}{A \LineEnding object may contain one and only
one instance of each of the RenderGroup and BoundingBox elements. No
other elements from the SBML Level 3 Render namespaces are permitted on
a \LineEnding object. (Reference: SBML Level~3 Specification for Render
Version~1, \sec{lineending-class}.)}

\validRule{render-21105}{The attribute
\token{render:\-enableRotationalMapping} on a \LineEnding must have a
value of data type \token{boolean}. (Reference: SBML Level~3
Specification for Render Version~1, \sec{lineending-class}.)}


\subsubsection*{Rules for \class{LinearGradient} object}

\validRule{render-21201}{A \LinearGradient object may have the optional
SBML Level~3 Core attributes \token{metaid} and \token{sboTerm}. No
other attributes from the SBML Level 3 Core namespaces are permitted on
a \LinearGradient. (Reference: SBML Level~3 Version~1 Core,
Section~3.2.)}

\validRule{render-21202}{A \LinearGradient object may have the optional
SBML Level~3 Core subobjects for notes and annotations. No other
elements from the SBML Level 3 Core namespaces are permitted on a
\LinearGradient. (Reference: SBML Level~3 Version~1 Core, Section~3.2.)}

\validRule{render-21203}{A \LinearGradient object may contain one and
only one instance of each of the RelAbsVector, RelAbsVector,
RelAbsVector, RelAbsVector, RelAbsVector and RelAbsVector elements. No
other elements from the SBML Level 3 Render namespaces are permitted on
a \LinearGradient object. (Reference: SBML Level~3 Specification for
Render Version~1, \sec{lineargradient-class}.)}


\subsubsection*{Rules for \class{LocalRenderInformation} object}

\validRule{render-21301}{A \LocalRenderInformation object may have the
optional SBML Level~3 Core attributes \token{metaid} and
\token{sboTerm}. No other attributes from the SBML Level 3 Core
namespaces are permitted on a \LocalRenderInformation. (Reference: SBML
Level~3 Version~1 Core, Section~3.2.)}

\validRule{render-21302}{A \LocalRenderInformation object may have the
optional SBML Level~3 Core subobjects for notes and annotations. No
other elements from the SBML Level 3 Core namespaces are permitted on a
\LocalRenderInformation. (Reference: SBML Level~3 Version~1 Core,
Section~3.2.)}

\validRule{render-21303}{A \LocalRenderInformation object may contain
one and only one instance of the \ListOfLocalStyles element. No other
elements from the SBML Level 3 Render namespaces are permitted on a
\LocalRenderInformation object. (Reference: SBML Level~3 Specification
for Render Version~1, \sec{localrenderinformation-class}.)}

\validRule{render-21304}{The \ListOfLocalStyles subobject on a
\LocalRenderInformation object is optional, but if present, this
container object must not be empty. (Reference: SBML Level~3
Specification for Render Version~1,
\sec{localrenderinformation-class}.)}

\validRule{render-21305}{Apart from the general notes and annotations
subobjects permitted on all SBML objects, a \ListOfLocalStyles container
object may only contain \LocalStyle objects. (Reference: SBML Level~3
Specification for Render Version~1,
\sec{localrenderinformation-class}.)}

\validRule{render-21306}{A \ListOfLocalStyles object may have the
optional SBML Level~3 Core attributes \token{metaid} and
\token{sboTerm}. No other attributes from the SBML Level 3 Core
namespaces are permitted on a \ListOfLocalStyles object. (Reference:
SBML Level~3 Specification for Render Version~1,
\sec{localrenderinformation-class}.)}


\subsubsection*{Rules for \class{LocalStyle} object}

\validRule{render-21401}{A \LocalStyle object may have the optional SBML
Level~3 Core attributes \token{metaid} and \token{sboTerm}. No other
attributes from the SBML Level 3 Core namespaces are permitted on a
\LocalStyle. (Reference: SBML Level~3 Version~1 Core, Section~3.2.)}

\validRule{render-21402}{A \LocalStyle object may have the optional SBML
Level~3 Core subobjects for notes and annotations. No other elements
from the SBML Level 3 Core namespaces are permitted on a \LocalStyle.
(Reference: SBML Level~3 Version~1 Core, Section~3.2.)}

\validRule{render-21403}{A \LocalStyle object may have the optional
attribute \token{render:\-idList}. No other attributes from the SBML
Level 3 Render namespaces are permitted on a \LocalStyle object.
(Reference: SBML Level~3 Specification for Render Version~1,
\sec{localstyle-class}.)}

\validRule{render-21404}{The attribute \token{render:\-idList} on a
\LocalStyle must have a value of data type \token{string}. (Reference:
SBML Level~3 Specification for Render Version~1,
\sec{localstyle-class}.)}


\subsubsection*{Rules for \class{Polygon} object}

\validRule{render-21501}{A \Polygon object may have the optional SBML
Level~3 Core attributes \token{metaid} and \token{sboTerm}. No other
attributes from the SBML Level 3 Core namespaces are permitted on a
\Polygon. (Reference: SBML Level~3 Version~1 Core, Section~3.2.)}

\validRule{render-21502}{A \Polygon object may have the optional SBML
Level~3 Core subobjects for notes and annotations. No other elements
from the SBML Level 3 Core namespaces are permitted on a \Polygon.
(Reference: SBML Level~3 Version~1 Core, Section~3.2.)}

\validRule{render-21503}{A \Polygon object must contain one and only one
instance of the \ListOfRenderPoints element. No other elements from the
SBML Level 3 Render namespaces are permitted on a \Polygon object.
(Reference: SBML Level~3 Specification for Render Version~1,
\sec{polygon-class}.)}

\validRule{render-21504}{Apart from the general notes and annotations
subobjects permitted on all SBML objects, a \ListOfRenderPoints
container object may only contain \RenderPoint objects. (Reference: SBML
Level~3 Specification for Render Version~1, \sec{polygon-class}.)}

\validRule{render-21505}{A \ListOfRenderPoints object may have the
optional SBML Level~3 Core attributes \token{metaid} and
\token{sboTerm}. No other attributes from the SBML Level 3 Core
namespaces are permitted on a \ListOfRenderPoints object. (Reference:
SBML Level~3 Specification for Render Version~1, \sec{polygon-class}.)}


\subsubsection*{Rules for \class{RadialGradient} object}

\validRule{render-21601}{A \RadialGradient object may have the optional
SBML Level~3 Core attributes \token{metaid} and \token{sboTerm}. No
other attributes from the SBML Level 3 Core namespaces are permitted on
a \RadialGradient. (Reference: SBML Level~3 Version~1 Core,
Section~3.2.)}

\validRule{render-21602}{A \RadialGradient object may have the optional
SBML Level~3 Core subobjects for notes and annotations. No other
elements from the SBML Level 3 Core namespaces are permitted on a
\RadialGradient. (Reference: SBML Level~3 Version~1 Core, Section~3.2.)}

\validRule{render-21603}{A \RadialGradient object may contain one and
only one instance of each of the RelAbsVector, RelAbsVector,
RelAbsVector, RelAbsVector, RelAbsVector, RelAbsVector and RelAbsVector
elements. No other elements from the SBML Level 3 Render namespaces are
permitted on a \RadialGradient object. (Reference: SBML Level~3
Specification for Render Version~1, \sec{radialgradient-class}.)}


\subsubsection*{Rules for \class{Rectangle} object}

\validRule{render-21701}{A \RenderRectangle object may have the optional
SBML Level~3 Core attributes \token{metaid} and \token{sboTerm}. No
other attributes from the SBML Level 3 Core namespaces are permitted on
a \RenderRectangle. (Reference: SBML Level~3 Version~1 Core,
Section~3.2.)}

\validRule{render-21702}{A \RenderRectangle object may have the optional
SBML Level~3 Core subobjects for notes and annotations. No other
elements from the SBML Level 3 Core namespaces are permitted on a
\RenderRectangle. (Reference: SBML Level~3 Version~1 Core,
Section~3.2.)}

\validRule{render-21703}{A \RenderRectangle object must contain one and
only one instance of each of the RelAbsVector, RelAbsVector and
\RelAbsVector elements, and may contain one and only one instance of
each of the RelAbsVector, RelAbsVector, RelAbsVector and RelAbsVector
elements. No other elements from the SBML Level 3 Render namespaces are
permitted on a \RenderRectangle object. (Reference: SBML Level~3
Specification for Render Version~1, \sec{renderrectangle-class}.)}


\subsubsection*{Rules for \class{RelAbsVector} object}

\validRule{render-21801}{A \RelAbsVector object may have the optional
SBML Level~3 Core attributes \token{metaid} and \token{sboTerm}. No
other attributes from the SBML Level 3 Core namespaces are permitted on
a \RelAbsVector. (Reference: SBML Level~3 Version~1 Core, Section~3.2.)}

\validRule{render-21802}{A \RelAbsVector object may have the optional
SBML Level~3 Core subobjects for notes and annotations. No other
elements from the SBML Level 3 Core namespaces are permitted on a
\RelAbsVector. (Reference: SBML Level~3 Version~1 Core, Section~3.2.)}

\validRule{render-21803}{A \RelAbsVector object may have the optional
attributes \token{render:\-abs} and \token{render:\-rel}. No other
attributes from the SBML Level 3 Render namespaces are permitted on a
\RelAbsVector object. (Reference: SBML Level~3 Specification for Render
Version~1, \sec{relabsvector-class}.)}

\validRule{render-21804}{The attribute \token{render:\-abs} on a
\RelAbsVector must have a value of data type \token{double}. (Reference:
SBML Level~3 Specification for Render Version~1,
\sec{relabsvector-class}.)}

\validRule{render-21805}{The attribute \token{render:\-rel} on a
\RelAbsVector must have a value of data type \token{double}. (Reference:
SBML Level~3 Specification for Render Version~1,
\sec{relabsvector-class}.)}


\subsubsection*{Rules for \class{RenderCubicBezier} object}

\validRule{render-21901}{A \RenderCubicBezier object may have the
optional SBML Level~3 Core attributes \token{metaid} and
\token{sboTerm}. No other attributes from the SBML Level 3 Core
namespaces are permitted on a \RenderCubicBezier. (Reference: SBML
Level~3 Version~1 Core, Section~3.2.)}

\validRule{render-21902}{A \RenderCubicBezier object may have the
optional SBML Level~3 Core subobjects for notes and annotations. No
other elements from the SBML Level 3 Core namespaces are permitted on a
\RenderCubicBezier. (Reference: SBML Level~3 Version~1 Core,
Section~3.2.)}

\validRule{render-21903}{A \RenderCubicBezier object must contain one
and only one instance of each of the RelAbsVector, RelAbsVector,
RelAbsVector and \RelAbsVector elements, and may contain one and only
one instance of each of the RelAbsVector and RelAbsVector elements. No
other elements from the SBML Level 3 Render namespaces are permitted on
a \RenderCubicBezier object. (Reference: SBML Level~3 Specification for
Render Version~1, \sec{rendercubicbezier-class}.)}


\subsubsection*{Rules for \class{RenderCurve} object}

\validRule{render-22001}{A \RenderCurve object may have the optional
SBML Level~3 Core attributes \token{metaid} and \token{sboTerm}. No
other attributes from the SBML Level 3 Core namespaces are permitted on
a \RenderCurve. (Reference: SBML Level~3 Version~1 Core, Section~3.2.)}

\validRule{render-22002}{A \RenderCurve object may have the optional
SBML Level~3 Core subobjects for notes and annotations. No other
elements from the SBML Level 3 Core namespaces are permitted on a
\RenderCurve. (Reference: SBML Level~3 Version~1 Core, Section~3.2.)}

\validRule{render-22003}{A \RenderCurve object may have the optional
attributes \token{render:\-startHead} and \token{render:\-endHead}. No
other attributes from the SBML Level 3 Render namespaces are permitted
on a \RenderCurve object. (Reference: SBML Level~3 Specification for
Render Version~1, \sec{rendercurve-class}.)}

\validRule{render-22004}{A \RenderCurve object must contain one and only
one instance of the \ListOfRenderPoints element. No other elements from
the SBML Level 3 Render namespaces are permitted on a \RenderCurve
object. (Reference: SBML Level~3 Specification for Render Version~1,
\sec{rendercurve-class}.)}

\validRule{render-22005}{The value of the attribute
\token{render:\-startHead} of a \RenderCurve object must be the
identifier of an existing \LineEnding object defined in the enclosing
\Model object. (Reference: SBML Level~3 Specification for Render
Version~1, \sec{rendercurve-class}.)}

\validRule{render-22006}{The value of the attribute
\token{render:\-endHead} of a \RenderCurve object must be the identifier
of an existing \LineEnding object defined in the enclosing \Model
object. (Reference: SBML Level~3 Specification for Render Version~1,
\sec{rendercurve-class}.)}

\validRule{render-22007}{Apart from the general notes and annotations
subobjects permitted on all SBML objects, a \ListOfRenderPoints
container object may only contain \RenderPoint objects. (Reference: SBML
Level~3 Specification for Render Version~1, \sec{rendercurve-class}.)}

\validRule{render-22008}{A \ListOfRenderPoints object may have the
optional SBML Level~3 Core attributes \token{metaid} and
\token{sboTerm}. No other attributes from the SBML Level 3 Core
namespaces are permitted on a \ListOfRenderPoints object. (Reference:
SBML Level~3 Specification for Render Version~1,
\sec{rendercurve-class}.)}


\subsubsection*{Rules for \class{RenderPoint} object}

\validRule{render-22101}{A \RenderPoint object may have the optional
SBML Level~3 Core attributes \token{metaid} and \token{sboTerm}. No
other attributes from the SBML Level 3 Core namespaces are permitted on
a \RenderPoint. (Reference: SBML Level~3 Version~1 Core, Section~3.2.)}

\validRule{render-22102}{A \RenderPoint object may have the optional
SBML Level~3 Core subobjects for notes and annotations. No other
elements from the SBML Level 3 Core namespaces are permitted on a
\RenderPoint. (Reference: SBML Level~3 Version~1 Core, Section~3.2.)}

\validRule{render-22103}{A \RenderPoint object must contain one and only
one instance of each of the RelAbsVector and \RelAbsVector elements, and
may contain one and only one instance of the RelAbsVector element. No
other elements from the SBML Level 3 Render namespaces are permitted on
a \RenderPoint object. (Reference: SBML Level~3 Specification for Render
Version~1, \sec{renderpoint-class}.)}


\subsubsection*{Rules for \class{Text} object}

\validRule{render-22201}{A \Text object may have the optional SBML
Level~3 Core attributes \token{metaid} and \token{sboTerm}. No other
attributes from the SBML Level 3 Core namespaces are permitted on a
\Text. (Reference: SBML Level~3 Version~1 Core, Section~3.2.)}

\validRule{render-22202}{A \Text object may have the optional SBML
Level~3 Core subobjects for notes and annotations. No other elements
from the SBML Level 3 Core namespaces are permitted on a \Text.
(Reference: SBML Level~3 Version~1 Core, Section~3.2.)}

\validRule{render-22203}{A \Text object may have the optional attributes
\token{render:\-font-family}, \token{render:\-font-weight},
\token{render:\-font-style}, \token{render:\-text-anchor} and
\token{render:\-vtext-anchor}. No other attributes from the SBML Level 3
Render namespaces are permitted on a \Text object. (Reference: SBML
Level~3 Specification for Render Version~1, \sec{text-class}.)}

\validRule{render-22204}{A \Text object must contain one and only one
instance of each of the RelAbsVector and \RelAbsVector elements, and may
contain one and only one instance of each of the RelAbsVector and
RelAbsVector elements. No other elements from the SBML Level 3 Render
namespaces are permitted on a \Text object. (Reference: SBML Level~3
Specification for Render Version~1, \sec{text-class}.)}

\validRule{render-22205}{The value of the attribute
\token{render:\-font-family} of a \Text object must conform to the
syntax of SBML data type \primtype{fontFamily} and may only take on the
allowed values of \primtype{fontFamily} defined in SBML; that is the
value must be one of the following "serif", "sans-serif" or "monospace".
(Reference: SBML Level~3 Specification for Render Version~1,
\sec{text-class}.)}

\validRule{render-22206}{The value of the attribute
\token{render:\-font-weight} of a \Text object must conform to the
syntax of SBML data type \primtype{fontWeight} and may only take on the
allowed values of \primtype{fontWeight} defined in SBML; that is the
value must be one of the following "normal" or "bold". (Reference: SBML
Level~3 Specification for Render Version~1, \sec{text-class}.)}

\validRule{render-22207}{The value of the attribute
\token{render:\-font-style} of a \Text object must conform to the syntax
of SBML data type \primtype{fontStyle} and may only take on the allowed
values of \primtype{fontStyle} defined in SBML; that is the value must
be one of the following "normal" or "italic". (Reference: SBML Level~3
Specification for Render Version~1, \sec{text-class}.)}

\validRule{render-22208}{The value of the attribute
\token{render:\-text-anchor} of a \Text object must conform to the
syntax of SBML data type \primtype{hTextAnchor} and may only take on the
allowed values of \primtype{hTextAnchor} defined in SBML; that is the
value must be one of the following "start", "middle" or "end".
(Reference: SBML Level~3 Specification for Render Version~1,
\sec{text-class}.)}

\validRule{render-22209}{The value of the attribute
\token{render:\-vtext-anchor} of a \Text object must conform to the
syntax of SBML data type \primtype{vTextAnchor} and may only take on the
allowed values of \primtype{vTextAnchor} defined in SBML; that is the
value must be one of the following "top", "middle", "bottom" or
"baseline". (Reference: SBML Level~3 Specification for Render Version~1,
\sec{text-class}.)}


\subsubsection*{Rules for \class{Transformation2D} object}

\validRule{render-22301}{A \TransformationTwoD object may have the
optional SBML Level~3 Core attributes \token{metaid} and
\token{sboTerm}. No other attributes from the SBML Level 3 Core
namespaces are permitted on a \TransformationTwoD. (Reference: SBML
Level~3 Version~1 Core, Section~3.2.)}

\validRule{render-22302}{A \TransformationTwoD object may have the
optional SBML Level~3 Core subobjects for notes and annotations. No
other elements from the SBML Level 3 Core namespaces are permitted on a
\TransformationTwoD. (Reference: SBML Level~3 Version~1 Core,
Section~3.2.)}

\validRule{render-22303}{A \TransformationTwoD object may have the
optional attribute \token{render:\-transform}. No other attributes from
the SBML Level 3 Render namespaces are permitted on a
\TransformationTwoD object. (Reference: SBML Level~3 Specification for
Render Version~1, \sec{transformationtwod-class}.)}

\validRule{render-22304}{The value of the attribute
\token{render:\-transform} of a \TransformationTwoD object must be an
array of values of type \token{double}. (Reference: SBML Level~3
Specification for Render Version~1, \sec{transformationtwod-class}.)}


\subsubsection*{Rules for \class{Transformation} object}

\validRule{render-22401}{A \Transformation object may have the optional
SBML Level~3 Core attributes \token{metaid} and \token{sboTerm}. No
other attributes from the SBML Level 3 Core namespaces are permitted on
a \Transformation. (Reference: SBML Level~3 Version~1 Core,
Section~3.2.)}

\validRule{render-22402}{A \Transformation object may have the optional
SBML Level~3 Core subobjects for notes and annotations. No other
elements from the SBML Level 3 Core namespaces are permitted on a
\Transformation. (Reference: SBML Level~3 Version~1 Core, Section~3.2.)}

\validRule{render-22403}{A \Transformation object must have the required
attribute \token{render:\-transform}. No other attributes from the SBML
Level 3 Render namespaces are permitted on a \Transformation object.
(Reference: SBML Level~3 Specification for Render Version~1,
\sec{transformation-class}.)}

\validRule{render-22404}{The value of the attribute
\token{render:\-transform} of a \Transformation object must be an array
of values of type \token{double}. (Reference: SBML Level~3 Specification
for Render Version~1, \sec{transformation-class}.)}


\subsubsection*{Rules for \class{GraphicalPrimitive1D} object}

\validRule{render-22501}{A \GraphicalPrimitiveOneD object may have the
optional SBML Level~3 Core attributes \token{metaid} and
\token{sboTerm}. No other attributes from the SBML Level 3 Core
namespaces are permitted on a \GraphicalPrimitiveOneD. (Reference: SBML
Level~3 Version~1 Core, Section~3.2.)}

\validRule{render-22502}{A \GraphicalPrimitiveOneD object may have the
optional SBML Level~3 Core subobjects for notes and annotations. No
other elements from the SBML Level 3 Core namespaces are permitted on a
\GraphicalPrimitiveOneD. (Reference: SBML Level~3 Version~1 Core,
Section~3.2.)}

\validRule{render-22503}{A \GraphicalPrimitiveOneD object may have the
optional attributes \token{render:\-id}, \token{render:\-stroke},
\token{render:\-stroke-width} and \token{render:\-stroke-dasharray}. No
other attributes from the SBML Level 3 Render namespaces are permitted
on a \GraphicalPrimitiveOneD object. (Reference: SBML Level~3
Specification for Render Version~1,
\sec{graphicalprimitiveoned-class}.)}

\validRule{render-22504}{The attribute \token{render:\-stroke} on a
\GraphicalPrimitiveOneD must have a value of data type \token{string}.
(Reference: SBML Level~3 Specification for Render Version~1,
\sec{graphicalprimitiveoned-class}.)}

\validRule{render-22505}{The attribute \token{render:\-stroke-width} on
a \GraphicalPrimitiveOneD must have a value of data type \token{string}.
(Reference: SBML Level~3 Specification for Render Version~1,
\sec{graphicalprimitiveoned-class}.)}

\validRule{render-22506}{The value of the attribute
\token{render:\-stroke-dasharray} of a \GraphicalPrimitiveOneD object
must be an array of values of type \token{unsigned integer}. (Reference:
SBML Level~3 Specification for Render Version~1,
\sec{graphicalprimitiveoned-class}.)}


\subsubsection*{Rules for \class{GraphicalPrimitive2D} object}

\validRule{render-22601}{A \GraphicalPrimitiveTwoD object may have the
optional SBML Level~3 Core attributes \token{metaid} and
\token{sboTerm}. No other attributes from the SBML Level 3 Core
namespaces are permitted on a \GraphicalPrimitiveTwoD. (Reference: SBML
Level~3 Version~1 Core, Section~3.2.)}

\validRule{render-22602}{A \GraphicalPrimitiveTwoD object may have the
optional SBML Level~3 Core subobjects for notes and annotations. No
other elements from the SBML Level 3 Core namespaces are permitted on a
\GraphicalPrimitiveTwoD. (Reference: SBML Level~3 Version~1 Core,
Section~3.2.)}

\validRule{render-22603}{A \GraphicalPrimitiveTwoD object may have the
optional attributes \token{render:\-fill} and
\token{render:\-fill-rule}. No other attributes from the SBML Level 3
Render namespaces are permitted on a \GraphicalPrimitiveTwoD object.
(Reference: SBML Level~3 Specification for Render Version~1,
\sec{graphicalprimitivetwod-class}.)}

\validRule{render-22604}{The attribute \token{render:\-fill} on a
\GraphicalPrimitiveTwoD must have a value of data type \token{string}.
(Reference: SBML Level~3 Specification for Render Version~1,
\sec{graphicalprimitivetwod-class}.)}

\validRule{render-22605}{The value of the attribute
\token{render:\-fill-rule} of a \GraphicalPrimitiveTwoD object must
conform to the syntax of SBML data type \primtype{fillRule} and may only
take on the allowed values of \primtype{fillRule} defined in SBML; that
is the value must be one of the following "nonzero" or "evenodd".
(Reference: SBML Level~3 Specification for Render Version~1,
\sec{graphicalprimitivetwod-class}.)}


\subsubsection*{Rules for \class{Style} object}

\validRule{render-22701}{A \Style object may have the optional SBML
Level~3 Core attributes \token{metaid} and \token{sboTerm}. No other
attributes from the SBML Level 3 Core namespaces are permitted on a
\Style. (Reference: SBML Level~3 Version~1 Core, Section~3.2.)}

\validRule{render-22702}{A \Style object may have the optional SBML
Level~3 Core subobjects for notes and annotations. No other elements
from the SBML Level 3 Core namespaces are permitted on a \Style.
(Reference: SBML Level~3 Version~1 Core, Section~3.2.)}

\validRule{render-22703}{A \Style object may have the optional
attributes \token{render:\-id}, \token{render:\-name},
\token{render:\-roleList} and \token{render:\-typeList}. No other
attributes from the SBML Level 3 Render namespaces are permitted on a
\Style object. (Reference: SBML Level~3 Specification for Render
Version~1, \sec{style-class}.)}

\validRule{render-22704}{A \Style object may contain one and only one
instance of the RenderGroup element. No other elements from the SBML
Level 3 Render namespaces are permitted on a \Style object. (Reference:
SBML Level~3 Specification for Render Version~1, \sec{style-class}.)}

\validRule{render-22705}{The attribute \token{render:\-name} on a \Style
must have a value of data type \token{string}. (Reference: SBML Level~3
Specification for Render Version~1, \sec{style-class}.)}

\validRule{render-22706}{The attribute \token{render:\-roleList} on a
\Style must have a value of data type \token{string}. (Reference: SBML
Level~3 Specification for Render Version~1, \sec{style-class}.)}

\validRule{render-22707}{The attribute \token{render:\-typeList} on a
\Style must have a value of data type \token{string}. (Reference: SBML
Level~3 Specification for Render Version~1, \sec{style-class}.)}


\subsubsection*{Rules for \class{RenderInformationBase} object}

\validRule{render-22801}{A \RenderInformationBase object may have the
optional SBML Level~3 Core attributes \token{metaid} and
\token{sboTerm}. No other attributes from the SBML Level 3 Core
namespaces are permitted on a \RenderInformationBase. (Reference: SBML
Level~3 Version~1 Core, Section~3.2.)}

\validRule{render-22802}{A \RenderInformationBase object may have the
optional SBML Level~3 Core subobjects for notes and annotations. No
other elements from the SBML Level 3 Core namespaces are permitted on a
\RenderInformationBase. (Reference: SBML Level~3 Version~1 Core,
Section~3.2.)}

\validRule{render-22803}{A \RenderInformationBase object must have the
required attribute \token{render:\-id}, and may have the optional
attributes \token{render:\-name}, \token{render:\-programName},
\token{render:\-programVersion},
\token{render:\-refenceRenderInformation} and
\token{render:\-backgroundColor}. No other attributes from the SBML
Level 3 Render namespaces are permitted on a \RenderInformationBase
object. (Reference: SBML Level~3 Specification for Render Version~1,
\sec{renderinformationbase-class}.)}

\validRule{render-22804}{A \RenderInformationBase object may contain one
and only one instance of each of the \ListOfColorDefinitions,
\ListOfGradientBases and \ListOfLineEndings elements. No other elements
from the SBML Level 3 Render namespaces are permitted on a
\RenderInformationBase object. (Reference: SBML Level~3 Specification
for Render Version~1, \sec{renderinformationbase-class}.)}

\validRule{render-22805}{The attribute \token{render:\-name} on a
\RenderInformationBase must have a value of data type \token{string}.
(Reference: SBML Level~3 Specification for Render Version~1,
\sec{renderinformationbase-class}.)}

\validRule{render-22806}{The attribute \token{render:\-programName} on a
\RenderInformationBase must have a value of data type \token{string}.
(Reference: SBML Level~3 Specification for Render Version~1,
\sec{renderinformationbase-class}.)}

\validRule{render-22807}{The attribute \token{render:\-programVersion}
on a \RenderInformationBase must have a value of data type
\token{string}. (Reference: SBML Level~3 Specification for Render
Version~1, \sec{renderinformationbase-class}.)}

\validRule{render-22808}{The attribute
\token{render:\-refenceRenderInformation} on a \RenderInformationBase
must have a value of data type \token{string}. (Reference: SBML Level~3
Specification for Render Version~1, \sec{renderinformationbase-class}.)}

\validRule{render-22809}{The attribute \token{render:\-backgroundColor}
on a \RenderInformationBase must have a value of data type
\token{string}. (Reference: SBML Level~3 Specification for Render
Version~1, \sec{renderinformationbase-class}.)}

\validRule{render-22810}{The \ListOfColorDefinitions,
\ListOfGradientBases and \ListOfLineEndings subobjects on a
\RenderInformationBase object is optional, but if present, these
container objects must not be empty. (Reference: SBML Level~3
Specification for Render Version~1, \sec{renderinformationbase-class}.)}

\validRule{render-22811}{Apart from the general notes and annotations
subobjects permitted on all SBML objects, a \ListOfColorDefinitions
container object may only contain \ColorDefinition objects. (Reference:
SBML Level~3 Specification for Render Version~1,
\sec{renderinformationbase-class}.)}

\validRule{render-22812}{Apart from the general notes and annotations
subobjects permitted on all SBML objects, a \ListOfGradientBases
container object may only contain \GradientBase objects. (Reference:
SBML Level~3 Specification for Render Version~1,
\sec{renderinformationbase-class}.)}

\validRule{render-22813}{Apart from the general notes and annotations
subobjects permitted on all SBML objects, a \ListOfLineEndings container
object may only contain \LineEnding objects. (Reference: SBML Level~3
Specification for Render Version~1, \sec{renderinformationbase-class}.)}

\validRule{render-22814}{A \ListOfColorDefinitions object may have the
optional SBML Level~3 Core attributes \token{metaid} and
\token{sboTerm}. No other attributes from the SBML Level 3 Core
namespaces are permitted on a \ListOfColorDefinitions object.
(Reference: SBML Level~3 Specification for Render Version~1,
\sec{renderinformationbase-class}.)}

\validRule{render-22815}{A \ListOfGradientBases object may have the
optional SBML Level~3 Core attributes \token{metaid} and
\token{sboTerm}. No other attributes from the SBML Level 3 Core
namespaces are permitted on a \ListOfGradientBases object. (Reference:
SBML Level~3 Specification for Render Version~1,
\sec{renderinformationbase-class}.)}

\validRule{render-22816}{A \ListOfLineEndings object may have the
optional SBML Level~3 Core attributes \token{metaid} and
\token{sboTerm}. No other attributes from the SBML Level 3 Core
namespaces are permitted on a \ListOfLineEndings object. (Reference:
SBML Level~3 Specification for Render Version~1,
\sec{renderinformationbase-class}.)}


