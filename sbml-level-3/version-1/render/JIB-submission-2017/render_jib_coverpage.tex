\documentclass{jib}
\newlength{\platz}
\setlength{\platz}{15pt}
\RequirePackage{listings}

\usepackage{changepage} %test, TODO remove

\lstset{%
  basicstyle=\ttfamily,
  fontadjust,
  flexiblecolumns=true,
  frame=L,
  xleftmargin=15pt,
  framesep=5pt,
  emphstyle=\rmfamily\itshape}
  

%%%%%%%%%%%%%%%%%%%%%%%%%%%%%%%%%%%%%%%%%%%%%%%%%%%%%%%%%%
% JIB Header/Footer
%%%%%%%%%%%%%%%%%%%%%%%%%%%%%%%%%%%%%%%%%%%%%%%%%%%%%%%%%%
%\jibvolume{XX} % insert volume
%\jibissue{X}   % insert issue
%\jibpages{XXX} % insert article ID
%\jibyear{XXXX} % insert year
%\makeHeaderFooter{} % leave as is
%%%%%%%%%%%%%%%%%%%%%%%%%%%%%%%%%%%%%%%%%%%%%%%%%%%%%%%%%%

\begin{document}

%%%%%%%%%%%%%%%%%%%%%%%%%%%%%%%%%%%%%%%%%%%%%%%%%%%%%%%%%%
%
% Title Page
%
%%%%%%%%%%%%%%%%%%%%%%%%%%%%%%%%%%%%%%%%%%%%%%%%%%%%%%%%%%

\begin{jibtitlepage}

\jibtitle{SBML Level 3 package: Render, Version 1, Release 1}


%We did not provide author(s) nor author footnote(s), please complete as applicable.
% Please make sure to use unique footnote characters for each author
\jibauthor{Frank T. Bergmann\iref{caltech}\iref{uh},
           Sarah M. Keating\iref{ebi},
           Ralph Gauges\iref{has},
           Sven Sahle\iref{uh},
           Katja Wengler\iref{uk}
}



\addjibinstitution{caltech}{Department of Computing and Mathematical Sciences,\\California Institute of Technology, Pasadena, CA, USA}
\addjibinstitution{uh}{BioQuant/COS,\\Heidelberg University, Heidelberg, DE}
\addjibinstitution{ebi}{European Bioinformatics Institute (EMBL-EBI),\\Hinxton, Cambrigeshire, UK}
\addjibinstitution{has}{Hochschule Albstadt-Sigmaringe,\\Sigmaringen, DE}
\addjibinstitution{uk}{University of Hertfordshire, \\ Hertfordshire, UK}


\end{jibtitlepage}


% \begin{abstract}
% 
% This document is a short description of the \LaTeX\ \emph{document class} \emph{jib} and its use. It shall be used to submit \LaTeX\ articles to the \emph{Journal of Integrative Bioinformatics} and at the same time may be used to check whether your \LaTeX\ installation can compile output files according to the guidelines of the journal.
% 
% \end{abstract}

% adjusts the width of the abstract, please do not change!
\begin{adjustwidth}{}{1cm}                 

\abstract{
Many software tools provide facilities for depicting reaction network diagrams in a
visual form. Two aspects of such a visual diagram can be distinguished: the layout
(i.e.: the positioning and connections) of the elements in the diagram, and the
graphical form of the elements (for example, the glyphs used for symbols, the
properties of the lines connecting them, and so on). This document describes the  SBML Level~3 \emph{Render} package that complements the SBML Level~3 Layout package and provides a means of capturing the precise rendering of the elements in a diagram.

The SBML Level~3 \emph{Render} package provides a flexible approach to rendering that is independent of both the underlying SBML model and the \emph{Layout} information.  There can be one 
block of render information that applies to all layouts or an additional block for each 
layout.

Many of the elements used in the current render specification are based on 
corresponding elements from the SVG specification. This allows us to easily convert a combination 
of layout information and render information into a SVG drawing.
}

\end{adjustwidth} % please sdo not change

\end{document}
