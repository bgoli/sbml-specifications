% -*- TeX-master: "main"; fill-column: 72 -*-

\section{Examples}
\label{examples}

This section contains a variety of examples of SBML Level~3 Version~1
documents employing the Hierarchical Model Composition package.

\subsection{Simple aggregate model}

The following is a simple aggregate model, with one defined model being
instantiated twice:

\exampleFile{figs/eg-simple-aggregate.xml}

In the model above, we defined a two-step enzymatic process, with
species \val{S} and \val{E} forming a complex, then dissociating to
\val{E} and \val{D}.  The aggregate model instantiates it twice, so the
resulting model \val{aggregate} has two parallel processes in two
parallel compartments performing the same reaction.


\subsection{Example of importing definitions from external files}

Now suppose that we have saved the above SBML content to a file named
\val{enzyme\_model.xml}.  The following example imports the model
\val{enzyme} from that file into a new model:

\exampleFile{figs/eg-import-external.xml}

In the model above, we import ``enzyme'' twice, create a compartment and
species local to the parent model, and replace the compartment
\val{comp} and species \val{S} from the two instantiations with the new
elements.  Now we have a model with a single compartment in which a
species \val{S} can either bind with enzyme \val{A.E} to form \val{A.D},
or bind with enzyme \val{B.E} to form \val{B.D}.


\subsubsection{Example of using ports}

In the following, we define one model (\val{simple}) with a single
reaction involving species \val{S} and \val{D}, and ports, and we
again import model \val{enzyme}:

\exampleFile{figs/eg-ports.xml}

In model \val{simple} above, we give ports to the compartment, the two
species, and the reaction.  Then, in model \val{complexified}, we import
both \val{simple} and the model \val{enzyme} from the file
\val{enzyme\_model.xml}, and replace the simple reaction with the more
complex two-step reaction by deleting reaction \val{J0} from model
\val{simple} and replacing \val{S} and \val{D} from both models with
local replacements.  Note that it is model \val{simple} that defines the
initial concentrations of \val{S} and \val{D}, so our modeler set the
attribute \token{identical} to \val{true} for those elements, faithfully
reproducing the values \val{5} and \val{10} in the local copy, and set
the attribute \token{identical} to \val{false} for the replacement of
those elements from model \val{enzyme}.  Also note that since
\val{simple} defines ports, the \val{port} attribute is used for the
subelements that referenced \val{simple} model elements, but
\val{symbol} still had to be used for subelements referencing
\val{enzyme}.

In the resulting model, \val{S} is converted to \val{D} by a two-step
enzymatic reaction defined wholly in \val{enzyme}, with the initial
concentrations of \val{S} and \val{D} set, in effect, in \val{simple}
(through the appropriate setting of the attribute \token{identical}).
If \val{simple} had other reactions that created \val{S} and destroyed
\val{D}, \val{S} would be created, would bind with \val{E} to form
\val{D}, and \val{D} would then be destroyed, even though those reaction
steps were defined in separate models.


\subsubsection{Example of replacement}

In reference to the previous example, what if we would like to annotate
that the deleted reaction had been \emph{replaced} by the two-step
enzymatic process?  To do that, we must move those reactions to the
parent model, and, since those reactions involve \val{E} and \val{ES},
we must also move those species as well.  The following SBML fragment
demonstrates one way of doing that.

\exampleFile{figs/eg-replacement.xml}

In the example above, we have recreated \val{enzyme} so as to give it
ports, then recreated basically the entire model in the parent
\val{complexified} so we can reference the deleted \val{oldrxn} in the
replacements lists.  Note that we list that reaction deletion both for
the two new reactions and for the two new species \val{E} and \val{ES},
since those species were themselves elided in the simple form of the
\val{S} to \val{D} reaction in \val{simple}.  The attribute
\token{identical} is used throughout, so that any visualization or
manipulation software knows that the only reason those elements exist in
the parent model is to create a reference, not to actually change the
element itself.
