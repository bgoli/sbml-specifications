% -*- TeX-master: "main"; fill-column: 72 -*-

\section{Background and context}
\label{background}

The focus of this section is prior work on the topic of model
composition in SBML.  We also explain how the current specification
relates to that prior work.


\subsection{Prior work on model composition in SBML}
\label{prior-work}

The SBML community has discussed the need to add model composition to
SBML since SBML's very beginning, some ten years ago.  The formulation
of model composition contained in the present document draws
substantially from prior work.  Before we turn to a narrative of the
history that led to the current specification, we want to highlight a
number of individuals for their inspirations and past work
in the development of precursors to this package.  These individuals are
listed in \tab{many-thanks}.

\begin{table}[hb]
  \centering
  \renewcommand{\arraystretch}{1.1}
  \rowcolors{2}{sbmlrowgray}{}
  \begin{edtable}{tabular}{lll}
    \toprule
    \textbf{Contributor}	& \textbf{Affiliation} & \textbf{City and Country}\\
    \midrule
    Michael Blinov		& University of Connecticut Health Center & Farmington, CT, US\\
    Nicolas Le Nov\`{e}re	& EMBL European Bioinformatics Institute & Hinxton, Cambridge, UK\\
    Chris J. Myers		& Electrical and Computer Engineering, University of Utah & Salt Lake City, UT, US\\
    Ranjit Randhawa		& Department of Computer Science, Virginia Tech. & Blacksburg, VA, US\\
    J\"{o}rg Stelling		& Max Planck Institute for Dynamics of Complex Technical Systems & Magdeburg, DE\\
    Jonathan Webb		& BBN Technologies & Cambridge, MA, US\\
    \bottomrule
  \end{edtable}
  \caption{List of individuals who made significant contributions to the
    development of prior SBML proposals that influenced the present version
    of hierarchical model composition.}
  \label{many-thanks}
\end{table}

The first known written proposal for composition in SBML appeared in an
internal discussion document titled \emph{Possible extensions to the
  Systems Biology Markup Language}~\citep{finney:2000} principally authored by
Andrew Finney (and, notably, written even before SBML Level~1 Version~1
was finalized in March of 2001).  The first of the four titular possible
extensions in that document concerns ``submodels'': the main model in a
file can contain a list of submodels, each of which are model
definitions only, and a list of submodel instantiations, each of which
are references to model definitions.  Finney's proposal also
extends the syntax of SBML identifiers (the \primtype{SId} data type) to
allow entity references using a dotted notation, in which \texttt{X.y}
signifies element \texttt{y} of submodel instance \texttt{X}; the
proposal also defines a form of linking model elements through
``substitutions''.  In addition, the proposal also introduces the
concept of validation through what it called the ``expanded'' version of
the model (now commonly referred to as the ``flattened'' form, meaning
translation to a plain SBML format that does not use composition
features): if the flat version of the model is valid, then the model as
a whole must also be valid.

In June of 2001, at the Third Workshop on Software Platforms for Systems
Biology, Martin Ginkel and J\"{o}rg Stelling presented their proposal
titled \emph{XML Notation for Modularity}~\citep{ginkel:2001}, complete with an
accompanying proposal document and sample XML file, partially in
response to deficiencies or missing elements they believed existed in
the proposal by Finney.  In their proposal, Ginkel and Stelling present
a ``classic view'' of modularity, where models are packaged as black
boxes with interfaces.  One of their design goals is to support the
substitution of one module for another with the same defined interface,
thereby supporting the simplification or elaboration of models as
needed.  Their proposal emphasizes the reuse of models and with the
possibility of developing libraries of models.

Martin Ginkel presented an expanded version of that proposal~\citep{ginkel:2002} at
in the July 2002 Fifth Workshop on Software Platforms for Systems
Biology, in the hope that it could be incorporated into the definition
of SBML Level 2 that was being developed at the time.  This proposal
clarified the need to separate model definitions from model
instantiations, and, further, the need to designate one model per
document as the ``main'' model.

In March of 2003, Jonathan Webb produced an independent
proposal~\citep{webb:2003} and circulated it on the mailing list
\link{http://sbml.org/Forums}{sbml-discuss@caltech.edu}.  This proposal
included a unified, generic approach to making links and references to
elements in submodels using XML XPath~\citep{xpath}.  Previous proposals
used separate mechanisms for species, parameters, compartments, and
reactions.  Webb also raised the issue of how to successfully resolve
conflicting attributes of linked elements, debated whether formal
interfaces were necessary or even preferable to directly access model
elements, discussed type-checking for linkages, and discussed issues
with unit incompatibilities.  Around this time, Martin Ginkel formed the
Model Composition Special Interest Group~\citep{comp-sig}, a group that
eventually reached 18 members (including Webb).

Model composition did not make it into SBML Level 2 when that
specification was released in June of 2003, because the changes between
SBML Level~1 and Level~2 were already substantial enough that software
developers at the time expressed a desire to delay the introduction of
composition to a later revision of SBML.  Andrew Finney (now the
co-chair of the Model Composition SIG) presented yet another
proposal~\citep{finney:2003} in May of 2003, even before SBML Level~2 Version~1 was
finalized, that aimed to add model composition to SBML Level~3.  With
only two years having passed between SBML Level~1 and Level~2, the
feeling at the time was that Level~3 was likely to be released in 2005
or 2006, and the model composition proposal would be ready when it was.
However, Level~2 ended up occupying the SBML community longer than
expected, with four versions of Level~2 produced to adjust features in
response to user feedback and developers' experiences.

In the interim, the desire to develop model composition features for
SBML continued unabated.  Finney revised his 2003 proposal in October
2003~\cite{finney:2003b}; this new version represented an attempt to synthesize the
earlier proposals by Ginkel and Webb, supplemented with his own original
submodel ideas, and was envisioned to exist in parallel with another
proposal by Finney, for arrays and sets of SBML elements (including
submodels)~\citep{finney:2003c}.  Finney attempted to resolve the differences in the
two basic philosophies (essentially, black-box versus white-box
encapsulation) by introducing optional ``ports'' as interfaces between a
submodel and its containing model, as well as including an XPath-based
method to allow referencing model entities.  The intention was that a
modeler who wanted to follow the classic modularity (black-box) approach
could do so, but other modelers could still use models in ways not
envisioned by the original modeler simply by accessing a model's
elements directly via XPath-based references.  In both schemes, elements
in the submodels were replaced by corresponding elements of the
containing model.  Finney's proposal also provided a direct link
facility that allows a containing model to refer directly to submodel
elements without providing placeholder elements in the containing
model.  For example, a containing model could have a reaction that
converts a species in one submodel to a species in a different submodel,
and in the direct-link approach, it would only need to define the
reaction, with the reactant and product being expressed as links
directly to the species defined in the submodels.

After Finney's last effort, activities in the SBML community focused on
updates to SBML Level~2, and since model composition was slated for
Level~3, not much progress was made for several years, apart from Finney
including a summary of his 2003 proposal and of some of the unresolved
issues in a poster~\citep{finney:2004} at the 2004 Intelligent Systems for Molecular
Biology (ISMB) conference held in Glasgow.

Finally, in June of 2007, unplanned discussions at the Fifth SBML
Hackathon~\citep{sbml5} prompted the convening of a workshop
specifically to revitalize the model composition package, and in
September of 2007, the SBML Composition Workshop~\citep{comp2007} was
held at the University of Connecticut Health Center, hosted by the
Virtual Cell group and organized by Ion Moraru and Michael Blinov.  The
event produced several artifacts:

\begin{enumerate}

\item Martin Ginkel provided a list of goals for model
  composition~\citep{ginkel:2007}, including use cases, and summarized many of the
  issues described above, including the notion of definition versus
  instantiation, linking, referencing elements that lack SBML
  identifiers, and the creation of optional interfaces.  The list of
  goals also mentioned the need of allowing parameterization of
  instances (i.e., setting new numerical values that override the
  defaults), and the need to be able to ``delete'' or elide elements out
  of submodels.  (He also provided a summary of ProMoT's model
  composition approach and a summary of other approaches.)

\item Andrew Finney wrote a list of issues and comments, recorded on the
  meeting wiki page~\citep{finney:2007}; these included some old issues
  as well as some new ones:

  \begin{itemize}

  \item There should perhaps be a flag for ports to indicate whether a
    given port must be overloaded.

  \item There should be support for N-to-M links, when a set of elements
    in one model are replaced as a group, conceptually, with one or more
    elements from a different model.
    
  \item The proposal should be generic enough to accommodate future
    updates and other Level~3 packages.

  \end{itemize}
  
\item Wolfram Liebermeister presented his group's experience with
  SBMLMerge~\citep{liebermeister:2007}, dealing with the pragmatics of
  merging multiple models.  He also noted that the annotations in a
  composed model need to be considered, particularly since they can be
  crucial to successfully merging models in the first place.

\item On behalf of Ranjit Randhawa, Cliff Shaffer summarized Ranjit's
  work in the JigCell group on model fusion, aggregation, and
  composition~\citep{randhawa:2007}.  Highlights of this presentation
  and work include the following:

  \begin{itemize}

  \item A description of different methods which all need some form of
    model composition, along with the realization that model fusion
    and model composition, though philosophically different, entail
    exactly the same processes and require the same information.

  \item A software application (the JigCell Composition Wizard) that
    can perform conversion between types.  The application can, for
    example, promote a parameter to a species, a concept which had
    been assumed to be impossible and undesirable in previous
    proposals.  

  \item The discovery that merging of SBML models should be done in
    the order Compartments $\rightarrow$ Species  $\rightarrow$
    Function Definitions  $\rightarrow$ Rules  $\rightarrow$ Events
    $\rightarrow$ Units  $\rightarrow$ Reactions  $\rightarrow$
    Parameters.  If done in this order, potential conflicts are
    resolved incrementally along the way.

  \end{itemize}

\item Nicolas Le~Nov\`{e}re created a proposal for SBML modularity in
  Core~\citep{lenov:2007}.  This is actually unrelated to the efforts
  described above; it is an attempt to modularize a ``normal'' SBML
  model in the sense of divvying up the information into modules or
  blocks stored in separate files, rather than composing a model from
  different chunks.  It was agreed at the workshop that this is a
  completely separate idea, and while it has merits, should be handled
  separately.

\item As a collective, the group produced an ``Issues to Address''
  document~\citep{various:2007}, with several conclusions:

  \begin{itemize}

  \item It should be possible to ``flatten'' a composed model to
    produce a valid SBML Level 3 Core model, and all questions of
    validity can then be simply applied to the flattened model.  If
    the Core-only version is valid, the composed model is valid.

  \item The model composition proposal should cover both
    designed-ahead-of-time as well as ad-hoc composition. (The latter
    refers to composing models out of components that were not
    originally developed with the use of ports or the expectation of
    being incorporated into other models.)

  \item The approach probably needs a mechanism for deleting SBML
    model elements.  The deletion syntax should be explicit, instead
    of being implied by (e.g.)\ using a generic replacement construct
    and omitting the target of the replacement.

  \item It should be possible to link any part of a model, not just
    (e.g.)\ compartments, species and parameters.

  \item The approach should support item ``object
    overloading''~\citep{various:2007b} and be generally applicable to
    all SBML objects.  However, contrary to what is provided in the
    JigCell Composition Wizard, changing SBML component types is not
    supported in object overloading.

  \item A proposition made during the workshop is that elements in the
    outer model always override elements in the submodels, and perhaps
    that sibling linking be disallowed.  This idea was hotly debated.

  \item Interfaces (ports) are indeed considered helpful, but should
    be optional.  They do not need to be directional as in the
    electrical engineering ``input'' and ``output'' sense---the outer
    element always overrides the inner element, but apart from that,
    biology does not tend to work in the directional way that
    electrical components do.

  \item The ability to refer to or import external files may need a
    mechanism to allow an application to check whether what is being
    imported is the same as it was when the modeler created the
    model.  The mechanism offered in this context was the use of MD5
    hashes.

  \item A model composition approach should probably only allow
    whole-model imports, not importing of individual SBML elements
    such as species or reactions.  The reason is that model components
    are invariably defined within a larger context, and attempting to
    pull a single piece out of a model is unlikely to be safe or
    desirable.

  \item The model composition approach must provide a means to handle
    the conversion of units, so that the units of entities defined in
    a submodel can be made congruent with the entities that refer to
    them in the enclosing model.

  \end{itemize}

\end{enumerate}

During the workshop, the attendees worked on a draft proposal.  Stefan
Hoops acted as principal editor.  The proposal for the SBML package
(which was renamed \emph{Hierarchical Model
  Composition}~\citep{hoops:2007}), was issued one day after the end of
the workshop.  It represented an attempt to summarize the workshop as a
whole, and provide a coherent whole, suitable as a Level~3 package.  It
provided a brief overview of the history and goals of the proposal, as
well as several UML diagrams of the proposed data structures.  Hoops
presented~\citep{hoops:2008} the proposal in August, 2008, at the 13th
SBML Forum, and again at the 7th SBML Hackathon in March of 2009 as well
as the 14th SBML Forum in September of 2009, in a continuing effort to
raise interest.

Roughly concurrently, Herbert Sauro, one of the original developers of
SBML, received a grant to develop a modular human-readable model
definition language, and hired Lucian Smith in November of 2007 to work
on the project.  Sauro and Frank Bergmann, then a graduate student with
Herbert, had previously written a proposal~\citep{bergmann:2006} for a
human-readable language that provided composition features, and this was
the design document Smith initially used to create a software system
that was eventually called \emph{Antimony}. Through a few iterations,
the design eventually settled on was very similar in concept (largely by
coincidence) to that developed by the group at the 2007 Connecticut
workshop: namely, with model definitions placed separately from their
instantiations in other models, and with the ability to link (or
``synchronize'', in Antimony terminology) elements of models with each
other.  Because Antimony was designed to be ``quick and dirty'', it
allowed type conversions much like the JigCell Composition Wizard,
whereby a parameter could become a species, compartment, or even
reaction.  Synchronized elements could end up with aspects of both
parent elements in their final definitions: if one element defined a
starting condition and the other how it changed in time, the final
element would have both.  If both elements defined the same aspect (like
a starting condition), the one designated the ``default'' would be used
in the final version.  Smith developed methods to import other Antimony
files and even SBML models, which could then be used as submodels of
other models and exported as flattened SBML.

At the SBML-BioModels.net Hackathon in 2010, in response to popular
demand from people who attended the workshop, Smith put together a short
presentation~\citep{smith:2010} about model composition and some of the
limitations he found with the 2007 proposal.  He proposed separating the
replacement concept (where old references to replaced values are still
valid) from the deletion concept (where old references to replaced
values are no longer valid).  Smith wrote a summary of that discussion,
added some more of thoughts, and posted it to the
\link{http://sbml.org/Forums}{sbml-discuss@caltech.edu} mailing
list~\citep{smith:2010b}.  In this posting, he proposed and/or reported
several possible modifications to the Hoops et al.\ 2007 proposal,
including the following:

\begin{itemize}\setlength{\parskip}{0ex}

\item Separation of \emph{replacement} from \emph{deletion}.

\item Separation of model definition from instantiation.

\item Elimination of ports, and the use of annotations instead.

\item Annotation for identifying N-to-M replacements, instead of giving
  them their own construct.

\end{itemize}

The message to \link{http://sbml.org/Forums}{sbml-discuss@caltech.edu}
was met with limited discussion.  However, it turns out that several of
the issues raised by Smith were brought up at the 2007 meeting, and had
simply been missed in the generation of the (incomplete) proposal after
the workshop.  The meeting attendees had, for example, originally
preferred to differentiate deletions from replacements more strongly
than by simply having an empty list of replacements, but omitted this
feature because no better method could be found.  Similarly, the
separation of definitions from instantiations had been in every proposal
up until 2007, and was mentioned in the notes for that meeting.  The
decision to merge the two was a last-minute design decision brought
about when the group noted that if the XInclude~\citep{xinclude}
construct was used, the separation was not strictly necessary from a
technical standpoint.

Smith joined the SBML team in September of 2010, and was tasked with
going through the old proposals and synthesizing from them a new version
that would work with the final incarnation of SBML Level 3.  That
version (the first version of this document) was presented at COMBINE in
October 2010~\citep{smith:2010c}, and further discussed on the
\link{http://sbml.org/Forums}{sbml-discuss@caltech.edu} mailing list.
At HARMONY in April of 2011, consensus was reached on a way forward for
resolving the remaining controversies surrounding the specification,
resulting in the first draft of this document.

Following these decisions, implementation of the proposed specification
followed, with Smith working on implementation in libSBML and integration
into \link{http://antimony.sourceforge.net/}{Antimony}, and Chris Myers 
working on integration into \link{http://www.async.ece.utah.edu/iBioSim/}{iBioSim}.  
A few more changes have been introduced as a result of issues discovered
during this implementation; they have been incorprated into the current
document.


\subsection{Genesis of the current formulation of the package}

The present specification for Hierarchical Model Composition is an
attempt to blend features of previous efforts into a concrete, Level
3-compatible syntax.  The specification has been written from scratch,
but draws strongly on the Hoops 2007 and Finney 2003 proposals, as well
as, to some degree, every one of the sources mentioned above.  Some
practical decisions are new to this proposal, sometimes due to
additional design constraints resulting from the final incarnation of
SBML Level~3, but all of them draw from a wealth of history and
experimentation by many different people over the last decade.  Where
this proposal differs from the historical consensus, the reasoning is
explained, but for the most part, the proposal follows the road most
traveled, and focuses on being clear, simple, only as complex as
necessary, and applicable to the largest number of situations.


\subsection{Design goals for the Hierarchical Model Composition package}
\label{sec:design-goals}

The following are the basic design goals followed in this package:

\begin{itemize}

\item \emph{Allow modelers to build models by aggregation, composition,
    or modularity}.  These methods are so similar to one another, and the
  process of creating an SBML Level 3 package is so involved, that we
  believe it is not advantageous to create one SBML package for
  aggregation and composition, and a separate package for modularity.
  Users of the hierarchical model composition package should be able to
  use and create models in the style that is best suited for their
  individual tasks, using any of these mechanisms, and to exchange and
  reuse models from other groups simply and straightforwardly.

\item \emph{Interoperate cleanly with other packages}. The rules of
  composition should be such that they could apply to any SBML element,
  even unanticipated elements not defined in SBML Level 3 Core and
  introduced by some future Level 3 package.

\item \emph{Allow models produced with these constructs to be valid SBML
    if the constructs are ignored}.  As proposed by Nicolas
  Le~Nov\`{e}re~\citep{lenov:2003} and affirmed by the SBML
  Editors~\citep{editors:2010}, whenever possible, ignoring elements
  defined in a Level 3 package namespace should result in
  syntactically-correct SBML models that can still be interpreted to
  some degree, even if it cannot produce the intended simulation results
  of the full (i.e., interpreting the package constructs) model.  For
  example, inspection and visualization of the Core model should still
  be possible.

\item \emph{Ignore verbosity of models}. We assume that software will
  deal with the ``nuts and bolts'' of reading and writing SBML.  If
  there are two approaches to designing a mechanism for this
  hierarchical composition package, where one approach is clear but
  verbose and the other approach is concise but complex or unobvious, we
  prefer the clear and verbose approach.  We assume that software tools
  can abstract away the verbosity for the user.  (However, tempering
  this goal is the next point.)

\item \emph{Avoid over-complicating the specification}. Apart from the
  base constructs defined by this specification, any new element or
  attribute introduced should have a clear use case that cannot be
  achieved in any other way.

\item \emph{Allow modular access to files outside the modeler's
    control}.  In order to encourage direct referencing of models (e.g.,
  to models hosted online on sites such as BioModels Database
  (\url{http://biomodels.net/database}), whenever possible, we will
  require referenced submodels only to be in SBML Level~3 Core format,
  and not require that they include constructs from this specification.

\item \emph{Incorporate most, if not all, of the desirable features of
    past proposals}. The names may change, but the aims of past efforts
  at SBML model composition should still be achievable with the present
  specification.

\end{itemize}

