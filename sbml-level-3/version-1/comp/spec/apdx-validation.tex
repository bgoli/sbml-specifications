% -*- TeX-master: "main"; fill-column: 72 -*-

\section{Validation of SBML documents}
\label{apdx-validation}

An important issue for software systems is being able to determine the
validity of a given SBML document that uses constructs from the
Hierarchical Model Composition package.  This section describes
operational rules for assessing validity.  


\subsection{Validation procedure}
\label{validation-procedure}

The validation rules below only apply to models
defined in the SBML document being validated.  Models defined in
external files are not required to be valid in and of themselves; the
only requirement is that the model containing the instantiation of an
externally-defined model must be the one that is valid.  That may seem
counterintuitive, but the reason is that replacements and deletions can
be used to render valid what might otherwise be invalid.  For example,
an external model that omits required attributes in some components
(which would be invalid according to SBML Level~3 Version~1 Core) may
become valid if those components are replaced by valid elements, or if
they are deleted entirely.  Similarly, references to nonexistent
components may themselves be deleted, or illegal combinations of objects
may be rectified, etc.


\subsubsection{The two-phase validation approach}

To understand the validation procedure for models that use Hierarchical
Model Composition, it is helpful to think in terms of an analogy to
baking.  To make a cake, one first assembles specific ingredients in a
certain way, and then one bakes the result to produce the final
product---the cake.  An SBML document using the Hierarchical Model
Composition package constructs is analogous to \emph{only a recipe}: it
is a description of how to assemble ingredients in a certain way to
create a ``cake'', but it is \emph{not the cake itself}.  The cake is
only produced after \emph{following} the instructions, which here
involves traversing the various model, submodel, deletion, and
replacement descriptions.

We decompose validation of such a composite model into two phases:

\begin{enumerate}\setlength{\parskip}{0ex}

\item \emph{Validate the ``recipe''}.  The submodel, deletion, and
  replacement constructs themselves must be valid.

\item \emph{Validate the ``cake''}.  The model produced by interpreting
  the various constructs must be valid SBML.

\end{enumerate}

The first phase involves checking the aggregation, deletion and linkage
instructions defined by the Hierarchical Model Composition constructs in
an SBML document.  The \Submodel, \Port, \Deletion, \ReplacedElement,
\ReplacedBy and other constructs defined in this specification must be
valid according to the rules defined in \sect{validation-rules}.
Passing this phase means that the constructs are well-formed, referenced
files and models and other entities exist, ports have identifiers in the
relevant namespaces, and so on.

The second validation phase takes place after interpreting the
Hierarchical Model Composition constructs.  The result of this phase
must be a valid SBML model.  This verification can in principle be
performed in various ways.  In this specification, we describe one
approach below that involves interpreting the Hierarchical Model
Composition constructs to produce a kind of ``flattened'' version of the
model devoid of the Hierarchical Model Composition package constructs.
The ``flattened'' version of the model only exists in memory: the
referenced files are not actually modified, but rather, the
interpretation of the package constructs leads to an in-memory
representation of a final, composite model implied by following the
recipe.  This generated model can then be tested against the rules for
SBML validity defined in the SBML Level~3 Version~1 Core specification.
Performing this ``flattening'' allows for the most straightforward way
of testing the validity of the resulting SBML model; however, it is
\emph{not part of the requirements for this package}.  The requirements
are only that the model implied by the package constructs is valid.


\subsubsection{Example algorithm for producing a ``flattened'' model}

\fig{flattening-algo} presents a possible algorithm for interpreting the
Hierarchical Model Composition constructs and creating a ``flattened''
SBML document.  As explained above, this procedure can be used as part
of a process to test the validity of an SBML document that uses
Hierarchical Model Composition.  After performing the steps of the
flattening algorithm, the result should be evaluated for validity
according to the normal rules of SBML Level~3 Version~1 Core and (if
applicable) the rules defined by any other Level~3 packages used in the
model.

\begin{figure}[thb]
  \renewcommand{\arraystretch}{1.2}
  \newcounter{rownum}
  \setcounter{rownum}{0}
  \begin{edtable}{tabular}{>{\stepcounter{rownum}\therownum.}rp{6in}}
    \toprule
    \multicolumn{1}{r}{\textbf{Step}} & \textbf{Procedure} \\
    \midrule
    & Examine the submodels of the model being validated.  If any
    submodel itself contains submodels, perform this algorithm in its
    entirety on each submodel before proceeding to the next step on the
    model.
    \\
    & For every submodel identifier ``\emph{M}'', verify that no
    component identifier or meta identifier (i.e., the \token{id} or
    \token{metaid} attribute values of the component) begin with the
    character sequence ``\emph{M\_}''.  For example, if a submodel has
    identifier \val{foo}, change it (for the purposes of this algorithm)
    to \val{foo\_}.  If a component has identifier ``\val{foo\_}''
    already, add another underscore.
    \\
    & Remove any submodel components that have been replaced or deleted.
    \\
    & Transform the remaining submodel components such that:
    \begin{enumerate}[label={\alph*})]

    \item Every identifier (\token{id} attribute) is changed to a value of
      the form ``\emph{M\_id}'', where ``\emph{id}'' is the original
      \token{id} value and ``\emph{M}'' is the submodel identifier.

    \item Every meta identifier (\token{metaid} attribute) is changed to a
      value of the form ``\emph{M\_metaid}'', where ``\emph{metaid}'' is
      the original \token{metaid} value and ``\emph{M}'' is the submodel
      identifier.

    \vspace*{-1em}
    \end{enumerate}
    \\
    & For every component that has not been removed in the submodel,
    adjust every reference to the identifiers that were transformed in
    the previous step as follows.  Change every \primtype{SIdRef} value
    to the form form ``\emph{M\_id}'', where ``\emph{id}'' is the
    original \primtype{SIdRef} value and ``\emph{M}'' is the submodel
    identifier, and change every XML \primtype{IDREF} value to the form
    ``\emph{M\_metaid}'', where ``\emph{metaid}'' is the original
    \primtype{IDREF} value.  \emph{However}, if the submodel has been
    replaced as a consequence of applying either a \ReplacedElement or
    \ReplacedBy construct, the \primtype{SIdRef} and \primtype{IDREF}
    values that pointed to the submodel need to be changed to the
    \primtype{SId} or \primtype{ID} value, respectively, of the object
    replacing it.
    \\
    & After performing the tasks above for all remaining components,
    merge the components of the remaining submodels into a single model.
    The various lists (list of species, list of compartments, etc.) are
    merged in this step.  Annotations must be preserved, as should
    constructs from other SBML Level~3 packages.
    \\
    \bottomrule
  \end{edtable}
  \caption{Example algorithm for ``flattening'' a model to remove
    Hierarchical Model Composition package constructs.}
  \label{flattening-algo}
\end{figure}


\subsubsection{Additional remarks about the validation procedure}

When instantiating a model, one does not have to first test the validity
of that model.  If it is in the same file as the containing model, it
will be tested anyway when the result of the ``flattening'' algorithm is
checked for validity in the second phase.  If it is in a different file,
that file's validity (or lack thereof) should not affect the validity of
the file being tested, though a validator may warn the user of this
situation if it desires.


\subsection{Validation and consistency rules}
\label{validation-rules}

This section summarizes all the conditions that must (or in some cases,
at least \emph{should}) be true of an SBML Level~3 Version~1 model that
uses the Hierarchical Model Composition package.  We use the same
conventions as are used in the SBML Level~3 Version~1 Core specification
document.  In particular, there are different degrees of rule
strictness.  Formally, the differences are expressed in the statement of
a rule: either a rule states that a condition \emph{must} be true, or a
rule states that it \emph{should} be true.  Rules of the former kind are
strict SBML validation rules---a model encoded in SBML must conform to
all of them in order to be considered valid.  Rules of the latter kind
are consistency rules.  To help highlight these differences, we use the
following three symbols next to the rule numbers:

\begin{description}

\item[\hspace*{6.5pt}\vSymbol\vsp] A \vSymbolName indicates a
  \emph{requirement} for SBML conformance. If a model does not follow
  this rule, it does not conform to the Hierarchical Model Composition
  specification.  (Mnemonic intention behind the choice of symbol:
  ``This must be checked.'')

\item[\hspace*{6.5pt}\cSymbol\csp] A \cSymbolName indicates a
  \emph{recommendation} for model consistency.  If a model does not
  follow this rule, it is not considered strictly invalid as far as
  the Hierarchical Model Composition specification is concerned;
  however, it indicates that the model contains a physical or
  conceptual inconsistency.  (Mnemonic intention behind the choice of
  symbol: ``This is a cause for warning.'')

\item[\hspace*{6.5pt}\mSymbol\msp] A \mSymbolName indicates a strong
  recommendation for good modeling practice.  This rule is not
  strictly a matter of SBML encoding, but the recommendation comes
  from logical reasoning.  As in the previous case, if a model does
  not follow this rule, it is not strictly considered an invalid SBML
  encoding.  (Mnemonic intention behind the choice of symbol: ``You're
  a star if you heed this.'')

\end{description}

The validation rules listed in the following subsections are all stated
or implied in the reset of this specification document.  They are
enumerated here for convenience.  Unless explicitly stated, all
validation rules concern objects and attributes specifically defined in
the Hierarchical Model Composition package.
\draftnote{[MH] I haven't looked at the rest of this section.}


\subsubsection*{General rules about this package} \begin{sbmlenum}

\validRule{comp-10101}{To conform to the  Hierarchical Model Composition
  package specification for SBML Level~3 Version~1, an SBML document must
  declare the use of the following XML Namespace:\\
  ``\uri{http://www.sbml.org/sbml/level3/version1/comp/version1}''.
  (References: SBML L3V1 comp V1 Section~\ref{}.)}
  

\validRule{comp-10102}{When appearing in an SBML document, all elements
  and attributes from the SBML Level~3 Version~1 Hierarchical Model
  Composition package must be placed in the XML namespace\\
  ``\uri{http://www.sbml.org/sbml/level3/version1/comp/version1}''.
  (References: SBML L3V1 comp V1 Section~\ref{}.) }


\end{sbmlenum} \subsubsection*{General rules about identifiers} \begin{sbmlenum}

\validRule{comp-10301}{(Extending the SBML Level~3 Version~1 Core
  validation rule \#10301) Within a \Model or \ExternalModelDefinition
  object, the value of the attribute \token{id} and \token{comp:id} on
  every instance of the following classes of objects must be unique
  across the set of all \token{id} and \token{comp:id} attribute values
  of all such objects in a model: the \Model itself, plus all contained
  \FunctionDefinition, \Compartment, \Species, \Reaction,
  \SpeciesReference, \ModifierSpeciesReference, \Event, and \Parameter
  objects, plus the newly-defined objects \Submodel and \Deletion.
  (References: SBML L3V1 comp V1 Section~\ref{namespaces}.) }


\validRule{comp-10302}{Within an SBMLDocument, the value of the
  attribute \token{id} and \token{comp:id} on every instance of all
  \Model and \ExternalModelDefinition objects must
  be unique across the set of all \token{id} and \token{comp:id}
  attribute values of such identifiers in the SBML document to which
  they belong.
  (References: SBML L3V1 comp V1 Section~\ref{namespaces}.) }


\validRule{comp-10303}{Within a \Model or \ExternalModelDefinition
  object, the value of the attribute \token{comp:id} on every instance
  of all \Port objects must be unique across the set of all
  \token{comp:id} attribute values of all such objects in the model.
  (References: SBML L3V1 comp V1 Section~\ref{namespaces}.) }


\validRule{comp-10304}{The value of a \token{comp:id} attribute must
  always conform to the syntax of the SBML data type \primtype{SId}.
  (References: SBML L3V1 Section~3.1.7.)}

  
\validRule{comp-10308}{The value of the \token{comp:submodelRef} attribute on
  \ReplacedElement objects must always conform to the syntax of
  the SBML data type \primtype{SId}.
  (References: SBML L3V1 comp V1 Section~\ref{replacedelement-submodelref}.) }
 
  
\validRule{comp-10309}{The value of the \token{comp:deletion} attribute on
  \ReplacedElement objects must always conform to the syntax of
  the SBML data type \primtype{SId}.
  (References: SBML L3V1 comp V1 Section~\ref{replacedelement-deletion}.) }

  
\validRule{comp-10310}{The value of the \token{comp:conversionFactor} attribute on
  \ReplacedElement objects must always conform to the syntax of
  the SBML data type \primtype{SId}.
  (References: SBML L3V1 comp V1 Section~\ref{replacedelement-conversionfactor}.) }
  

\end{sbmlenum} \subsubsection*{Rules for the extended \class{SBase} abstract object} \begin{sbmlenum}

\validRule{comp-20101}{Any \SBase objects (that is, all elements inheriting
  from the \SBase class, as defined in this package) may contain at most one 
  instance of the \ListOfReplacedElements object.
  (References: SBML L3V1 comp V1 Section~\ref{extended-sbase-class}.) }


\validRule{comp-20102}{Apart from the general notes and annotation
  subobjects permitted on all SBML components, a \ListOfReplacedElements
  container object may only contain \ReplacedElement objects. 
  (References: SBML L3V1 comp V1 Section~\ref{extended-sbase-class}.) }


\validRule{comp-20103}{A \ListOfReplacedElements object may have the optional 
  SBML core attributes \token{metaid} and \token{sboTerm}.  No other attributes 
  from the SBML Level~3 Core namespace or the comp namespace are permitted on 
  a \ListOfReplacedElements object. 
  (References: SBML L3V1 comp V1 Section~\ref{extended-sbase-class}.) }


\validRule{comp-20104}{The \ListOfReplacedElements in an \SBase object is
  optional, but if present, must not be empty.
  (References: SBML L3V1 comp V1 Section~\ref{extended-sbase-class}.) }


\end{sbmlenum} \subsubsection*{Rules for the extended \class{SBML} container object} \begin{sbmlenum}

\validRule{comp-20201}{The \token{sbml} container element must declare whether 
  the comp package alters the mathematics of the model using the attribute
  \token{required}.  
  (References: SBML L3V1 Section~4.1.2.) }
  

\validRule{comp-20202}{The attribute \token{comp:required} on the \token{sbml} 
  container element must have a value of type \token{boolean}.  
  (References: SBML L3V1 Section~4.1.2.) }
  

\validRule{comp-20203}{The attribute \token{comp:required} on the \token{sbml} 
  container element must be set to \token{true} if its
  \Model child contains any \Submodel objects with \Species, \Parameter,
  \Reaction, or \Event objects (directly or indirectly) that have not been
  replaced. 
  (References: SBML L3V1 comp V1 Section~\ref{???}.) }


\validRule{comp-20204}{There may be at most one instance of each of the
  following kind of object on the \token{sbml} container element:  
  \ListOfModelDefinitions and \ListOfExternalModelDefinitions. 
  (References: SBML L3V1 comp V1 Section~\ref{sbml-class}.) }


\validRule{comp-20205}{The various \token{listOf\_} container objects on 
  the \token{sbml} container element are optional, but if present, such 
  container elements must not be empty. Specifically, if any of the following 
  is present on the \token{sbml} container element, it must not be empty: 
  \ListOfModelDefinitions and \ListOfExternalModelDefinitions.
  (References: SBML L3V1 comp V1 Section~\ref{sbml-class}.) }
  

\validRule{comp-20206}{Apart from the general notes and annotation
  subobjects permitted on all SBML components, a \ListOfModelDefinitions
  container object may only contain ModelDefinition objects.
  (References: SBML L3V1 comp V1 Section~\ref{listofmodeldefinitions-class}.) }


\validRule{comp-20207}{Apart from the general notes and annotation
  subobjects permitted on all SBML components, a
  \ListOfExternalModelDefinitions container object may only contain
  \ExternalModelDefinition objects.
  (References: SBML L3V1 comp V1 Section~\ref{listofexternalmodeldefinitions-class}.) }
  

\validRule{comp-20208}{A \ListOfModelDefinitions object may have the optional 
  SBML core attributes \token{metaid} and \token{sboTerm}.  No other attributes 
  from the SBML Level~3 Core namespace or the comp namespace are permitted on 
  a \ListOfModelDefinitions object.
  (References: SBML L3V1 comp V1 Section~\ref{listofmodeldefinitions-class}.) }


\validRule{comp-20209}{A \ListOfExternalModelDefinitions object may have the optional 
  SBML core attributes \token{metaid} and \token{sboTerm}.  No other attributes 
  from the SBML Level~3 Core namespace or the comp namespace are permitted on 
  a \ListOfExternalModelDefinitions object.
  (References: SBML L3V1 comp V1 Section~\ref{listofexternalmodeldefinitions-class}.) }

\end{sbmlenum} \subsubsection*{Rules for \class{ExternalModelDefinition} objects} \begin{sbmlenum}

\validRule{comp-20301}{An \ExternalModelDefinition object derives from the SBML
  Level~3 Core \class{SBase} and as such may have the optional attributes
  \token{metaid} and \token{sboTerm}. No other attributes from the SBML
  Level~3 Core namespace are permitted on an \ExternalModelDefinition object.
  (References: SBML L3V1 Section~3.2.) }
   

\validRule{comp-20302}{An \ExternalModelDefinition object derives from the SBML
  Level~3 Core \class{SBase} and as such may have the optional subobjects
  notes and annotation. No other elements from the SBML
  Level~3 Core namespace are permitted on an \ExternalModelDefinition object.
  (References: SBML L3V1 Section~3.2.) }
   

\validRule{comp-20303}{An ExternalModelDefinition object must have the
  required attributes \token{comp:id} and \token{comp:source}, and may have
  the optional attributes \token{comp:modelRef} and \token{comp:md5}.
  No other attributes from the comp namespace are permitted on an
  \ExternalModelDefinition object.
  (References: SBML L3V1 comp V1 Section~\ref{externalmodeldefinition-class}.) }


\validRule{comp-20304}{The value of the \token{comp:source} attribute on an
  \ExternalModelDefinition object must point to a SBML Level 3 document.
  (References: SBML L3V1 comp V1 Section~\ref{externalmodeldefinition-class}.) }


\validRule{comp-20305}{The value of the \token{comp:modelRef} attribute on an
  \ExternalModelDefinition object must be the \token{id} of a \Model
  in the SBML document pointed to by the \token{comp:source} attribute.
  (References: SBML L3V1 comp V1 Section~\ref{externalmodeldefinition-class}.) }


\consistencyRule{comp-20306}{The value of the \token{comp:md5} attribute on an
  \ExternalModelDefinition object should match the
  calculated MD5 checksum of the SBML document pointed to by the 
  \token{comp:source} attribute.
  (References: SBML L3V1 comp V1 Section~\ref{externalmodeldefinition-class}.) }


\validRule{comp-20307}{The value of the \token{comp:source} attribute on
  \ExternalModelDefinition objects must always conform to the syntax of
  the XML Schema 1.0 data type \primtype{anyURI}.
  (References: SBML L3V1 comp V1 
  Section~\ref{listofexternalmodeldefinitions-class}.) }


\validRule{comp-20308}{The value of \token{comp:modelRef} attributes on
  \ExternalModelDefinition objects must always conform to the syntax of
  the SBML data type \primtype{SId}.
  (References: SBML L3V1 comp V1 
  Section~\ref{listofexternalmodeldefinitions-class}.) }


\validRule{comp-20309}{The value of the \token{comp:md5} attribute on
  \ExternalModelDefinition objects must always conform to the syntax of
  type \primtype{string}.
  (References: SBML L3V1 comp V1 
  Section~\ref{listofexternalmodeldefinitions-class}.) }

\validRule{comp-20310}{No \ExternalModelDefinition may reference an
  \ExternalModelDefinition in a different SBML document that in turn
  refers to the original \ExternalModelDefinition object, whether
  directly or indirectly through a chain of \ExternalModelDefinition
  objects.
  (References: SBML L3V1 comp V1 Section~\ref{???}.) }

\end{sbmlenum} \subsubsection*{Rules for \class{ModelDefinition} objects} \begin{sbmlenum}

\validRule{comp-20401}{A ModelDefinition object derives from the SBML
  Level~3 Core \class{Model} and as such must follow the same restrictions 
  present on Level~3 Core \Model objects.
  This includes any validation rules from the SBML Level 3 Version 1
  core specification as well as this document.
   (References: SBML L3V1 Section~4.2.) }

\end{sbmlenum} \subsubsection*{Rules for extended \class{Model} objects} \begin{sbmlenum}

\validRule{comp-20501}{There may be at most one instance of each of the
  following kind of object on the \Model: \ListOfSubmodels and \ListOfPorts. 
  (References: SBML L3V1 comp V1 Section~\ref{model-class}.) }


\validRule{comp-20502}{The various \token{listOf\_} container objects on 
  the \Model element are optional, but if present, such 
  container elements must not be empty. Specifically, if any of the following 
  is present on the \Model element, it must not be empty: 
  \ListOfSubmodels and \ListOfPorts.
  (References: SBML L3V1 comp V1 Section~\ref{model-class}.) }
  

\validRule{comp-20503}{Apart from the general notes and annotation
  subobjects permitted on all SBML components, a \ListOfSubmodels
  container object may only contain \Submodel objects.
  (References: SBML L3V1 comp V1 Section~\ref{model-class}.) }


\validRule{comp-20504}{Apart from the general notes and annotation
  subobjects permitted on all SBML components, a
  \ListOfPorts container object may only contain \Port objects.
  (References: SBML L3V1 comp V1 Section~\ref{model-class}.) }
  

\validRule{comp-20505}{A \ListOfSubmodels object may have the optional 
  SBML core attributes \token{metaid} and \token{sboTerm}.  No other attributes 
  from the SBML Level~3 Core namespace or the comp namespace are permitted on 
  a \ListOfSubmodels object.
  (References: SBML L3V1 comp V1 Section~\ref{model-class}.) }


\validRule{comp-20506}{A \ListOfPorts object may have the optional 
  SBML core attributes \token{metaid} and \token{sboTerm}.  No other attributes 
  from the SBML Level~3 Core namespace or the comp namespace are permitted on 
  a \ListOfPorts object.
  (References: SBML L3V1 comp V1 Section~\ref{model-class}.) }


\end{sbmlenum} \subsubsection*{Rules for \class{Submodel} objects} \begin{sbmlenum}

\validRule{comp-20601}{An \Submodel object derives from the SBML
  Level~3 Core \class{SBase} and as such may have the optional attributes
  \token{metaid} and \token{sboTerm}. No other attributes from the SBML
  Level~3 Core namespace are permitted on an \Submodel object.
  (References: SBML L3V1 Section~3.2.) }
   

\validRule{comp-20602}{An \Submodel object derives from the SBML
  Level~3 Core \class{SBase} and as such may have the optional subobjects
  notes and annotation. No other elements from the SBML
  Level~3 Core namespace are permitted on an \Submodel object.
  (References: SBML L3V1 Section~3.2.) }
   

\validRule{comp-20603}{There may be at most one ListOfDeletions container
  object on the \Submodel. 
  (References: SBML L3V1 comp V1 Section~\ref{listofdeletions-class}.) }


\validRule{comp-20604}{The ListOfDeletions container object on 
  the \Submodel element is optional, but if present, must not be empty. 
  (References: SBML L3V1 comp V1 Section~\ref{listofdeletions-class}.) }
  

\validRule{comp-20605}{Apart from the general notes and annotation
  subobjects permitted on all SBML components, a \ListOfDeletions
  container object may only contain \Deletion objects.
  (References: SBML L3V1 comp V1 Section~\ref{listofdeletions-class}.) }


\validRule{comp-20606}{A \ListOfDeletions object may have the optional 
  SBML core attributes \token{metaid} and \token{sboTerm}.  No other attributes 
  from the SBML Level~3 Core namespace or the comp namespace are permitted on 
  a \ListOfDeletions object.
  (References: SBML L3V1 comp V1 Section~\ref{listofdeletions-class}.) }


\validRule{comp-20607}{A \Submodel object must have the
  required attributes \token{comp:id} and \token{comp:modelRef}, and may have
  the optional attributes \token{comp:lengthConversionFactor},
  \token{comp:areaConversionFactor}, \token{comp:volumeConversionFactor},
  \token{comp:substanceConversionFactor}, \token{comp:timeConversionFactor}, and
  \token{comp:extentConversionFactor}.
  No other attributes from the comp namespace are permitted on a
  \Submodel object.
  (References: SBML L3V1 comp V1 Section~\ref{submodel-class}.) }


\validRule{comp-20608}{The value of the \token{comp:modelRef} attribute 
  on \Submodel objects must always conform to the syntax of
  the SBML data type \primtype{SId}.
  (References: SBML L3V1 comp V1 Section~\ref{submodel-modelref}.) }


\validRule{comp-20609}{The value of the \token{comp:lengthConversionFactor} attribute 
  on \Submodel objects must always conform to the syntax of
  the SBML data type \primtype{SId}.
  (References: SBML L3V1 comp V1 Section~\ref{submodel-conversion}.) }


\validRule{comp-20610}{The value of the \token{comp:areaConversionFactor} attribute 
  on \Submodel objects must always conform to the syntax of
  the SBML data type \primtype{SId}.
  (References: SBML L3V1 comp V1 Section~\ref{submodel-conversion}.) }


\validRule{comp-20611}{The value of the \token{comp:volumeConversionFactor} attribute 
  on \Submodel objects must always conform to the syntax of
  the SBML data type \primtype{SId}.
  (References: SBML L3V1 comp V1 Section~\ref{submodel-conversion}.) }


\validRule{comp-20612}{The value of the \token{comp:substanceConversionFactor} attribute 
  on \Submodel objects must always conform to the syntax of
  the SBML data type \primtype{SId}.
  (References: SBML L3V1 comp V1 Section~\ref{submodel-conversion}.) }


\validRule{comp-20613}{The value of the \token{comp:timeConversionFactor} attribute 
  on \Submodel objects must always conform to the syntax of
  the SBML data type \primtype{SId}.
  (References: SBML L3V1 comp V1 Section~\ref{submodel-conversion}.) }


\validRule{comp-20614}{The value of the \token{comp:extentConversionFactor} attribute 
  on \Submodel objects must always conform to the syntax of
  the SBML data type \primtype{SId}.
  (References: SBML L3V1 comp V1 Section~\ref{submodel-conversion}.) }


\validRule{comp-20615}{The \token{comp:modelRef} attribute on a \Submodel must
  be the identifier of a \Model, ModelDefinition, or
  \ExternalModelDefinition object in the same \class{sbml} element as the
  \Submodel. 
  (References: SBML L3V1 comp V1 Section~\ref{submodel-modelref}.) }

\validRule{comp-20616}{No \Model may contain a \Submodel which references
  that model itself.  That is, the \token{modelRef} attribute of a
  \Submodel 
  may not match the
  \token{id} attribute on the parent \Model. 
  (References: SBML L3V1 comp V1 Section~\ref{submodel-modelref}.) }

\validRule{comp-20617}{No \Model may contain a \Submodel which references
  that model indirectly.  That is, the \token{modelRef} attribute of a
  \Submodel may
  not point to the \token{id} of a \Model, any of whose \Submodel objects
  point to the original
  \Model, whether directly or indirectly through a chain of
  \Model/\Submodel pairs. 
  (References: SBML L3V1 comp V1 Section~\ref{submodel-modelref}.) }

\validRule{comp-20618}{The \token{comp:lengthConversionFactor} attribute
   on a \Submodel object must be the identifier of a \Parameter object in the 
   same \Model as the \Submodel.
  (References: SBML L3V1 comp V1 Section~\ref{submodel-conversion}.) }


\validRule{comp-20619}{The \token{comp:areaConversionFactor} attribute
   on a \Submodel object must be the identifier of a \Parameter object in the 
   same \Model as the \Submodel.
  (References: SBML L3V1 comp V1 Section~\ref{submodel-conversion}.) }


\validRule{comp-20620}{The \token{comp:volumeConversionFactor} attribute
   on a \Submodel object must be the identifier of a \Parameter object in the 
   same \Model as the \Submodel.
  (References: SBML L3V1 comp V1 Section~\ref{submodel-conversion}.) }


\validRule{comp-20621}{The \token{comp:substanceConversionFactor} attribute
   on a \Submodel object must be the identifier of a \Parameter object in the 
   same \Model as the \Submodel.
  (References: SBML L3V1 comp V1 Section~\ref{submodel-conversion}.) }


\validRule{comp-20622}{The \token{comp:timeConversionFactor} attribute
   on a \Submodel object must be the identifier of a \Parameter object in the 
   same \Model as the \Submodel.
  (References: SBML L3V1 comp V1 Section~\ref{submodel-conversion}.) }

\validRule{comp-20623}{The \token{comp:extentConversionFactor} attribute
   on a \Submodel object must be the identifier of a \Parameter object in the 
   same \Model as the \Submodel.
  (References: SBML L3V1 comp V1 Section~\ref{submodel-conversion}.) }


\end{sbmlenum} \subsubsection*{Rules for the \class{SBaseRef} abstract object} \begin{sbmlenum}

% was 20204 - question ??
\validRule{comp-20701}{The value of a \token{comp:portRef} attribute on an \SBaseRef
  object must be the identifier of a \Port object from the referenced
  \Model.  
  (References: SBML L3V1 comp V1 Section~\ref{sbaseref-portref}.) }


\validRule{comp-20702}{The value of an \token{comp:idRef} attribute on an \SBaseRef
  object must be the identifier of an object from the referenced \Model
  within the SId namespace for that model.  This includes elements with
  \token{id} attributes which are defined in packages other than Level 3 core or
  this comp package. 
  (References: SBML L3V1 comp V1 Section~\ref{sbaseref-idref}.) }


\validRule{comp-20703}{The value of a \token{comp:unitRef} attribute on an \SBaseRef
  object must be the identifier of a \\UnitDefinition object from the
  referenced \Model.  
  (References: SBML L3V1 comp V1 Section~\ref{sbaseref-unitref}.) }


\validRule{comp-20704}{The value of a \token{comp:metaIdRef} attribute on an \SBaseRef
  object must be the value of a \token{comp:metaid} attribute on any element
  contained in the referenced \Model.  This includes elements with \token{metaid}
  attributes which are defined in packages other than Level 3 core or
  this comp package. 
  (References: SBML L3V1 comp V1 Section~\ref{sbaseref-metaidref}.) }


\validRule{comp-20705}{If an \SBaseRef object contains an \SBaseRef child,
  it must point to a \Submodel element. 
  (References: SBML L3V1 comp V1 Section~\ref{sbaseref-recursive-sbaseref}.) }


\validRule{comp-20706}{The value of the \token{comp:portRef} attribute 
  on \SBaseRef objects must always conform to the syntax of
  the SBML data type \primtype{SId}.
  (References: SBML L3V1 comp V1 Section~\ref{sbaseref-portref}.) }


\validRule{comp-20707}{The value of the \token{comp:idRef}
  attribute on \SBaseRef objects must always conform to the syntax of
  the SBML data type \primtype{SId}.
  (References: SBML L3V1 comp V1 Section~\ref{sbaseref-idref}.) }


\validRule{comp-20708}{The value of the \token{comp:unitRef} attribute on
  \SBaseRef objects must always conform to the syntax of the SBML data
  type \primtype{UnitSId}.
  (References: SBML L3V1 comp V1 Section~\ref{sbaseref-unitref}.) }


\validRule{comp-20709}{The value of the \token{comp:metaIdRef} attributes on
  \SBaseRef objects must always conform to the syntax of the XML data
  type \primtype{ID}.
  (References: SBML L3V1 comp V1 Section~\ref{sbaseref-metaidref}.) }


\end{sbmlenum} \subsubsection*{Rules for \class{Port} objects} \begin{sbmlenum}

\validRule{comp-20801}{A \Port object must point to another object; that is,
  a \Port object must define one of the attributes 
  \token{comp:idRef}, \token{comp:unitRef}, or \token{comp:metaIdRef}. 
  (References: SBML L3V1 comp V1 Section~\ref{port-class}.) }


\validRule{comp-20802}{A \Port object can only point to one other object; that is,
  a \Port object can define only one of the attributes 
  \token{comp:idRef}, \token{comp:unitRef}, or \token{comp:metaIdRef}.
  (References: SBML L3V1 comp V1 Section~\ref{port-class}.) }


\validRule{comp-20803}{A \Port object must have the
  required attribute \token{comp:id}, and one, and only one, of the attributes  
  \token{comp:idRef}, \token{comp:unitRef}, or \token{comp:metaIdRef}.
  No other attributes from the comp namespace are permitted on a
  \Port object.
  (References: SBML L3V1 comp V1 Section~\ref{port-class}.) }


\validRule{comp-20804}{No two \Port objects in the same \Model may
  reference the same XML element.
  (References: SBML L3V1 comp V1 Section~\ref{port-class}.) }

\end{sbmlenum} \subsubsection*{Rules for \class{Deletion} objects} \begin{sbmlenum}

\validRule{comp-20901}{A \Deletion object must point to another object; that is,
  a \Deletion object must define one of the attributes \token{comp:portRef}, 
  \token{comp:idRef}, \token{comp:unitRef}, or \token{comp:metaIdRef}. 
  (References: SBML L3V1 comp V1 Section~\ref{deletion-class}.) }


\validRule{comp-20902}{A \Deletion object can only point to one other object; that is,
  a \Deletion object can define only one of the attributes  \token{comp:portRef}, 
  \token{comp:idRef}, \token{comp:unitRef}, or \token{comp:metaIdRef}.
  (References: SBML L3V1 comp V1 Section~\ref{deletion-class}.) }


\validRule{comp-20903}{A \Deletion object must have the
  required attribute \token{comp:id}, and one, and only one, of the attributes
  \token{comp:portRef}, 
  \token{comp:idRef}, \token{comp:unitRef}, or \token{comp:metaIdRef}.
  No other attributes from the comp namespace are permitted on a
  \Deletion object.
  (References: SBML L3V1 comp V1 Section~\ref{deletion-class}.) }

\end{sbmlenum} \subsubsection*{Rules for \class{ReplacedElement} objects} \begin{sbmlenum}

\validRule{comp-21001}{A \ReplacedElement object must point to another object; that is,
  a \ReplacedElement object must define one of the attributes \token{comp:portRef}, 
  \token{comp:idRef}, \token{comp:unitRef}, \token{comp:metaIdRef}, or \token{comp:deletion}. 
  (References: SBML L3V1 comp V1 Section~\ref{replacedelement-class}.) }


\validRule{comp-21002}{A \ReplacedElement object can only point to one other object; that is,
  a \ReplacedElement object can define only one of the attributes  \token{comp:portRef}, 
  \token{comp:idRef}, \token{comp:unitRef}, \token{comp:metaIdRef}, or \token{comp:deletion}.
  (References: SBML L3V1 comp V1 Section~\ref{replacedelement-class}.) }


\validRule{comp-21003}{A \ReplacedElement object must have the
  required attributes \token{comp:submodelRef} and \token{identical}, and one, 
  and only one, of the attributes \token{comp:portRef}, 
  \token{comp:idRef}, \token{comp:unitRef}, \token{comp:metaIdRef}, or \token{comp:deletion}.
  It may also have the optional attribute \token{comp:conversionFactor}.
  No other attributes from the comp namespace are permitted on a
  \ReplacedElement object.
  (References: SBML L3V1 comp V1 Section~\ref{replacedelement-class}.) }


\validRule{comp-21004}{The value of a \token{comp:submodelRef} attribute on a
  \ReplacedElement object must be the identifier of a \Submodel object
  from the parent \Model of the ReplacedElement.  
  (References: SBML L3V1 comp V1 Section~\ref{replacedelement-submodelref}.) }


\validRule{comp-21005}{The value of a \token{comp:deletion} attribute on a
  \ReplacedElement object must be the identifier of a \Deletion object
  from the parent \Model of the \ReplacedElement.  
  (References: SBML L3V1 comp V1 Section~\ref{replacedelement-identical}.) }


\validRule{comp-21006}{The value of a \token{conversionFactor} attribute on a
  \ReplacedElement object must be the identifier of a \Parameter object
  from the parent \Model of the ReplacedElement.  
  (References: SBML L3V1 comp V1 Section~\ref{replacedelement-conversionfactor}.) }


\validRule{comp-21007}{The value of the \token{comp:submodelRef} attribute on
  \ReplacedElement objects must always conform to the syntax of
  the SBML data type \primtype{SId}.
  (References: SBML L3V1 comp V1 Section~\ref{replacedelement-submodelref}.) }
  

\validRule{comp-21008}{The value of the \token{comp:deletion} attribute on
  \ReplacedElement objects must always conform to the syntax of
  the SBML data type \primtype{SId}.
  (References: SBML L3V1 comp V1 Section~\ref{replacedelement-deletion}.) }
  

\validRule{comp-21009}{The value of the \token{comp:conversionFactor} attribute on
  \ReplacedElement objects must always conform to the syntax of
  the SBML data type \primtype{SId}.
  (References: SBML L3V1 comp V1 Section~\ref{replacedelement-conversionfactor}.) }
  

\validRule{comp-21010}{The value of the \token{comp:identical} attribute on
  \ReplacedElement objects must always conform to the syntax of
  the SBML data type \primtype{bool}.
  (References: SBML L3V1 comp V1 Section~\ref{replacedelement-identical}.) }


\validRule{comp-21011}{No two \ReplacedElement objects in the same \Model
  may reference the same XML element unless that element is a \Deletion.
  (References: SBML L3V1 comp V1 Section~\ref{replacedelement-class}.) }

\end{sbmlenum}
