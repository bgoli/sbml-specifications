% -*- TeX-master: "main"; fill-column: 72 -*-

\section{Validation of SBML documents}
\label{apdx-validation}

An important concern is being able to determine the validity of a given
SBML document that uses constructs from the Hierarchical Model
Composition package.  This section describes operational rules for
assessing validity.


\subsection{Validation procedure}
\label{validation-procedure}

The validation rules below only apply to models defined in the SBML
document being validated.  Models defined in external files are not
required to be valid in and of themselves; the only requirement is that
the model containing the instantiation of an externally-defined model
must be the one that is valid.  That may seem counterintuitive, but the
reason is that replacements and deletions can be used to render valid
what might otherwise be invalid.  For example, an external model that
omits required attributes on some objects (which would be invalid
according to SBML Level~3 Version~1 Core) may become valid if those
objects are replaced by objects that \emph{are} valid, or if they are
deleted entirely.  Similarly, references to nonexistent objects may
themselves be deleted, or illegal combinations of objects may be
rectified, etc.


\subsubsection{The two-phase validation approach}

To understand the validation procedure for models that use Hierarchical
Model Composition, it is helpful to think in terms of an analogy to
baking.  To make a cake, one first assembles specific ingredients in a
certain way, and then one bakes the result to produce the final
product---the cake.  An SBML document using the Hierarchical Model
Composition package constructs is analogous to \emph{only a recipe}: it
is a description of how to assemble ingredients in a certain way to
create a ``cake'', but it is \emph{not the cake itself}.  The cake is
only produced after \emph{following} the instructions, which here
involves traversing the various model, submodel, deletion, and
replacement descriptions.

We decompose validation of such a composite model into two phases:

\begin{enumerate}\setlength{\parskip}{0ex}

\item \emph{Validate the ``recipe''}.  The submodel, deletion, and
  replacement constructs themselves must be valid.

\item \emph{Validate the ``cake''}.  The model produced by interpreting
  the various constructs must be valid SBML.

\end{enumerate}

The first phase involves checking the aggregation, deletion and linkage
instructions defined by the Hierarchical Model Composition constructs in
an SBML document.  The \Submodel, \Port, \Deletion, \ReplacedElement,
\ReplacedBy and other constructs defined in this specification must be
valid according to the rules defined in \sec{validation-rules}.
Passing this phase means that the constructs are well-formed, referenced
files and models and other entities exist, ports have identifiers in the
relevant namespaces, and so on.

The second validation phase takes place after interpreting the
Hierarchical Model Composition constructs.  The result of this phase
must be a valid SBML model.  This verification can in principle be
performed in various ways.  In this specification, we describe one
approach below that involves interpreting the Hierarchical Model
Composition constructs to produce a kind of ``flattened'' version of the
model devoid of the Hierarchical Model Composition package constructs.
The ``flattened'' version of the model only exists in memory: the
referenced files are not actually modified, but rather, the
interpretation of the package constructs leads to an in-memory
representation of a final, composite model implied by following the
recipe.  This generated model can then be tested against the rules for
SBML validity defined in the SBML Level~3 Version~1 Core specification.
Performing this ``flattening'' allows for the most straightforward way
of testing the validity of the resulting SBML model; however, it is
\emph{not part of the requirements for this package}.  The requirements
are only that the model implied by the package constructs is valid.


\subsubsection{Example algorithm for producing a ``flattened'' model}

\fig{flattening-algo} presents a possible algorithm for interpreting the
Hierarchical Model Composition constructs and creating a ``flattened''
SBML document.  As explained above, this procedure can be used as part
of a process to test the validity of an SBML document that uses
Hierarchical Model Composition.  After performing the steps of the
flattening algorithm, the result should be evaluated for validity
according to the normal rules of SBML Level~3 Version~1 Core and (if
applicable) the rules defined by any other Level~3 packages used in the
model.

\begin{figure}[thb]
  \renewcommand{\arraystretch}{0.85}
  \newcounter{rownum}
  \setcounter{rownum}{0}
  \setlength{\tabcolsep}{4pt}
  \begin{edtable}{tabular}{>{\stepcounter{rownum}\therownum.}rp{6.05in}}
    \toprule
    \multicolumn{1}{r}{\textbf{Step}} & \textbf{Procedure} \\
    \midrule

    & Examine all submodels of the model being validated.  For any
    submodel that itself contains submodels, perform this algorithm in
    its entirety on each of those inner submodels before proceeding to
    the next step.
    \\[6pt]

    & Let ``\emph{M}'' be the identifier of a given submodel.  Verify
    that no object identifier or meta identifier of objects in that
    submodel (i.e., the \token{id} or \token{metaid} attribute values)
    begin with the character sequence ``\emph{M\_\_}''; if there
    \emph{is} an existing identifier or meta identifier beginning with
    ``\emph{M\_\_}'', add an underscore to ``\emph{M\_\_}'' (i.e., to
    produce ``\emph{M\_\_\_}'') and again check that the sequence is
    unique.  Continue adding underscores until you find a unique prefix.
    Let ``\emph{P}'' stand for this final prefix.
    \\[6pt]

    & Remove all objects that have been replaced or deleted in the
    submodel.
    \\[6pt]

    & Transform the remaining objects in the submodel as follows:
    \begin{enumerate}[leftmargin=3ex,label={\alph*})]

    \item Change every identifier (\token{id} attribute) to a new value
      obtained by prepending ``\emph{P}'' to the original identifier.

    \item Change every meta identifier (\token{metaid} attribute) to a
      new value obtained by prepending ``\emph{P}'' to the original
      identifier.

    \vspace*{-0.5em}
    \end{enumerate}
    \\

    & Transform every \primtype{SIdRef} and \primtype{IDREF} type value
    in the remaining objects of the submodel as follows:
    \begin{enumerate}[leftmargin=3ex,label={\alph*})]

    \item If the referenced object has been replaced by the application
      of a \ReplacedBy or \ReplacedElement construct, change the
      \primtype{SIdRef} value (respectively, \primtype{IDREF} value) to
      the \primtype{SId} value (respectively, \primtype{ID} value) of
      the object replacing it.

    \item If the referenced object has not been replaced, change the
      \primtype{SIdRef} and \primtype{IDREF} value by prepending
      ``\emph{P}'' to the original value.

    \vspace*{-0.5em}
    \end{enumerate}
    \\

    & After performing the tasks above for all remaining objects, merge
    the objects of the remaining submodels into a single model.  Merge
    the various lists (list of species, list of compartments, etc.)  in
    this step, and preserve notes and annotations as well as constructs
    from other SBML Level~3 packages.
    \\
    \bottomrule
  \end{edtable}
  \caption{Example algorithm for ``flattening'' a model to remove
    Hierarchical Model Composition package constructs.}
  \label{flattening-algo}
\end{figure}


\subsubsection{Additional remarks about the validation procedure}

When \notice instantiating a model, it is not necessary to first test
the validity of that model.  If it is in the same file as the containing
model, it will be tested anyway when the result of the ``flattening''
algorithm is checked for validity in the second phase.  If it is in a
different file, that file's validity (or lack thereof) should not affect
the validity of the file being tested, though a validator may warn the
user of this situation if it desires.


\subsection{Validation and consistency rules}
\label{validation-rules}

This section summarizes all the conditions that must (or in some cases,
at least \emph{should}) be true of an SBML Level~3 Version~1 model that
uses the Hierarchical Model Composition package.  We use the same
conventions as are used in the SBML Level~3 Version~1 Core specification
document.  In particular, there are different degrees of rule
strictness.  Formally, the differences are expressed in the statement of
a rule: either a rule states that a condition \emph{must} be true, or a
rule states that it \emph{should} be true.  Rules of the former kind are
strict SBML validation rules---a model encoded in SBML must conform to
all of them in order to be considered valid.  Rules of the latter kind
are consistency rules.  To help highlight these differences, we use the
following three symbols next to the rule numbers:

\begin{description}

\item[\hspace*{6.5pt}\vSymbol\vsp] A \vSymbolName indicates a
  \emph{requirement} for SBML conformance. If a model does not follow
  this rule, it does not conform to the Hierarchical Model Composition
  specification.  (Mnemonic intention behind the choice of symbol:
  ``This must be checked.'')

\item[\hspace*{6.5pt}\cSymbol\csp] A \cSymbolName indicates a
  \emph{recommendation} for model consistency.  If a model does not
  follow this rule, it is not considered strictly invalid as far as the
  Hierarchical Model Composition specification is concerned; however, it
  indicates that the model contains a physical or conceptual
  inconsistency.  (Mnemonic intention behind the choice of symbol:
  ``This is a cause for warning.'')

\item[\hspace*{6.5pt}\mSymbol\msp] A \mSymbolName indicates a strong
  recommendation for good modeling practice.  This rule is not strictly
  a matter of SBML encoding, but the recommendation comes from logical
  reasoning.  As in the previous case, if a model does not follow this
  rule, it is not considered an invalid SBML encoding.  (Mnemonic
  intention behind the choice of symbol: ``You're a star if you heed
  this.'')

\end{description}

The validation rules listed in the following subsections are all stated
or implied in the rest of this specification document.  They are
enumerated here for convenience.  Unless explicitly stated, all
validation rules concern objects and attributes specifically defined in
the Hierarchical Model Composition package.

For \notice convenience and brievity, we use the shorthand
``\token{comp:x}'' to stand for an attribute or element name \token{x}
in the namespace for the Hierarchical Model Composition package, using
the namespace prefix \token{comp}.  In reality, the prefix string may be
different from the literal ``\token{comp}'' used here (and indeed, it
can be any valid XML namespace prefix that the modeler or software
chooses).  We use ``\token{comp:x}'' because it is shorter than to
write a full explanation everywhere we refer to an attribute or element
in the Hierarchical Model Composition package namespace.


\subsubsection*{General rules about this package}

\validRule{comp-10101}{To conform to Version 1 of the Hierarchical Model
  Composition package specification for SBML Level~3, an
  SBML document must declare the use of the following XML Namespace:\\
  \textls[-25]{\uri{http://www.sbml.org/sbml/level3/version1/comp/version1}}.
  (References: SBML Level~3 Package Specification for Hierarchical Model
  Composition, Version~1, \sec{xml-namespace}.)}
  
\validRule{comp-10102}{Wherever they appear in an SBML document,
  elements and attributes from the Hierarchical Model Composition
  package must be declared either implicitly or explicitly to be in the
  XML namespace
  \uri{http://www.sbml.org/sbml/level3/version1/comp/version1}.
  (References: SBML Level~3 Package Specification for Hierarchical Model
  Composition, Version~1, \sec{xml-namespace}.) }


\subsubsection*{General rules about identifiers} 

\validRule{comp-10301}{(Extends validation rule \#10301 in the SBML
  Level~3 Version~1 Core specification.) Within a \Model 
  object, the values of the attributes
  \token{id} and \token{comp:id} on every instance of the following
  classes of objects must be unique across the set of all \token{id} and
  \token{comp:id} attribute values of all such objects in a model: the
  \Model itself, plus all contained \FunctionDefinition, \Compartment,
  \Species, \Reaction, \SpeciesReference, \ModifierSpeciesReference,
  \Event, and \Parameter objects, plus the \Submodel and \Deletion
  objects defined by the Hierarchical Model Composition package, plus
  any objects defined by any other package with \token{package:id}
  attributes defined as falling in the 'SId' namespace.
  (References: SBML Level~3 Package Specification for Hierarchical Model
  Composition, Version~1, \sec{namespaces}.) }

\validRule{comp-10302}{The values of the attributes \token{id} and
  \token{comp:id} on every instance of all \Model and
  \ExternalModelDefinition objects must be unique across the set of all
  \token{id} and \token{comp:id} attribute values of such objects
  in the SBML document to which they belong.  (References: SBML Level~3
  Package Specification for Hierarchical Model Composition, Version~1,
  \sec{namespaces}.)}

\validRule{comp-10303}{Within a \Model object inside an SBML document,
  the value of the attribute \token{comp:id} on every instance of a
  \Port object must be unique across the set of all \token{comp:id}
  attribute values of all such objects in the model.  (References: SBML
  Level~3 Package Specification for Hierarchical Model Composition,
  Version~1, \sec{namespaces}.) }

\validRule{comp-10304}{The value of a \token{comp:id} attribute must
  always conform to the syntax of the SBML data type \primtype{SId}.
  (References: SBML Level~3 Version~1 Core, Section~3.1.7.)}

\validRule{comp-10308}{The value of a \token{comp:submodelRef} attribute
  on \ReplacedElement and \ReplacedBy objects must always conform to the
  syntax of the SBML data type \primtype{SId}.  (References: SBML
  Level~3 Package Specification for Hierarchical Model Composition,
  Version~1, \sec{replacedelement-submodelref}.) }
 
\validRule{comp-10309}{The value of a \token{comp:deletion} attribute
  on a \ReplacedElement object must always conform to the syntax of the
  SBML data type \primtype{SId}.  (References: SBML Level~3 Package
  Specification for Hierarchical Model Composition, Version~1,
  \sec{replacedelement-deletion}.) }

\clearpage

\validRule{comp-10310}{The value of a
  \token{comp:conversionFactor} attribute on a \ReplacedElement
  object must always conform to the syntax of the SBML data type
  \primtype{SId}.  (References: SBML Level~3 Package Specification for
  Hierarchical Model Composition, Version~1,
  \sec{replacedelement-conversionfactor}.) }
  

\validRule{comp-10311}{The value of a \token{comp:name} attribute
  must always conform to the syntax of type \primtype{string}.  
  (References: SBML Level~3 Version~1 Core, Section~3.1.1.)}



\begin{blockChanged}
\subsubsection*{General Rules for Units}

\consistencyRule{comp-10501}{If one element replaces another, whether it is the target of a \ReplacedBy element, or whether it has a child \ReplacedElement, the units of the replaced element, multiplied by the units of any applicable conversion factor, should equal the units of the replacement element.
  (References: SBML Level~3 Package Specification for Hierarchical Model
  Composition, Version~1, \sec{replacedelement-additional}.) }
\end{blockChanged}



\subsubsection*{Rules for the extended \class{SBase} abstract class} 

\validRule{comp-20101}{Any object derived from the extended \SBase class
  (defined in the Hierarchical Model Composition package) may contain
  at most one instance of a \ListOfReplacedElements subobject.
  (References: SBML Level~3 Package Specification for Hierarchical Model
  Composition, Version~1, \sec{extended-sbase-class}.) }

\validRule{comp-20102}{Apart from the general notes and annotation
  subobjects permitted on all SBML objects, a \ListOfReplacedElements
  container object may only contain \ReplacedElement objects.
  (References: SBML Level~3 Package Specification for Hierarchical Model
  Composition, Version~1, \sec{extended-sbase-class}.) }

\validRule{comp-20103}{A \ListOfReplacedElements object may have the
  optional attributes \token{metaid} and \token{sboTerm} defined by
  SBML Level~3 Core.  No other attributes from the SBML Level~3 Core
  namespace or the Hierarchical Model Composition namespace are
  permitted on a \ListOfReplacedElements object.  (References: SBML
  Level~3 Package Specification for Hierarchical Model Composition,
  Version~1, \sec{extended-sbase-class}.) }

\validRule{comp-20104}{The \ListOfReplacedElements in an \SBase object
  is optional, but if present, must not be empty.  (References: SBML
  Level~3 Package Specification for Hierarchical Model Composition,
  Version~1, \sec{extended-sbase-class}.) }

\validRule{comp-20105}{Any object derived from the extended \SBase class
  (defined in the Hierarchical Model Composition package) may contain
  at most one instance of a \ReplacedBy subobject.
  (References: SBML Level~3 Package Specification for Hierarchical Model
  Composition, Version~1, \sec{extended-sbase-class}.) }


\subsubsection*{Rules for the extended \class{SBML} class} 

\validRule{comp-20201}{In all SBML documents using the Hierarchical
  Model Composition package, the \SBML object must include a value for
  the attribute \token{comp:required} attribute.  
  (References: SBML Level~3 Version~1 Core, Section~4.1.2.) 
}
  
\validRule{comp-20202}{The value of attribute \token{comp:required} on
  the \SBML object must be of the data type \token{boolean}.
  (References: SBML Level~3 Version~1 Core, Section~4.1.2.) }
  
\validRule{comp-20203}{\begin{blockChanged}(Rule removed because of a change in the interpretation of the \token{required} attribute by the SBML Editors.)\end{blockChanged}}

\validRule{comp-20204}{\begin{blockChanged}(Rule removed because of a change in the interpretation of the \token{required} attribute by the SBML Editors.)\end{blockChanged}}

\validRule{comp-20205}{There may be at most one instance of the
  \ListOfModelDefinitions within an \SBML object that uses the SBML
  Level~3 Hierarchical Model Composition package.  (References: SBML
  Level~3 Package Specification for Hierarchical Model Composition,
  Version~1, \sec{sbml-class}.) }

\validRule{comp-20206}{The various
  \textsf{\textbf{ListOf\rule{0.15in}{0.5pt}}} subobjects within an
  \SBML object are optional, but if present, these container objects
  must not be empty.  Specifically, if any of the following classes of
  objects is present within the \SBML object, it must not be empty:
  \ListOfModelDefinitions and \ListOfExternalModelDefinitions.
  (References: SBML Level~3 Package Specification for Hierarchical Model
  Composition, Version~1, \sec{sbml-class}.) }
  
\validRule{comp-20207}{Apart from the general notes and annotation
  subobjects permitted on all SBML objects, a \ListOfModelDefinitions
  container may only contain extended \Model class objects.
  (References: SBML Level~3 Package Specification for Hierarchical Model
  Composition, Version~1, \sec{listofmodeldefinitions-class}.) }

\validRule{comp-20208}{Apart from the general notes and annotation
  subobjects permitted on all SBML objects, a
  \ListOfExternalModelDefinitions container may only contain
  \ExternalModelDefinition objects.  (References: SBML Level~3 Package
  Specification for Hierarchical Model Composition, Version~1,
  \sec{listofexternalmodeldefinitions-class}.) }
  
\validRule{comp-20209}{A \ListOfModelDefinitions object may have the
  optional attributes \token{metaid} and \token{sboTerm}.  No other
  attributes from the SBML Level~3 Core namespace or the Hierarchical
  Model Composition namespace are permitted on a \ListOfModelDefinitions
  object.  (References: SBML Level~3 Package Specification for
  Hierarchical Model Composition, Version~1,
  \sec{listofmodeldefinitions-class}.) }

\validRule{comp-20210}{A \ListOfExternalModelDefinitions object may have
  the optional SBML core attributes \token{metaid} and
  \token{sboTerm}.  No other attributes from the SBML Level~3 Core
  namespace or the Hierarchical
  Model Composition namespace are permitted on a
  \ListOfExternalModelDefinitions object.  (References: SBML Level~3
  Package Specification for Hierarchical Model Composition, Version~1,
  \sec{listofexternalmodeldefinitions-class}.) }

\validRule{comp-20211}{There may be at most one instance of the
  \ListOfExternalModelDefinitions within an \SBML object that uses the
  Hierarchical Model Composition package.  (References: SBML Level~3
  Package Specification for Hierarchical Model Composition, Version~1,
  \sec{sbml-class}.) }

\begin{blockChanged}\validRule{comp-20212}{The value of attribute \token{comp:required} on the \SBML object must be set to \val{true}.  (References: SBML Level~3 Package Specification for Hierarchical Model Composition, Version~1, \sec{xml-namespace}.)}\end{blockChanged}


\subsubsection*{Rules for \class{ExternalModelDefinition} objects} 

\validRule{comp-20301}{An \ExternalModelDefinition object may have the
  optional SBML Level~3 Core attributes \token{metaid} and
  \token{sboTerm}.  No other attributes from the SBML Level~3 Core
  namespace are permitted on an \ExternalModelDefinition.  (References:
  SBML Level~3 Version~1 Core, Section~3.2.)}

\validRule{comp-20302}{An \ExternalModelDefinition object may have the
  optional SBML Level~3 Core subobjects for notes and annotation.  No
  other subobjects from the SBML Level~3 Core namespace or the Hierarchical
  Model Composition namespace are permitted in
  an \ExternalModelDefinition.  (References: SBML Level~3 Version~1
  Core, Section~3.2.)  }
   
\validRule{comp-20303}{An \ExternalModelDefinition object must have the
  attributes \mbox{\token{comp:id}} and \token{comp:source} because
  they are required, and may have the optional attributes \token{comp:name},
  \token{comp:modelRef}, and \token{comp:md5}.  No other attributes from the Hierarchical
  Model Composition namespace are permitted on an
  \ExternalModelDefinition object.  (References: SBML Level~3 Package
  Specification for Hierarchical Model Composition, Version~1,
  \sec{externalmodeldefinition-class}.) }

\validRule{comp-20304}{The value of the \token{comp:source}
  attribute on an \ExternalModelDefinition object must reference an SBML
  Level~3 Version~1 document.  (References: SBML Level~3 Package Specification for
  Hierarchical Model Composition, Version~1,
  \sec{externalmodeldefinition-class}.) }

\validRule{comp-20305}{The value of the \token{comp:modelRef}
  attribute on an \ExternalModelDefinition object must be the value of
  an \token{id} attribute on a \Model or \ExternalModelDefinition 
  object in the SBML document
  referenced by the \token{comp:source} attribute.  (References:
  SBML Level~3 Package Specification for Hierarchical Model Composition,
  Version~1, \sec{externalmodeldefinition-class}.) }

\consistencyRule{comp-20306}{The value of the \token{comp:md5}
  attribute, if present on an \ExternalModelDefinition object, should
  match the calculated MD5 checksum of the SBML document referenced by
  the \token{comp:source} attribute.  (References: SBML Level~3
  Package Specification for Hierarchical Model Composition, Version~1,
  \sec{externalmodeldefinition-class}.) }

\validRule{comp-20307}{The value of a \token{comp:source} attribute
  on an \ExternalModelDefinition object must always conform to the
  syntax of the XML Schema~1.0 data type \primtype{anyURI}.
  (References: SBML Level~3 Package Specification for Hierarchical Model
  Composition, Version~1, \sec{externalmodeldefinition-class}.) }

\validRule{comp-20308}{The value of a \token{comp:modelRef}
  attribute on an \ExternalModelDefinition object must always conform to
  the syntax of the SBML data type \primtype{SId}.  (References: SBML
  Level~3 Package Specification for Hierarchical Model Composition,
  Version~1, \sec{externalmodeldefinition-class}.) }

\validRule{comp-20309}{The value of a \token{comp:md5} attribute on an
  \ExternalModelDefinition object must always conform to the syntax of
  type \primtype{string}.  (References: SBML Level~3 Package
  Specification for Hierarchical Model Composition Package Version~1
  \sec{externalmodeldefinition-class}.) }

\validRule{comp-20310}{An \ExternalModelDefinition object must not
  reference an \ExternalModelDefinition in a different SBML document
  that, in turn, refers back to the original \ExternalModelDefinition
  object, whether directly or indirectly through a chain of
  \ExternalModelDefinition objects.  (References: SBML Level~3 Package
  Specification for Hierarchical Model Composition, Version~1,
  \sec{externalmodeldefinition-class}.) }


% \subsubsection*{Rules for \class{ModelDefinition} objects} 

% \validRule{comp-20401}{A ModelDefinition object derives from the SBML
% Level~3 Core \class{Model} and as such must follow the same
% restrictions present on Level~3 Core \Model objects.  This includes
% any validation rules from the SBML Level 3 Version 1 core
% specification as well as this document.  (References: SBML Level~3
% Version~1 Core, Section~4.2.) }


\subsubsection*{Rules for extended \class{Model} objects} 

\validRule{comp-20501}{There may be at most one instance of each of the
  following kinds of objects within a \Model object using Hierarchical
  Model Composition: \ListOfSubmodels and \ListOfPorts.  (References:
  SBML Level~3 Package Specification for Hierarchical Model Composition,
  Version~1, \sec{model-class}.) }

\validRule{comp-20502}{The various
  \textsf{\textbf{ListOf\rule{0.15in}{0.5pt}}} subobjects with an \Model
  object are optional, but if present, these container object must not
  be empty.  Specifically, if any of the following classes of objects
  are present on the \Model, it must not be empty: \ListOfSubmodels and
  \ListOfPorts.  (References: SBML Level~3 Package Specification for
  Hierarchical Model Composition, Version~1, \sec{model-class}.) }

\validRule{comp-20503}{Apart from the general notes and annotation
  subobjects permitted on all SBML objects, a \ListOfSubmodels container
  object may only contain \Submodel objects.  (References: SBML Level~3
  Package Specification for Hierarchical Model Composition, Version~1,
  \sec{model-class}.) }

\validRule{comp-20504}{Apart from the general notes and annotation
  subobjects permitted on all SBML objects, a \ListOfPorts container
  object may only contain \Port objects.  (References: SBML Level~3
  Package Specification for Hierarchical Model Composition, Version~1,
  \sec{model-class}.) }
  
\validRule{comp-20505}{A \ListOfSubmodels object may have the optional
  attributes \token{metaid} and \token{sboTerm} defined by SBML Level~3
  Core.  No other attributes from the SBML Level~3 Core namespace or the
  Hierarchical Model Composition namespace are permitted on a
  \ListOfSubmodels object.  (References: SBML Level~3 Package
  Specification for Hierarchical Model Composition, Version~1,
  \sec{model-class}.) }

\validRule{comp-20506}{A \ListOfPorts object may have the optional
  attributes \token{metaid} and \token{sboTerm} defined by SBML
  Level~3 Core.  No other attributes from the SBML Level~3 Core
  namespace or the Hierarchical Model Composition namespace are
  permitted on a \ListOfPorts object.  (References: SBML Level~3 Package
  Specification for Hierarchical Model Composition, Version~1,
  \sec{model-class}.) }


\subsubsection*{Rules for \class{Submodel} objects} 

\validRule{comp-20601}{A \Submodel object may have the optional SBML
  Level~3 Core attributes \token{metaid} and \token{sboTerm}.  No
  other attributes from the SBML Level~3 Core namespace are permitted on
  an \Submodel object.  (References: SBML Level~3 Version~1 Core,
  Section~3.2.) }
   
\validRule{comp-20602}{An \Submodel object may have the optional SBML
  Level~3 Core subobjects for notes and annotation.  No other elements
  from the SBML Level~3 Core namespace are permitted on an \Submodel
  object.  (References: SBML Level~3 Version~1 Core, Section~3.2.) }
   
\validRule{comp-20603}{There may be at most one \ListOfDeletions
  container object within a \Submodel object.  (References: SBML Level~3
  Package Specification for Hierarchical Model Composition, Version~1,
  \sec{listofdeletions-class}.) }

\validRule{comp-20604}{A \ListOfDeletions container object within a
  \Submodel object is optional, but if present, must not be empty.
  (References: SBML Level~3 Package Specification for Hierarchical Model
  Composition, Version~1, \sec{listofdeletions-class}.) }
  
\validRule{comp-20605}{Apart from the general notes and annotation
  subobjects permitted on all SBML objects, a \ListOfDeletions container
  object may only contain \Deletion objects.  (References: SBML Level~3
  Package Specification for Hierarchical Model Composition, Version~1,
  \sec{listofdeletions-class}.) }

\validRule{comp-20606}{A \ListOfDeletions object may have the optional
  SBML core attributes \token{metaid} and \token{sboTerm}.  No other
  attributes from the SBML Level~3 Core namespace or the comp namespace
  are permitted on a \ListOfDeletions object.  (References: SBML Level~3
  Package Specification for Hierarchical Model Composition, Version~1,
  \sec{listofdeletions-class}.) }

\validRule{comp-20607}{A \Submodel object must have the attributes
  \token{comp:id} and \token{comp:modelRef} because they are required,
  and may also have the optional attributes \token{comp:name},
  \token{comp:timeConversionFactor}, and/or
  \token{comp:extentConversionFactor}.  No other attributes from the
  Hierarchical Model Composition namespace are permitted on a \Submodel
  object.  (References: SBML Level~3 Package Specification for
  Hierarchical Model Composition, Version~1, \sec{submodel-class}.)  }

\validRule{comp-20608}{The value of a \token{comp:modelRef}
  attribute on a \Submodel object must always conform to the syntax of
  the SBML data type \primtype{SId}.  (References: SBML Level~3 Package
  Specification for Hierarchical Model Composition, Version~1,
  \sec{submodel-modelref}.) }

\validRule{comp-20613}{The value of a
  \token{comp:timeConversionFactor} attribute on a \Submodel object
  must always conform to the syntax of the SBML data type
  \primtype{SId}.  (References: SBML Level~3 Package Specification for
  Hierarchical Model Composition, Version~1,
  \sec{submodel-timeconversionfactor}.) }

\validRule{comp-20614}{The value of a
  \token{comp:extentConversionFactor} attribute on a \Submodel object
  must always conform to the syntax of the SBML data type
  \primtype{SId}.  (References: SBML Level~3 Package Specification for
  Hierarchical Model Composition, Version~1,
  \sec{submodel-extentconversionfactor}.) }

\validRule{comp-20615}{The value of a \token{comp:modelRef}
  attribute on a \Submodel must be the identifier of a \Model or
  \ExternalModelDefinition object in the same \SBML object as the
  \Submodel.  (References: SBML Level~3 Package Specification for
  Hierarchical Model Composition, Version~1, \sec{submodel-modelref}.)
}

\validRule{comp-20616}{A \Model object must not contain a \Submodel
  which references that \Model object itself.  That is, the value of a
  \token{comp:modelRef} attribute on a \Submodel must not be the
  value of the parent \Model object's \token{id} attribute.
  (References: SBML Level~3 Package Specification for Hierarchical Model
  Composition, Version~1, \sec{submodel-modelref}.)}

\validRule{comp-20617}{A \Model object must not contain a \Submodel
  which references that \Model indirectly.  That is, the
  \token{comp:modelRef} attribute of a \Submodel may not point to
  the \token{id} of a \Model containing a \Submodel object that
  references the original \Model directly or indirectly through a chain
  of \Model/\Submodel pairs.  (References: SBML Level~3 Package
  Specification for Hierarchical Model Composition, Version~1,
  \sec{submodel-modelref}.) }

\validRule{comp-20622}{The value of a \token{comp:timeConversionFactor}
  attribute on a given \Submodel object must be the identifier of
  a \Parameter object defined in the same \Model containing the
  \Submodel.  (References: SBML Level~3 Package Specification for
  Hierarchical Model Composition, Version~1,
  \sec{submodel-timeconversionfactor}.) }

\validRule{comp-20623}{The value of a
  \token{comp:extentConversionFactor} attribute on a given \Submodel
  object must be the identifier of a \Parameter object defined in the
  same \Model containing the \Submodel.  (References: SBML Level~3
  Package Specification for Hierarchical Model Composition, Version~1,
  \sec{submodel-extentconversionfactor}.) }


\subsubsection*{Rules for the \class{SBaseRef} object} 

\validRule{comp-20701}{The value of a \token{comp:portRef} attribute
  on an \SBaseRef object must be the identifier of a \Port object in the
  \Model referenced by that \SBaseRef.  (References: SBML Level~3
  Package Specification for Hierarchical Model Composition, Version~1,
  \sec{sbaseref-portref}.) }

\validRule{comp-20702}{The value of a \token{comp:idRef} attribute
  on an \SBaseRef object must be the identifier of an object contained
  in (that is, within the \primtype{SId} namespace of) the \Model
  referenced by that \SBaseRef.  This includes objects with \token{id}
  attributes defined in packages other than SBML Level~3 Core or the
  Hierarchical Model Composition package.  (References: SBML Level~3
  Package Specification for Hierarchical Model Composition, Version~1,
  \sec{sbaseref-idref}.)  }

\validRule{comp-20703}{The value of a \token{comp:unitRef} attribute
  on an \SBaseRef object must be the identifier of a \UnitDefinition
  object contained in the \Model referenced by that \SBaseRef.
  (References: SBML Level~3 Package Specification for Hierarchical Model
  Composition, Version~1, \sec{sbaseref-unitref}.) }

\validRule{comp-20704}{The value of a \token{comp:metaIdRef}
  attribute on an \SBaseRef object must be the value of a
  \token{comp:metaid} attribute on an element contained in the \Model
  referenced by that \SBaseRef.  This includes elements with
  \token{metaid} attributes defined in packages other than SBML
  Level~3 Core or the Hierarchical Model Composition package.
  (References: SBML Level~3 Package Specification for Hierarchical Model
  Composition, Version~1, \sec{sbaseref-metaidref}.)}

\validRule{comp-20705}{If an \SBaseRef object contains an \SBaseRef
  child, the parent \SBaseRef must point to a \Submodel object,
  \begin{blockChanged} or a
  \Port that itself points to a \Submodel object. \end {blockChanged}
  (References: SBML Level~3 Package Specification for Hierarchical Model
  Composition, Version~1, \sec{sbaseref-recursive-sbaseref}.) }

\validRule{comp-20706}{The value of a \token{comp:portRef} attribute
  on an \SBaseRef object must always conform to the syntax of the SBML
  data type \primtype{SId}.  (References: SBML Level~3 Package
  Specification for Hierarchical Model Composition, Version~1,
  \sec{sbaseref-portref}.) }

\validRule{comp-20707}{The value of a \token{comp:idRef} attribute
  on an \SBaseRef object must always conform to the syntax of the SBML
  data type \primtype{SId}.  (References: SBML Level~3 Package
  Specification for Hierarchical Model Composition, Version~1,
  \sec{sbaseref-idref}.) }

\validRule{comp-20708}{The value of a \token{comp:unitRef} attribute
  on an \SBaseRef object must always conform to the syntax of the SBML
  data type \primtype{UnitSId}.  (References: SBML Level~3 Package
  Specification for Hierarchical Model Composition, Version~1,
  \sec{sbaseref-unitref}.) }

\validRule{comp-20709}{The value of a \token{comp:metaIdRef}
  attribute on an \SBaseRef object must always conform to the syntax of
  the XML data type \primtype{ID}.  (References: SBML Level~3 Package
  Specification for Hierarchical Model Composition, Version~1,
  \sec{sbaseref-metaidref}.) }

\validRule{comp-20710}{Apart from the general notes and annotation
  subobjects permitted on all SBML objects, an \SBaseRef
  object may only contain a single \SBaseRef child.
  (References: SBML Level~3 Package Specification for Hierarchical Model
  Composition, Version~1, \sec{sbaseref-recursive-sbaseref}.) }

\consistencyRule{comp-20711}{The 'sbaseRef' spelling of an \SBaseRef child 
  of an \SBaseRef object is considered deprecated, and 'sBaseRef' should be 
  used instead.
  (References: SBML Level~3 Package Specification for Hierarchical Model
  Composition, Version~1, \sec{sbaseref-deprecated-spelling}.) }

\begin{blockChanged}
\validRule{comp-20712}{An \SBaseRef object must point to another object; that
  is, a \SBaseRef object must always have a value for one of the attributes
  \token{comp:portRef}, \token{comp:idRef}, \token{comp:unitRef}, or
  \token{comp:metaIdRef}.
  (References: SBML Level~3 Package
  Specification for Hierarchical Model Composition, Version~1,
  \sec{sbaseref-class}.) }
\end{blockChanged}

\begin{blockChanged}
\validRule{comp-20713}{An \SBaseRef object can only point to \emph{one} other
  object; that is, a given \SBaseRef object can only have a value for one of
  the attributes \token{comp:portRef}, \token{comp:idRef}, \token{comp:unitRef}, or
  \token{comp:metaIdRef}.
  No other attributes from the Hierarchical
  Model Composition namespace are permitted on an \SBaseRef object.
  (References: SBML Level~3 Package
  Specification for Hierarchical Model Composition, Version~1,
  \sec{port-class}.) }
\end{blockChanged}

\begin{blockChanged}
\validRule{comp-20714}{Any one SBML object may only be referenced in one of the following ways:  referenced by a single \Port object; referenced by a single \Deletion object; referenced by a single \ReplacedElement; be the parent of a single \ReplacedBy child; be referenced by one or more \ReplacedBy  objects; or be referenced by one or more \ReplacedElement objects all using the \token{deletion} attribute.  Essentially, once an object has been referenced in one of these ways it cannot be referenced again.
  (References: SBML Level~3 Package
  Specification for Hierarchical Model Composition, Version~1,
  \sec{replacedelement-additional}.) }
\end{blockChanged}


\subsubsection*{Rules for \class{Port} objects} 

\validRule{comp-20801}{A \Port object must point to another object; that
  is, a \Port object must always have a value for one of the attributes
  \token{comp:idRef}, \token{comp:unitRef}, or
  \token{comp:metaIdRef}.  (References: SBML Level~3 Package
  Specification for Hierarchical Model Composition, Version~1,
  \sec{port-class}.) }

\validRule{comp-20802}{A \Port object can only point to \emph{one} other
  object; that is, a given \Port object can only have a value for one of
  the attributes \token{comp:idRef}, \token{comp:unitRef}, or
  \token{comp:metaIdRef}.  (References: SBML Level~3 Package
  Specification for Hierarchical Model Composition, Version~1,
  \sec{port-class}.) }

\validRule{comp-20803}{A \Port object must have a value for the required
  attribute \token{comp:id}, and one, and only one, of the
  attributes \token{comp:idRef}, \token{comp:unitRef}, or
  \token{comp:metaIdRef}.  It may also have the optional attribute \token{comp:name}.
  No other attributes from the Hierarchical
  Model Composition namespace are permitted on a \Port object.
  (References: SBML Level~3 Package Specification for Hierarchical Model
  Composition, Version~1, \sec{port-class}.) }

\validRule{comp-20804}{Port definitions must be unique; that is, no two
  \Port objects in a given \Model may reference the same object in that
  \Model.  (References: SBML Level~3 Package Specification for
  Hierarchical Model Composition, Version~1, \sec{port-class}.) }


\subsubsection*{Rules for \class{Deletion} objects} 

\validRule{comp-20901}{A \Deletion object must point to another object;
  that is, a \Deletion object must have a value for one of the
  attributes \token{comp:portRef}, \token{comp:idRef},
  \token{comp:unitRef}, or \token{comp:metaIdRef}.  (References:
  SBML Level~3 Package Specification for Hierarchical Model Composition,
  Version~1, \sec{deletion-class}.) }

\validRule{comp-20902}{A \Deletion object can only point to \emph{one}
  other object; that is, a given \Deletion object can only have a value
  for one of the attributes \token{comp:portRef},
  \token{comp:idRef}, \token{comp:unitRef}, or
  \token{comp:metaIdRef}.  (References: SBML Level~3 Package
  Specification for Hierarchical Model Composition, Version~1,
  \sec{deletion-class}.) }

\validRule{comp-20903}{A \Deletion object must have a value for one, 
  and only one, of the
  attributes \token{comp:portRef}, \token{comp:idRef},
  \token{comp:unitRef}, or \token{comp:metaIdRef}.  It may also have the
  optional attributes \token{comp:id} and \token{comp:name}.  No other
  attributes from the Hierarchical Model Composition namespace are
  permitted on a \Deletion object.  (References: SBML Level~3 Package
  Specification for Hierarchical Model Composition, Version~1,
  \sec{deletion-class}.) }


\subsubsection*{Rules for \class{ReplacedElement} objects} 

\validRule{comp-21001}{A \ReplacedElement object must point to another
  object; that is, a given \ReplacedElement object must have a value for
  one of the following attributes: \token{comp:portRef},
  \token{comp:idRef}, \token{comp:unitRef},
  \token{comp:metaIdRef}, or \token{comp:deletion}.  (References:
  SBML Level~3 Package Specification for Hierarchical Model Composition,
  Version~1, \sec{replacedelement-class}.) }

\validRule{comp-21002}{A \ReplacedElement object can only point to
  \emph{one} target object; that is, a given \ReplacedElement can only
  have a value for one of the following attributes:
  \token{comp:portRef}, \token{comp:idRef},
  \token{comp:unitRef}, \token{comp:metaIdRef}, or
  \token{comp:deletion}.  (References: SBML Level~3 Package
  Specification for Hierarchical Model Composition, Version~1,
  \sec{replacedelement-class}.) }

\validRule{comp-21003}{A \ReplacedElement object must have a value for
  the required attribute \token{comp:submodelRef}, and a value for one,
  and only one, of the following attributes: \token{comp:portRef},
  \token{comp:idRef}, \token{comp:unitRef}, \token{comp:metaIdRef}, or
  \token{comp:deletion}.  It may also have a value for the optional
  attribute \token{comp:conversionFactor}.  No other attributes from the
  Hierarchical Model Composition namespace are permitted on a
  \ReplacedElement object.  (References: SBML Level~3 Package
  Specification for Hierarchical Model Composition, Version~1,
  \sec{replacedelement-class}.) }

\validRule{comp-21004}{The value of a \token{comp:submodelRef}
  attribute on a \ReplacedElement object must be the identifier of a
  \Submodel present in the \ReplacedElement object's parent \Model.
  (References: SBML Level~3 Package Specification for Hierarchical Model
  Composition, Version~1, \sec{replacedelement-submodelref}.) }

\validRule{comp-21005}{The value of a \token{comp:deletion} attribute
  on a \ReplacedElement object must be the identifier of a \Deletion
  present in the \ReplacedElement object's parent \Model.  (References:
  SBML Level~3 Package Specification for Hierarchical Model Composition,
  Version~1, \sec{replacedelement-deletion}.) }

\validRule{comp-21006}{The value of a \token{comp:conversionFactor}
  attribute on a \ReplacedElement object must be the identifier of
  a \Parameter present in the \ReplacedElement object's parent \Model.
  (References: SBML Level~3 Package Specification for Hierarchical Model
  Composition, Version~1, \sec{replacedelement-conversionfactor}.) }

\validRule{comp-21007}{(Rule removed because it was 
  a duplicate of 10308.)}

\validRule{comp-21008}{(Rule removed because it was 
  a duplicate of 10309.) }
  
\validRule{comp-21009}{(Rule removed because it was 
  a duplicate of 10310.) }

\validRule{comp-21010}{No two \ReplacedElement objects in the same
  \Model may reference the same object unless that object is a \Deletion.
  (References: SBML Level~3 Package Specification for Hierarchical Model
  Composition, Version~1, \sec{replacedelement-class}.) }

\subsubsection*{Rules for \class{ReplacedBy} objects} 

\validRule{comp-21101}{A \ReplacedBy object must point to another
  object; that is, a given \ReplacedBy object must have a value for
  one of the following attributes: \token{comp:portRef},
  \token{comp:idRef}, \token{comp:unitRef}, or
  \token{comp:metaIdRef}.  (References:
  SBML Level~3 Package Specification for Hierarchical Model Composition,
  Version~1, \sec{replacedby-class}.) }

\validRule{comp-21102}{A \ReplacedBy object can only point to
  \emph{one} target object; that is, a given \ReplacedBy can only
  have a value for one of the following attributes:
  \token{comp:portRef}, \token{comp:idRef},
  \token{comp:unitRef}, or \token{comp:metaIdRef}.  (References: SBML Level~3 Package
  Specification for Hierarchical Model Composition, Version~1,
  \sec{replacedby-class}.) }

\validRule{comp-21103}{A \ReplacedBy object must have a value for the
  required attribute \token{comp:submodelRef}, and a value for one, and
  only one, of the following other attributes: \token{comp:portRef},
  \token{comp:idRef}, \token{comp:unitRef}, or \token{comp:metaIdRef}.
  No other attributes from the Hierarchical Model Composition namespace
  are permitted on a \ReplacedBy object.  (References: SBML Level~3
  Package Specification for Hierarchical Model Composition, Version~1,
  \sec{replacedby-class}.) }

\validRule{comp-21104}{The value of a \token{comp:submodelRef}
  attribute on a \ReplacedBy object must be the identifier of a
  \Submodel present in \ReplacedBy object's parent \Model.
  (References: SBML Level~3 Package Specification for Hierarchical Model
  Composition, Version~1, \sec{replacedby-submodelref}.) }

\validRule{comp-21105}{(Rule removed because it was 
  a duplicate of 10308.) }


\begin{blockChanged}
\subsubsection*{General Rules for Replacements}

\validRule{comp-21201}{If one element replaces another, whether it is the target of a \ReplacedBy element, or whether it has a child \ReplacedElement, the SBML class of the replacement element must match the SBML class of the replaced element, with two exceptions:  an element of a derived class may replace an object of its base class (for base classes other than \SBase), and any SBML class with mathematical meaning may replace a \Parameter.  A base class may not replace a derived class, however, nor may a \Parameter replace some other SBML element with mathematical meaning.
  (References: SBML Level~3 Package Specification for Hierarchical Model
  Composition, Version~1, \sec{replacedelement-additional}.) }

\validRule{comp-21202}{If one element replaces another, whether it is the target of a \ReplacedBy element, or whether it has a child \ReplacedElement, if the replaced element has the \token{id} attribute set, the replacement element must also have the \token{id} attribute set.
  (References: SBML Level~3 Package Specification for Hierarchical Model
  Composition, Version~1, \sec{replacedelement-additional}.) }

\validRule{comp-21203}{If one element replaces another, whether it is the target of a \ReplacedBy element, or whether it has a child \ReplacedElement, if the replaced element has the \token{metaid} attribute set, the replacement element must also have the \token{metaid} attribute set.
  (References: SBML Level~3 Package Specification for Hierarchical Model
  Composition, Version~1, \sec{replacedelement-additional}.) }

\validRule{comp-21204}{If one element replaces another, whether it is the target of a \ReplacedBy element, or whether it has a child \ReplacedElement, if the replaced element has an identifier attribute from some other SBML package set, the replacement element must also have that same identifier attribute set.
  (References: SBML Level~3 Package Specification for Hierarchical Model
  Composition, Version~1, \sec{replacedelement-additional}.) }
\end{blockChanged}

