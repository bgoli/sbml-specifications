% -*- TeX-master: "main"; fill-column: 72 -*-

\section{Best practices}
\label{best-practices}

\draftnote{This section is still unfinished.}
In this section, we recommend a number of practices for using and
interpreting various constructs in the Hierarchical Model Composition
package.  These recommendations are non-normative, but we advocate them
strongly; ignoring them will not render a model invalid, but may reduce
interoperability between software and models.


\subsection{Best practices for using \class{SBaseRef} for references}
\label{best-practices-sbaseref}

% There are clearly multiple approaches to using \SBaseRef objects to
% refer to SBML components.  To help increase interoperability of models
% that use Hierarchical Model Composition, we recommend the following
% order of preference in choosing the reference approach:

% \begin{enumerate}

% \item \emph{By port}.  The best-designed models intended for reuse and
%   composition will provide port definitions.  

% \item \emph{By SId}. Most elements one would want to replace (such as
%   species and reactions) will have SIds. If they do not have ports, you
%   can reference them by this. The SIdRef namespace is the namespace of
%   the submodel, not the parent model, and refers to that model's element
%   namespace (see section 3.7; identifier scoping).

% \item \emph{By UnitSId}. The SId of a UnitDefinition is defined in the
%   core specification to exist in its own namespace. Therefore, this
%   attribute is provided to be able to search that unit namespace of the
%   submodel (see section 3.7; identifier scoping). It should be noted
%   that even though this attribute is of type UnitSIdRef, the reserved
%   identifier names that are usually valid for other constructs (see
%   section 3.1.10 of the core specification) are not valid here, as these
%   may not be replaced or deleted.

% \item \emph{By MetaId}. Because some elements never have SIds and for
%   some they are only optional, the modeler may wish to replace or delete
%   an element that has no SId but does have a MetaID. (Since MetaIDs are
%   optional attributes of SBase, all SBML element have the potential to
%   have a MetaID.) If the element has no port or SId, you may use the
%   metaID, if present.

% \item \emph{By an element of a submodel}. The above four options will
%   all give you access to elements in a submodel, but cannot give you
%   access to elements in the submodel's submodels. If the object referred
%   to by one of the above methods is itself a submodel, adding an
%   SBaseRef child to the SBaseRef allows you to find elements 'buried' in
%   the hierarchy. This can, in turn, refer to a deeper submodel, allowing
%   access to any element of any arbitrary depth using this
%   construct. This is considered inelegant design: it is better to
%   'promote' any element in a submodel to a local element if it needs to
%   be referenced by a containing model, but if the submodel is fixed, no
%   other option is available.

% \end{enumerate}

% This does not actually allow access to any possible element in a model
% definition that you do not control: if the element in question has no
% port, id, or metaId, you must create a local copy of that model and add
% (for instance) a metaId to the element you wish to reference. In a
% future version of this specification, this restriction may be relaxed,
% allowing one to reference submodel elements by xpath:element. For now,
% this was considered to add too much of a burden to implementers of this
% specification, and was therefore delayed to a future version.
% Any element that has been replaced or deleted may not be referenced by
% an SBaseRef, including anything replaced or deleted within the submodel.

% If you replace or delete an element that itself has children, those
% children are considered to be deleted unless replaced. If you replace a
% KineticLaw, for example, any annotations that referred to the metaIDs of
% its LocalParameters will be invalid. To correct this situation, those
% annotations must themselves be deleted or replaced by valid elements, or
% the referenced
% LocalParameter must be explicitly replaced (by its equivalent in the new
% KineticLaw, presumably).

% It is legal to explicitly delete an element which was deleted by
% implication in this way if you need to refer to it elsewhere; the
% resulting model is exactly the same.


\subsection{Best practices for deletions and replacements}
\label{best-practices-deletions}
\label{best-practices-replacements}

Note that there may be model composition situations in which a model
contains elements that do not have an identifier, nor a meta identifier,
nor a port identifier.  In that case, there is no way to refer to it
using the with the \Deletion or \ReplacedElement objects defined in this specification.  A viable alternative to use in
that case is to copy the original model and modify it, either to perform
the desired deletions directly or to add the necessary identifiers so
that \Deletion objects can be defined and used in a submodel.
(Presumably, the original model was readable in the first place, or else
composition would have been impossible anyway.)  Copying a model and
making one's own version may have additional benefits, such as the
ability to control versions explicitly and references.  A second method may be to delete or replace the parent object of the element you wish to replace, assuming that element has an identifier, meta identifier, or port identifier.  When this is performed, the errant element will be deleted implicitly, allowing you to create replacements in the containing model without overlapping functionality.

% .... However, don't go overboard with this capability: it is legal in
% this scheme to replace an Event with a Species, but it is probably never
% wise.  We expect that tools written to produce hierarchical SBML will
% have their own restrictions that make sense in context.  This relaxation
% of the official validation allows freer intercompatibility with other
% package extensions-it may be that a Species could be validly replaced by
% a multi-component species, or it may not, but we will rely here on the
% normal validation rules that package supplies to dictate the results.


\subsection{Best practices for using ports}
\label{best-practices-ports}

Software developers who wish to include restrictions are encouraged to
experiment here, and add new attributes in a namespace of their own
devising.
