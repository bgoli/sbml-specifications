\section{Introduction}

With the Systems Biology Markup Language (SBML) there now is a common 
standard for the exchange of dynamical systems data which has already 
been adopted by many applications in this field 
\cite{SBMLWebsite,sbml3core,SBMLArticle}. 

In 2002 we worked on an automatic layout algorithm (\cite{Wegner2005}). 
Based on a given SBML file species and reactions should be placed 
automatically as a network. Since there was no way to save the final 
layout of the network, we started developing the layout extension. In 
2003 the first draft was done and presented on the SBML workshop in St. 
Louis in 2004. Based on the discussions in St. Louis, Ralph Gauges 
finalized a first specification and implementation for libSBML which was 
published in \cite{Gauges01082006}. During the next SBML meetings the 
layout extension was discussed heavily and also challenged by other 
proposals but due to the constant support from the community (e.g.: 
\cite{Deckard01122006}), it was finally accepted as a package of SBML 
Level 3. 

\subsection{Proposal corresponding to this package specification}

This specification for Layout in SBML Level~3
Version~1 is based on the proposal by the same authors, located at the
following URL:

\begin{center}
  \vspace*{1ex}\small
  \url{http://sbml.org/Community/Wiki/SBML_Level_3_Proposals/Layout}
  \vspace*{1ex}
\end{center}

The tracking number in the SBML issue tracking system~\citep{tracker}
for the Layout package activities is 233.  The
version of the proposal used as the starting point for this
specification is the version of April 2005. Previous versions of the current 
proposal are available from:
\begin{description}
  \item [] \small{\url{http://otto.bioquant.uni-heidelberg.de/sbml/}}  
\end{description}
Details of earlier independent proposals are provided in \ref{background}.

\subsection{Tracking number}
As initially listed in the SBML issue tracking system under: \\
\url{http://sourceforge.net/p/sbml/sbml-specifications/233/}.

\subsection{Package dependencies}

The Layout package adds additional classes to \sbmlthreecore and has no 
dependency on any other SBML Level~3 package.

%It is also designed to work seamlessly with other SBML Level~3 packages.

\subsection{Document conventions}

\label{conventions}

Following the precedent set by the \sbmlthreecore specification
document, we use UML~1.0 (Unified Modeling Language;
\citealt{eriksson:1998,oestereich:1999}) class diagram notation to
define the constructs provided by this package.  We also use color in
the diagrams to carry additional information for the benefit of those
viewing the document on media that can display color.  The following are
the colors we use and what they represent:

\begin{itemize}

\item[\raisebox{2.75pt}{\colorbox{black}{\rule{0.8pt}{0.8pt}}}]
  \emph{Black}: Items colored black in the UML diagrams are components
  taken unchanged from their definition in the \sbmlthreecore
  specification document.

\item[\raisebox{2.75pt}{\colorbox{mediumgreen}{\rule{0.8pt}{0.8pt}}}]
  \emph{\textcolor{mediumgreen}{Green}}: Items colored green are
  components that exist in \sbmlthreecore, but are extended by this
  package.  Class boxes are also drawn with dashed lines to further
  distinguish them.

\item[\raisebox{2.75pt}{\colorbox{darkblue}{\rule{0.8pt}{0.8pt}}}]
  \emph{\textcolor{darkblue}{Blue}}: Items colored blue are new
  components introduced in this package specification.  They have no
  equivalent in the \sbmlthreecore specification.

\end{itemize}

We also use the following typographical conventions to distinguish the
names of objects and data types from other entities; these conventions
are identical to the conventions used in the \sbmlthreecore specification
document:

\begin{description}

\item \abstractclass{AbstractClass}: Abstract classes are classes that
  are never instantiated directly, but rather serve as parents of other
  object classes.  Their names begin with a capital letter and they are
  printed in a slanted, bold, sans-serif typeface.  In electronic
  document formats, the class names defined within this document are
  also hyperlinked to their definitions; clicking
  on these items will, given appropriate software, switch the view to
  the section in this document containing the definition of that class.
  (However, for classes that are unchanged from their definitions in
  \sbmlthreecore, the class names are not hyperlinked because they
  are not defined within this document.)

\item \class{Class}: Names of ordinary (concrete) classes begin with a
  capital letter and are printed in an upright, bold, sans-serif
  typeface.  In electronic document formats, the class names are also
  hyperlinked to their definitions in this specification document.
  (However, as in the previous case, class names are not hyperlinked if
  they are for classes that are unchanged from their definitions in the
  \sbmlthreecore specification.)

\item \token{SomeThing}, \token{otherThing}: Attributes of classes, data
  type names, literal XML, and generally all tokens \emph{other} than
  SBML UML class names, are printed in an upright typewriter typeface.
  Primitive types defined by SBML begin with a capital letter; SBML also
  makes use of primitive types defined by XML
  Schema~1.0~\citep{biron:2000,fallside:2000,thompson:2000}, but
  unfortunately, XML~Schema does not follow any capitalization
  convention and primitive types drawn from the XML~Schema language may
  or may not start with a capital letter.

\end{description}

The UML diagrams in this document show the name of the class on top. 
Below are the attributes specific to that class. Optional attributes 
have some default value which may be NULL. Arrays are written in square 
brackets where with the valid array length within those brackets. So an 
array $[2..]$ would mean that it can hold from 2 to $\infty$ number of 
objects. 

For other matters involving the use of UML and XML, we follow the 
conventions used in the \sbmlthreecore specification document. 



