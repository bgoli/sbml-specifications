% -*- TeX-master: "main"; fill-column: 72 -*-

\section{Introduction}
\label{intro}

\subsection{Motivation}

Quantitative methods for modelling biological networks require an
in-depth knowledge of the biochemical reactions and their stoichiometric
and kinetic parameters. In many practical cases, this knowledge is
missing. This has led to the development of several qualitative
modelling methods using information such as gene expression data coming
from functional genomic experiments.

The qualitative models contemplated in this package are essentially
based on the definition of \emph{regulatory} or \emph{influence
  graphs}. The components of these models differ from species and
reactions used in current SBML models. For example, qualitative models
typically associate discrete levels of activities with entity pools; the
processes involving them cannot be described as reactions per se but
rather as transitions between states. These systems can be viewed as
reactive systems, which dynamics are represented by means of state
transition graphs (or other Kripke structures representing, in the form
of a graph, which nodes are the reachable states and the edges are the
state transitions). In this context, logical regulatory networks
(Boolean or multi-valued) \cite{kauffman69,thomas91} and standard Petri
nets \cite{chaouiya07} are the two formalisms mostly used in biology
that give rise to such behaviours. Published models using these approaches cover, far from exhaustiveness, gene regulatory networks and signalling pathways (e.g. \cite{thieffry95,sanchez03,albert03,faure06,mendoza06,helikar08,naldi10,calzone10}), metabolic pathways (see review in \cite{chaouiya07}). 

Finally, because their dynamics can be abstracted by Kripke structures, models expressed as systems of piece-wide linear differential equations \cite{batt05}, may be covered by this package, provided some extension. Specific classes of high-level Petri nets may also be contemplated in the future (see \ref{apdx-future}).

 Despite differences from traditional SBML models, it is desirable to
 bring these classes of models under a common format scheme. The purpose
 of this Qualitative Models package for SBML Level 3 is to support
 encoding qualitative models in SBML.

\subsection{Package dependencies}

The QualitativeModels package has no dependencies on other
SBML Level~3 packages.  
(If you find incompatibilities with other packages, please contact the
Package Working Group.  Contact information is shown on the front page
of this document.)


\subsection{Document conventions}
\label{conventions}

Following the precedent set by the SBML Level~3 Core specification
document \citep{l3v1c}, we use UML~1.0 (Unified Modeling Language;
\citealt{eriksson:1998,oestereich:1999}) class diagram notation to
define the constructs provided by this package.  We also use color in
the diagrams to carry additional information for the benefit of those
viewing the document on media that can display color.  The following are
the colors we use and what they represent:

\begin{itemize}

\item[\raisebox{2.75pt}{\colorbox{black}{\rule{0.8pt}{0.8pt}}}]
  \emph{Black}: Items colored black in the UML diagrams are components
  taken unchanged from their definition in the SBML Level~3 Core
  specification document.

\item[\raisebox{2.75pt}{\colorbox{mediumgreen}{\rule{0.8pt}{0.8pt}}}]
  \emph{\textcolor{mediumgreen}{Green}}: Items colored green are
  components that exist in SBML Level~3 Core, but are extended by this
  package.  Class boxes are also drawn with dashed lines to further
  distinguish them.

\item[\raisebox{2.75pt}{\colorbox{darkblue}{\rule{0.8pt}{0.8pt}}}]
  \emph{\textcolor{darkblue}{Blue}}: Items colored blue are new
  components introduced in this package specification.  They have no
  equivalent in the SBML Level~3 Core specification.

\end{itemize}

We also use the following typographical conventions to distinguish the
names of objects and data types from other entities; these conventions
are identical to the conventions used in the SBML Level~3 Core specification
document:

\begin{description}
  
\item \abstractclass{AbstractClass}: Abstract classes are never
  instantiated directly, but rather serve as parents of other classes.
  Their names begin with a capital letter and they are printed in a
  slanted, bold, sans-serif typeface.  In electronic document formats,
  the class names defined within this document are also hyperlinked to
  their definitions; clicking on these items will, given appropriate
  software, switch the view to the section in this document containing
  the definition of that class.  (However, for classes that are
  unchanged from their definitions in SBML Level~3 Core, the class names
  are not hyperlinked because they are not defined within this
  document.)
  
\item \class{Class}: Names of ordinary (concrete) classes begin with a
  capital letter and are printed in an upright, bold, sans-serif
  typeface.  In electronic document formats, the class names are also
  hyperlinked to their definitions in this specification document.
  (However, as in the previous case, class names are not hyperlinked if
  they are for classes that are unchanged from their definitions in the
  SBML Level~3 Core specification.)

\item \token{SomeThing}, \token{otherThing}: Attributes of classes, data
  type names, literal XML, and generally all tokens \emph{other} than
  SBML UML class names, are printed in an upright typewriter typeface.
  Primitive types defined by SBML begin with a capital letter; SBML also
  makes use of primitive types defined by XML
  Schema~1.0~\citep{biron:2000,fallside:2000,thompson:2000}, but
  unfortunately, XML~Schema does not follow any capitalization
  convention and primitive types drawn from the XML~Schema language may
  or may not start with a capital letter.

\end{description}

For other matters involving the use of UML and XML, we follow the
conventions used in the SBML Level~3 Core specification document.  






