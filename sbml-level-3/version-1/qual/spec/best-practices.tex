% -*- TeX-master: "main"; fill-column: 72 -*-

\section{Best practices}
\label{best-practices}




\ALL\footnote{\ALL all, \PN Petri nets, \LRG logical regulatory networks} To be valid, the SBML root element must express the requirement of this package: \texttt{<sbml $\dots$ qual:required="true" $\dots$ >}.

\medskip
\PN In Petri nets the initial conditions are part of the model, meaning that the \attr{initialLevel} must be defined.

\medskip
\PN To represent unbounded places, the \attr{maxLevel} should be not specified.

\medskip
\LRG Discussions are still ongoing about the possible (but sometimes convenient to avoid cumbersome descriptions) incoherency of the \qual{FunctionTerm} elements. For now, here are the guidelines to ensure coherent definitions:
\begin{itemize}
\item The \qual{FunctionTerm} elements of all the transitions targeting the same output should be "coherent": the conditions of two \qual{FunctionTerm} elements, leading to different effects on the level/symbol of the output, should not be fulfilled at the same time( i.e. they should be exclusive).
\item If several \qual{FunctionTerm} elements lead to the same effect on the level/symbol of the same output, then the importing tool should consider the disjunction (OR) on the conditions of the terms. 
\end{itemize}
\medskip

\LRG Any qualitativeSpecies which attribute \qual{constant} is set to "false" should appear as the output of a transition (meaning there is a process governing its evolution). Conversely, any qualitativeSpecies that appears as the output of a transition should have its attribute \qual{constant} set to false.

\medskip
\PN  The \qual{transitionEffect} of an \qual{Input} is set to "consumption", unless this input is connected to the transition by a test arc (meaning the transition has no effect on its marking). The \qual{thresholdLevel} indicates the weight of the arc from this place to the transition and is required. It is used to specify the enabling conditions of the transition (and to indicate the number of tokens consumed by the firing of this transition). The \qual{sign} attribute should not be defined.

\medskip
\LRG The \qual{transitionEffect} of an \qual{Input} is set to "none". The \qual{thresholdLevel}, when specified, indicates the level for which the species participates to the transition (in this case, any reference to the input \qual{id} attribute in a <ci> element within a functionTerm of the transition refers to the value of this  \qual{thresholdLevel}). The \qual{sign} attribute indicates the type of effect on the output of the transition (the regulated species): "positive" (activation), "negative" (inhibition), "dual" (positive or negative depending e.g. on co-factors) or "unknown". It is optional and mainly used for graphical purposes.

\medskip
\PN  The \qual{transitionEffect} of an \qual{Output} is set to "production". The \qual{outputLevel} indicates the weight of the arc from the transition to this place, it should be defined and is interpreted as  the number of tokens produced by the firing of this transition.

\medskip
\LRG The \qual{transitionEffect} of an \qual{Output} is set to "assignmentLevel". The \qual{outputLevel} should not be defined, the level assigned to this species being defined by the \qual{resultLevel} of the transition.



%\medskip
%\LRG To declare external nodes (ones that have no Boolean expression/truth table associated with them), one should set the attribute \attr{boundaryCondition} of the  \qual{QualitativeSpecies} to TRUE: 
%\begin{verbatim}
%<qualitativeSpecies id="EGF" maxLevel="1" boundaryCoundition="true" 
%                                                  compartment="extracellular"/> 
%\end{verbatim}

%Thus, the  \qual{QualitativeSpecies} is defined as being in the boundary on the  system, and cannot be used as an output in any transition."
%\medskip
%\LRG To declare a "delay" node, which is specified to delay its state update for $k$ iterations, one should set, for all the \qual{Transition} elements having this node as their (unique) output, the attribute \attr{temporisationType} to the value \const{timer} and the \attr{temporisationValue} to $k$. 

%\medskip
%\LRG To declare a "sustain" node, which is specified to sustain (i.e., to remain in) its latest state for the next $k$ iterations, one should set, for all the \qual{Transition} elements having this node as their (unique) output, the attribute \attr{temporisationType} to the value \const{sustain} and the \attr{temporisationValue} to $k$. 


% section recommended Practices (end)
