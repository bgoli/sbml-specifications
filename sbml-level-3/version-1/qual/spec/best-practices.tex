% -*- TeX-master: "main"; fill-column: 72 -*-

\section{Best practices}
\label{best-practices}

The aim of this specification is provide a common basis for modelling several different types of \textit{qualitative} models. To facilitate this goal the elements defined have attributes that are optional in some types of models but not in others; or indeed have meaning in some contexts but not in others.  Here we outline the two cases that can be modelled using the syntax of this specification.

\subsection{Logical Regulatory Networks}

In these models a \QualitativeSpecies is listed as an \Input to a \Transition if it acts as an activator/inhibitor or regulator in that \Transition and as an \Output if the evolution of that \QualitativeSpecies is governed by the \Transition i.e. it is a \textit{regulated} species. A \QualitativeSpecies that is an \Output must have a \token{constant} attribute set to \val{false}.

The \token{maxLevel} attribute on the \QualitativeSpecies can be set to indicate the possible values that the species can take. For example a boolean would have a \token{maxLevel} of \val{1} indicating that it could have only values of 0 and 1. 


In these classes of models the \token{transitionEffect} of the \Input species should be set to \val{none} as the \Input species are not altered by the transition.




The \qual{thresholdLevel}, when specified, indicates the level at which the species participates in the transition. Any reference to the \Input \qual{id} attribute in a <ci> element within a functionTerm of the transition refers to the value of this \qual{thresholdLevel}. This provides a means of encoding the statement: A transition occurs when the level of A exceeds the threshold level of B; as $A > thresholdB$ rather than $A > 1$; which may provide a modeller with additional information about the number.


The \qual{sign} attribute indicates the type of effect on the \Output of the \Transition (the regulated species): "positive" (activation), "negative" (inhibition), "dual" (positive or negative depending e.g. on co-factors) or "unknown". It is optional and mainly used for graphical purposes.

Discussions are still ongoing about the possible incoherency of the \qual{FunctionTerm} elements and the need to avoid cumbersome descriptions. As of the writing of this specification, the following guidelines are recommended to ensure coherent definitions:
\begin{itemize}
\item The \qual{FunctionTerm} elements of all the transitions targeting the same output should be "coherent": the conditions of two \qual{FunctionTerm} elements, leading to different effects on the level of the output, should not be fulfilled at the same time( i.e. they should be exclusive).
\item If several \qual{FunctionTerm} elements lead to the same effect on the level of the same output, then the importing tool should consider the disjunction (OR) on the conditions of the terms. 
\end{itemize}
 
\subsection{Petri Nets}

In Petri Net models the \QualitativeSpecies represent places within the model. Since the initial conditions are part of the model the \token{initialLevel} attribute for each \QualitativeSpecies must be set.

In order to represent an unbound place the \token{maxLevel} attribute of the \QualitativeSpecies is left unset.

The \qual{transitionEffect} of an \qual{Input} is set to "consumption", unless this input is connected to the transition by a test arc (meaning the transition has no effect on its marking). 

The \qual{thresholdLevel} of an \Input indicates the weight of the arc from this place to the transition and is required. It is used to specify the enabling conditions of the transition (and to indicate the number of tokens consumed by the firing of this transition). 

In this class of models the \qual{sign} attribute on an \Input should not be defined.

The \qual{transitionEffect} of an \qual{Output} is set to "production". 

The \qual{outputLevel} of an \Output indicates the weight of the arc from the transition to this place, it should be defined and is interpreted as  the number of tokens produced by the firing of this transition.




%\medskip
%\LRG To declare external nodes (ones that have no Boolean expression/truth table associated with them), one should set the attribute \attr{boundaryCondition} of the  \qual{QualitativeSpecies} to TRUE: 
%\begin{verbatim}
%<qualitativeSpecies id="EGF" maxLevel="1" boundaryCoundition="true" 
%                                                  compartment="extracellular"/> 
%\end{verbatim}

%Thus, the  \qual{QualitativeSpecies} is defined as being in the boundary on the  system, and cannot be used as an output in any transition."
%\medskip
%\LRG To declare a "delay" node, which is specified to delay its state update for $k$ iterations, one should set, for all the \qual{Transition} elements having this node as their (unique) output, the attribute \attr{temporisationType} to the value \const{timer} and the \attr{temporisationValue} to $k$. 

%\medskip
%\LRG To declare a "sustain" node, which is specified to sustain (i.e., to remain in) its latest state for the next $k$ iterations, one should set, for all the \qual{Transition} elements having this node as their (unique) output, the attribute \attr{temporisationType} to the value \const{sustain} and the \attr{temporisationValue} to $k$. 


% section recommended Practices (end)
