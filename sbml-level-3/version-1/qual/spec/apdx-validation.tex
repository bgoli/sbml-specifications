% -*- TeX-master: "main"; fill-column: 72 -*-

\section{Validation of SBML documents}
\label{apdx-validation}

\subsection{Validation and consistency rules}
\label{validation-rules}

This section summarizes all the conditions that must (or in some cases,
at least \emph{should}) be true of an SBML Level~3 Version~1 model that
uses the Qualitative Models package.  We use the same
conventions as are used in the SBML Level~3 Version~1 Core specification
document.  In particular, there are different degrees of rule
strictness.  Formally, the differences are expressed in the statement of
a rule: either a rule states that a condition \emph{must} be true, or a
rule states that it \emph{should} be true.  Rules of the former kind are
strict SBML validation rules---a model encoded in SBML must conform to
all of them in order to be considered valid.  Rules of the latter kind
are consistency rules.  To help highlight these differences, we use the
following three symbols next to the rule numbers:

\begin{description}

\item[\hspace*{6.5pt}\vSymbol\vsp] A \vSymbolName indicates a
  \emph{requirement} for SBML conformance. If a model does not follow
  this rule, it does not conform to the Qualitative Models
  specification.  (Mnemonic intention behind the choice of symbol:
  ``This must be checked.'')

\item[\hspace*{6.5pt}\cSymbol\csp] A \cSymbolName indicates a
  \emph{recommendation} for model consistency.  If a model does not
  follow this rule, it is not considered strictly invalid as far as
  the Qualitative Models specification is concerned;
  however, it indicates that the model contains a physical or
  conceptual inconsistency.  (Mnemonic intention behind the choice of
  symbol: ``This is a cause for warning.'')

\item[\hspace*{6.5pt}\mSymbol\msp] A \mSymbolName indicates a strong
  recommendation for good modeling practice.  This rule is not
  strictly a matter of SBML encoding, but the recommendation comes
  from logical reasoning.  As in the previous case, if a model does
  not follow this rule, it is not strictly considered an invalid SBML
  encoding.  (Mnemonic intention behind the choice of symbol: ``You're
  a star if you heed this.'')

\end{description}

The validation rules listed in the following subsections are all stated
or implied in the rest of this specification document.  They are
enumerated here for convenience.  Unless explicitly stated, all
validation rules concern objects and attributes specifically defined in
the Qualitative Models package.

For \notice convenience and brievity, we use the shorthand
``\token{qual:\-x}'' to stand for an attribute or element name \token{x}
in the namespace for the Qualitative Models package, using
the namespace prefix \token{qual}.  In reality, the prefix string may be
different from the literal ``\token{qual}'' used here (and indeed, it
can be any valid XML namespace prefix that the modeler or software
chooses).  We use ``\token{qual:\-x}'' because it is shorter than to
write a full explanation everywhere we refer to an attribute or element
in the Qualitative Models package namespace.


\subsubsection*{General rules about this package}

\validRule{qual-10101}{To conform to the Qualitative Models package specification for SBML Level~3 Version~1, an
  SBML document must declare the use of the following XML Namespace:\\
  \textls[-25]{\uri{http://www.sbml.org/sbml/level3/version1/qual/version1}}.
  (References: SBML Level~3 Package Specification for Qualitative Models, Version~1, \sec{xml-namespace}.)}

\validRule{qual-10102}{Wherever they appear in an SBML document,
  elements and attributes from the Qualitative Models
  package must be declared either implicitly or explicitly to be in the
  XML namespace
  \uri{http://www.sbml.org/sbml/level3/version1/comp/version1}.
  (References: SBML Level~3 Package Specification for Qualitative Models, Version~1, \sec{xml-namespace}.) }

\subsubsection*{General rules about identifiers} 

\validRule{qual-10301}{(Extends validation rule \#10301 in the SBML
  Level~3 Version~1 Core specification.) Within a \Model the values of the attributes
  \token{id} and \token{qual:\-id} on every instance of the following
  classes of objects must be unique across the set of all \token{id} and
  \token{qual:\-id} attribute values of all such objects in a model: the
  \Model itself, plus all contained \FunctionDefinition, \Compartment,
  \Species, \Reaction, \SpeciesReference, \ModifierSpeciesReference,
  \Event, and \Parameter objects, plus the \QualitativeSpecies, \Transition, 
  \Input and \Output
  objects defined by the Qualitative Models package.
  (References: SBML Level~3 Package Specification for Qualitative Models, Version~1, \sec{}.) }

\subsubsection*{Rules for the extended \class{SBML} class} 

\validRule{qual-20101}{In all SBML documents using the Qualitative Models package, 
  the \SBML object must include a value for
  the attribute \token{qual:\-required} attribute. 
   (References: SBML Level~3 Version~1 Core, Section~4.1.2.) }
  
\validRule{qual-20102}{The value of attribute \token{qual:\-required} on
  the \SBML object must be of the data type \primtype{boolean}.
  (References: SBML Level~3 Version~1 Core, Section~4.1.2.) }
  
\validRule{qual-20103}{The value of attribute \token{qual:\-required} on
  the \SBML object must be set to \val{true}.  (References: SBML Level~3 Package
  Specification for Qualitative Models, Version~1,
  \sec{xml-namespace}.) }

\subsubsection*{Rules for extended \class{Model} object} 

\validRule{qual-20201}{There may be at most one instance of each of the
  following kinds of objects within a \Model object using Qualitative Models: 
  \ListOfTransitions and \ListOfQualitativeSpecies.  (References:
  SBML Level~3 Package Specification for Qualitative Models,
  Version~1, \sec{model-class}.) }

\validRule{qual-20202}{The various
  \textsf{\textbf{ListOf\rule{0.15in}{0.5pt}}} subobjects with an \Model
  object are optional, but if present, these container object must not
  be empty.  Specifically, if any of the following classes of objects
  are present on the \Model, it must not be empty: \ListOfQualitativeSpecies and
  \ListOfTransitions.  (References: SBML Level~3 Package Specification for
  Qualitative Models, Version~1, \sec{model-class}.) }

\validRule{qual-20203}{Apart from the general notes and annotation
  subobjects permitted on all SBML objects, a \ListOfTransitions container
  object may only contain \Transition objects.  (References: SBML Level~3
  Package Specification for Qualitative Models, Version~1,
  \sec{model-class}.) }

\validRule{qual-20204}{Apart from the general notes and annotation
  subobjects permitted on all SBML objects, a \ListOfQualitativeSpecies container
  object may only contain \QualitativeSpecies objects.  (References: SBML Level~3
  Package Specification for Qualitative Models, Version~1,
  \sec{model-class}.) }
  
\validRule{qual-20205}{A \ListOfQualitativeSpecies object may have the optional
  \token{meta\-id} and \token{sboTerm} defined by SBML Level~3 Core.  No
  other attributes from the SBML Level~3 Core namespace or the
  Qualitative Models namespace are permitted on a
  \ListOfQualitativeSpecies object.  (References: SBML Level~3 Package
  Specification for Qualitative Models, Version~1,
  \sec{model-class}.) }

\validRule{qual-20206}{A \ListOfTransitions object may have the optional
  attributes \token{meta\-id} and \token{sboTerm} defined by SBML
  Level~3 Core.  No other attributes from the SBML Level~3 Core
  namespace or the Qualitative Models namespace are
  permitted on a \ListOfTransitions object.  (References: SBML Level~3 Package
  Specification for Qualitative Models, Version~1,
  \sec{model-class}.) }

\subsubsection*{Rules for \class{QualitativeSpecies} object} 

\validRule{qual-20301}{A \QualitativeSpecies object may have the optional SBML Level 3 Core attributes \token{metaid} and \token{sboTerm}. No other attributes from the SBML Level 3 Core namespaces are permitted on a \QualitativeSpecies. (References: SBML Level~3 Version~1 Core, Section~3.2.) }

\validRule{qual-20302}{A \QualitativeSpecies object may have the optional SBML Level 3 Core subobjects for notes and annotations. No other elements from the SBML Level 3 Core namespaces are permitted on a \QualitativeSpecies. (References: SBML Level~3 Version~1 Core, Section~3.2.) }

\validRule{qual-20303}{A \QualitativeSpecies object must have the required attributes \token{qual:\-id},
\token{qual:\-compartment} and \token{qual:\-constant}, and may have the optional
attributes \token{qual:\-name}, \token{qual:\-initialLevel} 
and \token{qual:\-maxLevel}.  No other attributes from the SBML Level~3
Qualitative Models namespace are permitted on a \QualitativeSpecies object. (References: SBML Level~3 Package
  Specification for Qualitative Models, Version~1,
  \sec{qualSpecies-class}.) }


\validRule{qual-20304}{The attribute \token{qual:\-constant} in \QualitativeSpecies must be of the data type \token{boolean}.    (References: SBML Level~3 Package
  Specification for Qualitative Models, Version~1,
  \sec{qualSpecies-class}.) }


\validRule{qual-20305}{The attribute \token{qual:\-name} in \QualitativeSpecies must be of the data type \token{string}.    (References: SBML Level~3 Package
  Specification for Qualitative Models, Version~1,
  \sec{qualSpecies-class}.) }

\validRule{qual-20306}{The attribute \token{qual:\-initialLevel} in \QualitativeSpecies must be of the data type \token{integer}.    (References: SBML Level~3 Package
  Specification for Qualitative Models, Version~1,
  \sec{qualSpecies-class}.) }

\validRule{qual-20307}{The attribute \token{qual:\-maxLevel} in \QualitativeSpecies must be of the data type \token{integer}.    (References: SBML Level~3 Package
  Specification for Qualitative Models, Version~1,
  \sec{qualSpecies-class}.) }


\validRule{qual-20308}{The value of the attribute \token{qual:\-compartment} in a \QualitativeSpecies object must be
the identifier of an existing \Compartment object defined in the enclosing
\Model object.  (References: SBML Level~3 Package
  Specification for Qualitative Models, Version~1,
  \sec{qualSpecies-class}.) }

\validRule{qual-20309}{The value of the attribute \token{qual:\-initialLevel} in a \QualitativeSpecies object cannot be greater than the value of the \token{qual:\-maxLevel} attribute for the given \QualitativeSpecies object.  (References: SBML Level~3 Package
  Specification for Qualitative Models, Version~1,
  \sec{qualSpecies-class}.) }
  
