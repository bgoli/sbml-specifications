% -*- TeX-master: "main"; fill-column: 72 -*-

\section{Background and context}
\label{background}

It is possible to represent some qualitative models using SBML Level~2 or indeed SBML Level~3~Core. However,  after several attempts, experience showed that the possible confusion caused by the presence of irrelevant attributes and the need to reinterpret the semantics of some SBML elements could lead to ambiguity. At this point the decision was made to develop an SBML Level~3 package that captured the nature of qualitative models.


A first proposal was written in August 2008 by Duncan Berenguier and Nicolas Le Nov\`ere and discussed during a dedicated meeting on the 12th and 13th of August 2008. This meeting brought together a number of people who specialised in qualitative modelling. A summary of the meeting is available at \url{http://www.ebi.ac.uk/compneur/xwiki/bin/view/SBML/L3F} which also provides a link to the revised proposal document that was produced as a result of this meeting.

A secondary, but very valuable, outcome of this meeting was the formation of the Common Logical Modelling Toolbox (CoLoMoTo) community. A community that focuses on logical modelling but who are committed to making their models exchangeable and reusable as widely as possible.  This small focussed community then took control of developing the SBML L3 Qualitaive Models package.

The first CoLoMoTo meeting was held at in November 2010 (see \url{http://compbio.igc.gulbenkian.pt/nmd/node/30}, for the program and participants). A revised version of the proposal was discussed during this meeting and a formal SBML L3 proposal document was written and circulated as a result of these and other discussions.
This document is available at \url{http://sbml.org/images/6/61/SBML-L3-qual-proposal_2.1.pdf}.

The proposal was voted on and accepted by the SBML community (June 2011) and a dedicated discussion list set up (\url{https://lists.sourceforge.net/lists/listinfo/sbml-qual}). The package was presented at COMBINE 2011.

A second CoLoMoTo meeting took place in March 2012 (see \url{http://co.mbine.org/colomoto/meetings/2012} ). During this meeting it was decided that there were parts of the proposal that had been introduced in anticipation of the future development of models. Whilst these are valuable aspects of the proposal there is no software supporting these features as yet. It was therefore decided to remove these features from a version 1 specification and reconsider them in the future for subsequent versions of the Qualitative Models package. A summary of these features is given in Appendix B \ref{} of this document.


