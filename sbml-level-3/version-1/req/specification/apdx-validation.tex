% -*- TeX-master: "main" -*-

\section{\texorpdfstring{\textls[-15]{Validation of SBML documents using Required Elements constructs}}{Validation of SBML documents using Required Elements constructs}}
\label{apdx-validation}
\label{validation-rules}

This section summarizes all the conditions that must (or in some cases, at least \emph{should}) be true of an SBML Level~3 Version~1 model that uses the Required Elements package.  We use the same conventions that are used in the SBML Level~3 Version~1 Core specification document.  In particular, there are different degrees of rule strictness.  Formally, the differences are expressed in the statement of a rule: either a rule states that a condition \emph{must} be true, or a rule states that it \emph{should} be true.  Rules of the former kind are strict SBML validation rules---a model encoded in SBML must conform to all of them in order to be considered valid.  Rules of the latter kind are consistency rules.  To help highlight these differences, we use the following three symbols next to the rule numbers:

\begin{description}

\item[\hspace*{6.5pt}\vSymbol\vsp] A \vSymbolName indicates a \emph{requirement} for SBML conformance. If a model does not follow this rule, it does not conform to the Required Elements package specification.  (Mnemonic intention behind the choice of symbol: ``This must be checked.'')

\item[\hspace*{6.5pt}\cSymbol\csp] A \cSymbolName indicates a \emph{recommendation} for model consistency.  If a model does not follow this rule, it is not considered strictly invalid as far as the Required Elements package specification is concerned; however, it indicates that the model contains a physical or conceptual inconsistency.  (Mnemonic intention behind the choice of symbol: ``This is a cause for warning.'')

\item[\hspace*{6.5pt}\mSymbol\msp] A \mSymbolName indicates a strong recommendation for good modeling practice.  This rule is not strictly a matter of SBML encoding, but the recommendation comes from logical reasoning.  As in the previous case, if a model does not follow this rule, it is not considered an invalid SBML encoding.  (Mnemonic intention behind the choice of symbol: ``You're a star if you heed this.'')

\end{description}

The validation rules listed in the following subsections are all stated or implied in the rest of this specification document.  They are enumerated here for convenience.  Unless explicitly stated, all validation rules concern objects and attributes specifically defined in the Required Elements package.

For \notice convenience and brievity, we use the shorthand ``\token{req:x}'' to stand for an attribute or element name \token{x} in the namespace for the Required Elements package, using the namespace prefix \token{req}.  In reality, the prefix string may be different from the literal ``\token{req}'' used here (and indeed, it can be any valid XML namespace prefix that the modeler or software chooses).  We use ``\token{req:x}'' because it is shorter than to write a full explanation everywhere we refer to an attribute or element in the Required Elements package namespace.


\subsection*{General rules about the Required Elements package}

\validRule{req-10101}{To conform to Version 1 of the Required Elements package specification for SBML Level~3, an
  SBML document must declare the use of the following XML Namespace:\\
  \textls[-12]{\uri{http://www.sbml.org/sbml/level3/version1/req/version1}}.  (References: SBML Level~3 Package Specification for Required Elements, Version~1, \sec{xml-namespace}.)}
  
\validRule{req-10102}{Wherever they appear in an SBML document, elements and attributes from the Required Elements package must be declared either implicitly or explicitly to be in the XML namespace\\ \textls[-12]{\uri{http://www.sbml.org/sbml/level3/version1/req/version1}}.  (References: SBML Level~3 Package Specification for Required Elements, Version~1, \sec{xml-namespace}.) }

\validRule{req-10103}{The value of attribute \token{req:required} on the \SBML object must be set to \val{false}.  (References: SBML Level~3 Package Specification for Required Elements, Version~1, \sec{xml-namespace}.) }


\subsection*{Rules for Required Elements attributes}

\validRule{req-20101}{The attribute \token{mathOverridden}, if present, must have a value of type \primtype{string}.  (References: SBML Level~3 Package Specification for Required Elements, Version~1, \sec{extended-sbase-class}.) }

\clearpage

\validRule{req-20102}{The attribute \token{mathOverridden}, if present, must be the label of an SBML Level~3 package present in the same document.  (References: SBML Level~3 Package Specification for Required Elements, Version~1, \sec{extended-sbase-class}.) }

\validRule{req-20103}{The attribute \token{coreHasAlternativeMath}, if present, must have a value of type \primtype{boolean}.  (References: SBML Level~3 Package Specification for Required Elements, Version~1, \sec{extended-sbase-class}.) 
}

\validRule{req-20104}{If the value of the attribute \token{coreHasAlternativeMath} is set \primtype{true}, any \primtype{math} child of the parent element must be present and complete.  (References: SBML Level~3 Package Specification for Required Elements, Version~1, \sec{extended-sbase-class}.) }
