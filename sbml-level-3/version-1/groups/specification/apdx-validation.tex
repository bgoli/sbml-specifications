% -*- TeX-master: "main"; fill-column: 72 -*-

\section{Validation of SBML documents}
\label{apdx-validation}

\subsection{Validation and consistency rules}
\label{validation-rules}

This section summarizes all the conditions that must (or in some cases, at least \emph{should}) be true of an SBML Level~3 Version~1 model that uses the Groups package.  We use the same conventions that are used in the SBML Level~3 Version~1 Core specification document.  In particular, there are different degrees of rule strictness.  Formally, the differences are expressed in the statement of a rule: either a rule states that a condition \emph{must} be true, or a rule states that it \emph{should} be true.  Rules of the former kind are strict SBML validation rules---a model encoded in SBML must conform to all of them in order to be considered valid.  Rules of the latter kind are consistency rules.  To help highlight these differences, we use the following three symbols next to the rule numbers:

\begin{description}

\item[\hspace*{6.5pt}\vSymbol\vsp] A \vSymbolName indicates a \emph{requirement} for SBML conformance. If a model does not follow this rule, it does not conform to the Groups package specification.  (Mnemonic intention behind the choice of symbol: ``This must be checked.'')

\item[\hspace*{6.5pt}\cSymbol\csp] A \cSymbolName indicates a \emph{recommendation} for model consistency.  If a model does not follow this rule, it is not considered strictly invalid as far as the Groups package specification is concerned; however, it indicates that the model contains a physical or conceptual inconsistency.  (Mnemonic intention behind the choice of symbol: ``This is a cause for warning.'')

\item[\hspace*{6.5pt}\mSymbol\msp] A \mSymbolName indicates a strong recommendation for good modeling practice.  This rule is not strictly a matter of SBML encoding, but the recommendation comes from logical reasoning.  As in the previous case, if a model does not follow this rule, it is not considered an invalid SBML encoding.  (Mnemonic intention behind the choice of symbol: ``You're a star if you heed this.'')

\end{description}

The validation rules listed in the following subsections are all stated or implied in the rest of this specification document.  They are enumerated here for convenience.  Unless explicitly stated, all validation rules concern objects and attributes specifically defined in the Groups package.

For \notice convenience and brievity, we use the shorthand ``\token{groups:x}'' to stand for an attribute or element name \token{x} in the namespace for the Groups package, using the namespace prefix \token{groups}.  In reality, the prefix string may be different from the literal ``\token{groups}'' used here (and indeed, it can be any valid XML namespace prefix that the modeler or software chooses).  We use ``\token{groups:x}'' because it is shorter than to write a full explanation everywhere we refer to an attribute or element in the Groups package namespace.

\begin{blockChanged}
\subsubsection*{General rules about this package}

\validRule{groups-10101}{To conform to the \GroupsPackage specification
for SBML Level~3 Version~1, an SBML document must declare 
\uri{http://www.sbml.org/sbml/level3/version1/groups/version1} as the
XML Namespace to use for elements of this package.
(Reference: SBML Level~3 Package specification for Groups, Version~1
\sec{xml-namespace}.)}

\validRule{groups-10102}{Wherever they appear in an SBML document,
elements and attributes from the \GroupsPackage must use the 
\uri{http://www.sbml.org/sbml/level3/version1/groups/version1} namespace, declaring so either implicitly or explicitly.
(Reference: SBML Level~3 Package specification for Groups, Version~1
\sec{xml-namespace}.)}

\subsubsection*{General rules about identifiers}

\validRule{groups-10301}{(Extends validation rule \#10301 in the SBML Level~3 Version~1 Core specification.) Within a \Model object, the values of the attributes \token{id} and \token{groups:id} on every instance of the following classes of objects must be unique across the set of all \token{id} and \token{groups:id} attribute values of all such objects in a model: the \Model itself, plus all contained \FunctionDefinition, \Compartment, \Species, \Reaction, \SpeciesReference, \ModifierSpeciesReference, \Event, and \Parameter objects, plus the \Group, \ListOfMembers, and \Member objects defined by the Groups package, plus any objects defined by any other package with \token{package:id} attributes defined as falling in the 'SId' namespace. (References: \sbmlthreegroups, \sec{group-class}.) }

\validRule{groups-10302}{The value of a \token{groups:\-id} must conform
to the syntax of the \class{SBML} data type \primtype{SId} (Reference:
SBML Level~3 Package specification for Groups, Version~1
\sec{group-idname-attributes}.)}

\subsubsection*{Rules for the extended \class{SBML} class}

\validRule{groups-20101}{In all SBML documents using the \GroupsPackage,
the \class{SBML} object must have the \token{groups:\-required}
attribute. (Reference: SBML Level~3 Version~1 Core, Section~4.1.2.)}

\validRule{groups-20102}{The value of attribute
\token{groups:\-required} on the \class{SBML} object must be of data
type \primtype{boolean}. (Reference: SBML Level~3 Version~1 Core,
Section~4.1.2.)}

\validRule{groups-20103}{The value of attribute
\token{groups:\-required} on the \class{SBML} object must be set to
\val{false}. (Reference: SBML Level~3 Package specification for Groups,
Version~1 \sec{xml-namespace}.)}


\subsubsection*{Rules for extended \class{Model} object}

\validRule{groups-20201}{A \Model object may contain one and only one
instance of the \ListOfGroups element. No other elements from the SBML
Level 3 Groups namespaces are permitted on a \Model object. (Reference:
SBML Level~3 Package specification for Groups, Version~1,
\sec{model-class}.)}

\validRule{groups-20202}{The \ListOfGroups subobject on a \Model object
is optional, but if present, this container object must not be empty.
(Reference: SBML Level~3 Specification for Groups Version~1,
\sec{model-class}.)}

\validRule{groups-20203}{Apart from the general notes and annotations
subobjects permitted on all SBML objects, a \ListOfGroups container
object may only contain \Group objects. (Reference: SBML Level~3 Package
specification for Groups, Version~1, \sec{model-class}.)}

\validRule{groups-20204}{A \ListOfGroups object may have the optional
SBML Level~3 Core attributes \token{metaid} and \token{sboTerm}. No
other attributes from the SBML Level 3 Core namespaces are permitted on
a \ListOfGroups object. (Reference: SBML Level~3 Package specification
for Groups, Version~1, \sec{model-class}.)}


\subsubsection*{Rules for \class{Group} object}

\validRule{groups-20301}{A \Group object may have the optional SBML
Level~3 Core attributes \token{metaid} and \token{sboTerm}. No other
attributes from the SBML Level 3 Core namespaces are permitted on a
\Group. (Reference: SBML Level~3 Version~1 Core, Section~3.2.)}

\validRule{groups-20302}{A \Group object may have the optional SBML
Level~3 Core subobjects for notes and annotations. No other elements
from the SBML Level 3 Core namespaces are permitted on a \Group.
(Reference: SBML Level~3 Version~1 Core, Section~3.2.)}

\validRule{groups-20303}{A \Group object must have the required
attribute \token{groups:\-kind}, and may have the optional attributes
\token{groups:\-id} and \token{groups:\-name}. No other attributes from
the SBML Level 3 Groups namespaces are permitted on a \Group object.
(Reference: SBML Level~3 Package specification for Groups, Version~1,
\sec{group-class}.)}

\validRule{groups-20304}{A \Group object may contain one and only one
instance of the \ListOfMembers element. No other elements from the SBML
Level 3 Groups namespaces are permitted on a \Group object. (Reference:
SBML Level~3 Package specification for Groups, Version~1,
\sec{group-class}.)}

\validRule{groups-20305}{The value of the attribute
\token{groups:\-kind} of a \Group object must conform to the syntax of
SBML data type \primtype{groupKind} and may only take on the allowed
values of \primtype{groupKind} defined in SBML; that is the value must
be one of the following: 'classification', 'partonomy' or 'collection'.
(Reference: SBML Level~3 Package specification for Groups, Version~1,
\sec{group-class}.)}

\validRule{groups-20306}{The attribute \token{groups:\-name} on a \Group
must have a value of data type \token{string}. (Reference: SBML Level~3
Package specification for Groups, Version~1, \sec{group-class}.)}

\validRule{groups-20307}{The \ListOfMembers subobject on a \Group object
is optional, but if present, this container object must not be empty.
(Reference: SBML Level~3 Package specification for Groups, Version~1,
\sec{group-class}.)}

\validRule{groups-20308}{Apart from the general notes and annotations
subobjects permitted on all SBML objects, a \ListOfMembers container
object may only contain \Member objects. (Reference: SBML Level~3
Package specification for Groups, Version~1, \sec{group-class}.)}

\validRule{groups-20309}{A \ListOfMembers object may have the optional
SBML Level~3 Core attributes \token{metaid} and \token{sboTerm}. No
other attributes from the SBML Level 3 Core namespaces are permitted on
a \ListOfMembers object. (Reference: SBML Level~3 Package specification
for Groups, Version~1, \sec{group-class}.)}

\consistencyRule{groups-20310}{If multiple \ListOfMembers objects contain \Member elements that reference the same SBML object, the \token{sboterm} and any child \Notes or \Annotation elements set for those \ListOfMembers should be consistent, as they all should apply to the same referenced object.  (References: \sbmlthreegroups, \sec{listofmembers-class}.) }



\subsubsection*{Rules for \class{Member} object}

\validRule{groups-20401}{A \Member object may have the optional SBML
Level~3 Core attributes \token{metaid} and \token{sboTerm}. No other
attributes from the SBML Level 3 Core namespaces are permitted on a
\Member. (Reference: SBML Level~3 Version~1 Core, Section~3.2.)}

\validRule{groups-20402}{A \Member object may have the optional SBML
Level~3 Core subobjects for notes and annotations. No other elements
from the SBML Level 3 Core namespaces are permitted on a \Member.
(Reference: SBML Level~3 Version~1 Core, Section~3.2.)}

\validRule{groups-20403}{A \Member object may have the optional
attributes \token{groups:\-id} and \token{groups:\-name} and must have a value for one (and exactly one) of attributes
\token{groups:\-idRef} and \token{groups:\-metaIdRef}. No other
attributes from the SBML Level 3 Groups namespaces are permitted on a
\Member object. (Reference: SBML Level~3 Package specification for
Groups, Version~1, \sec{member-class}.)}

\validRule{groups-20404}{The attribute \token{groups:\-name} on a
\Member must have a value of data type \token{string}. (Reference: SBML
Level~3 Package specification for Groups, Version~1,
\sec{member-class}.)}


\validRule{groups-20405}{The value of the attribute
\token{groups:\-idRef} of a \Member object must be the identifier of an
existing \SBase object defined in the enclosing \Model object.
(Reference: SBML Level~3 Package specification for Groups, Version~1,
\sec{member-class}.)}

\validRule{groups-20406}{The value of the attribute
\token{groups:\-metaIdRef} of a \Member object must be the \token{metaid} of an
existing \SBase object defined in the enclosing \Model object.
(Reference: SBML Level~3 Package specification for Groups, Version~1,
\sec{member-class}.)}

\validRule{groups-20407}{The value of a \token{groups:idRef} attribute on \Member objects must conform to the syntax of the SBML data type \primtype{SIdRef}.  (References: \sbmlthreegroups, \sec{member-class}.) }

\validRule{groups-20408}{The value of a \token{groups:metaIdRef} attribute on \Member objects must conform to the syntax of the SBML data type \primtype{IDREF}.  (References:\sbmlthreegroups, \sec{member-class}.) }

\end{blockChanged}
