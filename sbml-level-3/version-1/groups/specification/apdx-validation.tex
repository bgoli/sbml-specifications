% -*- TeX-master: "main" -*-

\section{Validation of SBML documents using Groups constructs}
\label{apdx-validation}
\label{validation-rules}

This section summarizes all the conditions that must (or in some cases, at least \emph{should}) be true of an SBML Level~3 Version~1 model that uses the Groups package.  We use the same conventions that are used in the SBML Level~3 Version~1 Core specification document.  In particular, there are different degrees of rule strictness.  Formally, the differences are expressed in the statement of a rule: either a rule states that a condition \emph{must} be true, or a rule states that it \emph{should} be true.  Rules of the former kind are strict SBML validation rules---a model encoded in SBML must conform to all of them in order to be considered valid.  Rules of the latter kind are consistency rules.  To help highlight these differences, we use the following three symbols next to the rule numbers:

\begin{description}

\item[\hspace*{6.5pt}\vSymbol\vsp] A \vSymbolName indicates a \emph{requirement} for SBML conformance. If a model does not follow this rule, it does not conform to the Groups package specification.  (Mnemonic intention behind the choice of symbol: ``This must be checked.'')

\item[\hspace*{6.5pt}\cSymbol\csp] A \cSymbolName indicates a \emph{recommendation} for model consistency.  If a model does not follow this rule, it is not considered strictly invalid as far as the Groups package specification is concerned; however, it indicates that the model contains a physical or conceptual inconsistency.  (Mnemonic intention behind the choice of symbol: ``This is a cause for warning.'')

\item[\hspace*{6.5pt}\mSymbol\msp] A \mSymbolName indicates a strong recommendation for good modeling practice.  This rule is not strictly a matter of SBML encoding, but the recommendation comes from logical reasoning.  As in the previous case, if a model does not follow this rule, it is not considered an invalid SBML encoding.  (Mnemonic intention behind the choice of symbol: ``You're a star if you heed this.'')

\end{description}

The validation rules listed in the following subsections are all stated or implied in the rest of this specification document.  They are enumerated here for convenience.  Unless explicitly stated, all validation rules concern objects and attributes specifically defined in the Groups package.

For \notice convenience and brievity, we use the shorthand ``\token{groups:x}'' to stand for an attribute or element name \token{x} in the namespace for the Groups package, using the namespace prefix \token{groups}.  In reality, the prefix string may be different from the literal ``\token{groups}'' used here (and indeed, it can be any valid XML namespace prefix that the modeler or software chooses).  We use ``\token{groups:x}'' because it is shorter than to write a full explanation everywhere we refer to an attribute or element in the Groups package namespace.


\subsection*{General rules about the Groups package}

\validRule{groups-10101}{To conform to Version 1 of the Groups package specification for SBML Level~3, an
  SBML document must declare the use of the following XML Namespace:\\
  \uri{http://www.sbml.org/sbml/level3/version1/groups/version1}.  (References: SBML Level~3 Package Specification for Groups, Version~1, \sec{xml-namespace}.)}
  
\validRule{groups-10102}{Wherever they appear in an SBML document, elements and attributes from the Groups package must be declared either implicitly or explicitly to be in the XML namespace\\ \uri{http://www.sbml.org/sbml/level3/version1/groups/version1}.  (References: SBML Level~3 Package Specification for Groups, Version~1, \sec{xml-namespace}.) }


\subsection*{General rules about identifiers} 

\validRule{groups-10301}{(Extends validation rule \#10301 in the SBML Level~3 Version~1 Core specification.) The values of the attribute \token{groups:id} on every instance of \Group objects must be unique across the set of all \token{groups:id} attribute values of all such objects in a model.  (References: SBML Level~3 Package Specification for Groups, Version~1, \sec{group-class}.) }

\clearpage

\validRule{groups-10302}{The value of a \token{groups:symbol} attribute on \Member objects must conform to the syntax of the SBML data type \primtype{SIdRef}.  (References: SBML Level~3 Package Specification for Groups, Version~1, \sec{member-class}.) }


\subsection*{Rules for extended \class{Model} objects} 

\validRule{groups-20101}{There may be at most one \ListOfGroups container object within a \Model object.  (References: SBML Level~3 Package Specification for Groups, Version~1, \sec{model-class}.) }

\validRule{groups-20102}{A \ListOfGroups container object within a \Model object is optional, but if present, must not be empty.  (References: SBML Level~3 Package Specification for Groups, Version~1, \sec{model-class}.) }
  
\validRule{groups-20103}{Apart from the general notes and annotation subobjects permitted on all SBML objects, a \ListOfGroups container object may only contain \Group objects.  (References: SBML Level~3 Package Specification for Groups, Version~1, \sec{model-class}.) }

\validRule{groups-20104}{A \ListOfGroups object may have the optional SBML core attributes \token{metaid} and \token{sboTerm}.  No other attributes from the SBML Level~3 Core namespace or the comp namespace are permitted on a \ListOfGroups object.  (References: SBML Level~3 Package Specification for Groups, Version~1, \sec{model-class}.) }


\subsection*{Rules for \class{Group} objects} 

\validRule{groups-20201}{A \Group object may have the optional SBML Level~3 Core attributes \token{metaid} and \token{sboTerm}.  No other attributes from the SBML Level~3 Core namespace are permitted on a \Group object.  (References: SBML Level~3 Version~1 Core, Section~3.2.) }

\validRule{groups-20202}{A \Group object must have the attribute \mbox{\token{groups:kind}} because it is a required attribute. (References: SBML Level~3 Package Specification for Groups, Version~1, \sec{group-class}.) }

\validRule{groups-20203}{The value of the \token{kind} attribute on a \Group object must have one of the following values: \val{classification}, \val{partonomy}, or \val{collection}.  These are the only permitted values for the \token{kind} attribute on \Group. (References: SBML Level~3 Package Specification for Groups, Version~1, \sec{group-class}.) }

\validRule{groups-20204}{There may be at most one \ListOfMembers container object within a \Group object.  (References: SBML Level~3 Package Specification for Groups, Version~1, \sec{group-class}.) }

\validRule{groups-20205}{A \ListOfMembers container object within a \Group object is optional, but if present, must not be empty.  (References: SBML Level~3 Package Specification for Groups, Version~1, \sec{group-class}.) }

\validRule{groups-20206}{Apart from the general notes and annotation subobjects permitted on all SBML objects, a \ListOfMembers container object may only contain \Member objects.  (References: SBML Level~3 Package Specification for Groups, Version~1, \sec{group-class}.) }

\validRule{groups-20207}{A \ListOfMembers object may have the optional SBML core attributes \token{metaid} and \token{sboTerm}.  No other attributes from the SBML Level~3 Core namespace or the comp namespace are permitted on a \ListOfMembers object.  (References: SBML Level~3 Package Specification for Groups, Version~1, \sec{group-class}.) }


\subsection*{Rules for \class{Member} objects} 

\validRule{groups-20201}{A \Member object must have the attribute \mbox{\token{groups:symbol}} because it is a required attribute.  No other attributes from the Groups namespace are permitted on a \Member object. (References: SBML Level~3 Package Specification for Groups, Version~1, \sec{member-class}.) }

\validRule{groups-20202}{The value of the \token{groups:symbol} attribute on a \Member object must be the value of an \token{id} attribute on an existing object in the enclosing SBML model.  (References: SBML Level~3 Package Specification for Groups, Version~1, \sec{member-class}.) }
