% -*- TeX-master: "main" -*-

\section{Introduction}
\label{intro}

The SBML Level~3 Groups package offers a more flexible mechanism for indicating that components of an SBML model are related in some way.  The nature of the relationship is left up to the modeler, and can be clarified by means of annotations on model components.  Groups may contain either the same or different types of SBML objects, and groups may be nested if desired.  There are no predefined behavioral semantics associated with groups.


\subsection{Proposal corresponding to this package specification}

This specification for Groups in SBML Level~3 Version~1 is based on the proposal located at the following URL:

\begin{center}
  \vspace*{1ex}\small
  \url{https://sbml.svn.sf.net/svnroot/sbml/trunk/specifications/sbml-level-3/version-1/groups/proposal}
  \vspace*{1ex}
\end{center}

The tracking number in the SBML issue tracking system~\citep{tracker} for Groups package activities is 2847474.  The version of the proposal used as the starting point for this specification is the version of June, 2012~\citep{hucka_2012}.


\subsection{Package dependencies}

The Groups package has no dependencies on other SBML Level~3 packages.


\subsection{Document conventions}
\label{conventions}

UML~1.0 (Unified Modeling Language; \citealt{eriksson:1998, oestereich:1999}) notation is used in this document to define the constructs provided by this package.  Colors in the diagrams carry the following additional information for the benefit of those viewing the document on media that can display color:

\begin{itemize}

\item[\raisebox{2.75pt}{\colorbox{black}{\rule{0.8pt}{0.8pt}}}]
  \emph{Black}: Items colored black in the UML diagrams are components
  taken unchanged from their definition in the SBML Level~3 Core
  specification document.

\item[\raisebox{2.75pt}{\colorbox{mediumgreen}{\rule{0.8pt}{0.8pt}}}]
  \emph{\textcolor{mediumgreen}{Green}}: Items colored green are
  components that exist in SBML Level~3 Core, but are extended by this
  package.  Class boxes are also drawn with dashed lines to further
  distinguish them.

\item[\raisebox{2.75pt}{\colorbox{darkblue}{\rule{0.8pt}{0.8pt}}}]
  \emph{\textcolor{darkblue}{Blue}}: Items colored blue are new
  components introduced in this package specification.  They have no
  equivalent in the SBML Level~3 Core specification.

\end{itemize}

The following typographical conventions distinguish the names of objects and data types from other entities; these conventions are identical to the conventions used in the SBML Level~3 Core specification document:

\begin{description}
  
\item \abstractclass{AbstractClass}: Abstract classes are never instantiated directly, but rather serve as parents of other classes.  Their names begin with a capital letter and they are printed in a slanted, bold, sans-serif typeface.  In electronic document formats, the class names defined within this document are also hyperlinked to their definitions; clicking on these items will, given appropriate software, switch the view to the section in this document containing the definition of that class.  (However, for classes that are unchanged from their definitions in SBML Level~3 Core, the class names are not hyperlinked because they are not defined within this document.)
  
\item \class{Class}: Names of ordinary (concrete) classes begin with a capital letter and are printed in an upright, bold, sans-serif typeface.  In electronic document formats, the class names are also hyperlinked to their definitions in this specification document.  (However, as in the previous case, class names are not hyperlinked if they are for classes that are unchanged from their definitions in the SBML Level~3 Core specification.)

\item \token{SomeThing}, \token{otherThing}: Attributes of classes, data type names, literal XML, and tokens \emph{other} than SBML class names, are printed in an upright typewriter typeface.

\end{description}

For other matters involving the use of UML and XML, this document follows the conventions used in the SBML Level~3 Core specification document.
