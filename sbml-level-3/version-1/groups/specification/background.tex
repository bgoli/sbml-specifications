% -*- TeX-master: "main" -*-

\section{Background and context}
\label{background}

SBML Level 2 Versions~2--4 provides two object classes, \CompartmentType and \SpeciesType, meant to allow the definition of types of compartments and species. The original motivation for the introduction of these two classes of objects was the expected introduction of a facility for defining generalized reactions, a scheme that would have allowed reactions to be defined on whole classes of entities in a compact format. However, generalized reactions never ended up being introduced in SBML Level~2, and the notion of generalized reactions has been superceded by the effort to support rule-based models using the Level~3 Multistate and Multicomponent Species package.  Moreover, the existence of just two types was never satisfactory in Level~2, because it did not satisfy the occasional desire to have other types of objects, such as parameters.  For these reasons, when SBML Level~3 was developed, \CompartmentType and \SpeciesType were removed, with the expectation that SBML Level~3 packages would be developed to take their place.

%VERSION2
%The SBML Level~3 Groups package is intended to fill the gaps left by the absence of \CompartmentType and \SpeciesType from \sbmlthreecore, while offering a more flexible mechanism for indicating that components of an SBML model are related.  The nature of the relationship is left up to the modeler.  This package can entirely take the place of \SpeciesType, but the modeler must explicitly create the restrictions previously automatically associated with that element; as a trade-off, the Groups package offers modelers and software tools the ability to indicate relationships and restrictions between any type of SBML component, not just species and compartments.
The SBML Level~3 Groups package is intended to fill, in at least some capacity, the gaps left by the absence of \CompartmentType and \SpeciesType from \sbmlthreecore, while also offering a more flexible mechanism for indicating that components of an SBML model are related.  The nature of the relationship is left up to the modeler.  The Version~1 definition of this SBML Level~3 package, cannot, unfortunately, entirely take the place of SBML Level~2's \SpeciesType, since the definition of \SpeciesType includes usage and validation rules that are not present the final Version~1 of the Groups package.  Constructs providing the ability to define such validation rules have been designed, but ultimately were removed from the Version~1 specification because (1) their implementation proved to be a challenge and (2) none of the developers who implemented preliminary support for the Groups specification voiced a need or desire for these additional capabilities.  Those constructs have been removed from the final Version~1 of the Groups package, but the potential remains to add them back in a Version~2 specification.

The term \emph{groups} is used in this package rather than \emph{types}, because the latter would imply stronger behavioral constraints on objects than what Groups provides. This package only provides a way of conceptually grouping components of a model.  It does not provide a way to define types in the computer science sense; therefore, a different term is appropriate.


\subsection{Prior work}

The earliest relevant work is the development of the \CompartmentType and \SpeciesType object classes in SBML Level~2 beginning with Version~2~\citep{l2v2}. The original design was based on Andrew Finney's proposal for these object classes, which was made in the context of Finney's proposal for multicomponent species for SBML Level~3~\citep{finney_2004}.  The \SpeciesType and \CompartmentType classes were included in SBML Level~2 Version~2; however, a community vote held in 2006~\citep{vote_2006b} resulted in the decision that corresponding changes to \Reaction be postponed to SBML Level~3.  Level~2 was thus left only with the typing mechanism for species and compartments.  Eventually, further work on generalized reactions and rule-based modeling was moved to an SBML Level~3 package, and \SpeciesType and \CompartmentType were removed from the core of SBML Level~3.  This left the SBML Level~3 core without an explicit mechanism for defining species types or any other types.

The first version of the Groups proposal was produced by the first author in 2009~\citep{hucka_2009}, after discussions with the SBML Editors during the 2009 SBML Hackathon held at the European Bioinformatics Institute~\citep{sbml_hackathon_2009}.  The original proposal differed from the current proposal in several respects. The most notable architectural difference is that membership in groups was done in an inverted fashion compared to the current specification: model entities contained structures that indicated which groups they belonged to, rather than the current scheme, in which group definitions include lists of their members.  The current scheme was developed during discussions with the SBML Editors during HARMONY 2011 in New York City~\citep{harmony_2011} and HARMONY 2012 in Maastricht, The Netherlands~\citep{harmony_2012}.

The idea for using an attribute to indicate the type of a group (i.e., a whether it is a classification, partonomy or simply a collection) was put forward by Nicolas Le~Nov\`{e}re during the discussions at HARMONY 2012.  The resulting attribute \token{kind} on the \Group object class solved this problem.

%VERSION2
%Finally, the idea of adding constraints to a Group in order to fully recapture the original validation restrictions of \SpeciesType was proposed by Lucian Smith, and incorporated into Release~0.2 of the specification.
