% -*- TeX-master: "main" -*-

\section{Background and context}
\label{background}

Currently SBML restricts the subset of MathML that is deemed 'valid' for use within an SBML model.  It was considered that L3 packages that needed additional math would add this as required. However, this limits the use of the additional math to the use of the package - which in some cases may not be necessary. For example the arrays package has suggested including the mathematical operation 'mean' - but it would be perfectly feasible to use 
'mean' without using arrays. 

Discussions of additional math constructs that should be added to core SBML inevitably introduces difference of opinion as there are varying views on what is considered useful and/or necessary. 

It has also been suggested that we relax the restriction and just allow all of MathML to be used within SBML. However, learning from the experience of CellML, this may produce a situation where many software packages do not support all mathematical operations. This can lead to reduced interoperability with some models only being simulatable by one software package. 

The conclusion of the discussions was that we produce a number of small math only packages that group together math constructs that would be added by that package e.g. statistical functions, array functions, linear algebra functions.  Each package would have its own sbml namespace and use the required attribute to indicate that it has been used in a fashion consistent with the current use of packages. 

These packages can be used with all SBML Level 3 versions and thus facilitates modelers in their need to use mathematical constructs whilst maintaining a position where software supports this and we can continue to have the level of interoperability that SBML has currently achieved. 