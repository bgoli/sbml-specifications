% -*- TeX-master: "main" -*-

\section{Package syntax and semantics}
\label{sec:syntax}

This section contains a definition of the syntax and semantics of the \sbmlthreepkg.  The \ThisPackage involves the following MathML Version 2 \cite{w3c:2000b} constructs.

\TODO{list mathml elements used}

% --------------------------------------------------------------------------
\subsection{Namespace URI and other declarations necessary for using this package}
\label{xml-namespace}

Every SBML Level~3 package is identified uniquely by an XML namespace URI.  For an SBML document to be able to use a given Level~3 package, it must declare the use of that package by referencing its URI.  The following is the namespace URI for this version of the \ThisPackage for \sbmlthreecorenoversion:
\begin{center}
\PackageURL
\end{center}

In addition, SBML documents using a given package must indicate whether the package can be used to change the mathematical interpretation of a model.  This is done using the attribute \token{required} on the \token{<sbml>} element in the SBML document.  For the \ThisPackage, the value of this attribute must be \val{true}, because the use of the \ThisPackage will change the mathematical meaning of a model.

The following fragment illustrates the beginning of a typical SBML model using \sbmlthreecore and this version of the \ThisPackage:

\TODO{adjust example}

\begin{example}
<?xml version="1.0" encoding="UTF-8"?>
<sbml xmlns="http://www.sbml.org/sbml/level3/version1/core" level="3" version="1"
      xmlns:math_xxx="http://www.sbml.org/sbml/level3/math/xxx/version1"
      math_xxx:required="true">
\end{example}

The use of the package namespace differs from other Level~3 packages. In this package it is only be used to indicate the use of additional constructs. The MathML constructs themselves remain in the MathML namespace in accordance with the current use of MathML within SBML. 


%------------------------------------------------------------------------------
\subsection{MathML elements}
\label{new-primitive-types}



%-------------------------------------------------------------------------------------
\subsection{Element}
\label{group-class}

\subsubsection{MathML definition}
\label{group-idname-attributes}



