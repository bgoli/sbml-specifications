% -*- TeX-master: "main"; fill-column: 72 -*-

\section{Validation of SBML documents}
\label{apdx-validation}

\subsection{Validation and consistency rules}
\label{validation-rules}

This section summarizes all the conditions that must (or in some cases, at least \emph{should}) be true of an SBML Level~3 Version~1 model that uses the \ThisPackage.  We use the same conventions that are used in the SBML Level~3 Version~1 Core specification document.  In particular, there are different degrees of rule strictness.  Formally, the differences are expressed in the statement of a rule: either a rule states that a condition \emph{must} be true, or a rule states that it \emph{should} be true.  Rules of the former kind are strict SBML validation rules---a model encoded in SBML must conform to all of them in order to be considered valid.  Rules of the latter kind are consistency rules.  To help highlight these differences, we use the following three symbols next to the rule numbers:

\begin{description}

\item[\hspace*{6.5pt}\vSymbol\vsp] A \vSymbolName indicates a \emph{requirement} for SBML conformance. If a model does not follow this rule, it does not conform to the \ThisPackage specification.  (Mnemonic intention behind the choice of symbol: ``This must be checked.'')

\item[\hspace*{6.5pt}\cSymbol\csp] A \cSymbolName indicates a \emph{recommendation} for model consistency.  If a model does not follow this rule, it is not considered strictly invalid as far as the \ThisPackage specification is concerned; however, it indicates that the model contains a physical or conceptual inconsistency.  (Mnemonic intention behind the choice of symbol: ``This is a cause for warning.'')

\item[\hspace*{6.5pt}\mSymbol\msp] A \mSymbolName indicates a strong recommendation for good modeling practice.  This rule is not strictly a matter of SBML encoding, but the recommendation comes from logical reasoning.  As in the previous case, if a model does not follow this rule, it is not considered an invalid SBML encoding.  (Mnemonic intention behind the choice of symbol: ``You're a star if you heed this.'')

\end{description}

The validation rules listed in the following subsections are all stated or implied in the rest of this specification document.  They are enumerated here for convenience.  Unless explicitly stated, all validation rules concern objects and attributes specifically defined in the \ThisPackage.

%For \notice convenience and brievity, we use the shorthand ``\token{groups:x}'' to stand for an attribute or element name \token{x} in the namespace for the \ThisPackage, using the namespace prefix \token{groups}.  In reality, the prefix string may be different from the literal ``\token{groups}'' used here (and indeed, it can be any valid XML namespace prefix that the modeler or software chooses).  We use ``\token{groups:x}'' because it is shorter than to write a full explanation everywhere we refer to an attribute or element in the \ThisPackage namespace.

\subsubsection*{General rules about this package}

\validRule{\LowerPackage-10101}{To conform to the \ThisPackage specification
for SBML Level~3, an SBML document must declare \PackageURL as the
XMLNamespace  that permits the use of the elements detailed in this package in the MathML Namespace. (Reference: \sbmlthreepkg
\sec{xml-namespace}.)}


\subsubsection*{General rules for MathML content}

\TODO{add rules for each of the math elements used in this package: type and number of arguments; any specific restrictions etc}


