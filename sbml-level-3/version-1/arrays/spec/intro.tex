% -*- TeX-master: "main"; fill-column: 72 -*-

\section{Introduction and motivation}
\label{intro}

This document describes a package for arrays to be used with the Systems Biology Markup Language (SBML) Level 3 Version 1.  This package enables SBML elements such as compartments, parameters, species, reactions, etc. to be arrays.  While this package proposal is independent of all other package proposals, it is designed to work with and complement the hierarchical model composition and distributions packages.  Indeed, the need to specify probability mass functions is a significant motivator for needing arrays.

\subsection{Proposal corresponding to this package specification}

NEED A PROPOSAL

% This specification for Flux Balance Constraints in SBML Level~3
% Version~1 is based on the proposal by the same authors, located at the
% following URL:

% \begin{center}
%   \vspace*{1ex}\small
%   \url{http://sbml.org/Community/Wiki/SBML_Level_3_Proposals/Flux_Balance_Constraints_Proposal_(2012)}
%   \vspace*{1ex}
% \end{center}

% The tracking number in the SBML issue tracking system~\citep{tracker}
% for Flux Balance Constraints package activities is 3154219.  The
% version of the proposal used as the starting point for this
% specification is the version of March 2012. Previous versions of the current proposal are:
% \begin{description}
%   \item[Version 3 (March 2012)]
%   \item [] \small{\url{http://sbml.org/Community/Wiki/SBML_Level_3_Proposals/Flux_Balance_Constraints_Proposal_(2012)}}
%   \item[Version 2 (March 2011)]
%   \item [] \small{\url{http://sbml.org/Community/Wiki/SBML_Level_3_Proposals/Flux_Constraints_Proposal}}
%   \item[Version 1 (February 2010)]
%   \item [] \small{\url{http://precedings.nature.com/documents/4236/version/1}}
% \end{description}
% Details of earlier independent proposals are provided in \ref{background}.

%\subsection{ Tracking number }
%As initially listed in the SBML issue tracking system. \url{http://sourceforge.net/tracker/?func=detail&aid=3154219&group_id=71971&atid=894711
%3154219}

\subsection{Package dependecies}

The arrays package has no dependencies on other SBML Level~3 packages.  It is also designed with the goal of being able to work seamlessly with other SBML Level~3 packages. For example, any objects are entirely encapsulated and any extensions to existing SBML classes are defined as optional.

\subsection{Document conventions}
\label{conventions}

Following the precedent set by the SBML Level~3 Core specification
document, we use UML~1.0 (Unified Modeling Language;
\citealt{eriksson:1998,oestereich:1999}) class diagram notation to
define the constructs provided by this package.  We also use color in
the diagrams to carry additional information for the benefit of those
viewing the document on media that can display color.  The following are
the colors we use and what they represent:

\begin{itemize}

\item[\raisebox{2.75pt}{\colorbox{black}{\rule{0.8pt}{0.8pt}}}]
  \emph{Black}: Items colored black in the UML diagrams are components
  taken unchanged from their definition in the SBML Level~3 Core
  specification document.

\item[\raisebox{2.75pt}{\colorbox{mediumgreen}{\rule{0.8pt}{0.8pt}}}]
  \emph{\textcolor{mediumgreen}{Green}}: Items colored green are
  components that exist in SBML Level~3 Core, but are extended by this
  package.  Class boxes are also drawn with dashed lines to further
  distinguish them.

\item[\raisebox{2.75pt}{\colorbox{darkblue}{\rule{0.8pt}{0.8pt}}}]
  \emph{\textcolor{darkblue}{Blue}}: Items colored blue are new
  components introduced in this package specification.  They have no
  equivalent in the SBML Level~3 Core specification.

\end{itemize}

We also use the following typographical conventions to distinguish the
names of objects and data types from other entities; these conventions
are identical to the conventions used in the SBML Level~3 Core specification
document:

\begin{description}

\item \abstractclass{AbstractClass}: Abstract classes are classes that
  are never instantiated directly, but rather serve as parents of other
  object classes.  Their names begin with a capital letter and they are
  printed in a slanted, bold, sans-serif typeface.  In electronic
  document formats, the class names defined within this document are
  also hyperlinked to their definitions; clicking
  on these items will, given appropriate software, switch the view to
  the section in this document containing the definition of that class.
  (However, for classes that are unchanged from their definitions in
  SBML Level~3 Core, the class names are not hyperlinked because they
  are not defined within this document.)

\item \class{Class}: Names of ordinary (concrete) classes begin with a
  capital letter and are printed in an upright, bold, sans-serif
  typeface.  In electronic document formats, the class names are also
  hyperlinked to their definitions in this specification document.
  (However, as in the previous case, class names are not hyperlinked if
  they are for classes that are unchanged from their definitions in the
  SBML Level~3 Core specification.)

\item \token{SomeThing}, \token{otherThing}: Attributes of classes, data
  type names, literal XML, and generally all tokens \emph{other} than
  SBML UML class names, are printed in an upright typewriter typeface.
  Primitive types defined by SBML begin with a capital letter; SBML also
  makes use of primitive types defined by XML
  Schema~1.0~\citep{biron:2000,fallside:2000,thompson:2000}, but
  unfortunately, XML~Schema does not follow any capitalization
  convention and primitive types drawn from the XML~Schema language may
  or may not start with a capital letter.

\end{description}

For other matters involving the use of UML and XML, we follow the
conventions used in the SBML Level~3 Core specification document.
