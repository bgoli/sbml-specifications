% -*- TeX-master: "main"; fill-column: 72 -*-
%
\section{Examples}
\label{examples}

This section contains a variety of examples of SBML Level~3 Version~1
documents employing the Arrays package.

\subsection{Array of reactions}

This example creates an array {\tt Cell} of 100 compartments, arrays for species {\tt A}, {\tt B}, and {\tt C} also of size 100 with each one placed in the corresponding compartment {\tt cell[i]}, and an array of 100 reactions with one within each {\tt Cell[d0]} converting {\tt A[d0]} plus {\tt B[d0]} into {\tt C[d0]} with kinetic rate law of $k \cdot$ {\tt A[d0]} $\cdot$ {\tt B[d0]}.

\begin{example}
<listOfCompartments>
    <!-- Create an array of n compartments -->
    <compartment id="Cell" constant="true" spatialDimensions="3" size="1">
        <arrays:listOfDimensions
            xmlns:arrays="http://www.sbml.org/sbml/level3/version1/arrays/version1">
            <arrays:dimension arrays:id="d0" arrays:size="n" arrays:arrayDimension="0"/>
        </arrays:listOfDimensions>
    </compartment>
</listOfCompartments>
<listOfSpecies>
     <!-- Create array of n species A with A[d0] placed in Cell[d0] -->
    <species id="B" constant="false" initialAmount="0" hasOnlySubstanceUnits="true"
	 boundaryCondition="false" compartment="Cell">
        <arrays:listOfDimensions
            xmlns:arrays="http://www.sbml.org/sbml/level3/version1/arrays/version1">
            <arrays:dimension arrays:id="d0" arrays:size="n" arrays:arrayDimension="0"/>
        </arrays:listOfDimensions>
        <arrays:listOfIndices
            xmlns:arrays="http://www.sbml.org/sbml/level3/version1/arrays/version1">
            <arrays:index arrays:referencedAttribute="compartment" arrays:arrayDimension="0">
                <math xmlns="http://www.w3.org/1998/Math/MathML">
                    <ci> d0 </ci>
                </math>
            </arrays:index>
        </arrays:listOfIndices>
    </species>
    <!-- Create array of n species B with B[d0] placed in Cell[d0] -->
    <species id="B" constant="false" initialAmount="0" hasOnlySubstanceUnits="true"
	 boundaryCondition="false" compartment="Cell">
        <arrays:listOfDimensions
            xmlns:arrays="http://www.sbml.org/sbml/level3/version1/arrays/version1">
            <arrays:dimension arrays:id="d0" arrays:size="n" arrays:arrayDimension="0"/>
        </arrays:listOfDimensions>
        <arrays:listOfIndices
            xmlns:arrays="http://www.sbml.org/sbml/level3/version1/arrays/version1">
            <arrays:index arrays:referencedAttribute="compartment" arrays:arrayDimension="0">
                <math xmlns="http://www.w3.org/1998/Math/MathML">
                    <ci> d0 </ci>
                </math>
            </arrays:index>
        </arrays:listOfIndices>
    </species>
    <!-- Create array of n species C with C[d0] placed in cell[d0] -->
    <species id="C" constant="false" initialAmount="0" hasOnlySubstanceUnits="true" 
	boundaryCondition="false" compartment="Cell">
        <arrays:listOfDimensions
            xmlns:arrays="http://www.sbml.org/sbml/level3/version1/arrays/version1">
            <arrays:dimension arrays:id="d0" arrays:size="n" arrays:arrayDimension="0"/>
        </arrays:listOfDimensions>
        <arrays:listOfIndices
            xmlns:arrays="http://www.sbml.org/sbml/level3/version1/arrays/version1">
            <arrays:index arrays:referencedAttribute="compartment" arrays:arrayDimension="0">
                <math xmlns="http://www.w3.org/1998/Math/MathML">
                    <ci> d0 </ci>
                </math>
            </arrays:index>
        </arrays:listOfIndices>
    </species>
</listOfSpecies>
<listOfParameters>
    <parameter id="k" constant="false" value="1"/>
    <!-- Specifies size of all arrays (i.e., n:=100) -->
    <parameter id="n" constant="true" value="100"/>
</listOfParameters>
<listOfReactions>
    <!-- Create array of n reactions r with r[d0] converting A[d0] and B[d0] into C[d0]-->
    <reaction id="r" reversible="false" fast="false" compartment="Cell">
        <arrays:listOfDimensions
            xmlns:arrays="http://www.sbml.org/sbml/level3/version1/arrays/version1">
            <arrays:dimension arrays:id="d0" arrays:size="n" arrays:arrayDimension="0"/>
        </arrays:listOfDimensions>
        <arrays:listOfIndices
            xmlns:arrays="http://www.sbml.org/sbml/level3/version1/arrays/version1">
            <arrays:index arrays:referencedAttribute="species" arrays:arrayDimension="0">
                <math xmlns="http://www.w3.org/1998/Math/MathML">
                    <ci> d0 </ci>
                </math>
            </arrays:index>
        </arrays:listOfIndices>
        <listOfReactants>
            <speciesReference constant="true" species="B" stoichiometry="1">
                <arrays:listOfIndices
                    xmlns:arrays="http://www.sbml.org/sbml/level3/version1/arrays/version1">
                    <arrays:index arrays:referencedAttribute="species" arrays:arrayDimension="0">
                        <math xmlns="http://www.w3.org/1998/Math/MathML">
                            <ci> d0 </ci>
                        </math>
                    </arrays:index>
                </arrays:listOfIndices>
            </speciesReference>
            <speciesReference constant="true" species="A" stoichiometry="1">
                <arrays:listOfIndices
                    xmlns:arrays="http://www.sbml.org/sbml/level3/version1/arrays/version1">
                    <arrays:index arrays:referencedAttribute="species" arrays:arrayDimension="0">
                        <math
                            xmlns="http://www.w3.org/1998/Math/MathML">
                            <ci> d0 </ci>
                        </math>
                    </arrays:index>
                </arrays:listOfIndices>
            </speciesReference>
        </listOfReactants>
        <listOfProducts>
            <speciesReference constant="true" species="C" stoichiometry="1">
                <arrays:listOfIndices
                    xmlns:arrays="http://www.sbml.org/sbml/level3/version1/arrays/version1">
                    <arrays:index arrays:referencedAttribute="species" arrays:arrayDimension="0">
                        <math xmlns="http://www.w3.org/1998/Math/MathML">
                            <ci> d0 </ci>
                        </math>
                    </arrays:index>
                </arrays:listOfIndices>
            </speciesReference>
        </listOfProducts>
        <kineticLaw>
            <math xmlns="http://www.w3.org/1998/Math/MathML">
                <apply>
                    <times/>
                    <apply>
                        <times/>
                        <ci> k </ci>
                        <apply>
                            <selector/>
                            <ci> A </ci>
                            <ci> d0 </ci>
                        </apply>
                    </apply>
                    <apply>
                        <selector/>
                        <ci> B </ci>
                        <ci> d0 </ci>
                    </apply>
                </apply>
            </math>
        </kineticLaw>
    </reaction>
</listOfReactions>
\end{example}

% \subsection{Array of Parameters}

% \begin{verbatim}
% <listOfCompartments> 
%  <compartment id="cell"...>
%   <arrays:orderedListOfDimensions>
%    <arrays:dimension id="i" lowerLimit="1" upperLimit="100"/>
%   </arrays:orderedListOfDimensions>
% </compartment>
% </listOfCompartments> 
% <listOfParameters>
%  <parameter id="radius" ...> 
%   <arrays:orderedListOfDimensions>
%    <arrays:dimension id="i" lowerLimit="1" upperLimit="100"/>
%  </arrays:orderedListOfDimensions>
%  <parameter id="position" ...>
%   <arrays:orderedListOfDimensions>
%    <arrays:dimension id="i" lowerLimit="1" upperLimit="100"/>
%    <arrays:dimension id="j" lowerLimit="1" upperLimit="3"/>
%   </arrays:orderedListOfDimensions> 
%  </parameter>
% </listOfParameters>
% \end{verbatim}

\subsection{Array of rate rules}

\begin{eqnarray*}
\frac{dX[d0]}{dt} & = & \left\{ \begin{array}{l}
  y,~~i = 0, 1, 2, 3, 4 \\
 2y,~~i = 5, 6, 7 
\end{array}
\right.
\end{eqnarray*}
\begin{example}
<listOfParameters>
    <!-- Create size variables for arrays -->
    <parameter id="n" constant="true" value="8"/>
    <!-- Create array X of size n -->
    <parameter id="X" constant="false" metaid="iBioSim4" value="0">
        <arrays:listOfDimensions
            xmlns:arrays="http://www.sbml.org/sbml/level3/version1/arrays/version1">
            <arrays:dimension arrays:id="d0" arrays:size="n" arrays:arrayDimension="0"/>
        </arrays:listOfDimensions>
    </parameter>
    <!-- Create scalar parameter y -->
    <parameter id="y" constant="false" metaid="iBioSim5" value="0"/>
</listOfParameters>
<listOfRules>
    <rateRule metaid="rule" variable="X">
        <math xmlns="http://www.w3.org/1998/Math/MathML">
            <piecewise>
                <piece>
                    <ci> y </ci>
                    <apply>
                        <lt/>
                        <ci> d0 </ci>
                        <cn type="integer"> 5 </cn>
                    </apply>
                </piece>
                <otherwise>
                    <apply>
                        <times/>
                        <cn type="integer"> 2 </cn>
                        <ci> y </ci>
                    </apply>
                </otherwise>
            </piecewise>
        </math>
        <arrays:listOfDimensions
            xmlns:arrays="http://www.sbml.org/sbml/level3/version1/arrays/version1">
            <arrays:dimension arrays:id="d0" arrays:size="n" arrays:arrayDimension="0"/>
        </arrays:listOfDimensions>
        <arrays:listOfIndices
            xmlns:arrays="http://www.sbml.org/sbml/level3/version1/arrays/version1">
            <arrays:index arrays:referencedAttribute="variable" arrays:arrayDimension="0">
                <math xmlns="http://www.w3.org/1998/Math/MathML">
                    <ci> d0 </ci>
                </math>
            </arrays:index>
        </arrays:listOfIndices>
    </rateRule>
</listOfRules>
\end{example}

\subsection{Array of events}

\begin{eqnarray*}
\textup{If }X[d0] > 1\textup{ then set }X[d0] & = & \left\{ \begin{array}{l}
0.5,~~i = 0, 1, 2, 3, 4 \\
0.75,~~i = 6, 7, 8
\end{array}
\right.
\end{eqnarray*}

\begin{example}
<listOfParameters>
    <!-- Create size variables for arrays -->
    <parameter id="n" constant="true" metaid="iBioSim2" value="8"/>
    <!-- Create array x of size n -->
    <parameter id="X" constant="false" metaid="iBioSim1" value="0">
        <arrays:listOfDimensions
            xmlns:arrays="http://www.sbml.org/sbml/level3/version1/arrays/version1">
            <arrays:dimension arrays:id="d0" arrays:size="n" arrays:arrayDimension="0"/>
        </arrays:listOfDimensions>
    </parameter>
</listOfParameters>
<listOfEvents>
    <event id="event0" useValuesFromTriggerTime="false">
        <arrays:listOfDimensions
            xmlns:arrays="http://www.sbml.org/sbml/level3/version1/arrays/version1">
            <arrays:dimension arrays:id="d0" arrays:size="n" arrays:arrayDimension="0"/>
        </arrays:listOfDimensions>
        <trigger persistent="false" initialValue="false">
            <math xmlns="http://www.w3.org/1998/Math/MathML">
                <apply>
                    <gt/>
                    <apply>
                        <selector/>
                        <ci> X </ci>
                        <ci> d0 </ci>
                    </apply>
                    <cn type="integer"> 1 </cn>
                </apply>
            </math>
        </trigger>
        <listOfEventAssignments>
            <eventAssignment variable="X">
                <math xmlns="http://www.w3.org/1998/Math/MathML">
                    <piecewise>
                        <piece>
                            <cn> 0.5 </cn>
                            <apply>
                                <lt/>
                                <ci> d0 </ci>
                                <cn type="integer"> 5 </cn>
                            </apply>
                        </piece>
                        <otherwise>
                            <cn> 0.75 </cn>
                        </otherwise>
                    </piecewise>
                </math>
                <arrays:listOfIndices
                    xmlns:arrays="http://www.sbml.org/sbml/level3/version1/arrays/version1">
                    <arrays:index arrays:referencedAttribute="variable" arrays:arrayDimension="0">
                        <math
                            xmlns="http://www.w3.org/1998/Math/MathML">
                            <ci> d0 </ci>
                        </math>
                    </arrays:index>
                </arrays:listOfIndices>
            </eventAssignment>
        </listOfEventAssignments>
    </event>
</listOfEvents>
\end{example}

\subsection{Initial assignment arrays}

This will set an the same initial value to all 10 elements of the x array.

\begin{example}
<listOfParameters>
    <!-- Set size n=10 -->
    <parameter id="n" constant="true" value="10"/>
    <!-- Set array parameters X[d0]=5.7 for all i=1,...,10 -->
    <parameter id="X" constant="false" value="5.7">
        <arrays:listOfDimensions
            xmlns:arrays="http://www.sbml.org/sbml/level3/version1/arrays/version1">
            <arrays:dimension arrays:id="d0" arrays:size="n" arrays:arrayDimension="0"/>
        </arrays:listOfDimensions>
    </parameter>
</listOfParameters>
\end{example}

This could also be done with an initial assignment.

\begin{example}
<listOfParameters>
    <parameter id="X" constant="false" value="0">
        <arrays:listOfDimensions
            xmlns:arrays="http://www.sbml.org/sbml/level3/version1/arrays/version1">
            <arrays:dimension arrays:id="d0" arrays:size="n" arrays:arrayDimension="0"/>
        </arrays:listOfDimensions>
    </parameter>
    <!-- Set size n=10 -->
    <parameter id="n" constant="true" value="10"/>
</listOfParameters>
<listOfInitialAssignments>
     <!-- Set array parameters X[d0]=5.7 for all i=1,...,10 -->
    <initialAssignment symbol="X">
        <math xmlns="http://www.w3.org/1998/Math/MathML">
            <cn type="integer"> 5.7 </cn>
        </math>
        <arrays:listOfDimensions
            xmlns:arrays="http://www.sbml.org/sbml/level3/version1/arrays/version1">
            <arrays:dimension arrays:id="d0" arrays:size="n" arrays:arrayDimension="0"/>
        </arrays:listOfDimensions>
        <arrays:listOfIndices
            xmlns:arrays="http://www.sbml.org/sbml/level3/version1/arrays/version1">
            <arrays:index arrays:referencedAttribute="symbol" arrays:arrayDimension="0">
                <math xmlns="http://www.w3.org/1998/Math/MathML">
                    <ci> d0 </ci>
                </math>
            </arrays:index>
        </arrays:listOfIndices>
    </initialAssignment>
</listOfInitialAssignments>
<listOfParameters> 
  <!-- Set size n=10 -->
  <parameter id="n" value="10"/>
  <!-- Create an array X of size n -->
  <parameter id="X"...>
    <arrays:listOfDimensions>
      <arrays:dimension id="i" size="n" arrayDimension="0"/> 
    </arrays:listOfDimensions>
  </parameter> ...
</listOfParameters> ...
<listOfInitialAssignments>
  <!-- Set array parameters X[d0]=5.7 for all i=0,...,9 -->
  <initialAssignment variable="X">
    <math xmlns="http://www.w3.org/1998/Math/MathML">
      <cn type="real">5.7</cn>
    </math>
  </initialAssignment>
</listOfInitialAssignments>
\end{example}

Here is an example where half of the array is assigned 5.7 and the other half is 3.2. 

\begin{example}
<listOfParameters>
 <!-- Create an array X of size n -->
    <parameter id="X" constant="false" value="0">
        <arrays:listOfDimensions
            xmlns:arrays="http://www.sbml.org/sbml/level3/version1/arrays/version1">
            <arrays:dimension arrays:id="d0" arrays:size="n" arrays:arrayDimension="0"/>
        </arrays:listOfDimensions>
    </parameter>
    <!-- Set size n=10 -->
    <parameter id="n" constant="true" value="10"/>
</listOfParameters>
<listOfInitialAssignments>
    <initialAssignment symbol="X" metaid="init__X">
        <math
            xmlns="http://www.w3.org/1998/Math/MathML">
            <piecewise>
                <piece>
                    <cn> 5.7 </cn>
                    <apply>
                        <lt/>
                        <ci> d0 </ci>
                        <cn type="integer"> 6 </cn>
                    </apply>
                </piece>
                <otherwise>
                    <cn> 3.2 </cn>
                </otherwise>
            </piecewise>
        </math>
        <arrays:listOfDimensions
            xmlns:arrays="http://www.sbml.org/sbml/level3/version1/arrays/version1">
            <arrays:dimension arrays:id="d0" arrays:size="n" arrays:arrayDimension="0"/>
        </arrays:listOfDimensions>
        <arrays:listOfIndices
            xmlns:arrays="http://www.sbml.org/sbml/level3/version1/arrays/version1">
            <arrays:index arrays:referencedAttribute="symbol" arrays:arrayDimension="0">
                <math
                    xmlns="http://www.w3.org/1998/Math/MathML">
                    <ci> d0 </ci>
                </math>
            </arrays:index>
        </arrays:listOfIndices>
    </initialAssignment>
</listOfInitialAssignments>
\end{example}

This could also be done using the {\tt vector} and {\tt selector}  operators.

\begin{example}
    
<listOfParameters>
    <parameter id="X" constant="false" value="0">
        <arrays:listOfDimensions
            xmlns:arrays="http://www.sbml.org/sbml/level3/version1/arrays/version1">
            <arrays:dimension arrays:id="d0" arrays:size="n" arrays:arrayDimension="0"/>
        </arrays:listOfDimensions>
    </parameter>
    <parameter id="n" constant="true" value="10"/>
</listOfParameters>
<listOfInitialAssignments>
    <initialAssignment symbol="X" metaid="init__X">
        <math xmlns="http://www.w3.org/1998/Math/MathML">
            <vector>
                <cn> 5.7 </cn>
                <cn> 5.7 </cn>
                <cn> 5.7 </cn>
                <cn> 5.7 </cn>
                <cn> 5.7 </cn>
                <cn> 5.7 </cn>
                <cn> 3.2 </cn>
                <cn> 3.2 </cn>
                <cn> 3.2 </cn>
                <cn> 3.2 </cn>
                <cn> 3.2 </cn>
            </vector>
        </math>
        <arrays:listOfDimensions
            xmlns:arrays="http://www.sbml.org/sbml/level3/version1/arrays/version1">
            <arrays:dimension arrays:id="d0" arrays:size="n" arrays:arrayDimension="0"/>
        </arrays:listOfDimensions>
        <arrays:listOfIndices
            xmlns:arrays="http://www.sbml.org/sbml/level3/version1/arrays/version1">
            <arrays:index arrays:referencedAttribute="symbol" arrays:arrayDimension="0">
                <math xmlns="http://www.w3.org/1998/Math/MathML">
                    <ci> d0 </ci>
                </math>
            </arrays:index>
        </arrays:listOfIndices>
    </initialAssignment>
</listOfInitialAssignments>
\end{example}

% The {\tt matrix} and {\tt matrixrow} operators can also be used for initial assignments.

% \begin{verbatim}
% <listOfParameters>
%  <!-- Create size variable n=3 -->
%  <parameter id="n" value="3"> 
%  <!-- Create a two dimensional array of size n by n -->
%  <parameter id="Ident" value="0"> 
%   <arrays:listOfDimensions>
%    <arrays:dimension id="i" size="n" arrayDimension="0"/>
%    <arrays:dimension id="j" size="n" arrayDimension="1"/>
%   </arrays:listOfDimensions>
%  </parameter>
% </listOfParameters>
% <listOfInitialAssignments>
%  <!-- Assign Ident to the identity matrix -->
%  <initialAssignment variable="Ident">
%   <math>
%    <matrix> 
%     <matrixrow> <cn>1</cn> <cn>0</cn> <cn>0</cn> </matrixrow>
%     <matrixrow> <cn>0</cn> <cn>1</cn> <cn>0</cn> </matrixrow> 
%     <matrixrow> <cn>0</cn> <cn>0</cn> <cn>1</cn> </matrixrow> 
%    </matrix>
%   </math> 
%  </initialAssignment>
% </listOfInitialAssignments>
% \end{verbatim}

% Here is an example to assign a single value. 

% \begin{example}
    
% <listOfParameters>
%     <parameter id="n" constant="true" value="10"/>
%     <parameter id="X" constant="false" value="0">
%         <arrays:listOfDimensions
%             xmlns:arrays="http://www.sbml.org/sbml/level3/version1/arrays/version1">
%             <arrays:dimension arrays:id="d0" arrays:size="n" arrays:arrayDimension="0"/>
%         </arrays:listOfDimensions>
%     </parameter>
% </listOfParameters>
% <listOfInitialAssignments>
%     <initialAssignment symbol="X" metaid="init__X">
%         <math xmlns="http://www.w3.org/1998/Math/MathML">
%             <cn type="integer"> 14 </cn>
%         </math>
%         <arrays:listOfDimensions
%             xmlns:arrays="http://www.sbml.org/sbml/level3/version1/arrays/version1">
%             <arrays:dimension arrays:id="d0" arrays:size="n" arrays:arrayDimension="0"/>
%         </arrays:listOfDimensions>
%         <arrays:listOfIndices
%             xmlns:arrays="http://www.sbml.org/sbml/level3/version1/arrays/version1">
%             <arrays:index arrays:referencedAttribute="symbol" arrays:arrayDimension="0">
%                 <math xmlns="http://www.w3.org/1998/Math/MathML">
%                     <ci> 0 </ci>
%                 </math>
%             </arrays:index>
%         </arrays:listOfIndices>
%     </initialAssignment>
% </listOfInitialAssignments>
% \end{example}

\subsection{Examples for array referencing}

Here is an example array reference using {\tt selector}.
\begin{displaymath}
0.1 * S[x]
\end{displaymath}
\begin{example}
<math xmlns="http://www.w3.org/1998/Math/MathML">
    <apply>
        <times/>
        <apply>
            <selector/>
            <ci> S </ci>
            <ci> x </ci>
        </apply>
        <cn> 0.1 </cn>
    </apply>
</math>


\end{example}

Here is a more complicated example of array referencing.
\begin{displaymath}
W[d0]= A[d0][0]V[0]+ A[d0][1]V[1]+ A[d0][2]V[2]
\end{displaymath}

\begin{example}
    
<listOfParameters>
    <!-- Create size variable n=3 -->
    <parameter id="n" constant="true" value="3"/>
    <!-- Create 2-dimensional array A of size n by n -->
    <parameter id="A" constant="false" value="0">
        <arrays:listOfDimensions
            xmlns:arrays="http://www.sbml.org/sbml/level3/version1/arrays/version1">
            <arrays:dimension arrays:id="d0" arrays:size="n" arrays:arrayDimension="0"/>
            <arrays:dimension arrays:id="d1" arrays:size="n" arrays:arrayDimension="1"/>
        </arrays:listOfDimensions>
    </parameter>
    <!-- Create an array W of size n -->
    <parameter id="W" constant="false" value="0">
        <arrays:listOfDimensions
            xmlns:arrays="http://www.sbml.org/sbml/level3/version1/arrays/version1">
            <arrays:dimension arrays:id="d0" arrays:size="n" arrays:arrayDimension="0"/>
        </arrays:listOfDimensions>
    </parameter>
    <!-- Create an array V of size n -->
    <parameter id="V" constant="false" value="0">
        <arrays:listOfDimensions
            xmlns:arrays="http://www.sbml.org/sbml/level3/version1/arrays/version1">
            <arrays:dimension arrays:id="d0" arrays:size="n" arrays:arrayDimension="0"/>
        </arrays:listOfDimensions>
    </parameter>
</listOfParameters>
<listOfRules>
     <!-- W[d0] = A[d0][0]V[0] + A[d0][1]V[1] + A[d0][2]V[2] -->
    <assignmentRule metaid="rule" variable="W">
        <math xmlns="http://www.w3.org/1998/Math/MathML">
            <apply>
                <plus/>
                <apply>
                    <plus/>
                    <apply>
                        <times/>
                        <apply>
                            <selector/>
                            <ci> A </ci>
                            <ci> d0 </ci>
                            <cn type="integer"> 0 </cn>
                        </apply>
                        <apply>
                            <selector/>
                            <ci> V </ci>
                            <cn type="integer"> 1 </cn>
                        </apply>
                    </apply>
                    <apply>
                        <times/>
                        <apply>
                            <selector/>
                            <ci> A </ci>
                            <ci> d0 </ci>
                            <cn type="integer"> 1 </cn>
                        </apply>
                        <apply>
                            <selector/>
                            <ci> V </ci>
                            <cn type="integer"> 1 </cn>
                        </apply>
                    </apply>
                </apply>
                <apply>
                    <times/>
                    <apply>
                        <selector/>
                        <ci> A </ci>
                        <ci> d0 </ci>
                        <cn type="integer"> 2 </cn>
                    </apply>
                    <apply>
                        <selector/>
                        <ci> V </ci>
                        <cn type="integer"> 2 </cn>
                    </apply>
                </apply>
            </apply>
        </math>
        <arrays:listOfDimensions
            xmlns:arrays="http://www.sbml.org/sbml/level3/version1/arrays/version1">
            <arrays:dimension arrays:id="d0" arrays:size="n" arrays:arrayDimension="0"/>
        </arrays:listOfDimensions>
        <arrays:listOfIndices
            xmlns:arrays="http://www.sbml.org/sbml/level3/version1/arrays/version1">
            <arrays:index arrays:referencedAttribute="variable" arrays:arrayDimension="0">
                <math xmlns="http://www.w3.org/1998/Math/MathML">
                    <ci> d0 </ci>
                </math>
            </arrays:index>
        </arrays:listOfIndices>
    </assignmentRule>
</listOfRules>
\end{example}

% This could also be done with {\tt scalarproduct} as follows:

% \begin{example}
% <assignmentRule variable="w">
%  <math xmlns="http://www.w3.org/1998/Math/MathML">
%   <apply>
%    <ci>scalarproduct</ci>
%    <ci>A</ci>
%    <ci>v</ci>
%   </apply>
%  </math>
% </assignmentRule>
% \end{example}

\subsection{Array references in functions}

Functions can also make reference to array variables, but in this case, it is not necessary to declare arrays as such within the function or to declare the array indices or limits within functions.  The following defines a function on two vectors, 
\begin{eqnarray*}
f(x,y) & = & X[1]Y[0] - Y[1]X[0]
\end{eqnarray*}
The arguments are declared as vectors using the type field of the ci command.

\begin{example}
<!-- f(x,y) = X[1]Y[0] - Y[1]X[0] -->
<functionDefinition id="f">
    <math
        xmlns="http://www.w3.org/1998/Math/MathML">
        <lambda>
            <bvar>
                <ci> X </ci>
            </bvar>
            <bvar>
                <ci> Y </ci>
            </bvar>
            <apply>
                <minus/>
                <apply>
                    <times/>
                    <apply>
                        <selector/>
                        <ci> X </ci>
                        <cn type="integer"> 1 </cn>
                    </apply>
                    <apply>
                        <selector/>
                        <ci> Y </ci>
                        <cn type="integer"> 0 </cn>
                    </apply>
                </apply>
                <apply>
                    <times/>
                    <apply>
                        <selector/>
                        <ci> Y </ci>
                        <cn type="integer"> 1 </cn>
                    </apply>
                    <apply>
                        <selector/>
                        <ci> X </ci>
                        <cn type="integer"> 0 </cn>
                    </apply>
                </apply>
            </apply>
        </lambda>
    </math>
</functionDefinition>
\end{example}
