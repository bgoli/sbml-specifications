% -*- TeX-master: "main"; fill-column: 72 -*-

\section{ Background }
\label{background}

\subsection{ Problems with current SBML approaches }

All mathematical operations in SBML are currently restricted to operations on scalar values.  There is currently no way to express a vector or matrix.  If one desires such a structure, they must first flatten it into individual scalar values.  This approach is not appealing for expressing things such as a probability mass function for a distribution.  A second problem in SBML is that regular structures cannot be represented efficiently.  While the hierarchical modeling packages has made this somewhat easier, it is still necessary to instantiate each submodel individually and make connections between the submodels explicitly.  

\subsection{ Past work on this problem or similar topics }

The idea of adding arrays to SBML has been around for more than 10 years when it when a Dynamic Arrays package was proposed by Bruce Shapiro, Victoria Gor, and Eric Mjolsness.  This specification adopts several of the ideas proposed there with some exceptions.  In particular, this specification restricts arrays to be statically sized, it limits arrays to one or two dimensions, and it requires all arrays to be explicitly defined.  

%% TODO: get year of this document and I believe there was one other document but can't find it.

\subsection{Array notation used in this document}

Elements of vectors are referenced as $x(i)$ in this document to refer to the  $i^{th}$ element of the vector $x$ while $x(i,j)$ is used to reference the $i^{th}$ row and $j^{th}$ column of a matrix.
%In terms of SBML, this means an array whose id is $x$ whose first index has id i, second index id j, etc.
This notation is used to describe the meaning of certain features, but it is not intended to be used explicitly anywhere in SBML. 
In some cases, this specification may refer to the $i^{th}$ element of an array of objects (such as rules) that do not have object ids and cannot be referenced in SBML. 
%Furthermore, this specification sometimes uses the same notation $x(i,j)$ to refer to the array itself with indices i, j,... rather than a specific array element; the correct interpretation should be clear from the context in which it is used. 
%The notation A[i..j, k..m, ..., p..q] will be used to refer to the array 
%i.e, the array whose first dimension ranges from i to j, second dimension ranges from k to m, and whose last dimension ranges from p to q. For example A[0..5,0..7] refers to 6 by 8 array whose indices both start at 0. 


%\subsection{ Design goals of the current \FBC Package }
%\label{design-goals}

