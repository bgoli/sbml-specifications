% -*- TeX-master: "main"; fill-column: 72 -*-

\section{Examples}
\label{examples}

This section contains a variety of examples of SBML Level~3 Version~1
documents employing the Arrays package.

\subsection{Array of Reactions}

\begin{verbatim}
<listOfCompartments> 
 <parameter id="n" value="100"...>
 <compartment id="cell"...>
  <orderedListOfDimensions>
   <dimension id="i" size="n"/>
  </orderedListOfDimensions>
</compartment>
</listOfCompartments> 
<listOfSpecies>
 <species id="A" compartment="cell" ... > 
  <orderedListOfDimensions>
   <dimension id="i" size="n"/>
  </orderedListOfDimensions>
 </species>
 <species id="B" compartment="cell" ... > 
  <orderedListOfDimensions>
   <dimension id="i" size="n"/>
  </orderedListOfDimensions>
 </species>
 <species id="C" compartment="cell" ... >
  <orderedListOfDimensions>
   <dimension id="i" size="n"/>
  </orderedListOfDimensions>
</species>
</listOfSpecies>
<reaction id="reaction1" ...> 
  <orderedListOfDimensions>
   <dimension id="i" size="n"/>
  </orderedListOfDimensions>
 <listOfReactants>
  <speciesReference species="A">
   <orderedListOfIndices>
    <index>
     <math><apply><selector/><ci>A<ci><ci>i</ci></apply></math>
    </index>
  </orderedListOfIndices>
 </speciesReference>
  <speciesReference species="B"> 
   <orderedListOfIndices>
    <index>
     <math><apply><selector/><ci>B<ci><ci>i</ci></apply></math>
    </index>
   </orderedListOfIndices>
</speciesReference>
</listOfReactants> 
 <listOfProducts>
  <speciesReference species="C"> 
   <orderedListOfIndices>
    <index>
     <math><apply><selector/><ci>C<ci><ci>i</ci></apply></math>
    </index>
   </orderedListOfIndices>
 </speciesReference>
 </listOfProducts>
</reaction>
\end{verbatim}

% \subsection{Array of Parameters}

% \begin{verbatim}
% <listOfCompartments> 
%  <compartment id="cell"...>
%   <orderedListOfDimensions>
%    <dimension id="i" lowerLimit="1" upperLimit="100"/>
%   </orderedListOfDimensions>
% </compartment>
% </listOfCompartments> 
% <listOfParameters>
%  <parameter id="radius" ...> 
%   <orderedListOfDimensions>
%    <dimension id="i" lowerLimit="1" upperLimit="100"/>
%  </orderedListOfDimensions>
%  <parameter id="position" ...>
%   <orderedListOfDimensions>
%    <dimension id="i" lowerLimit="1" upperLimit="100"/>
%    <dimension id="j" lowerLimit="1" upperLimit="3"/>
%   </orderedListOfDimensions> 
%  </parameter>
% </listOfParameters>
% \end{verbatim}

\subsection{Array of Rate Rules}

\begin{eqnarray*}
\frac{dx_i}{dt} & = & \left\{ \begin{array}{l}
  y, i = 1,2,3,4,5 \\
 2y, i = 6, 7, 8 
\end{array}
\right.
\end{eqnarray*}
\begin{verbatim}
<listOfParameters>
 <parameter id="n" value="8"/>
 <parameter id="m" value="5"/>
 <parameter id="o" value="3"/>
 <parameter id="x" ...> 
  <orderedListOfDimensions>
   <dimension id="i" size="n"/>
 </orderedListOfDimensions>
 <parameter id="y" .../>
</listOfParameters>
<rateRule variable="x">
 <orderedListOfDimensions>
  <dimension id="i" size="m"/>
 </orderedListOfDimensions>
 <orderedListOfIndices>
  <index>
   <math><apply><selector/><ci>x<ci><ci>i</ci></apply></math>
  </index>
 </orderedListOfIndices>
<math xmlns="http://www.w3.org/1998/Math/MathML">
  <ci>y</ci></math>
</rateRule>
<rateRule variable="x">
 <orderedListOfDimensions>
  <dimension id="i" size="o"/>
 </orderedListOfDimensions>
 <orderedListOfIndices>
  <index>
   <math>
    <apply>
     <selector/>
      <ci>x<ci>
      <apply>
       <plus/>
        <ci>i</ci>
        <ci>5</ci>
      </apply>
     </apply>
    </math>
  </index>
 </orderedListOfIndices>
<math xmlns="http://www.w3.org/1998/Math/MathML">
 <apply><times/>
   <cn type="integer">2</cn> <ci>y</ci>
 </apply></math> 
</rateRule>
\end{verbatim}

\subsection{Array of Events}

\begin{eqnarray*}
\textup{If }x_i > 1\textup{ then set }x_i & = & \left\{ \begin{array}{l}
0.5, i = 1,2,3,4,5 \\
0.75, i = 6,7,8
\end{array}
\right.
\end{eqnarray*}

\begin{verbatim}
<listOfParameters>
 <parameter id="n" value="8"/>
 <parameter id="m" value="5"/>
 <parameter id="o" value="3"/>
 <parameter id="x" ...> 
  <orderedListOfDimensions>
   <dimension id="i" size="n"/>
 </orderedListOfDimensions>
</listOfParameters>
<event id="foosmall">
 <orderedListOfDimensions>
  <dimension id="i" size="m"/>
 </orderedListOfDimensions>
 <trigger>
  <math xmlns="http://www.w3.org/1998/Math/MathML">
   <list>
    <apply>
     <gt/>
      <apply>
       <selector/>
       <ci>x</ci>
       <ci>i</ci>
      </apply>
     <cn type="integer">1</cn>
   </apply>
   </list>
  </math>
 </trigger>
 <listOfEventAssignments>
  <eventAssignment variable="x">
   <orderedListOfIndices>
    <index>
     <math><apply><selector/><ci>x<ci><ci>i</ci></apply></math>
    </index>
   </orderedListOfIndices>
<math xmlns="http://www.w3.org/1998/Math/MathML">
    <cn type="real">0.5</cn>
   </math>
  </eventAssignment>
 </listOfEventAssignments>
</event>
<event id="foobig">
<orderedListOfDimensions>
  <dimension id="i" size="o"/>
 </orderedListOfDimensions>
 <trigger>
  <math xmlns="http://www.w3.org/1998/Math/MathML">
   <list>
    <apply>
     <gt/>
      <apply>
       <selector/>
        <ci>x</ci>
        <apply>
         <plus/>
          <ci>i</ci>
          <cn>5</cn>
        </apply>
     </apply>
     <cn type="integer">1</cn>
   </apply>
   </list>
  </math>
 </trigger>
 <listOfEventAssignments>
  <eventAssignment variable="x">
   <orderedListOfIndices>
    <index>
     <math>
      <apply>
       <selector/>
        <ci>x<ci>
        <apply>
         <plus/>
          <ci>i</ci>
          <cn>5</cn>
        </apply>
      </apply>
      </math>
    </index>
   </orderedListOfIndices>
 <math xmlns="http://www.w3.org/1998/Math/MathML">
    <cn type="real">0.75</cn>
   </math>
  </eventAssignment>
 </listOfEventAssignments>
</event>
\end{verbatim}

\subsection{Initial Assignment Arrays}

This will set an the same initial value to all 10 elements of the x array.

\begin{verbatim}
<listOfParameters>
 <parameter id="n" value="10"/>
 <parameter id="x" value="5.7"...>
  <orderedListOfDimensions>
   <dimension id="i" size="n">
  </orderedListOfDimensions> 
 </parameter>
</listOfParameters>
\end{verbatim}

This could also be done with an initial assignment.

\begin{verbatim}
<listOfParameters> 
 <parameter id="n" value="10"/>
 <parameter id="x"...>
  <orderedListOfDimensions>
   <dimension id="i" size="n"/> 
  </orderedListOfDimensions>
 </parameter> ...
</listOfParameters> ...
<listOfInitialAssignments>
 <initialAssignment variable="x">
  <math>
   <cn>5.7<cn>
  </math>
 </initialAssignment>
</listOfInitialAssignments>
\end{verbatim}

Here is an example where half of the array is assigned 5.7 and the other half is 3.2. 

\begin{verbatim}
<listOfParameters> 
 <parameter id="n" value="10"/>
 <parameter id="m" value="5"/>
 <parameter id="x"...>
  <orderedListOfDimensions>
   <dimension id="i" size="n"/> 
  </orderedListOfDimensions>
 </parameter> ...
</listOfParameters> ...
<initialAssignment variable="x"> 
 <orderedListOfDimensions>
  <dimension id="i" size="m"/>
 </orderedListOfDimensions>
<orderedListOfIndices>
  <index>
   <math><apply><selector/><ci>x<ci><ci>i</ci></apply></math>
  </index>
 </orderedListOfIndices>
 <math><cn>5.7<cn></math>
</initialAssignment>
<initialAssignment variable="x">
 <orderedListOfDimensions>
  <dimension id="i" size="m"/>
 </orderedListOfDimensions>
 <orderedListOfIndices>
  <index>
   <math>
    <apply>
     <selector/>
      <ci>x<ci>
      <apply>
       <plus/>
        <ci>i</ci>
        <ci>m</ci>
      </apply>
    </apply>
    </math>
  </index>
 </orderedListOfIndices>
 <math><cn>3.2<cn></math>
</initialAssignment>
\end{verbatim}

This could also be done using the {\tt vector} operator.

\begin{verbatim}
<initialAssignment variable="x"> 
 <math>
  <vector> 
   <cn>5.7</cn> 
   <cn>5.7</cn> 
   <cn>5.7</cn> 
   <cn>5.7</cn> 
   <cn>5.7</cn> 
   <cn>3.2</cn> 
   <cn>3.2</cn> 
   <cn>3.2</cn> 
   <cn>3.2</cn> 
   <cn>3.2</cn>
  </vector>
 </math>
</initialAssignment>
\end{verbatim}

The {\tt matrix} and {\tt matrixrow} operators can also be used for initial assignments.

\begin{verbatim}
<parameter id="n" value="3"> 
<parameter id="Ident" value="0"> 
<orderedListOfDimensions>
  <dimension id="i" size="n"/>
  <dimension id="j" size="n"/>
 </orderedListOfDimensions>
</parameter>
<initialAssignment variable="Ident">
 <math>
  <matrix> 
   <matrixrow> <cn>1</cn> <cn>0</cn> <cn>0</cn> </matrixrow>
   <matrixrow> <cn>0</cn> <cn>1</cn> <cn>0</cn> </matrixrow> 
   <matrixrow> <cn>0</cn> <cn>0</cn> <cn>1</cn> </matrixrow> 
  </matrix>
 </math> 
</initialAssignment>
\end{verbatim}

Here is an example to assign a single value. 

\begin{verbatim}
<initialAssignment variable="Ident"> 
 <orderedListOfIndices>
  <index>
   <math><apply><selector/><ci>Ident<ci><ci>2</ci></apply></math>
  </index>
  <index>
   <math><apply><selector/><ci>Ident<ci><ci>1</ci></apply></math>
  </index>
 </orderedListOfIndices>
 <math>
  <cn>14<cn>
 </math>
<initialAssignemnt>
\end{verbatim}

\subsection{Examples for array referencing}

Here is an example array reference using {\tt selector}.
\begin{displaymath}
0.1 * s1[x]
\end{displaymath}
\begin{verbatim}
<math xmlns="http://www.w3.org/1998/Math/MathML">
 <apply>
  <times/>
   <apply>
    <selector/>
     <ci> s1 </ci>
     <ci> x </ci> 
   </apply>
   <cn> 0.1 </cn>
 </apply>
</math>
\end{verbatim}

%{\bf QUESTION: what is the best way to access elements in arrays with more than 2 dimensions?}

Here is a more complicated example of array referencing.
\begin{displaymath}
w[i]= A[i,1]v[1]+ A[i,2]v[2]+ A[i,3]v[3]
\end{displaymath}

\begin{verbatim}
<listOfParameters> 
 <parameter id="n" value="3".../>
 <parameter id="A">
  <orderedListOfDimensions>
   <dimension id="i" size="n"/> 
   <dimension id="j" size="n"/>
  </orderedListOfDimensions> 
 </parameter> 
 <parameter id="v">
  <orderedListOfDimensions>
   <dimension id="i" size="n"/>
  </orderedListOfDimensions> 
 </parameter> 
 <parameter id="w">
  <orderedListOfDimensions>
   <dimension id="i" size="n"/>
  </orderedListOfDimensions> 
</parameter>
</listOfParameters>
<assignmentRule variable="w">
 <orderedListOfDimensions>
  <dimension id="i" size="n"/>
 </orderedListOfDimensions> 
 <orderedListOfIndices>
  <index>
   <math><apply><selector/><ci>w<ci><ci>i</ci></apply></math>
  </index>
 </orderedListOfIndices>
 <math xmlns="http://www.w3.org/1998/Math/MathML">
  <apply><plus/>
    <apply><times/>
     <apply><selector/><ci>A</ci><ci>i</ci><cn type="integer">1</cn></apply>
     <apply><selector/><ci>v</ci><cn type="integer">1</cn></apply>
    </apply>
    <apply><times/>
     <apply><selector/><ci>A</ci><ci>i</ci><cn type="integer">2</cn></apply>
     <apply><selector/><ci>v</ci><cn type="integer">2</cn></apply>
    </apply>
    <apply><times/>
     <apply><selector/><ci>A</ci><ci>i</ci><cn type="integer">3</cn></apply>
     <apply><selector/><ci>v</ci><cn type="integer">3</cn></apply>
    </apply>
   </apply>
 </math>
</assignmentRule>
\end{verbatim}

This could also be done with {\tt scalarproduct} as follows:

\begin{verbatim}
<assignmentRule variable="w">
 <math xmlns="http://www.w3.org/1998/Math/MathML">
  <apply>
   <ci>scalarproduct</ci>
   <ci>A</ci>
   <ci>v</ci>
  </apply>
 </math>
</assignmentRule>
\end{verbatim}

\subsection{Array references in functions}

Functions can also make reference to array variables, but in this case, it is not necessary to declare arrays as such within the function or to declare the array indices or limits within functions.  The following defines a function on two vectors, 
\begin{eqnarray*}
f(x,y) & = & x[2]y[1] - y[2]x[1]
\end{eqnarray*}
The arguments are declared as vectors using the type field of the ci command.
\begin{verbatim}
<functionDefinition id="f" />
 <math xmlns="http://www.w3.org/1998/Math/MathML">
  <lambda>
   <bvar>
    <ci>x</ci>
   </bvar>
   <bvar>
    <ci>y</ci>
   </bvar>
   <apply>
    <plus/>
    <apply>
     <times/>
     <apply>
      <ci>selector</ci>
      <ci>x</ci>
      <cn type="integer">2</cn>
     </apply>
     <apply>
      <ci>selector</ci>
      <ci>y</ci>
      <cn type="integer">1</cn>
     </apply>
    </apply>
    <apply>
     <times/>
     <cn type="integer">-1</cn>
     <apply>
      <times/>
      <apply>
       <ci>selector</ci>
       <ci>x</ci>
       <cn type="integer">1</cn>
      </apply>
      <apply>
       <ci>selector</ci>
       <ci>y</ci>
       <cn type="integer">2</cn>
      </apply>
     </apply>
    </apply>
   </apply>
  </lambda>
 </math>
</functionDefinition>
\end{verbatim}

% \subsection{\FBC syntax examples}

% \paragraph{Encoding the \FluxBound}
% As described in \ref{fluxbound-class} the flux bound represents a mathematical (in)equality of the form <\token{reaction}> <\token{operator}> <\token{value}>. In SBML Level~3 Version~1 with \FBC this is encoded as:
% %
% \exampleFile{examples/ex_fb_fbc.txt}
% %
% This example illustrates two things: the encoding of $\infty$ and that care should be used when selecting inequalities such as \val{less} or \val{greater}. While mathematically there is a difference, this difference is only practically relevant when working with rational arithmetic (solvers).

% \paragraph{Encoding the \Objective}
% The \FBC allows for the definition of multiple `objective functions' with one being designated as active (see \ref{objective-class}) the following example illustrates this:
% %
% \exampleFile{examples/ex_objf_fbc.txt}
% %
% Note how both \Objective instances differ in \token{type} and each contains different set of \class{FluxObjectives}.

% \paragraph{The extended \Species}
% The \FBC package extend the \SBML \Species of \sbmlthreecore by providing attributes for storing \token{charge} and \token{chemicalFormula}
% %
% \exampleFile{examples/ex_spec_l3.txt}