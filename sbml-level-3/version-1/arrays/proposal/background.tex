% -*- TeX-master: "main"; fill-column: 72 -*-

\section{ Background }
\label{background}

\subsection{ Problems with current SBML approaches }

% While there is currently no official way of encoding constraint based models in SBML L2 there have been pragmatic approaches used by a variety of groups and applications. Arguably the best and most widely used format is that used by the \textsf{COBRA toolbox}~\citep{cobra} where the metabolic network is well defined using SBML \Reaction and \Species classes. However, flux bounds and reactions that take part in the objective function are defined as \LocalParameter objects and (implicitly) rely on all tools using the same naming convention. Similarly, reaction annotations are generally stored as key-value pairs in HTML \Notes objects which has routinely led to different groups and software using in-house key definitions describing the same entity. While a step in the right direction this format is not suitable for implementation in SBML Level 3.


\subsection{ Past work on this problem or similar topics }
% The problem of describing and annotating `FBA' models in SBML has been raised  at various times in the past few years. In this regard there are two known putative proposals one by Karthik Raman and the other by the Church Laboratory. As far as we are aware these proposals never developed beyond their initial presentation at SBML forum/hackathons. In 2009 the discussion was reopened at the SBML Forum held in Stanford and has subsequently developed into the current active package proposal and this document (see \ref{intro}).

% \begin{description}
%   \item[Brett Olivier (2009)] SBML Level 3 FBA package discussion
%   \item[]\url{http://sbml.org/images/4/4a/Olivier_sbml_forum_2009_09_04.pdf}
%   \item[Karthik Raman (2005)] Flux annotations in SBML
%   \item[]\url{http://sbml.org/images/d/d9/Raman-flux-annotations.pdf}
%   \item[Church laboratory (pre 2005)] Metabolic flux model annotations
%   \item[]\url{http://sbml.org/Community/Wiki/Old_known_SBML_annotations_list}
% \end{description}

\subsection{Array notation used in this document}

Elements of arrays will be referenced as A[i,j,...] in this document to refer to, i.e,  the  th element of the mathematical matrix A with indices i, j, .... 
In terms of SBML this means an array whose id is A whose first index has id i, second index id j, etc. This notation is used to describe the meaning of certain features, but it is not intended to be used explicitly anywhere in SBML. 
In some cases we may refer to the ijk...th element of an array of objects (such as rules) that do not have object ids and can not be referenced in SBML. 
Furthermore, we will sometimes use the same notation A[i,j,...] to refer to the array itself with indices i, j,... rather than a specific array element; the correct interpretation should be clear from the context in which it is used. 
The notation A[i..j, k..m, ..., p..q] will be used to refer to the array 
	 
i.e, the array whose first dimension ranges from i to j, second dimension ranges from k to m, and whose last dimension ranges from p to q. For example A[0..5,0..7] refers to 6 by 8 array whose indices both start at 0. 


%\subsection{ Design goals of the current \FBC Package }
%\label{design-goals}

