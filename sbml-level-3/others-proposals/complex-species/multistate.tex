\documentclass{cekarticle}
\usepackage[latin1]{inputenc}
\usepackage[T1]{fontenc}
\usepackage{color}
\usepackage{amsmath}
\usepackage{amssymb}
\usepackage{array}


%%%% Good for A4 paper
\setlength{\textheight}{240mm}
\setlength{\topmargin}{-40pt}
\setlength{\headheight}{12pt}
\setlength{\headsep}{35pt}

%-------------------------------
% to modify locally the margins
%-------------------------------
\newenvironment{changemargin}[2]{\begin{list}{}{%
%\setlength{\topsep}{0pt}%
\setlength{\leftmargin}{0pt}%
\setlength{\rightmargin}{0pt}%
\setlength{\listparindent}{\parindent}%
\setlength{\itemindent}{\parindent}%
%\setlength{\parsep}{0pt plus 1pt}%
\addtolength{\leftmargin}{#1}%
\addtolength{\rightmargin}{#2}%
}\item }{\end{list}}

% The following does not work: somehow locked in cekarticle.cls 
\addtocounter{tocdepth}{4}
\setcounter{secnumdepth}{5}


%=============================================================================
% Title page
%=============================================================================

\title{Systems Biology Markup Language (SBML) Level~2 Proposal: Multistate Features}

\author{Nicolas Le Nov\`{e}re, Thomas Simon Shimizu, Andrew Finney}

\begin{document}

\authoremail{
\begin{minipage}{\textwidth}\centering
lenov@pasteur.fr,tss26@cus.cam.ac.uk,afinney@cds.caltech.edu
\end{minipage}}

\maketitlepage

%=============================================================================
\section{Introduction}
\label{sec:introduction}
%=============================================================================

This document describes a proposed extension for inclusion in Systems Biology
Markup Language (SBML) Level~2. It describes features enabling the inclusion of
complexes with several alternative states in models.

This document is not a definition of SBML Level~2 or part of it.  This document
simply presents various features which could be incorporated into SBML Level~2
as the Systems Biology community wishes.  This document is intended for detailed
review by that community and to provoke alternative proposals.  Throughout this
document issues that the authors believe will require further discussion have
been highlighted.

For brevity the text of this document is with reference to SBML Level~1
\citep{Hucka:2001}, i.e. features are described in terms of changes to SBML
Level 1.  This document uses UML diagrams in the same way except that new
features are shown in red.

All types proposed in this document will be derived from the \texttt{SBase}
type.

%=============================================================================
\section{Why this extension?}
\label{sec:why}
%=============================================================================

An alternative introduction to the problem of multistate reactants can be found
in Andrew Finney's initial proposal ``Complex Species:species with multiple
states'' \citep{Finney:2001a}.

Many biological macromolecules possess multiple internal states which can affect
reaction rates.  Typical examples are:

\begin{itemize}
 \item Different relative atomic coordinates $\Rightarrow$ Conformational changes or folding
 \item Covalent modification $\Rightarrow$ glycosylation, phosphorylation, methylation
 \item Non-covalent modification $\Rightarrow$ metal or ligand binding
\end{itemize}

In modelling reaction systems that involve such molecules, it is possible to
treat all the different states as separate species. However the number of
possible reactions increases exponentially with the number of reacting species
(of course in most cases, not all states will affect every reaction, so the
number of reactions which need to be computed separately will be somewhat less
than this).  Writing out all of these reactions separately is tedious at best,
and devastating at worst.  It is desirable to have an efficient notation which
compresses the redundant information.  In addition, some simulators (e.g.
\textsc{StochSim} \citep{Mortonfirth:1998} or MCell \citep{Stiles:1996})
explicitly consider individual molecules, not populations of molecular species.
This calls for a mechanism in SBML which can distinguish between specific
instances of the same specie.

%=============================================================================
\section{Specie}
\label{sec:species}
%=============================================================================

Andrew's initial proposal \citep{Finney:2001a} allowed a subset of the
state-dependent reaction instances that involve different reactant states, but
have identical rates, to be grouped together and expressed as a single reaction.
This was achieved by defining several new SBML elements such as
\texttt{complexSpecie} (for defining species with multiple states) and
\texttt{complexSpecieInstance} (for distinguishing between specific instances of
the same specie that take part in a reaction).  However, in the case that
different states have a different effect on the reaction rate, these had to be
defined as separate reactions.

We take a similar, but slightly different approach that introduces some new
elements, but attempts to incorporate much of the multistate-specific
information by extending the existing SBML~1 elements with optional attributes.

The proposed structure of the \texttt{Specie} type is shown in
figure~\ref{fig:specie}.

\begin{figure}[h]
  \vspace*{8pt}
  \centering
  \begin{tabular}{|l|}
    \hline
    \multicolumn{1}{|c|}{\rule[-3mm]{0mm}{8mm}{\textsf{Specie}}}\\
    \hline
    \rule[0mm]{0mm}{5mm}{\textsf{\small name: SName }}\\
    \textsf{\small compartment: SName }\\
    \textsf{\small initialAmount: double }\\
    \textsf{\small units: SName \{use=``optional''\}  }\\
    \textsf{\small boundaryCondition: boolean \{use=``default'' value=``false''\} }\\
    \textsf{\small charge: integer \{use=``optional''\} }\\
    \textsf{\small \textcolor{red}{feature: Feature[0..*] \{use=``optional''\}} }\\
    \rule[-3mm]{0mm}{5mm}{\textsf{\small \textcolor{red}{initialState: InitialState[0..*] \{use=``optional''\}} }}\\
    \hline
  \end{tabular}
  \caption{The definition of the proposed extended \texttt{Specie} type. Addenda are shown in red.}
  \label{fig:specie}
\end{figure}

The ``state'' of a multi-state molecule is defined collectively by the states of
all ``features'' that it possesses.  A ``feature'' here is a characteristic of the
specie which can be in one of at least two states that affect certain reaction
rates.  Therefore the extended \texttt{specie} element posesses now two new attributes, a
\texttt{listOfFeatures} and a \texttt{listOfInitialStates}.

\subsection{Feature}\label{sec:feature}

The \texttt{listOfFeatures} lists all the features of the specie which can
possess several alternative states.

\begin{figure}[h]
  \vspace*{8pt}
  \centering
  \textcolor{red}{%
  \begin{tabular}{|l|}
    \hline
    \multicolumn{1}{|c|}{\rule[-3mm]{0mm}{8mm}{\textsf{Feature}}}\\
    \hline
    \rule[0mm]{0mm}{5mm}{\textsf{\small name: SName }}\\
    \rule[-3mm]{0mm}{5mm}{\textsf{\small state: State[2..*] }}\\
    \hline
  \end{tabular}
}
  \caption{The definition of a specific Feature attached to a Specie type.}
  \label{fig:feature}
\end{figure}

\paragraph{State}\label{sec:state}

Each \texttt{feature} element contains a \texttt{listOfStates} attribute,
containing at least two \texttt{state} element (otherwise one doesn't need this
feature, do we?).

\begin{figure}[h]
  \vspace*{8pt}
  \centering
  \textcolor{red}{%
  \begin{tabular}{|l|}
    \hline
    \multicolumn{1}{|c|}{\rule[-3mm]{0mm}{8mm}{\textsf{State}}}\\
    \hline
    \rule[-3mm]{0mm}{8mm}{\textsf{\small name: SName }}\\
    \hline
  \end{tabular}
}
  \caption{The definition of one of the states possibly taken by a specific 
\texttt{feature} of a \texttt{specie} type.}
  \label{fig:state}
\end{figure}

\subsubsection{Example of a Feature}\label{sec:examplefeature}

The following example describes a protein which can exist under various
conformations, according to its degree of folding. Therefore we define a feature
named ``Folding'', which here can take three alternative values: ``unfolded'',
``folded'' and ``inactivated'' (the latter can correspond to a degradation, a
misfolding, or even to an interaction with some kind of other molecule such as a
chaperone).

\begin{example}
<feature name="Folding">
    <listOfStates>
        <state name="unfolded">
        <state name="folded">
        <state name="inactivated">
    </listOfStates>
</feature>
\end{example}


\subsection{InitialState}\label{sec:initialstate}

The \texttt{listOfInitialStates} expresses the initial amount of each state of
the specie (that is specific sets of values taken by each of the features) which
is present at the beginning of the simulation.  All features with an
\texttt{initialAmount} different of zero must be listed in the
\texttt{listOfInitialStates}.

\begin{figure}[h]
  \vspace*{8pt}
  \centering
  \textcolor{red}{%
  \begin{tabular}{|l|}
    \hline
    \multicolumn{1}{|c|}{\rule[-3mm]{0mm}{8mm}{\textsf{InitialState}}}\\
    \hline
    \rule[0mm]{0mm}{5mm}{\textsf{\small name: SName }}\\
    \textsf{\small initialAmount: double }\\
    \textsf{\small units: SName \{use=``optional''\}  }\\
    \rule[-3mm]{0mm}{5mm}{\textsf{\small featureState: FeatureState[1..*] }}\\
    \hline
  \end{tabular}
}
  \caption{The definition of a specific \texttt{InitialState} of a given \texttt{Specie} type}
  \label{fig:initialstate}
\end{figure}

The sum of the \texttt{initialAmount} of all the \texttt{initialStates}
(Figure~\ref{fig:initialstate}) in a \texttt{listOfInitialStates} must equal the
\texttt{initialAmount} of the corresponding specie (i.e. all instances of the
specie are under one state or another).

\subsubsection{FeatureState}\label{sec:featurestate}

The \texttt{listOfFeatureState} element describes one state of the specie, i.e.
a unique list of values taken by all the features.

\begin{figure}[h]
  \vspace*{8pt}
  \centering
  \textcolor{red}{%
  \begin{tabular}{|l|}
    \hline
    \multicolumn{1}{|c|}{\rule[-3mm]{0mm}{8mm}{\textsf{FeatureState}}}\\
    \hline
    \rule[0mm]{0mm}{5mm}{\textsf{\small feature: SName }}\\
    \rule[-3mm]{0mm}{5mm}{\textsf{\small state: SName }}\\
    \hline
  \end{tabular}
}
  \caption{The definition of a particular state taken by a specific \texttt{feature}
    of a \texttt{specie} type}
  \label{fig:featurestate}
\end{figure}

\subsubsection{Example of an InitialState}\label{sec:exampleinitialstate}

\begin{example}
<initialState name="nonactivated" initialAmount="500">
    <listOfFeatureStates>
        <featureState name="Folding" state="unfolded">
        <featureState name="Phosphorylation" state="noPhosphate">     
   </listOfFeatureStates>
</initialState>
\end{example}

\subsection{Complete example of a specie element}\label{sec:examplespecie}

\begin{example}
<specie name="Specie1" initialAmount="1000">
    <listOfFeatures>
        <feature name="Folding">
            <listOfStates>
                <state name="unfolded">
                <state name="folded">
                <state name="inactivated">
            </listOfStates>
        </feature>
        <feature name="Phosphorylation">
            <listOfStates>
                <state name="noPhosphate">
                <state name="Phosphate">
            </listOfStates>
        </feature>
    </listOfFeatures> 
    <listOfInitialStates>
        <initialState name="nonactivated" initialAmount="500">
            <listOfFeatureStates>
                <featureState name="Folding" state="unfolded">
                <featureState name="Phosphorylation" state="noPhosphate">     
            </listOfFeatureStates>
        </initialState>
        <initialState name="activated" initialAmount="500">
            <listOfFeatureStates>
                <featureState name="Folding" state="folded">
                <featureState name="Phosphorylation" state="Phosphate">       
            </listOfFeatureStates>
        </initialState>
    </listOfInitialStates>
</specie>
\end{example}

\subsection{Issues}\label{sec:issuespecie}

In this proposal, we have attempted to express multistate molecules by extending
the \texttt{complex} element present in SBML~1.  This contrasts from Andrew's
initial approach of introducing a novel \texttt{complexSpecie} element.
We think our approach works quite well, but have we left anything out?  Are
there any objections to extending the \texttt{complex} element using optional
attributes?

Confusion could arise from the dual meaning we use for the word ``state''. A
\emph{feature} is one of the many characteristics of a specie, which can take a
discrete number of \textbf{states}. At the same time, a \textbf{state} of the
\emph{specie} itself is defined by a specific set of feature \textbf{states}.

The \texttt{compartment} element could be conserved for SBML~1 compatibility.
However its removal would be coherent with the extension proposed by the ECell
group.  If a specie is defined at the root of the model, it is defined for every
compartment (The initialAmount has to be expressed as a concentration then).
If a specie is defined in a subset of the various compartments, we don't need to
specify the compartments since all the definitions are located within the
compartments themselves.

%=============================================================================
\section{Reactions}
\label{sec:reactions}
%=============================================================================

We present two alternatives of how to extend the \texttt{reaction} element to
enable the distinction between specific instances of species.  One does not
require any change of the \texttt{reaction} element, but affects the
\texttt{specieReference} instead, while the other affects
the \texttt{reaction} element itself.  We would appreciate feedback on which
you think is better.

Before that, we will present parts of the extension which are common to both
options, dealing with the modification of reaction rates, and with the states
of nascent species, created or modified by reactions. We utilise the concept of
a ``reaction modifier'', which defines the effect of the reactant state on a
reaction.  This allows all state-dependent instances of a reaction to be
expressed as a single reaction.


\subsection{kineticLaw}\label{sec:kineticLaw}

Because the value of a reaction rate modifier can depend on the
state of more than one species, we need to tie it to kinetic
parameters rather than individual specie concentrations.  

In this proposal a \texttt{listOfStateEffects} is an optional
attribute of the \texttt{kineticLaw} element
(Figure~\ref{fig:kineticlaw}).

\begin{figure}[h]
  \vspace*{8pt}
  \centering
  \begin{tabular}{|l|}
    \hline
    \multicolumn{1}{|c|}{\rule[-3mm]{0mm}{8mm}{\textsf{KineticLaw}}}\\
    \hline
    \rule[0mm]{0mm}{5mm}{\textsf{\small formula: string}}\\
    \textsf{\small parameter: Parameter[0..*]} \\
    \textsf{\small timeUnits: SName \{use=``optional''\}}\\
    \textsf{\small substanceUnits: SName \{use=``optional''\}}\\
    \rule[-3mm]{0mm}{5mm}{\textsf{\small \textcolor{red}{stateEffect: StateEffect[0..*] \{use=``optional''\}}}}\\
    \hline
  \end{tabular}
  \caption{The definition of the proposed extended \texttt{kineticLaw} element. Addenda are shown in red.}
  \label{fig:kineticlaw}
\end{figure}

\subsubsection{stateEffect}\label{sec:stateeffect}

The \texttt{stateEffect} element specifies which parameter it will modify (by
name), the actual value of the modifier (as a real-valued number) and a
\texttt{listOfSpecieStates}.  \texttt{specieState} (there might be a better name
for this) is a new element used to specify the set of conditions under which
this \texttt{stateEffect} applies.  The elements of the
\texttt{listOfSpecieStates} are interpreted using the AND operator, so the
conditions in all elements of the list must be satisfied for the
\texttt{stateEffect} to apply (figure~\ref{fig:stateeffect}).

\begin{figure}[h]
  \vspace*{8pt}
  \centering
  \textcolor{red}{%
  \begin{tabular}{|l|}
    \hline
    \multicolumn{1}{|c|}{\rule[-3mm]{0mm}{8mm}{\textsf{StateEffect}}}\\
    \hline
    \rule[0mm]{0mm}{5mm}{\textsf{\small \textcolor{red}{parameter: SName }}}\\
    \textsf{\small \textcolor{red}{modifier: double \{use=``default'' value=``0''\}}}\\
    \rule[-3mm]{0mm}{5mm}{\textsf{\small \textcolor{red}{specieState: SpecieState[1..*] }}}\\
    \hline
  \end{tabular}
}
  \caption{The definition of the stateEffect element.}
  \label{fig:stateeffect}
\end{figure}

The ``modifier'' attribute is the coefficient which modulates the reaction
velocity.  It is multiplied with the term of the kinetic law which involves the
specie to which it is associated.  Within the framework of the mass action law,
this is effectively equivalent to modifying the quantity of the reacting specie.

\paragraph{specieState}

Each \texttt{specieState} element specifies a \texttt{specieReference} (or a
\texttt{specieInstance}, see section~\ref{sec:reactsecond}) by its name and a
\texttt{listOfFeatureConditions}.

\begin{figure}[h]
  \vspace*{8pt}
  \centering
  \textcolor{red}{%
  \begin{tabular}{|l|}
    \hline
    \multicolumn{1}{|c|}{\rule[-3mm]{0mm}{8mm}{\textsf{SpecieState}}}\\
    \hline
    \rule[0mm]{0mm}{5mm}{\textsf{\small \textcolor{red}{specie: SName }}}\\
    \rule[-3mm]{0mm}{5mm}{\textsf{\small \textcolor{red}{featureCondition: FeatureCondition[1..*] }}}\\
    \hline
  \end{tabular}
}
  \caption{The definition of the specieState element.}
  \label{fig:speciestate}
\end{figure}

\subparagraph{featureCondition}

\begin{figure}[h]
  \vspace*{8pt}
  \centering
  \textcolor{red}{%
  \begin{tabular}{|l|}
    \hline
    \multicolumn{1}{|c|}{\rule[-3mm]{0mm}{8mm}{\textsf{FeatureCondition}}}\\
    \hline
    \rule[0mm]{0mm}{5mm}{\textsf{\small feature: SName }}\\
    \rule[-3mm]{0mm}{5mm}{\textsf{\small condition: formula }}\\
    \hline
  \end{tabular}
}
  \caption{The definition of the \texttt{featureCondition} element.}
  \label{fig:featurecondition}
\end{figure}

Each \texttt{featureCondition} element uses logic expressions (consisting of
parentheses and the operators AND, OR and NOT) to define the states of each
feature that the \texttt{stateEffect} applies to.  For instance, if a feature
has five states A, B, C, D and E, and a \texttt{stateEffect} applies to states A
or B, one could either write ``A OR B'', or ``NOT (C OR D OR E)''.

(Comment: The data type "LogicExpression" could be called LogExpression,
LExpression or even LExp, or again BooleanExpression or BoolExpression.)

States which do not match the \texttt{listOfFeatureConditions} of any of the
\texttt{stateEffects} of a reaction are assumed to have a modifier of 0 (i.e.
they cannot take part in the reaction).

\subsubsection{example}

Reactants A and B bind to produce reactant C.  Both A and B can be in
one of two states, active, or inactive (denoted by a subscript of 1 or
0, respectively).  The rates are:

\begin{itemize}
\item $A_0 + B_0 \rightarrow C$   ---  modifier = 0
\item $A_0 + B_1 \rightarrow C$   ---  modifier = 0.6
\item $A_1 + B_0 \rightarrow C$   ---  modifier = 0.2
\item $A_1 + B_1 \rightarrow C$   ---  modifier = 1
\end{itemize}

\begin{example}
<kineticLaw formula="kf * A * B - kr * C">
    <listOfParameters>
        <parameter name ="kf" value="1000">
        <parameter name ="kr" value="1">
    </listOfParameters>
    <listOfStateEffect>
        <stateEffect parameter="kf" modifier="0.6">
            <listOfSpecieStates>
                <specieState specie="A">
                    <listOfFeatureConditions>
                        <featureCondition feature="Activity" state="inactive">
                    </listOfFeatureConditions>
                </specieState>
                <specieState specie="B">
                    <listOfFeatureConditions>
                        <featureCondition feature="Activity" state="active">
                    </listOfFeatureConditions>
                </specieState>
            </listOfSpecieStates>
        </stateEffect>
        <stateEffect parameter="kf" modifier="0.2">
            <listOfSpecieStates>
                <specieState specie="A">
                    <listOfFeatureConditions>
                        <featureCondition feature="Activity" state="active">
                    </listOfFeatureConditions>
                </specieState>
                <specieState specie="B">
                    <listOfFeatureConditions>
                        <featureCondition feature="Activity" state="inactive">
                    </listOfFeatureConditions>
                </specieState>
            </listOfSpecieStates>
        </stateEffect>
        <stateEffect parameter="kf" modifier="1.0">
            <listOfSpecieStates>
                <specieState specie="A">
                    <listOfFeatureConditions>
                        <featureCondition feature="Activity" state="active">
                    </listOfFeatureConditions>
                </specieState>
                <specieState specie="B">
                    <listOfFeatureConditions>
                        <featureCondition feature="Activity" state="active">
                    </listOfFeatureConditions>
                </specieState>
            </listOfSpecieStates>
        </stateEffect>
    </listOfStateEffects>
</kineticLaw>
\end{example}


\subsection{States of nascent molecules}\label{sec:nascentstate}

If the newly created molecule is also a multistate complex, it is
necessary to specify all the state of all of its features.  This is
easily implemented by creating a new element called
\texttt{nascentState} (figure~\ref{fig:nascentstate}) which has a
\texttt{listOfFeatureStates} as an attribute (see
section~\ref{sec:featurestate}). \texttt{nascentState} in turn becomes
an optional attribute of the \texttt{specieReference} element (option~1
section~\ref{sec:reactfirst}) or the \texttt{specieInstance} element
(option~2 section~\ref{sec:reactsecond}).

\begin{figure}[h]
  \vspace*{8pt}
  \centering
  \textcolor{red}{%
  \begin{tabular}{|l|}
    \hline
    \multicolumn{1}{|c|}{\rule[-3mm]{0mm}{8mm}{\textsf{NascentState}}}\\
    \hline
    \rule[0mm]{0mm}{5mm}{\textsf{\small name: SName }}\\
    \textsf{\small proportion: double }\\
    \rule[-3mm]{0mm}{5mm}{\textsf{\small featureState: FeatureState[1..*] }}\\
    \hline
  \end{tabular}
}
  \caption{The definition of a specific \texttt{nascentState} of a given \texttt{specie} type}
  \label{fig:nascentstate}
\end{figure}


If the created molecules come to existence under a specific set of nascent
states, with well defined probabilities, we fill the \texttt{nascentState}
elements. Each element contains an attribute which specifies the probability of
that state (the default is 1. It is up to the parser to rescale everything if
the sum of the \texttt{proportion} elements over all the \texttt{nascentState}
elements is more than 1).

\subsubsection{Example}

Here is the result of a ligand binding affecting the activity of a specie. The
specie has two features: the presence of ligand, and the activity.

\begin{example}
<listOfNascentState>
    <nascentState name="A1" proportion="0.9">
        <listOfFeatureState>
            <featureState name="ligand" state="bound"/>
            <featureState name="activity" state="active"/>
        </listOfFeatureState>
    </nascentState>
    <nascentState name="I1" proportion="0.1">
        <listOfFeatureState>
            <featureState name="ligand" state="bound"/>
            <featureState name="activity" state="inactive"/>
        </listOfFeatureState>
    </nascentState>
</listOfNascentState>

\end{example}

In case the nascent state depends on the state(s) of the reactants, we have to
enumerate the various reactions, with each \texttt{listOfNascentState} limited
to one element.

If no nascent states are defined for a multistate product, the feature
values are allocated randomly with a stochastic algorithm, or evenly
with a deterministic algorithm.

We now turn to the two alternative proposals for the \texttt{reaction}
element and its attributes.

\subsection{First possibility: \texttt{reaction} unchanged}\label{sec:reactfirst}

\subsubsection{specieReference}

In a first option, the \texttt{reaction} element is not modified, but the
\texttt{specieReference} element, as shown in figure~\ref{fig:speciereference1}.

\begin{figure}[h]
  \vspace*{8pt}
  \centering
  \begin{tabular}{|l|}
    \hline
    \multicolumn{1}{|c|}{\rule[-3mm]{0mm}{8mm}{\textsf{SpecieReference}}}\\
    \hline
    \rule[0mm]{0mm}{5mm}{\textsf{\small \textcolor{red}{name: SName }}}\\
    \textsf{\small specie: SName }\\
    \textsf{\small stoichiometry: integer \{use=``default'' value=``1''\}  }\\
    \textsf{\small denominator: integer \{use=``default'' value=``1''\} }\\
    \rule[-3mm]{0mm}{5mm}{\textsf{\small \textcolor{red}{nascentState: NascentState[0..*] \{use=``optional''\}} }}\\
    \hline
  \end{tabular}
  \caption{The definition of the extended specieReference element under the first option. Addenda are shown in red.}
  \label{fig:speciereference1}
\end{figure}

The addition of the \texttt{name} attribute to the \texttt{specieReference} element
allows different instances of the same specie to be distinguished
within a reaction.  This name would be used instead of the specie name
in the \texttt{kineticLaw}.  It takes the same value as the \texttt{specie} by
default, so it need not be explicitly defined for reactants that don't
need specific instances identified (i.e. if the reactant and product species do not have
multiple states, as in SBML~1).

\subsubsection{example}

An example of a reaction which requires specific instances to be
identified is the reversible transfer of a phosphate group between
two molecules of specie A:
\begin{displaymath}
  A1P + A2 \rightleftharpoons A1 + A2P
\end{displaymath}

For this reaction, \texttt{listOfReactants} and \texttt{listOfProducts} would
look like this (see figure~\ref{fig:stateeffect} for the definition of
\texttt{stateEffect} and figure~\ref{fig:featurecondition} for the definition of
\texttt{featureCondition}):

\begin{example}
<listOfReactants>
    <specieReference name="A1" specie="A" />
    <specieReference name="A2" specie="A" />
</listOfReactants>
<listOfProducts>
    <specieReference name="A1" specie="A" />
    <specieReference name="A2" specie="A"/>
</listOfReactants>
<kineticLaw formula="(kf * Specie1A * Specie1B - kr * Specie1A * Specie1B) * compOne * N_A">
    <listOfParameters>
        <parameter name="kf" value="100" />
        <parameter name="kr" value="0" />
    </listOfParameters>
    <listOfStateEffects>
        <stateEffect parameter="kf" modifier="1">
            <listOfSpecieStates>
                <specieState specie="A1">
                    <listOfFeatureConditions>
                        <featureCondition feature="Phosphorylation" state="Phosphate"/>
                    </listOfFeatureConditions>
                </specieState>
                <specieState specie="A2">
                    <listOfFeatureConditions>
                        <featureCondition feature="Phosphorylation" state="noPhosphate">
                    </listOfFeatureConditions>
                </specieState>
            </listOfSpecieState>
        </stateEffect>
         <stateEffect parameter="kr" modifier="1">
            <listOfSpecieStates>
                <specieState specie="A1">
                    <listOfFeatureConditions>
                        <featureCondition feature="Phosphorylation" state="noPhosphate"/>
                    </listOfFeatureConditions>
                </specieState>
                <specieState specie="A2">
                    <listOfFeatureConditions>
                        <featureCondition feature="Phosphorylation" state="Phosphate">
                    </listOfFeatureConditions>
                </specieState>
            </listOfSpecieState>
        </stateEffect>
    </listOfStateEffects>
</kineticLaw>
\end{example}

Note that both \texttt{specieReference} elements A1 and A2 point to the same
specie A. To be compared with the second possibility described in the section
\ref{sec:reactsecond}

\subsection{Second possibility: extension of the \texttt{reaction} element}\label{sec:reactsecond}

The second option requires modifications of the \texttt{reaction} element as
shown in figure~\ref{fig:reaction2}.

\begin{figure}[h]
  \vspace*{8pt}
  \centering
  \begin{tabular}{|l|}
    \hline
    \multicolumn{1}{|c|}{\rule[-3mm]{0mm}{8mm}{\textsf{Reaction}}}\\
    \hline
    \rule[0mm]{0mm}{5mm}{\textsf{\small name: SName }}\\
    \textsf{\small \textcolor{red}{specieInstance: SpecieInstance[0..*] \{use=``optional''\} }}\\
    \textsf{\small reactant: SpecieReference[1..*]  }\\
    \textsf{\small product: SpecieReference[1..*]  }\\
    \textsf{\small kineticLaw: KineticLaw \{minOccurs=``0''\}  }\\
    \textsf{\small reversible: boolean \{use=``default'' value=``true''\} }\\
    \rule[-3mm]{0mm}{5mm}{\textsf{\small fast: boolean \{use=``default'' value=``false''\} }}\\
    \hline
  \end{tabular}
  \caption{The definition of the proposed extended \texttt{reaction} element. Addenda are shown in red.}
  \label{fig:reaction2}
\end{figure}

\subsubsection{specieInstance}

This solution follows Andrew's idea of explicitly representing
individual instances of species with a new element, which we call
\texttt{specieInstance} (depicted in figure \ref{fig:specieinstance}).  This is used to distinguish between individual
instances of the species involved in a reaction.  The
\texttt{listOfSpecieInstance} attribute of the reaction element is optional,
and would only need to be used in reactions where specific instances
of reacting species need to be distinguished. 

\begin{figure}[h]
  \vspace*{8pt}
  \centering
  \textcolor{red}{%
  \begin{tabular}{|l|}
    \hline
    \multicolumn{1}{|c|}{\rule[-3mm]{0mm}{8mm}{\textsf{SpecieInstance}}}\\
    \hline
    \rule[0mm]{0mm}{5mm}{\textsf{\small name: SName }}\\
    \textsf{\small specie: SName } \\
    \rule[-3mm]{0mm}{5mm}{\textsf{\small nascentState: NascentState[0..*]  \{use=``optional''\}}}\\
    \hline
  \end{tabular}
}
  \caption{The definition of specieInstance}
  \label{fig:specieinstance}
\end{figure}

\subsubsection{example}

Under this scheme, to express the reactants and products of the reversible
phosphorylation example presented above, one would also need a
\texttt{listOfSpecieInstances}.  Note that the \texttt{specieReference} elements
point to different \texttt{specie} elements, in fact \texttt{specieInstance}
elements (to be compared with the first possibility presented in section
\ref{sec:reactfirst}). The definition would then looks like (see
figure~\ref{fig:stateeffect} for the definition of \texttt{stateEffect} and
figure~\ref{fig:featurecondition} for the definition of
\texttt{featureCondition}):

\begin{example}
<listOfSpecieInstances>
    <SpecieInstance name="A1" specie="A" />
    <SpecieInstance name="A2" specie="A" />
</listOfSpecieInstances>
<listOfReactants>
    <specieReference specie="A1" />
    <specieReference specie="A2" />
</listOfReactants>
<listOfProducts>
    <specieReference specie="A1" />
    <specieReference specie="A2" />
</listOfReactants>
<kineticLaw formula="(kf * Specie1A * Specie1B - kr * Specie1A * Specie1B) * compOne * N_A">
    <listOfParameters>
        <parameter name="kf" value="100" />
        <parameter name="kr" value="0" />
    </listOfParameters>
    <listOfStateEffects>
        <stateEffect parameter="kf" modifier="1">
            <listOfSpecieStates>
                <specieState specie="A1">
                    <listOfFeatureConditions>
                        <featureCondition feature="Phosphorylation" state="Phosphate"/>
                    </listOfFeatureConditions>
                </specieState>
                <specieState specie="A2">
                    <listOfFeatureConditions>
                        <featureCondition feature="Phosphorylation" state="noPhosphate">
                    </listOfFeatureConditions>
                </specieState>
            </listOfSpecieState>
        </stateEffect>
         <stateEffect parameter="kr" modifier="1">
            <listOfSpecieStates>
                <specieState specie="A1">
                    <listOfFeatureConditions>
                        <featureCondition feature="Phosphorylation" state="noPhosphate"/>
                    </listOfFeatureConditions>
                </specieState>
                <specieState specie="A2">
                    <listOfFeatureConditions>
                        <featureCondition feature="Phosphorylation" state="Phosphate">
                    </listOfFeatureConditions>
                </specieState>
            </listOfSpecieState>
        </stateEffect>
    </listOfStateEffects>
</kineticLaw>
\end{example}


\section{Complete example}\label{sec:completeexample}

Note that this example has intentionally been made independent of
StochSim. In StochSim, the features of multistate complexes are
represented by binary flags, so each can only have two states. In this
example the feature ``Folding'' possess three states (this could be
encoded in StochSim using two binary flags).

\begin{changemargin}{-2cm}{0cm}
\begin{example}
<sbml version="1" level="1">
    <model name="ExampleMultipleState" />
    <notes>
        <body xmlns="http://www.w3.org/1999/xhtml">
            <p>This model exemplifies the use of the extension proposed by the StochSim team.</p>
            <p>The main reaction is Specie1 + Specie2 <=> Specie3</p>
            <p>Specie1 possesses 2 features affecting the reaction rate. 
               The "Folding" exists under three states "unfolded", "folded" and "inactivated", 
               the "Phosphorylation" under two "noPhosphate" and "Phosphate".</p>
            <p>The features are regulated by two interconversions and one cross-phosphorylation.</p>
            <p>Specie1-unfolded <=> Specie1-folded</p>
            <p>Specie1-folded <=> Specie1-inactivated</p>
            <p>Specie1-folded + Specie1-inactivated-P <=> Specie1-folded-P + Specie1-inactivated </p>
        </body>
    </notes>

    <listOfParameters>
        <parameter name="N_A" value="6.022e23" />
    </listOfParameters>

    <listOfCompartments>
        <compartment name="compOne">
            <listOfSpecies>
                <specie name="Specie1" initialAmount="1000">
                    <listOfFeatures>
                        <feature name="Folding">
                            <listOfStates>
                                <state name="unfolded"/>
                                <state name="folded"/>
                                <state name="inactivated"/>
                            </listOfStates>
                        </feature>
                        <feature name="Phosphorylation">
                            <listOfStates>
                                <state name="noPhosphate"/>
                                <state name="Phosphate"/>
                            </listOfStates>
                        </feature>
                    </listOfFeatures>   
                    <listOfInitialStates>
                        <initialState name="nonactivated" initialAmount="500">
                            <listOfFeatureStates>
                                <featureState feature="Folding" state="unfolded"/>
                                <featureState feature="Phosphorylation" state="noPhosphate"/>
                            </listOfFeatureStates>
                        </initialState>
                        <initialState name="activated" initialAmount="500">
                            <listOfFeatureStates>
                                <featureState feature="Folding" state="folded"/>
                                <featureState feature="Phosphorylation" state="Phosphate"/>
                            </listOfFeatureStates>
                        </initialState>
                    </listOfInitialStates>
                </specie>
                <specie name="Specie2" initialAmount="1000" />
                <specie name="Specie3" initialAmount="0" />
            </listOfSpecies>

            <listOfReactions>
                <reaction  name="main">
                    <listOfReactants>
                        <specieReference specie="Specie1" stoichiometry="1"/>
                        <specieReference specie="Specie2" stoichiometry="1"/>
                    </listOfReactants>
                    <listOfProducts>
                        <specieReference name="Specie3" stoichiometry="1"/>
                    </listOfProducts>
                    <kineticLaw formula="(kf * Specie1 * Specie2 - kr * Specie3) * compOne * N_A">
                        <listOfParameters>
                            <parameter name="kf" value="2500" />
                            <parameter name="kr" value="100" />
                        </listOfParameters>
                        <listOfStateEffects>
                            <stateEffect parameter="kf" modifier="0.5">
                                <listOfSpecieState>
                                    <specieState specie="specie1">
                                        <listOfFeatureConditions>
                                            <featureCondition feature="Folding" condition="folded"/>
                                            <featureCondition feature="Phosphorylation" condition="noPhosphate"/>
                                        </listOfFeatureConditions>
                                    </specieState>
                                </listOfSpecieState>
                            </stateEffect>  
                            <stateEffect parameter="kf" modifier="1">
                                <listOfSpecieState>
                                    <specieState specie="specie1">
                                        <listOfFeatureConditions>
                                            <featureCondition feature="Folding" condition="folded"/>
                                            <featureCondition feature="Phosphorylation" condition="Phosphate"/>
                                        </listOfFeatureConditions>
                                    </specieState>
                                </listOfSpecieState>
                            </stateEffect>  

<!-- This is optional (since the default modifier is 0), and will be omitted in later listOfStateEffects -->

                            <stateEffect parameter="kf" modifier="0">
                                 <listOfSpecieState>
                                    <specieState specie="specie1">
                                        <listOfFeatureConditions>
                                            <featureCondition feature="Folding" condition="unfolded or inactivated"/>
                                        </listOfFeatureConditions>
                                    </specieState>
                                </listOfSpecieState>
                            </stateEffect>

<!-- =================================================================================================== -->

                        </listOfStateEffects>
                    </kineticLaw>
                </reaction>
                <reaction name="folding">
                    <listOfReactants>
                        <specieReference specie="Specie1" stoichiometry="1" />
                    </listOfReactants>
                    <listOfProducts>
                        <specieReference specie="Specie1" stoichiometry="1">
                    </listOfProducts>
                    <kineticLaw formula="(kf * Specie1 - kr * Specie1) * compOne * N_A">
                        <listOfParameters>
                            <parameter name="kf" value="1000" />
                            <parameter name="kr" value="10" />
                        </listOfParameters>
                        <listOfStateEffects>
                            <stateEffect parameter="kf" modifier="1">
                                <listOfSpecieState>
                                    <specieState specie="specie1">
                                        <listOfFeatureConditions>
                                            <featureCondition feature="Folding" condition="unfolded"/>
                                        </listOfFeatureConditions>
                                    </specieState>
                                </listOfSpecieState>
                            </stateEffect>  
                            <stateEffect parameter="kr" modifier="1">
                                <listOfSpecieState>
                                    <specieState specie="specie1">
                                        <listOfFeatureConditions>
                                            <featureCondition feature="Folding" condition="folded"/>
                                        </listOfFeatureConditions>
                                    </specieState>
                                </listOfSpecieState>
                           </stateEffect>  
                        </listOfStateEffects>
                    </kineticLaw>
                </reaction>

                <reaction name="inactivation">
                    <listOfReactants>
                        <specieReference specie="Specie1" stoichiometry="1" />
                    </listOfReactants>
                    <listOfProducts>
                        <specieReference specie="Specie1" stoichiometry="1">
                    </listOfProducts>
                    <kineticLaw formula="(kf * Specie1 - kr * Specie1) * compOne * N_A">
                        <listOfParameters>
                            <parameter name="kf" value="100" />
                            <parameter name="kr" value="0" />
                        </listOfParameters>
                        <listOfStateEffects>
                            <stateEffect parameter="kf" modifier="1">
                                <listOfSpecieState>
                                    <specieState specie="specie1">
                                        <listOfFeatureConditions>
                                            <featureCondition feature="Folding" condition="folded"/>
                                        </listOfFeatureConditions>
                                    </specieState>
                                </listOfSpecieState>
                            </stateEffect>  
                            <stateEffect parameter="kr" modifier="1">
                                <listOfSpecieState>
                                    <specieState specie="specie1">
                                        <listOfFeatureConditions>
                                            <featureCondition feature="Folding" condition="inactivated"/>
                                        </listOfFeatureConditions>
                                    </specieState>
                                </listOfSpecieState>
                            </stateEffect>  
                        </listOfStateEffects>
                    </kineticLaw>
                </reaction>

                <reaction name="trans-phosphorylation">

<!-- ============================================ Second solution ====================================== -->

                    <listOfSpecieInstances>           
                        <specieInstance name="Specie1A" specie="Specie1" />
                        <specieInstance name="Specie1B" specie="Specie1" />
                    </listOfSpecieInstances>

                    <listOfReactants>

<!-- ============================================ First solution ======================================= -->

                        <specieReference name="Specie1A" specie="Specie1" stoichiometry="1" />
                        <specieReference name="Specie1B" specie="Specie1" stoichiometry="1" />

<!-- ============================================ Second solution ====================================== -->

                        <specieReference specie="Specie1A" stoichiometry="1" />
                        <specieReference specie="Specie1B" stoichiometry="1" />

                    </listOfReactants>
                    <listOfProducts>
<!-- ============================================ First solution ======================================= -->

                        <specieReference name="Specie1A" specie="Specie1" stoichiometry="1" />
                        <specieReference name="Specie1B" specie="Specie1" stoichiometry="1" />

<!-- ============================================ Second solution ====================================== -->

                        <specieReference specie="Specie1A" stoichiometry="1">
                        <specieReference specie="Specie1B" stoichiometry="1">

                    </listOfProducts>
                    <kineticLaw formula="(kf * Specie1A * Specie1B - kr * Specie1A * Specie1B) * compOne * N_A">
                        <listOfParameters>
                            <parameter name="kf" value="100" />
                            <parameter name="kr" value="0" />
                        </listOfParameters>
                        <listOfStateEffects>
                            <stateEffect parameter="kf" modifier="1">
                                <listOfSpecieState>
                                    <specieState specie="specie1A">
                                        <listOfFeatureConditions>
                                            <featureCondition feature="Folding" condition="folded"/>
                                            <featureCondition feature="Phosphorylation" condition="noPhosphate"/>
                                        </listOfFeatureConditions>
                                    </specieState>
                                </listOfSpecieState>
                            </stateEffect>  
                            <stateEffect parameter="kf" modifier="1">
                                <listOfSpecieState>
                                    <specieState specie="specie1B">
                                        <listOfFeatureConditions>
                                            <featureCondition feature="Folding" condition="inactivated"/>
                                            <featureCondition feature="Phosphorylation" condition="Phosphate"/>
                                        </listOfFeatureConditions>
                                    </specieState>
                                </listOfSpecieState>
                            </stateEffect>  
                            <stateEffect parameter="kr" modifier="1">
                                <listOfSpecieState>
                                    <specieState specie="specie1A">
                                        <listOfFeatureConditions>
                                            <featureCondition name="Folding"  condition="folded"/>
                                            <featureCondition name="Phosphorylation" condition="Phosphate"/>
                                        </listOfFeatureConditions>
                                    </specieState>
                                </listOfSpecieState>
                            </stateEffect>  
                            <stateEffect parameter="kr" modifier="1">
                                <listOfSpecieState>
                                    <specieState specie="specie1B">                                    
                                        <listOfFeatureConditions>
                                            <featureCondition feature="Folding" condition="inactivated"/>
                                            <featureCondition feature="Phosphorylation" condition="noPhosphate"/>
                                        </listOfFeatureConditions>
                                    </specieState>
                                </listOfSpecieState>
                            </stateEffect>  
                        </listOfStateEffects>                         
                    </kineticLaw>
                </reaction>
            </listOfReactions>
        </compartment>
    </listOfCompartments>
</model>
\end{example}
\end{changemargin}

%=============================================================================
% References
%=============================================================================

\bibliographystyle{apalike}
\bibliography{multistates}
\end{document}


