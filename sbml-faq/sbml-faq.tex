% To to:
% - mention SBML icons

\documentclass{cekfaq}

\newcommand{\yes}{\raisebox{1pt}{\rule{3.5pt}{3.5pt}}}

\begin{document}

\title{SBML Frequently Asked Questions}

\author{Andrew Finney, Michael Hucka}

\authoremail{\{afinney,mhucka\}@caltech.edu}

\address{Systems Biology Workbench Development Group\\
  ERATO Kitano Symbiotic Systems Project\\
  Control and Dynamical Systems, MC 107-81\\
  California Institute of Technology, Pasadena, CA 91125, USA\\[3pt]
  \url{http://www.sbml.org/}}

\acknowledge{Principal Investigators: John Doyle and Hiroaki Kitano}

\date{{\normalsize$Revision$}\\[5pt]\today{}}

\small
\maketitlepage
\normalsize

\section{Disclaimer}

This is a non-normative document; i.e., it does not define any aspect of
the SBML standard.  Only the SBML specification documents define SBML (see
Section~\ref{sec:docs}).


\section{SBML Introduction and Background}
\label{sec:intro}


\subsection{What is SBML?}

The Systems Biology Markup Language (SBML) is a machine-readable format for
describing qualitative and quantitative models of biochemical networks.  It
can also be used to express the interactions of biochemical networks with
other phenomena.  By a ``biochemical network'', we mean a system consisting
of biochemical entities linked by chemical reactions that alter and
transform the entities.  The primary encoding of SBML is
\href{http://www.w3.org/XML/}{XML}.


\subsection{What is the purpose of SBML?}

SBML is designed to enable the exchange of biochemical network models
between different software packages.  The aim is to (1) enable the enable
the exchange of models between software tools with little or no human
intervention, thus allowing the tools to be properly integrated; and (2)
enable these models to be published in electronic form alongside, for
example, peer-reviewed journal articles.  The emphasis is on supporting
quantitative models.


\subsection{What can be represented in SBML?}

An SBML model consists of a set of chemical entities linked by reactions
that can transform one entity into another.  Entities are located in one or
more volumetric compartments.  As well as representing biochemical
networks, SBML can represent quantitative models of interaction between
these networks and other phenomena.  SBML can describe discrete events that
are triggered by state changes in the modelled system.  The scope of SBML is
constantly evolving through a community led development.

\subsection{Is SBML a database schema?}

Not specifically.  There is no reason why SBML models could not be stored
in a database, nor is there any reason why you could not use SBML as a
schema.  However, this was not the motivation for the creation of SBML.  An
SBML model is meant to encode a consistent view of knowledge of a
biological system.  SBML is not meant to encode a large set of potentially
conflicting knowledge about such a system.

\subsection{Who developed SBML?}

SBML originated out of a workshop on software platforms for systems biology
held in the year 2000 and funded by the Japan Science and Technology
Corporation.  The original authors were Michael Hucka, Andrew Finney and
Herbert Sauro, but SBML is now very much a community effort.  Please see
Section~\ref{sec:history} for more details about the history of SBML.
See Section~\ref{sec:organization}
about how you can participate in SBML's evolution.


\subsection{Where is SBML defined?}

The Systems Biology Markup Language is formally defined in specification
documents available from the SBML web site (\url{http://www.sbml.org}).
The web site also provides supporting documents, examples, and sample SBML
models.


\subsection{Where is the SBML web site?}

The home web site for SBML is \url{http://www.sbml.org/}.


\section{SBML Support}
\label{sec:sbml-support}

\subsection{Which applications support SBML?}

The following matrix (Table~\ref{tab:apps}) lists the software tools known
to support SBML, along with information about which Level of SBML they
support.  The names of the packages are clickable URLs pointing to more
information about them.  (If you know of other SBML-compatible software not
listed here, please inform the editors of this FAQ.)

\newcommand{\indev}{\multicolumn{2}{c}{(in development)}}

\begin{table}[htb]
  \centering
  \caption{Table of applications known to support SBML.}
  \label{tab:apps}
  \vspace*{1pt}
  \begin{tabular}{l!{\hspace{10pt}}c!{\hspace{10pt}}c!{\hspace{10pt}}c!{\hspace{10pt}}c}
    \toprule
                        & \multicolumn{2}{c}{\textbf{SBML Level 1}} & \multicolumn{2}{c}{\textbf{SBML Level 2}} \\
    \textbf{Application} & \textbf{Read} & \textbf{Write} & \textbf{Read} & \textbf{Write}\\
    \midrule
    \href{http://www.basis.ncl.ac.uk/technology.html}{BASIS} & & \yes\\
    \href{http://bio.bbn.com/biospice/biosketchpad/index.html}{BioSketchPad} & & & & \yes\\
    \href{http://biocomp.ece.utk.edu/}{BioSpreadsheet} & & & \yes & \yes\\
    \href{http://www-aig.jpl.nasa.gov/public/mls/cellerator/}{Cellerator} & & \yes \\
    \href{http://sbserv.symbio.jst.go.jp/002/001.html}{CellDesigner} & & & \yes & \yes\\
    \href{http://www.bii.a-star.edu.sg/sbg/cellware}{Cellware} & & & \indev\\
    \href{http://www.cytoscape.org/}{Cytoscape} & \yes & \yes \\
    \href{http://www.systemsbiology.org/}{Dizzy} \\
    \href{http://ecell.sourceforge.net/}{E-Cell} & & & \indev \\
    \href{http://biocomp.ece.utk.edu/}{ESS} & & & \yes & \yes\\
    \href{http://www.gepasi.org/}{Gepasi} & \yes & \yes \\
    \href{http://www.sys-bio.org/}{Jarnac} & \yes & \yes \\
    \href{http://www.sys-bio.org/}{JDesigner} & \yes & \yes \\
    \href{http://gnida.cs.vt.edu/~cellcyclepse/}{JigCell} & & & \yes & \yes\\
    \href{http://nsr.bioeng.washington.edu/Software/JSIM/}{JSIM} & \yes\\
    \href{http://www.sf.net/projects/sbml}{\textsc{libsbml}} & \yes & \yes & \yes & \yes\\
    \href{http://www.sf.net/projects/sbml}{mathSBML} & \yes & \yes\\
    \href{http://arep.med.harvard.edu/moma/}{MOMA} &\\ 
    \href{http://monod.molsci.org/}{Monod} & & & & \yes\\ 
    \href{http://strc.herts.ac.uk/bio/maria/NetBuilder/}{NetBuilder} & & \yes\\
    \href{http://simpath.lionbioscience.com:8080/documentation/pathscout11}{PathScout} & & \yes\\
    \href{http://www.cds.caltech.edu/~hsauro/sbml.htm}{SBML DLL Library} & \yes & \yes\\
    \href{http://www.sigpath.org}{SigPath} & & \yes\\
    \href{http://www.zoo.cam.ac.uk/comp-cell/StochSim.html}{StochSim} & & \yes\\
    \href{http://www.nrcam.uchc.edu/vcell_development/vcell_dev.html}{Virtual Cell} & \yes & \yes\\
    \href{http://www.sys-bio.org/}{WinSCAMP (beta)} & \yes & \yes\\
    \bottomrule
  \end{tabular}
\end{table}


\subsection{Are software libraries available for programming with SBML?}

The following matrix (Table~\ref{tab:libs}) lists software libraries known
to support SBML.  (If you know of others not listed here, please inform the
editors of this FAQ.)

\begin{table}[htb]
  \centering
  \caption{Table of software libraries for SBML.}
  \label{tab:libs}
  \vspace*{2pt}
  \begin{tabular}{ll!{\hspace{10pt}}c!{\hspace{10pt}}c!{\hspace{10pt}}c!{\hspace{10pt}}c}
    \toprule
                       &         & \multicolumn{2}{c}{\textbf{SBML Level 1}} & \multicolumn{2}{c}{\textbf{SBML Level 2}} \\
    \textbf{Library} & \textbf{Language}    & \textbf{Read} & \textbf{Write} & \textbf{Read} & \textbf{Write}\\
    \midrule
    \textsc{libsbml}        & C/C++, others     & \yes      & \yes      & \yes  & \yes\\
    mathSBML            & Mathematica       & \yes      & \yes\\
    SBML Windows DLL        & (Windows DLL)     & \yes      & \yes\\
    JigCell's \texttt{sbml.jar} & Java          &           &       & \yes  & \yes\\
    \bottomrule
  \end{tabular}
\end{table}


% 2003-06-09 <mhucka@caltech.edu>
% It is probably not appropriate to include this item.  We probably
% shouldn't make claims about other people's plans.
%
%\subsection{What groups plan to add support for SBML in the future?}
%Charon
%Copasi
%ProMote/DIVA
%MONOD

\subsection{Are there large groups using SBML?}

The following large consortia are known to us to be using SBML as their
standard model definition language.  (If you know of others, please inform
the editors of this FAQ.)

\begin{itemize}
\item DARPA BioSPICE
\item International E. coli Alliance (IECA)
\end{itemize}


\subsection{Where can I find examples of SBML models?}

The specification documents for SBML, available from the project web site
(\url{http://www.sbml.org/}), includes numerous simple examples.  In
addition, there is a repository of SBML Level~1 models at the following
URL: \url{http://www.sbml.org/ModelsWebPages/ModelRepository.htm}


\section{SBML Levels}
\label{sec:sbml-levels}

\subsection{What are SBML ``Levels''?}

SBML is being developed in a series of \emph{levels}, where each level adds
new features and fixes problems with the previous level.  The
lowest-numbered levels provide fundamental features that are common to all
biochemical network models.  Higher-numbered levels add more features that
are specific to particular classes of tools.  Any level can be used as a
standard for interchanging models.

\subsection{What is the current SBML Level?}

SBML Level 2 Version 1 will be finalized in June 2003.  All significant new
development will be incorporated into Level~3.

\subsection{What are the differences between Levels 1 and 2?}

Among other things, Level 2 replaces the infix string formula notation in
Level 1 with MathML expressions.  Level 2 incorporates CellML metadata.
Level~2 also introduces \emph{event} structures.  There are several other
small changes introduced in Level~2.  The complete list of changes is
documented in the SBML Level~2 specification.


\subsection{What features are anticipated in Level 3?}

People interested in SBML have organized themselves into a number of
working groups focused on different topics.  The topics of these working
groups give an indication of the features anticipated for SBML Level 3
(see Section~\ref{sec:workinggroups} for a list of working groups).

\section{Specifications and Documentation}
\label{sec:docs}

\subsection{Are there publications about SBML?}

There is currently one, published in
\href{http://bioinformatics.oupjournals.org/cgi/reprint/19/4/524?ijkey=BzZTZ.dDZEXp0U&keytype=ref&siteid=bioinfo}{\emph{Bioinformatics}, 2003,
  vol. 19, no. 4, pp. 524--531}:
\begin{quote}
\emph{The Systems Biology Markup Language (SBML): A medium for
  representation and exchange of biochemical network models}, 

Hucka, M., Finney, A., Sauro, H.~M., Bolouri, H., Doyle, J.~C., Kitano,
H., Arkin, A.~P., Bornstein, B.~J., Bray, D., Cornish-Bowden, A. , Cuellar,
A.~A., Dronov, S., Gilles, E.~D., Ginkel, M., Gor, V., Goryanin, I.~I.,
Hedley, W.~J., Hodgman, T.~C., Hofmeyr, J.-H., Hunter, P.~J., Juty, N.~S.,
Kasberger, J.~L., Kremling, A., Kummer, U., Le Nov\`{e}re, N., Loew,
L.~M., Lucio, D., Mendes, P., Minch, E., Mjolsness, E.~D., Nakayama, Y.,
Nelson, M.~R., Nielsen, P.~F., Sakurada, T., Schaff, J.~C., Shapiro,
B.~E., Shimizu, T.~S., Spence, H.~D., Stelling, J., Takahashi, K.,
Tomita, M., Wagner, J., Wang, J.
\end{quote}

\subsection{Where is SBML defined?}

The following are the specification documents for the different levels of
SBML:

\begin{itemize}
  
\item \href{http://www.sbw-sbml.org/sbml/docs/index.html}{SBML Level 1 Version 1}.  (Please note that SBML Level 1
  Version~1 is now deprecated in favor of Version 2; we encourage all new
  developers to use SBML Level 2 or if really necessary Level 1 Version 2.)

\item \href{http://www.sbw-sbml.org/sbml/docs/index.html}{SBML Level 1 Version 2}.

\item \href{http://www.sbw-sbml.org/sbml/docs/index.html}{SBML Level 2}.

\end{itemize}

SBML Level 2 makes use of other standards and specifications.  In
particular, the following are important references:

\begin{itemize}

\item \href{http://www.cellml.org/public/metadata/}{CellML Metadata specification}

\item \href{http://www.w3.org/TR/2003/WD-MathML2-20030411/}{MathML 2.0 specification}

\end{itemize}


\section{The Design of SBML}

\subsection{What is the basic idea behind the SBML units system?}

The idea is that units of every math entity in a model is precisely defined
whilst at the same time allowing for reasonable default unit definition.  The set of math entities includes
variables, parameters and the result of equations.

The motivation for having a units system in SBML is three fold: it allows the semantics
of math entities to be defined precisely, it allows a consistent method for handling multi-compartmental models and in the longer term it will allow
for consistency checking and unit conversions in models that are composed from submodels.

\subsection{Unless a model contains unit attribute values the units used in a model are not well defined?}

In a model that doesn't contain unit attribute values only the units of parameters (and their rules) are undefined.
The units of species, compartments and kinetic laws are well defined via built-in units which have default unit definitions.

\subsection{Allowing default units is a bit of cop out?}

If you read the specification you will see that all math entities apart from parameters have precise units.
The majority of tools offer little or no support for units.  We wouldn't have interoperability
between tools if we imposed a mandatory explicit units system.

\subsection{Not having a precise definition of parameter units is a big hole isn't it?}

In Level 1 and Level 2 the only practical use for these would be for documentary purposes.
We don't and can't expect software to determine and check the effective units of parameters by analyzing
equations.  In addition many tools don't capture the units parameters so we allow parameter
units to be optional.

In Level 3 in it anticipated that checking parameter units will be present
 in systems supporting model composition.  However it's still an
open question to many people whether tools should impose constraints based
on parameter units and/or perform conversions between math entities
as part of the composition process.

\subsection{What are the 'built-in' units exactly?}

There are built-in units for substance, length, area, volume and time.  These units
are, by default, used to form the units of various math entities in a SBML model.
For example a species symbol in a kinetic law equation has substance/volume 
units (assuming that the species' compartment has 3 spatial dimensions).

These built-in units are in turn defaulted to specific units, for example
substance defaults to moles.  An SBML model can explicitly set the underlying
units of a built-in unit, for example, it is possible to redefine substance to be
millimole.

\subsection{What are the benefits of the built-in units?}

The great benefit of the built-in units is that it allows
a model to redefine the underlying units of a whole model
precisely without having to assign units to all math entities explicitly.

The SBML specification ensures that the built-in units operate in a consistent
and reasonable way across math entities.

\subsection{Why are reaction rates in ``substance/time'' units instead of
  ``substance/volume/time'' units?} 

\label{sec:reaction-units}

(In the explanation below, a symbol such as $A$ represents the substance of
a species and $[A]$ represents the concentration of a species.)

In both Level 1 and Level 2, the formula used to define the rate of a given
reaction is defined to be substance/time (moles/second by default).  This
is reasonable since reactions change species amount directly and only
species concentration indirectly.  We are aware that many modelers use
concentration/time or substance/volume/time (moles/liter/second by default)
when defining rate of change formulae for species.  There are reasons for
SBML to adopt substance/time for kinetic laws:
\begin{enumerate}
  
\item Kinetic laws on reactions don't define the rate of change of a single
  species.  In fact the rate of change of species is composed from the rate
  laws of all reactions in which the species is either a reactant or
  product.  The result is a formula that defines a rate of change of amount
  for the species.
  
\item A reaction can have a transport component i.e. move species from one
  compartment to another where each compartment can have a different
  volume.

\end{enumerate}

Consider the situation where there are 3 species $A$, $B$ and $C$ all in
separate compartments with volumes $V_a$, $V_b$, and $V_c$ respectively.
We define a transporting reaction $A \rightarrow B + C$ with a rate law
$k[A]$ .  The rates of change of amount are
\[
  \begin{array}{lll}
    dA/dt & = & -k[A]\\
    dB/dt & = & k[A]\\
    dC/dt & = & k[A]
  \end{array}
\]

and thus the rates of change of concentration are:
\[
  \begin{array}{lll}
    d[A]/dt & = & -k[A]/V_a\\
    d[B]/dt & = & k[A]/V_b\\
    d[C]/dt & = & k[A]/V_c
  \end{array}
\]

One immediate observation of this is if you don't care about volume and
compartments: you just locate everything inside a single unit compartment
and then $[A] = A$, thus returning the math to the conventional
representation.

Let's now consider an alternative definition of rate laws that may seem
rational: the rate law defines the rate of change of concentration of the
species in the reaction.  Thus we have:
\[
  \begin{array}{lll}
    d[A]/dt & = & -k[A]\\
    d[B]/dt & = & k[A]\\
    d[C]/dt & = & k[A]
  \end{array}
\]

Therefore, the rates of change of amount are:
\[
  \begin{array}{lll}
    dA/dt & = & -k[A]V_a\\
    dB/dt & = & k[A]V_b\\
    dC/dt & = & k[A]V_c
  \end{array}
\]

Hopefully you'd agree that this is incorrect---this suggests that you can
increase the rate of increase of amount of $C$ simply by increasing the
volume of $C$'s compartment.

\subsection{How should models without compartments be encoded?}

The models should locate all species in a compartment with unit volume.
The default units system of SBML will ensure that this unit volume
representation is exactly equivalent to a model dealing with concentrations,
including rate laws defined in substance/volume/time units.

\subsection{Why isn't there a default compartment with unit volume?}

There are several reasons:
\begin{enumerate}
  
\item It would be a special case that all SBML parsing programs would have
  to deal with.
  
\item We would have to invent a special reserved name to refer to the
  default compartment.
  
\item It would only save effort on the SBML writing component of a software
  tool.  The writing component is the easy part.
  
\item A model which uses a single unit volume compartment is making explicit an important underlying assumption about the model.
\end{enumerate}


\subsection{Why does SBML include ``low-level'' features such as rules in
  combination with biochemical concepts like reactions and species?} 

The aim of SBML is to enable the construction of quantitative models which
describe both the activity of biochemical networks and interaction of
biochemical networks and other phenomena.  SBML allows the declaration of
variables (non-constant parameters) and associated ODEs and DAEs to
describe these phenomena.  Examples of these phenomena include the
mechanical force generated by muscle cells or the electrical potential
across a synapse.

\subsection{Why is there a distinction between assignment and algebraic rules?
  Aren't they equivalent?}

Although it is typically easy to transform between assignment and algebraic
rules we make the distinction because:
\begin{itemize}

\item Algebraic rules define the point in the model where there is a �loop�
  dependency between variables.  It is not possible to form such a loop in
  scalar rules (see SBML Level 2 specification).
  
\item Many tools are not capable of supporting algebraic rules (DAEs)
  
\item Those tools that do support make the distinction between assignment rules
  and algebraic rules.

\end{itemize}

\subsection{Why can't functions be recursive in Level 2?}

Functions definitions in SBML Level 2 are designed to be substituted in
place of the function call operator, i.e., they operate like macros rather
than functions.  This can't occur if a function is recursive.

\subsection{What on earth are ``events''?}

�Events� are discrete discontinuous events that can be fired in response to
state transitions in the model.


\subsection{Is it possible to represent an entirely
  event-driven, deterministic model in SBML Level~2?} 

Yes, although we are not aware of any simulators that can support that kind
of model or of models of this type.


\subsection{What's the difference between the \textup{\attrib{boundaryCondition}}
  and \textup{\attrib{constant}} attributes on species in Level~2?}

One might expect these attributes to both define that the species doesn't
vary during simulation however there is a difference in their semantics.
The \attrib{constant} attribute indicates that the species concentration
doesn't vary during simulation not matter what reactions it occurs in.
Such a constant species can't have an associated rule.  The
�boundaryCondition� attribute defines only that the species concentration
is not defined by the set of reactions.

\subsection{Why isn't there an explicit definition of SBML in terms of an
  ODE equivalent form?} 

On one hand, we were worried that a focus on the ODE representation would
deter developers that employ other forms, for example stochastic discrete
event simulation, from supporting SBML.

On the other hand, we simply have not had time to work on this area.  We
would welcome volunteers to work on this.


\section{Implementing SBML support}

\subsection{Which level of SBML should I use in my software?}

We recommend SBML Level 2, despite the fact that at the moment more tools
support Level 1, because Level 2 fixes known problems with Level 1 and
Level 2 will be forwards compatible with Level 3.

\subsection{Is it a good idea to use SBML as my software's native model format?}

Depending on the needs of your software, yes, this may be a good idea.
JigCell uses SBML as its native format.

\subsection{What software libraries are available to help me program
  support for SBML in my software?}  

There are two libraries that you could use: libSBML and Marc Vass' SBML Java parser library.

libSBML supports Level 1 will support Level 2 shortly.  libSBML is a native
library and provides C and C++ APIs.  libSBML is available from the
\href{http://sourceforge.net/project/showfiles.php?group_id=71971&release_id=164086}{SBML
  Sourceforge project}.
 
Marc Vass' SBML Java parser library supports level 2 and is avaliable from \url{???}.

\subsection{The unit system seems to imply some implicit unit conversions may be required in various places when parsing a model.
Can you give an overview of what you expect a model parser to do?}

The first step is to establish what the units are of the various entities in your object model.
These units may be generic and thus can be derived from the SBML built-in units.

There are 3 places where unit conversion may have to occur depending on the units used for variables internal to a simulation:

\begin{itemize}

\item \emph{symbols in formulae}

Variables that are used in formulae may require a conversion from the simulation units to the units associated with the SBML symbol
before the SBML formulae are applied to the variables.  

\item \emph{result of formulae}

The result of SBML formulae may require a conversion from SBML units associated with the formulae result to the units associated
with the simulation variables that are assigned the result of formulae.  Particular attention should be placed on the
results of kinetic laws which have substance/time units in SBML (see Section~\ref{sec:reaction-units})

\item \emph{initial conditions}

The initial values of variables may require conversion from the SBML units associated with the symbol to the simulation units of
the variable.

\end{itemize}

\subsection{My software cannot support all the features of SBML---what should I do?}

Don't Panic.  Few if any packages support all aspects of SBML.  
Depending on what you wish do with a model you can either
ignore features that are not relevant to your tool or report errors
when certain features are used.  In most cases a simulator will 
simply report an error if it encounters a feature
that it is unable to support.  Ignoring quantitative aspects of an SBML
model will make it unlikely a user will be able to reproduce 
results seen in the tool which generated the model.

\subsection{What do I do about the fact that SBML does not encode all the
  information that I need to encapsulate in a model?}

You should use \class{Annotation} elements.  These are described in some detail in the Level~1 and Level~2
specification documents.  \class{Annotation} elements can be enclosed within any SBML element and can contain
elements of any namespace. Elements that are not in the SBML namespace should either locally redefine the default namespace
or use namespace prefixes to correctly separate SBML data from application specific data.  Data stored in annotation
elements should not contain data that could be or is encoded in SBML.  

\subsection{How should I structure annotations?}

The annotation data enclosed in a specific SBML element is
assumed by other applications to be directly associated with that specific element.
Therefore it is important to decompose and locate annotation data appropriately in an SBML document.
Avoid, for example, encoding all annotation data in a top level attribute instead encode data associated
with individual species in separate annotation elements enclosed in each SBML species element.

\subsection{What should I do when I encounter an incorrect SBML file or
  stream?} 

Whilst it is not expected that an application will detect all errors in an parsed SBML document it is
expected that a parser will not simply ignore or fudge discrepancies between the document and the relevant
SBML standard.  The parser should report the line number and a description of the error to the user
i.e. a parser should not assume that an incorrect SBML file is an extremely unusual event or that
the details of a parsing error are irrelevant to the user.

If you encounter consistent inconsistencies between the standard specification and documents that
claim to be compliant then please report this to the sbml-discuss mailing list. 

\subsection{There are several different ways in which I could encode my
  models in SBML. Which forms are more interoperable?}
\label{sec:bestpractice}
  
It is true that there are several ways in which to encode a given model in SBML, deciding which form to use in
various contexts is a big subject.  Although this section describes which SBML features are commonly used tools should
support as wide a set of features of SBML when reading SBML as possible.  Here is a rough guide:

\subsubsection{Identifiers}

Although \attrib{id} fields have restricted syntax do assume that tools could potentially display the contents
of this field to users even when \attrib{name} attributes are available.  This means that if you
capture names from a UI that this data is placed in both the \attrib{name} and \attrib{id} fields
(with the \attrib{id} field value perhaps slightly mangled).  Avoid using crude automatically generated values
in the \attrib{id} field.

\subsubsection{Stoichiometry}

Whilst it is possible to encode complex math expressions to specify stoichiometry it is best to restrict these
expressions to being just positive integer values.

\subsubsection{Species initial value}

A species initial value can be specified using \attrib{initialAmount} or \attrib{initialConcentration}.
The attribute most commonly used is \attrib{initialConcentration}.

\subsubsection{Compartment spatial dimensions}

Common practice is to allow the \attrib{spatialDimensions} attribute to default to 3.  In fact the majority of simulators
only support the value 3 for this attribute.

\subsubsection{Species References}

A reaction's product and reactant lists can contain more than one species reference structures referring to the same species.
This is not however good practice for interoperability. 

\subsubsection{Rules verses Reactions}

In all cases where possible a reaction structure is preferable to the equivalent rule structures even if this
means the use of reactions which have no reactants or products.

\subsubsection{Algebraic Rules verses Assignment Rules}

If it is possible to avoid using algebraic rules then do not use them.

\subsubsection{Discontinuities}

Avoid the use of discontinuous operators like \texttt{piecewise}.

\subsubsection{Delay}

Avoid the use of the built-in delay operator.

\subsubsection{Events}

Avoid the use of the events.

\subsubsection{Units}

Do not expect tools to interpret units.  Ensure your tool can parse models which use entirely default units.

\subsection{How much effort should I invest in preserving the SBML form
  when round-tripping models through my software?} 

The first priority should be to support as much of the SBML standard as possible both for reading and writing.
You should write using the most interoperable form as possible as described in Section~\ref{sec:bestpractice}.
To maximize interoperability beyond this requires
trying to include as much of an imported SBML model as possible when rewriting it in SBML.  This includes preserving
annotation data and avoiding mangling \attrib{id} and \attrib{name} fields.  The order of structures, other than rules,
and the white space between elements do not require preservation.

\subsection{How do I interpret SBML for stochastic simulation purposes?} 

Although SBML takes a differential equation approach to describing models
it is possible to parse an SBML document into a stochastic modelling environment.
The most common type of stochastic simulator in use are derivatives of the Gillespie
non-deterministic discrete event algorithm.  This the type assumed in the section.
Ideally the SBML parsing process is performed from a model with most if not all units defaulted.
This requires the species amount in moles or species concentrations in moles per litre to
be translated into number of molecules.  The parameters of simple mass action kinetics can be extracted
from kinetic rate laws of reactions.

\section{Organization}
\label{sec:organization}

\subsection{What is the overall SBML development process?}

All SBML development is and has been motivated and directed by the System
Biology community.  However the process is managed under the communities control by the SBML Editors.
Proposals from working groups and individuals for inclusion of features
into future SBML Levels are collected by the editors.  The editors then establish what the consensus is
among the community about how to proceed with these proposals.  With this information the editors assemble
some of the proposals into an specification of a new SBML level.  After this specification has been reviewed
by the community the specification becomes a new standard.


\subsection{Who are the SBML Editors?}

\href{mailto://afinney@cds.caltech.edu}{Andrew Finney} and \href{mailto://mhucka@cds.caltech.edu}{Mike Hucka}

\subsection{What do SBML Editors do?}

The editors function is as follows:

\begin{itemize}

\item assemble proposals into SBML Level specifications
\item organize SBML forum meetings
\item maintain the \url{http://www.sbml.org} website
\item publicize and document the SBML standard
\item support developers that wish to support SBML
\item support working groups developing proposals

\end{itemize}

\subsection{What are SBML Forum Meetings?}
\label{sec:forums}

These are biannual face-to-face meetings organized by the editors with the title \emph{Workshops on Software Platforms for Systems Biology}.
These are held in early summer and early winter.  The winter meeting is normally held at the
International Conference for Systems Biology.  These meetings allow for significant discussion of new
proposals and interoperability issues.

An archive of information on these meetings is available at
\url{http://www.sbw-sbml.org/workshops/index.html}

\subsection{What are the Workshops on Software Platforms for Systems Biology?}

They are the same as the SBML forum meetings (see Section~\ref{sec:forums}).

\subsection{SBML doesn't encode all the types of representation that are
  being used in Systems Biology modelling.  What is being done to address
  this problem?}
  
\label{sec:workinggroups}

The SBML community has established working groups to develop proposals for
extending SBML in future levels.  The working groups are:

\begin{description}
\item \href{http://www.mpi-magdeburg.mpg.de/zlocal/martins/sbml-comp/}{\emph{The model composition group}}
  Support for composition a model from submodels.

\item \href{http://caboy.uchc.edu/wagner/Science/VirtualCell/SBML-DWG/Default.htm}{\emph{The diagrams/graphical layout group}}
  Support for recording in SBML the graphical network
  diagrams of a model created by many contemporary network-oriented
  modelling tools.

\item \href{http://www.sbw-sbml.org/sbml-discuss/archive/msg00166.html} {\emph{The arrays group}}
  Extension of SBML data structures to permit arrays of items
  (e.g., species, compartments) to be grouped and manipulated \emph{en
    masse}.

\item \href{http://www.sbw-sbml.org/sbml-discuss/archive/msg00174.html} {\emph{The complexes/multistate species group}}
  Support for a compact representation of species
  with multiple possible ``states'' (e.g., due to phosphorylation).

\item \href{http://www.sbw-sbml.org/sbml-discuss/archive/msg00423.html} {\emph{The parameter sets group}}
  A separate standard for defining alternative initial
  values for variables in a model.
  
\item \href{http://www.sbw-sbml.org/sbml-discuss/archive/msg00235.html} {\emph{The alternative reactions group}}
  Extension of SBML reactions to
  support more directly the expression of stochastic
  
\item \emph{The dynamic structures group}
  Support for enabling model structures to vary
  during simulation.  One aspect of aspect of this is allowing rules and
  reactions to have effects that depend on the state of the model system.
  
\item \emph{The spatial features group}
  Support for describing 2-D and 3-D spatial
  characteristics of models, the geometry of compartments, the diffusion
  properties of species, and the specification of different species
  concentrations across different regions of a cell.

%\item \emph{The hybrid group}
%
%\item \emph{The constraints group}

\item \emph{The controlled vocabularies group}
Support for
  indicating that a model, its reactions, or other components of a model
  are meant to be interpreted in certain ways (e.g., to be simulated in a
  stochastic framework).

\end{description}

\subsection{How do members of the working groups communicate?}
Discussions within the group should, at least initially,
   take place on sbml-discuss, rather than on a separate
   mailing list.  This will avoid the need for working group
   organizer(s) to create new archived mailing lists right
   from the start, and will also allow other people to see
   some of the discussions even if they do not want to be
   involved to the level of officially being in the working
   group.  If the traffic for a particular group becomes too
   much for other people on sbml-discuss, that working group
   may create its own archived mailing list at that time.

\subsection{I have identified a feature or features that are missing from
  SBML.  How to do I start a working group to address this issue?}  
Any interested person or group can send mail to
   sbml-discuss to propose forming a working group.  The
   announcement should include the following information:

\begin{itemize}
   \item A short statement of the purpose or goal of the effort
     (typically, an extension to SBML).

   \item The initial contact person for the group.  That person
     may or may not be the eventual chairperson.

   \item The list of the initial members of the group.

   \item A description of the expected form of the outcome from
     the effort.  This will typically be a document
     describing some proposed changes to SBML.

   \item An estimate of how long the group expects to stay in
     existence (meaning, how long it will give itself to
     accomplish its goals before reevaluating its status and
     expectations.)
\end{itemize}

Being a member of a working group implies a desire and
   willingness to work on actual software underlying the
   focus of the group's effort.  Proposals for SBML changes need
   to be accompanied by examples of one or more software
   implementations of the proposed changes.  To make this
   possible, the implementations should add proposed XML
   extensions within the SBML 'annotation' element.  This
   will allow tools to exchange models with the proposed
   extensions in a semi-organized fashion.  Once the
   proposed changes have been debugged to the satisfaction
   of the group(s) using them, the proposal can be
   introduced to the SBML community at large.  When
   accepted, the new extensions will be migrated out of the
   'annotation' element and made a part of SBML.

\subsection{What is the relationship between SBML and CellML?}

CellML is built around an approach of composing systems of equations
by linking together the variables in those equations; this is augmented
by features for declaring biochemical reactions explicitly, as well as
encapsulating arbitrary components into modules.  By contrast, SBML
provides constructs that are
more similar to the internal object models used in many simulation/analysis
packages specialized for biochemical networks.  This reflects its history
of having been developed in cooperation with software developers who write
these tools.  In practice, SBML appears to be better suited to the purpose
of enabling interoperability with existing simulation tools focused on the
biochemical reaction level.  These differences notwithstanding, the SBML and
CellML efforts share much
in common, and the development of SBML has benefited from discussions and
interactions with the developers of CellML.  (the CellML
team has been an active participant in the SBML development process.)  The editors view CellML and
SBML as somewhat different approaches being investigated as solutions to the
same general problems.  One of the aims of the
\href{http://www.mpi-magdeburg.mpg.de/zlocal/martins/sbml-comp/}
{model composition working group}
is to ensure that the two representations can be integrated with each other to create
a single standard in the future.  The result will be compositional compatibility
between CellML and SBML, such that models expressed in one language can be
used as components or submodels in the other.

\subsection{What is the relationship between SBML and BioPAX?}

\href{http://www.biopax.org/}{BioPax} is a consortium that is developing a format for
the exchange of pathway data between bioinformatics databases.  The editors have a good
working relationship with the group doing the majority of the work on this standard:
Chris Sander's group at MSKCC.  The aim is to ensure that large subsets of data represented in
one format can be converted into the other format.
The BioPAX standard is designed to support a large set of biochemical entity types and
types of relationships between these types.  These types will be arranged into ontologies.

\subsection{What is the relationship between SBML and I3C?}

\href{http://www.i3c.org/}{Interoperable Informatics Infrastructure Consortium (I3C)} is an organization 
which promotes the use of standards in the life sciences research software.  The I3C has a 
\href{http://www.i3c.org/wgr/psb/psb.asp}{pathways/systems biology working group}.  Mike Hucka, SBML co-editor,
regularly attends meetings of this working group.  

The I3C doesn't create standards instead it encourages their development and then makes recommendations
on standards to its members.  I3C has a close relationship with OMG.

\subsection{Why isn't SBML being developed under the auspices of a
  standards body like the OMG?} 

In general terms the SBML process meets the requirements of the community it serves.

The editors were considering submitting SBML as a proposal to the
\href{http://www.omg.org/}{Object Management Group (OMG)}
in response to a request for proposals (RFP) for pathways representations.
The community decided at the 7th Forum meeting in May 2003 that whilst it would be useful
to have the endorsement of a standards body like the OMG
our time would be better spent working on standards development rather conforming to all the standards
requirements of the OMG.  The editors will not be submitting a response to the RFP.

\subsection{Which research groups are involved in SBML development?}

All the groups that have produced software list in Section~\ref{sec:sbml-support} are involved
to some extent in the development of SBML.

\section{History}
\label{sec:history}

\subsection{Who proposed the creation of SBML?}

Dr.\ Hamid Bolouri initially suggested it at the First Workshop on Software
Platforms for Systems Biology in April~2000.  The delegates there supported
the idea enthusiastically.

\subsection{Who were the original SBML editors?}

Michael Hucka, Herbert Sauro and Andrew Finney wrote the original SBML
proposal in the second half of the year 2000.

\subsection{What was the organizational context of the original editors?}

This work was initially performed in John Doyle's group by Michael Hucka,
Herbert Sauro and Andrew Finney in the Control and Dynamical Systems
Department (CDS) at the \href{http://www.caltech.edu}{California Institute
  of Technology (Caltech)}.  John Doyle and Hiroaki Kitano were the joint
Principle Investigators.  Hamid Bolouri managed the process until May 2003.

\subsection{Where are they now?}

Hamid Bolouri is Professor of Computational Biology at the
\href{http://www.systemsbiology.org}{Institute of Technology}.  Herbert
Sauro is an Assistant Professor at the \href{http://www.kgi.edu}{Keck
  Graduate Institute}.  Mike Hucka remains in CDS and Caltech.  Andrew
Finney is a Senior Research Fellow at the
\href{http://strc.herts.ac.uk/}{Science and Technology Research Centre at
  the University of Hertfordshire}.  Mike and Andrew are now joint
chairs/editors of the SBML process.

\subsection{Who funded the initial editorial work?}

\href{http://www.jst.go.jp/EN/}{Japan Science And Technology Corporation}'s
\href{http://www.jst.go.jp/erato/}{Exploratory Research for Advanced
  Technology program (JST ERATO)}.

\subsection{Who funded many of the SBML Forum meetings?}

UK's \href{http://www.bbsrc.ac.uk/}{Biotechnology and Biological Sciences
  Research Council}.

\section{Help}

\subsection{My question is not answered in this FAQ list. Who should I contact?}

\href{mailto:afinney\@cds.caltech.edu}{Andrew Finney}
\href{mailto:mhucka\@cds.caltech.edu}{Mike Hucka}


\end{document}
