\documentclass{cekfaq}
\usepackage{amsmath}
\usepackage{array}

\begin{document}

\title{SBML Frequently Asked Questions}

\author{Andrew Finney, Michael Hucka}

\authoremail{\{afinney,mhucka\}@caltech.edu}

\address{Systems Biology Workbench Development Group\\
  ERATO Kitano Symbiotic Systems Project\\
  Control and Dynamical Systems, MC 107-81\\
  California Institute of Technology, Pasadena, CA 91125, USA\\[3pt]
  \url{http://www.sbw-sbml.org/}}

\acknowledge{Principal Investigators: John Doyle and Hiroaki Kitano}

\date{}

\maketitlepage


\section{SBML Introduction and Background}
\label{sec:intro}


\subsection{What is SBML?}

The Systems Biology Markup Language (SBML) is an open standard for encoding
qualitative and quantitative biochemical network models.  The primary
encoding for SBML is in XML.  By a biochemical network we mean a set of
chemical entities linked by the reactions that transform one entity into
another.  SBML can and should be able to represent models describing the
interaction of biochemical networks with other phenomena.  SBML is being
developed by a community of Systems Biology software developers.

\subsection{What is the purpose of SBML?}

SBML is designed to enable the exchange of biochemical network models
between different software packages.  The aim is to (1) enable the enable
the exchange of models between software tools with little or no human
intervention thus allowing the tools to be properly integrated; and (2)
enable these models to be published in electronic form alongside, for
example, peer reviewed journal articles.  The emphasis is on supporting
quantitative models.


\subsection{What can be represented in SBML?}

An SBML model consists of a set of chemical entities linked by reactions
that can transform one entity into another.  Entities are located in one or
more volumetric compartments.  As well as representing biochemical networks
SBML can represent quantitative models of interaction between these
networks and other phenomena.  SBML can describe discrete events that are
triggered by state changes in the modeled system.  The scope of SBML is
constantly evolving through a community led development.

\subsection{Is SBML a database schema?}

Not specifically.  There is no reason why SBML models could not be stored
in a database, nor is there any reason why you could not use SBML as a
schema.  However, this was not the motivation for the creation of SBML.  An
SBML model is meant to encode a consistent view of knowledge of a
biological system.  SBML is not meant to encode a large set of potentially
conflicting knowledge about such a system.

\subsection{What is the SBML home page?}

The home web site for SBML is \url{http://www.sbml.org/}.



\section{SBML Support}
\label{sec:sbml-support}

\subsection{Which applications support SBML?}

The following matrix lists the software tools known to support SBML, along
with information about which Level of SBML they support.  (If you know of
others not listed here, please inform the editors of this FAQ.)

%\begin{table}[h]
%  \begin{tabular}{lll}
%    \toprule
%    \textbf{Application} & \textbf{Level 1?} & \textbf{Level 2?} \\	
%    \midrule
%    BASIS	 & 
%    BioSketchPad &
%    Cellerator	 & 
%    CellDesigner &
%    Cytoscape	 &
%    Dizzy	 &
%    E-Cell	 &
%    ESS		 &
%    Gepasi	 &
%    Jarnac	 &
%    JDesigner	 &
%    JigCell	 &
%    JSim	 &
%    libsbml	 &
%    mathSBML	 &
%    MOMA	 &
%    NetBuilder	 & 
%    sbml.dll	 &
%    SigPath	 &
%    StochSim	 &
%    Virtual Cell &
%    WinSCAMP	 &
%    \bottomrule
%  \end{tabular}
%\end{table}


\subsection{Are there software libraries for programming with SBML?}

The following matrix lists software libraries known to support SBML.  (If
you know of others not listed here, please inform the editors of this
FAQ.)

%\begin{table}[h]
%  \begin{tabular}{lll}
%    \toprule
%    \textbf{Library} & \textbf{Language} & \textbf{Level 1?} & \textbf{Level 2?}
%    JigCell's sbml.jar	& Java		& 		& 
%    libsbml		& C/C++, others	&
%    mathSBML		& Mathematica	&
%    sbml.dll		& Windows	& 
%    \bottomrule
%  \end{tabular}
%\end{table}


% 2003-06-09 <mhucka@caltech.edu>
% It is probably not appropriate to include this item.  We probably
% shouldn't make claims about other people's plans.
%
%\subsection{What groups plan to add support for SBML in the future?}
%Charon
%Copasi
%ProMote/DIVA
%MONOD

\subsection{What consortia are using SBML?}

The following consortia are known to us to be using SBML as their standard
model definition language.  (If you know of others, please inform the
editors of this FAQ.)

\begin{itemize}
\item DARPA BioSPICE
\item International E. coli Alliance (IECA)
\end{itemize}


\subsection{Where can I find examples of SBML models?}

There is a repository of Level 1 models at the following URL: 
\url{http://www.sbw-sbml.org/ModelsWebPages/ModelRepository.htm}



\section{SBML Levels}

\subsection{What are SBML Levels?}

SBML is being developed in a series of levels where each level adds new
features and fixes problems with the previous level.  Initial levels
provide fundamental features that are common to all biochemical network
models.  Subsequent levels add more features that are specific to
particular classes of tools.  Any level can be used as a standard for
interchanging models.

\subsection{What Level are you at now?}

SBML Level 2 Version 1 will be finalized in June 2003.  All significant new
development will be incorporated into Level 3.

\subsection{What are the differences between Levels 1 and 2?}

Level 2 replaces the infix string formula notation in Level 1 with MathML
expressions.  Level 2 incorporates CellML metadata.  Level 2 introduces
�event� structures.  There several small changes introduced in Level 2.
The complete

List of changes is documented in the Level 2 specification.


\subsection{What features are anticipated in Level 3?}

People interested in SBML have organized themselves into a number of
working groups focused on different topics.  The topics of these working
groups give an indication of the features anticipated for SBML Level 3.

\begin{description}
\item[Diagrams]: Support for recording in SBML the graphical network
  diagrams of a model created by many contemporary network-oriented
  modeling tools.

\item[Model Composition]: Support for composition a model from submodels. 

\item[Complex Species]: Support for a compact representation of species
  with multiple possible ``states'' (e.g., due to phosphorylation).

\item[Arrays]: Extension of SBML data structures to permit arrays of items
  (e.g., species, compartments) to be grouped and manipulated \emph{en masse}.
  
\item[Controlled vocabularies for models and reactions]: Support for
  indicating that a model, its reactions, or other components of a model
  are meant to be intrepreted in certain ways (e.g., to be simulated in a
  stochastic framework).

\item[Spatial features]: 

\item[Alternative reaction representations]

\item[Dynamic structures]

\item[Parameter Sets]

\item[Constraints]

\item[Documentation]

\end{description}


\subsection{What Publications Describe SBML?}

There is currently one, published in \href{FIXME}{\emph{Bioinformatics}, 2003,
  vol. 19, no. 4, pp. 524--531}:
\begin{quote}
\emph{The Systems Biology Markup Language (SBML): A medium for
  representation and exchange of biochemical network models}, 

Hucka, M., Finney, A., Sauro, H.~M., Bolouri, H., Doyle, J.~C., Kitano,
H., Arkin, A.~P., Bornstein, B.~J., Bray, D., Cornish-Bowden, A. , Cuellar,
A.~A., Dronov, S., Gilles, E.~D., Ginkel, M., Gor, V., Goryanin, I.~I.,
Hedley, W.~J., Hodgman, T.~C., Hofmeyr, J.-H., Hunter, P.~J., Juty, N.~S.,
Kasberger, J.~L., Kremling, A., Kummer, U., Le Nov\`{e}re, N., Loew,
L.~M., Lucio, D., Mendes, P., Minch, E., Mjolsness, E.~D., Nakayama, Y.,
Nelson, M.~R., Nielsen, P.~F., Sakurada, T., Schaff, J.~C., Shapiro,
B.~E., Shimizu, T.~S., Spence, H.~D., Stelling, J., Takahashi, K.,
Tomita, M., Wagner, J., Wang, J.
\end{quote}

\subsection{What other documentation is there?}

The following are the specification documents for the differents levels of
SBML:

\begin{itemize}
  
\item \href{FIXME}{SBML Level 1 Version 1}.  (Please note that SBML Level 1
  Version~1 is now deprecated in favor of Version 2; we encourage all new
  developers to use SBML Level 2 or Level 1 Version 2.)

\item \href{FIXME}{SBML Level 1 Version 2}.

\item \href{FIXME}{SBML Level 2}.

\end{itemize}

SBML Level 2 makes use of other standards and specifications.  In
particular, the following are important references:

\begin{itemize}

\item \href{FIXME}{CellML Metadata specification}

\item \href{FIXME}{Math 2.0 specification}

\end{itemize}




\section{The Design of SBML}

\subsection{SBML does not enforce units on mathematical entities---does
  that mean units for these entities are not defined?}

No.  MORE HERE

\subsection{Why are the units of reaction rates in terms of ``substance/time''
  as opposed to ``substance/volume/time''?}

(In the explanation below, a symbol such as $A$ represents the substance of
a species and $[A]$ represents the concentration of a species.)

In both Level 1 and Level 2, the formula used to define the rate of a given
reaction is defined to be substance/time (moles/second by default).  This
is reasonable since reactions change species amount directly and only
species concentration indirectly.  We are aware that many modelers use
concentration/time or substance/volume/time (moles/liter/second by default)
when defining rate of change formulae for species.  There are reasons for
SBML to adopt substance/time for kinetic laws:
\begin{enumerate}
  
\item Kinetic laws on reactions don't define the rate of change of a single
  species.  In fact the rate of change of species is composed from the rate
  laws of all reactions in which the species is either a reactant or
  product.  The result is a formula that defines a rate of change of amount
  for the species.
  
\item A reaction can have a transport component i.e. move species from one
  compartment to another where each compartment can have a different
  volume.

\end{enumerate}

Consider the situation where there are 3 species $A$, $B$ and $C$ all in
separate compartments with volumes $V_a$, $V_b$, and $V_c$ respectively.
We define a transporting reaction $A \rightarrow B + C$ with a rate law
$k[A]$ .  The rates of change of amount are
\begin{equation*}
  \begin{array}{lll}
    dA/dt & = & -k[A]\\
    dB/dt & = & k[A]\\
    dC/dt & = & k[A]
  \end{array}
\end{equation*}

and thus the rates of change of concentration are:
\begin{equation*}
  \begin{array}{lll}
    d[A]/dt & = & -k[A]/V_a\\
    d[B]/dt & = & k[A]/V_b\\
    d[C]/dt & = & k[A]/V_c
  \end{array}
\end{equation*}

One immediate observation of this is if you don't care about volume and
compartments: you just locate everything inside a single unit compartment
and then $[A] = A$, thus returning the math to the conventional
representation.

Let's now consider an alternative definition of rate laws that may seem
rational: the rate law defines the rate of change of concentration of the
species in the reaction.  Thus we have:
\begin{equation*}
  \begin{array}{lll}
    d[A]/dt & = & -k[A]\\
    d[B]/dt & = & k[A]\\
    d[C]/dt & = & k[A]
  \end{array}
\end{equation*}

Therefore, the rates of change of amount are:
\begin{equation*}
  \begin{array}{lll}
    dA/dt & = & -k[A]V_a\\
    dB/dt & = & k[A]V_b\\
    dC/dt & = & k[A]V_c
  \end{array}
\end{equation*}

Hopefully you'd agree that this is incorrect---this suggests that you can
increase the rate of increase of amount of $C$ simply by increasing the
volume of $C$'s compartment.

\subsection{Many models deal with species concentrations without an explicit compartment---how should those models be encoded?}

The models should locate all species in a compartment with unit volume.
The default units system of SBML will ensure that this unit volume
representation is exactly equivalent to model dealing with concentrations
including rate laws defined in substance/volume/time units.

\subsection{Why isn't there a default compartment with unit volume?}

There are several reasons:
\begin{enumerate}
  
\item It would be a special case that all SBML parsing programs would have
  to deal with.
  
\item We would have to invent a special reserved name to refer to the
  default compartment.
  
\item It would only save effort on the SBML writing component of a software
  tool.  The writing component is the easy part.
\end{enumerate}


\subsection{Why does SBML include ``low-level'' features like rules in combination with biochemical concepts like reactions and species?}

The aim of SBML is to enable the construction of quantitative models which
describe both the activity of biochemical networks and interaction of
biochemical networks and other phenomena.  SBML allows the declaration of
variables (non-constant parameters) and associated ODEs and DAEs to
describe these phenomena.  Examples of these phenomena include the
mechanical force generated by muscle cells or the electrical potential
across a synapse.

\subsection{Why is there a distinction between scalar and algebraic rules,
  when they appear to be mathematically equivalent?}

Although it is typically easy to transform between scalar and algebraic
rules we make the distinction because:
\begin{itemize}

\item Algebraic rules define the point in the model where there is a �loop�
  dependency between variables.  It is not possible to form such a loop in
  scalar rules (see SBML Level 2 specification).
  
\item Many tools are not capable of supporting algebraic rules (DAEs)
  
\item Those tools that do support make the distinction between scalar rules
  and algebraic rules.

\end{itemize}

\subsection{Why can't functions be recursive in Level 2?}

Functions definitions in SBML Level 2 are designed to be substituted in
place of the function call operator, i.e., they operate like macros rather
than functions.  This can't occur if a function is recursive.

\subsection{What on earth are ``events''?}

�Events� are discrete discontinuous events that can be fired in response to
state transitions in the model.

\subsection{Is it possible to represent an entirely deterministic event-driven model in SBML Level 2?}

Yes, although we are not aware of any simulators that can support that kind
of model or of models of this type.

\subsection{What is the difference, with respect to species, between the
  \attrib{boundaryCondition} and \attrib{constant} attributes in SBML
  Level~2?}

One might expect these attributes to both define that the species doesn't
vary during simulation however there is a difference in their semantics.
The \attrib{constant} attribute indicates that the species concentration
doesn't vary during simulation not matter what reactions it occurs in.
Such a constant species can't have an associated rule.  The
�boundaryCondition� attribute defines only that the species concentration
is not defined by the set of reactions.

\subsection{Why isn't there an explicit definition of SBML in terms of an ODE equivalent form?}

On one hand, we were worried that a focus on the ODE representation would
deter developers that employ other forms, for example stochastic discrete
event simulation, from supporting SBML.

On the other hand, we simply have not had time to work on this area.  We
would welcome volunteers to work on this.


\section{Implementing SBML support}

\subsection{Which SBML Level should I implement?}

We recommend SBML Level 2 despite the fact that at the moment more tools
support Level 1 because Level 2 fixes known problems with Level 1 and
Level2 will be forwards compatible with Level 3.

\subsection{I could use SBML as my software's native model format---is that a good idea?}

Yes.  JigCell uses SBML as its native format. 

\subsection{My software cannot support all the features of SBML---what should I do?}

Don�t Panic.  Few if any packages support all aspects of SBML.  When
parsing SBML simply stop and report an error if you encounter a feature
that your package is unable to support.

\subsection{SBML seems complicated---what software libraries can I use to help me support SBML in my software?}

There are two libraries that you could use: libSBML

\subsection{How much effort should I invest in preserving the SBML form when round tripping models through my model?}

\subsection{SBML does not encode all the information that I need to encapsulate in a model, how do I encode this information?}

\subsection{How should I structure annotations?}

\subsection{What should I do when I encounter an incorrect SBML file or stream?}

\subsection{There are several different ways in which I could encode my models in SBML, what SBML forms are more interoperable?}

\subsection{How do I interpret SBML for stochastic simulation purposes?}


\section{Organization}

\subsection{SBML doesn't encode all the types of representation that are being used in Systems Biology modeling, what is being done to address this problem?}

\subsection{What is the SBML development process?}

\subsection{How do members of the working groups communicate?}

\subsection{I have identified a feature or features that are missing from SBML, how to do I start a working group to address this issue?}

\subsection{What is the relationship between SBML and CellML?}

\subsection{What is the relationship between SBML and BioPAX?}

\subsection{What is the relationship between SBML and I3C?}

\subsection{Why isn't SBML being developed under the auspices of a standards body like the OMG?}

\subsection{Which research groups are involved in SBML development?}



\section{History}

\subsection{Who drew up the original SBML specification?}

\subsection{Who funded the initial development work?}

\subsection{Where are they now?}



\section{Help}

\subsection{My question is not answered by this FAQ who should I contact?}




\end{document}
