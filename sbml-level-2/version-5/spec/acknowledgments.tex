% -*- TeX-master: "sbml-level-2-version-5"; fill-column: 66 -*-
% ----------------------------------------------------------------

\section{Acknowledgments}
\label{sec:acknowledgements}
\label{sec:acknowledgments}

The development of SBML was originally funded entirely by the
Japan Science and Technology Agency (JST) under the ERATO Kitano
Symbiotic Systems Project during the years 2000--2003.  From 2003
to today, general support for development of SBML and associated
software such as libSBML and the SBML Test Suite has been provided
by the National Institute of General Medical Sciences (USA) via
grant numbers GM070923 and GM077671.

We gratefully acknowledge additional sponsorship from the
following funding agencies: the National Human Genome Research
Institute (USA); the International Joint Research Program of NEDO
(Japan); the JST ERATO-SORST Program (Japan); the Japanese
Ministry of Agriculture; the Japanese Ministry of Education,
Culture, Sports, Science and Technology; the BBSRC e-Science
Initiative (UK); the DARPA IPTO Bio-Computation Program (USA); and
the Air Force Office of Scientific Research (USA).

Additional support has been or continues to be provided by the
following institutions: the California Institute of Technology
(USA), EML Research gGmbH (Germany), the European Molecular
Biology Laboratory's European Bioinformatics Institute (UK), the
Molecular Sciences Institute (USA), the University of Heidelberg
(Germany), the University of Hertfordshire (UK), the University of
Newcastle (UK), the Systems Biology Institute (Japan), and the
Virginia Bioinformatics Institute (USA).

SBML was first conceived at the JST/ERATO-sponsored \emph{First
  Workshop on Software Platforms for Systems Biology}, held in
April, 2000, at the California Institute of Technology in
Pasadena, California, USA.  The participants collectively decided
to begin developing a common XML-based declarative language for
representing models.  A draft version of the Systems Biology
Markup Language was developed by the Caltech ERATO team and
delivered to all collaborators in August, 2000.  This draft
version underwent extensive discussion over mailing lists and then
again during the \emph{Second Workshop on Software Platforms for
  Systems Biology} held in Tokyo, Japan, November 2000.  A revised
version of SBML was issued by the Caltech ERATO team in December,
2000, and after further discussions over mailing lists and in
meetings, we produced a specification for SBML
Level~1~\citep{hucka:2001}.

\sbmltwo was conceived at the \emph{5th Workshop on Software
  Platforms for Systems Biology}, held in July 2002, at the
University of Hertfordshire, UK.  The participants collectively
decided to revise the form of SBML in \sbmltwo.  The first draft
of the Level~2 Version~1 document was released in August 2002. The
final set of features in \sbmltwoone was finalized in May 2003 at
the \emph{7th Workshop on Software Platforms for Systems Biology}
in Ft.\ Lauderdale, Florida.

\sbmltwotwo was largely finalized after the 2005 SBML Forum
meeting in Boston and a final document was issued in September
2006.  \sbmltwothree was finalized after the 2006 SBML Forum
meeting in Yokohama, Japan, and the 2007 SBML Hackathon in
Newcastle, UK.  \sbmltwofour was finalized after the 2008 SBML
Forum in G\"{o}teborg, Sweden.  They were developed with
contributions from so many people constituting the worldwide
\emph{SBML Forum} that we regret it has become infeasible to list
individuals by name.  For discussions and help developing SBML,
and for feedback about this specification, we are grateful to
everyone on the
\link{http://sbml.org/Forums}{sbml-discuss@caltech.edu} and
\link{http://sbml.org/Forums}{sbml-interoperability@caltech.edu}
mailing lists, and many other groups and developers at large,
notably the creators of CellML~\citep{hedley:2001b}, the members
of the DARPA Bio-SPICE project, and the authors of all of the
software systems that support SBML.

A guide to software known to support SBML is provided on the
SBML.org website at the following URL:
\link{http://sbml.org/SBML\_Software\_Guide}{http://sbml.org/SBML\_Software\_Guide}.
