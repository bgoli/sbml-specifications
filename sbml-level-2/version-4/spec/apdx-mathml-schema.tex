% -*- TeX-master: "sbml-level-2-version-3"; fill-column: 66 -*-
% $Id$
% $Source$
% ----------------------------------------------------------------

\section{XML Schema for MathML subset}
\label{apdx:mathml-subset-schema}

The following XML schema defines the syntax of the MathML syntax
that is used in SBML Level 2.

\begin{example}\begin{footnotesize}
<?xml version="1.0" encoding="UTF-8"?>
<!--
 * Filename    : sbml-mathml.xsd
 * Description : Schema for the MathML subset used by SBML L2.
 * Author(s)   : Andrew Finney, Michael Hucka
 * Organization: SBML Team <sbml-team@caltech.edu>
 * Revision    : $Id$
 * Source      : $Source$
 *
 * Copyright 2003-2006 California Institute of Technology, the Japan Science
 * and Technology Corporation, and the University of Hertfordshire.
 *
 * This software is licensed according to the terms described in the file
 * named "LICENSE.txt" included with this distribution and available
 * online at http://sbml.org/xml-schemas/LICENSE.txt
 *
 * Summary:
 * 
 * This is a reduced version of the XML Schema for MathML 2.0.  It 
 * corresponds to the subset of MathML 2.0 used in SBML Level 2, and
 * should be used by validating XML parsers instead of the actual
 * MathML XML Schema when validating SBML files.  To accomplish that,
 * changed the value of the attribute 'schemaLocation' in the SBML
 * XML Schema file (sbml.xsd) to refer to a copy of this Schema file
 * on your computer's local hard disk.
-->
<xs:schema xmlns="http://www.w3.org/1998/Math/MathML" 
           xmlns:xsi="http://www.w3.org/2001/XMLSchema-instance" 
           xmlns:xs="http://www.w3.org/2001/XMLSchema" 
           targetNamespace="http://www.w3.org/1998/Math/MathML" 
           elementFormDefault="qualified" 
           attributeFormDefault="unqualified">
    <xs:attributeGroup name="MathAttributes">
        <xs:attribute name="class" type="xs:NMTOKENS" use="optional"/>
        <xs:attribute name="style" type="xs:string" use="optional"/>
        <xs:attribute name="id" type="xs:ID" use="optional"/>
    </xs:attributeGroup>
    <xs:complexType name="MathBase">
        <xs:attributeGroup ref="MathAttributes"/>
    </xs:complexType>
    <xs:attributeGroup name="CnAttributes">
        <xs:attribute name="type">
            <xs:simpleType>
                <xs:restriction base="xs:NMTOKEN">
                    <xs:enumeration value="e-notation"/>
                    <xs:enumeration value="integer"/>
                    <xs:enumeration value="rational"/>
                    <xs:enumeration value="real"/>
                </xs:restriction>
            </xs:simpleType>
        </xs:attribute>
        <xs:attributeGroup ref="MathAttributes"/>
    </xs:attributeGroup>
    <xs:complexType name="SepType"/>
    <xs:complexType name="Cn" mixed="true">
        <xs:choice minOccurs="0">
            <xs:element name="sep" type="SepType"/>
        </xs:choice>
        <xs:attributeGroup ref="CnAttributes"/>
    </xs:complexType>
    <xs:complexType name="Ci">
        <xs:simpleContent>
            <xs:extension base="xs:string">
                <xs:attributeGroup ref="MathAttributes"/>
            </xs:extension>
        </xs:simpleContent>
    </xs:complexType>
    <xs:simpleType name="CsymbolURI">
        <xs:restriction base="xs:string">
            <xs:enumeration value="http://www.sbml.org/sbml/symbols/time"/>
            <xs:enumeration value="http://www.sbml.org/sbml/symbols/delay"/>
        </xs:restriction>
    </xs:simpleType>
    <xs:complexType name="Csymbol">
        <xs:simpleContent>
            <xs:extension base="xs:string">
                <xs:attribute name="encoding" use="required" fixed="text"/>
                <xs:attribute name="definitionURL" type="CsymbolURI" use="required"/>
                <xs:attributeGroup ref="MathAttributes"/>
            </xs:extension>
        </xs:simpleContent>
    </xs:complexType>
    <xs:complexType name="NodeContainer">
        <xs:complexContent>
            <xs:extension base="MathBase">
                <xs:group ref="Node"/>
            </xs:extension>
        </xs:complexContent>
    </xs:complexType>
    <xs:complexType name="Apply">
        <xs:complexContent>
            <xs:extension base="MathBase">
                <xs:sequence>
                    <xs:choice>
                        <xs:element name="ci" type="Ci"/>
                        <xs:element name="csymbol" type="Csymbol"/>
                        <xs:element name="eq" type="MathBase"/>
                        <xs:element name="neq" type="MathBase"/>
                        <xs:element name="gt" type="MathBase"/>
                        <xs:element name="lt" type="MathBase"/>
                        <xs:element name="geq" type="MathBase"/>
                        <xs:element name="leq" type="MathBase"/>
                        <xs:element name="plus" type="MathBase"/>
                        <xs:element name="minus" type="MathBase"/>
                        <xs:element name="times" type="MathBase"/>
                        <xs:element name="divide" type="MathBase"/>
                        <xs:element name="power" type="MathBase"/>
                        <xs:sequence>
                            <xs:element name="root" type="MathBase"/>
                            <xs:element name="degree" type="NodeContainer" minOccurs="0"/>
                        </xs:sequence>
                        <xs:element name="abs" type="MathBase"/>
                        <xs:element name="exp" type="MathBase"/>
                        <xs:element name="ln" type="MathBase"/>
                        <xs:sequence>
                            <xs:element name="log" type="MathBase"/>
                            <xs:element name="logbase" type="NodeContainer" minOccurs="0"/>
                        </xs:sequence>
                        <xs:element name="floor" type="MathBase"/>
                        <xs:element name="ceiling" type="MathBase"/>
                        <xs:element name="factorial" type="MathBase"/>
                        <xs:element name="and" type="MathBase"/>
                        <xs:element name="or" type="MathBase"/>
                        <xs:element name="xor" type="MathBase"/>
                        <xs:element name="not" type="MathBase"/>
                        <xs:element name="sin" type="MathBase"/>
                        <xs:element name="cos" type="MathBase"/>
                        <xs:element name="tan" type="MathBase"/>
                        <xs:element name="sec" type="MathBase"/>
                        <xs:element name="csc" type="MathBase"/>
                        <xs:element name="cot" type="MathBase"/>
                        <xs:element name="sinh" type="MathBase"/>
                        <xs:element name="cosh" type="MathBase"/>
                        <xs:element name="tanh" type="MathBase"/>
                        <xs:element name="sech" type="MathBase"/>
                        <xs:element name="csch" type="MathBase"/>
                        <xs:element name="coth" type="MathBase"/>
                        <xs:element name="arcsin" type="MathBase"/>
                        <xs:element name="arccos" type="MathBase"/>
                        <xs:element name="arctan" type="MathBase"/>
                        <xs:element name="arcsec" type="MathBase"/>
                        <xs:element name="arccsc" type="MathBase"/>
                        <xs:element name="arccot" type="MathBase"/>
                        <xs:element name="arcsinh" type="MathBase"/>
                        <xs:element name="arccosh" type="MathBase"/>
                        <xs:element name="arctanh" type="MathBase"/>
                        <xs:element name="arcsech" type="MathBase"/>
                        <xs:element name="arccsch" type="MathBase"/>
                        <xs:element name="arccoth" type="MathBase"/>
                    </xs:choice>
                    <xs:group ref="Node" maxOccurs="unbounded"/>
                </xs:sequence>
            </xs:extension>
        </xs:complexContent>
    </xs:complexType>
    <xs:complexType name="Piece">
        <xs:complexContent>
            <xs:extension base="MathBase">
                <xs:group ref="Node" minOccurs="2" maxOccurs="2"/>
            </xs:extension>
        </xs:complexContent>
    </xs:complexType>
    <xs:complexType name="Otherwise">
        <xs:complexContent>
            <xs:extension base="MathBase">
                <xs:group ref="Node"/>
            </xs:extension>
        </xs:complexContent>
    </xs:complexType>
    <xs:complexType name="Piecewise">
        <xs:complexContent>
            <xs:extension base="MathBase">
                <xs:sequence>
                    <xs:element name="piece" type="Piece" minOccurs="0" maxOccurs="unbounded"/>
                    <xs:element name="otherwise" type="Otherwise" minOccurs="0" maxOccurs="1"/>
                </xs:sequence>
            </xs:extension>
        </xs:complexContent>
    </xs:complexType>
    <xs:attributeGroup name="AnnotationAttributes">
        <xs:attributeGroup ref="MathAttributes"/>
        <xs:attribute name="encoding" type="xs:string" use="required"/>
    </xs:attributeGroup>
    <xs:complexType name="Annotation">
        <xs:simpleContent>
            <xs:extension base="xs:string">
                <xs:attributeGroup ref="AnnotationAttributes"/>
            </xs:extension>
        </xs:simpleContent>
    </xs:complexType>
    <xs:complexType name="Annotation-xml">
        <xs:sequence maxOccurs="unbounded">
            <xs:any processContents="skip"/>
        </xs:sequence>
        <xs:attributeGroup ref="AnnotationAttributes"/>
    </xs:complexType>
    <xs:complexType name="Semantics">
        <xs:complexContent>
            <xs:extension base="MathBase">
                <xs:sequence>
                    <xs:group ref="Node"/>
                    <xs:sequence maxOccurs="unbounded">
                        <xs:choice>
                            <xs:element name="annotation" type="Annotation"/>
                            <xs:element name="annotation-xml" type="Annotation-xml"/>
                        </xs:choice>
                    </xs:sequence>
                </xs:sequence>
                \changed{<xs:attribute name="definitionURL" type="xs:anyURI" use="optional"/>}
            </xs:extension>
        </xs:complexContent>
    </xs:complexType>
    <xs:group name="Node">
        <xs:choice>
            <xs:element name="apply" type="Apply"/>
            <xs:element name="cn" type="Cn"/>
            <xs:element name="ci" type="Ci"/>
            <xs:element name="csymbol" type="Csymbol"/>
            <xs:element name="true" type="MathBase"/>
            <xs:element name="false" type="MathBase"/>
            <xs:element name="notanumber" type="MathBase"/>
            <xs:element name="pi" type="MathBase"/>
            <xs:element name="infinity" type="MathBase"/>
            <xs:element name="exponentiale" type="MathBase"/>
            <xs:element name="semantics" type="Semantics"/>
            <xs:element name="piecewise" type="Piecewise"/>
        </xs:choice>
    </xs:group>
    <xs:complexType name="Bvar">
        <xs:complexContent>
            <xs:extension base="MathBase">
                <xs:sequence>
                    <xs:element name="ci" type="Ci"/>
                </xs:sequence>
            </xs:extension>
        </xs:complexContent>
    </xs:complexType>
    <xs:complexType name="Lambda">
        <xs:complexContent>
            <xs:extension base="MathBase">
                <xs:sequence>
                    <xs:element name="bvar" type="Bvar" minOccurs="0" maxOccurs="unbounded"/>
                    <xs:group ref="Node"/>
                </xs:sequence>
            </xs:extension>
        </xs:complexContent>
    </xs:complexType>
    <xs:complexType name="Math">
        <xs:complexContent>
            <xs:extension base="MathBase">
                <xs:choice>
                    <xs:group ref="Node"/>
                    <xs:element name="lambda" type="Lambda"/>
                </xs:choice>
            </xs:extension>
        </xs:complexContent>
    </xs:complexType>
    <xs:element name="math" type="Math">
        <!--This is the top-level element for a 'math' container in SBML.-->
    </xs:element>
</xs:schema>
\end{footnotesize}
\end{example}
