% -*- TeX-master: "sbml-level-2-version-3"; fill-column: 66 -*-
% $Id$
% $Source$
% ----------------------------------------------------------------

\begin{blockChanged}

\renewcommand{\thesection}{\changed{\Alph{section}}}
\section{\changed{Major changes between versions of SBML Level~2 and implications for backward compatibility}}
\label{apdx:changes}
\renewcommand{\thesection}{\Alph{section}}

In Section~\ref{sec:deprecated-features}, we summarized the
language features removed and deprecated in \sbmltwothree.  Here
we describe the changes in more detail.  We describe the
cumulative changes introduced since \sbmltwoone.


\subsection{Between Version 2 and Version 1}

The following features were removed between \sbmltwoone and
Version~2:
\begin{itemize}
  
\item The \token{offset} field on \UnitDefinition.  (See
  Section~\ref{sec:unit-structure}.)  The definition of offsets in
  \sbmltwoone was in fact incorrect; moreover, a proper
  implementation would have required a complete change in the SBML
  unit scheme.  Few models appeared to use offsets on unit
  definitions, so the impact of this change on models is expected
  to be small.
  
\item The \val{Celsius} predefined unit.  (See
  Section~\ref{sec:unit-structure}.)  The removal of offsets on
  unit definitions meant an inconsistency existed if the Celsius
  predefined unit was left in the system.  Removing Celsius
  removes the inconsistency.  Alternative ways of using Celsius
  are discussed in Section~\ref{sec:unit-structure}.
  
\item The \token{substanceUnit} and \token{timeUnits} fields on
  \KineticLaw.  (See Section~\ref{subsec:kinetic-law}.)  The
  ability to redefine the substance units on each reaction
  separately, coupled with other features in \sbmltwoone, created
  the opportunity for defining a valid system of reactions which
  potentially could not be combined into a consistent system of
  equations without external knowledge.

\end{itemize}

The following features were deprecated between \sbmltwoone and
Version~2:
\begin{itemize}
  
\item The \token{charge} field on \Species.  (See
  Section~\ref{sec:charge}.)  This field does not appear to be
  supported by any existing software, and moreover, since its
  value cannot be accessed from mathematical formulas in SBML, the
  impact of this change is expected to be small.

\end{itemize}

The following additional changes were made between \sbmltwoone and
Version~2:
\begin{itemize}

\item \sbmltwoone did not clearly specify the value space of
  integer and floating-point numbers permitted in the MathML
  expressions in SBML; moreover, it used the XML Schema type
  \val{integer} instead of \sbmltwotwo's \val{int}.  Although
  extremely unlikely, some previously valid \sbmltwoone documents
  \emph{may} not be valid in Version~2 and Version~3 as a result
  of these changes.  See Sections~\ref{sec:integer},
  \ref{sec:double} and~\ref{sec:cn-token} for more information.

\item \sbmltwoone did not define a default value for the field
  \token{fast} on \Reaction.  \sbmltwotwo introduced a default
  value, and the value is \val{false}.  Further, SBML now requires
  that software tools \emph{must} respect the value or indicate to
  the user that they do not have the capacity to do so.  See
  Section~\ref{sec:fast}.
  
\item As of \sbmltwotwo, SBML is somewhat stricter about how the
  content of \token{annotation} elements must be organized and
  written..  Previously valid \sbmltwoone documents \emph{may}
  need changes to their \token{annotation} elements to comply with
  the specification beginning with Version~2 and Version~3.  See
  Section~\ref{sec:annotation-use} for more details.
  
\item As of \sbmltwotwo, SBML is slightly stricter about how the
  content of \token{notes} elements must be organized.  Previously
  valid \sbmltwoone documents \emph{may} need changes to their
  \token{notes} elements to comply with the specification
  beginning with Version~2 and Version~3.  See
  Section~\ref{sec:notes} for more details.
  
\item \sbmltwotwo corrected numerous errata and ambiguities
  discovered in \sbmltwoone.  These errata are listed on the
  project web site at \url{http://sbml.org}.  As a result of
  changes to \sbmltwo implied by these errata, some existing
  \sbmltwoone models, even when modified as explained above, may
  still not be compliant with Version~2 or Version~3.  The
  ultimate impact of the changes depends on the specific features
  used by a given model and the assumptions under which the model
  was created.

\end{itemize}




\subsection{Between Version 3 and Version 2}

The following features were removed between \sbmltwotwo and
Version~3:
\begin{itemize}

\item The \token{spatialSizeUnits} field on \Species.  (See
  Section~\ref{sec:species-units}.)  This field introduced an
  implicit unit conversion between the spatial units used in
  defining the quantity of a species and the size of the
  compartment in which the species was located.  Moreover, the
  semantics of \token{spatialSizeUnits} were confusing and
  required complicated unit conversions to be written explicitly
  into reaction rate expressions by either the modeler or the
  software.  Although the conversions could be worked out
  unambiguously, the potential for error was judged to exceed by
  far the utility of this feature.

\item The \token{timeUnits} field \Event.  (See
  Section~\ref{sec:events}.)  This field was judged to add
  needless complexity and inconsistency.  For instance, the
  ability to change the time units of the delay of an \Event to be
  different from the units of time for the whole model meant that
  computing an \Event's time of triggering and its delay might
  have to be done using two different sets of units.  The ability
  to redefine the units of time for the delay of an \Event was
  also inconsistent with the lack of such a field on other SBML
  components involving an element of time; for example, \RateRule,
  and now \KineticLaw, have no such fields.

\end{itemize}

The following additional changes were made between \sbmltwotwo and
Version~2:
\begin{itemize}

\item The definition of the XML type \primtype{ID} was incorrectly
  given in the \sbmltwotwo specification.  This type is used as
  the type of the attribute \token{metaid} on \SBase.  The error
  in the definition of \primtype{ID} was such that the type did
  not include the colon (\token{:}) character and all Unicode
  characters actually permitted in XML \primtype{ID}.  This change
  is therefore entirely backward compatible: all models with valid
  \token{metaid} values valid prior to \sbmltwothree are still
  valid under the new definition.

\item The SBML specifications prior to \sbmltwothree did not
  indicate what units are assumed for literal numbers appearing in
  MathML expressions (\ie numbers inside MathML's \token{cn}
  elements).  \sbmltwothree stipulates that there are no units
  associated with numbers (not even \val{dimensionless}), and
  provides suggestions for how to associate units with numbers
  (Section~\ref{sec:units-of-mathml}).

\item The SBML specifications prior to \sbmltwothree did not make
  clear what units are required by the arguments to various MathML
  operators and other constructs.  \sbmltwothree clarifies this
  (Section~\ref{sec:operator-arg-types}).

\item The \primtype{UnitKind} enumeration previously defined in
  the context of \Unit and \UnitDefinition has been eliminated in
  favor of simply defining the symbols as reserved words in the
  value space of \primtype{UnitSId}.  This has no effect on
  written models and is completely backward compatible.  It was
  done to resolve the problem that, previously, the values in
  \primtype{UnitKind} were technically inaccessible from fields
  whose data type was \primtype{UnitSId}.

\item The SBML specification did not point out that the value
  space of the data type \primtype{boolean} is different in
  \mathmltwo and \xmlschemaone.  This means that the permitted
  values of attributes on SBML objects is different from the
  values permitted in MathML formulas.  (Section~\ref{sec:boolean}
  explains the difference.)

\item The SBML specifications were inconsistent about the
  permitted number of items inside
  \token{listOf\rule{0.5in}{0.5pt}} lists: the text portion of the
  specifications claimed the lists could have zero length, but the
  XML Schema definition for SBML required one or more items.  As
  of \sbmltwothree, the specification is consistent on requiring
  one or more items inside these lists.

\item The SBO term hierarchy (Section~\ref{sec:sbo}) has grown in
  the time intervening between \sbmltwotwo and Version~3, and the
  mapping of terms between SBO and SBML components was revised as
  a result of community discussions during the 2006 SBML Forum
  meeting.

\item A number of validation rules have been introduced; some were
  missing from previous specifications, and some were added to
  cover  changes introduced in \sbmltwothree (for example, for
  validation SBO terms assigned to various SBML model components).

\item The \token{sboTerm} field, introduced on many components in
  \sbmltwotwo, has been moved to \SBase as an optional attribute
  and removed from the individual components.  The result is that
  all model components may now have SBO terms associated with
  them.  This change is completely backward compatible.

\item The SBML specifications prior to \sbmltwothree did not
  adequately explain the assumptions regarding XML namespace
  declarations with the \token{annotation} element on SBML
  components.  \sbmltwothree makes these assumptions more
  explicit, include the assumption that applications may not
  preserve another applications annotations unless those
  annotations are self-contained with the XML namespace
  declaration.  See Section~\ref{sec:annotation-use} for more
  details.


\end{itemize}








\end{blockChanged}
