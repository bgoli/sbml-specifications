% -*- TeX-master: "sbml-level-2-version-3"; fill-column: 66 -*-
% $Id$
% $Source$
% ----------------------------------------------------------------

\section{Introduction}
\label{sec:introduction}

We present the \textbf{S}ystems \textbf{B}iology \textbf{M}arkup
\textbf{L}anguage (SBML) \changed{\sbmltwothree}, a model
representation formalism for systems biology.  SBML is oriented
towards describing systems of biochemical reactions of the sort
common in research on a number of topics, including cell signaling
pathways, metabolic pathways, biochemical reactions, gene
regulation, and many others.  SBML is defined in a neutral fashion
with respect to programming languages and software encoding;
however, it is primarily oriented towards allowing models to be
encoded using XML, the eXtensible Markup
Language~\citep{bosak:1999,bray:2000}.  This document contains
many examples of SBML models written in XML, as well as the text
of an XML Schema~\citep{biron:2000,fallside:2000,thompson:2000}
that defines \changed{\sbmltwothree}.  A downloadable copy of the XML Schema
and other related documents and software are also available from
the SBML project web site, \url{http://sbml.org/}.

The SBML project is not an attempt to define a universal language
for representing quantitative models.  The rapidly evolving views
of biological function, coupled with the vigorous rates at which
new computational techniques and individual tools are being
developed today, are incompatible with a one-size-fits-all idea of
a universal language. A more realistic alternative is to
acknowledge the diversity of approaches and methods being explored
by different software tool developers, and seek a common
intermediate format---a \emph{lingua franca}---enabling
communication of the most essential aspects of the models.

The definition of the model description language presented here
does not specify \emph{how} programs should communicate or
read/write SBML.  We assume that for a simulation program to
communicate a model encoded in SBML, the program will have to
translate its internal data structures to and from SBML, use a
suitable transmission medium and protocol, etc., but these issues
are outside of the scope of this document.

%-----------------------------------------------------------------------------
\subsection{Developments, discussions, and notifications of updates}
%-----------------------------------------------------------------------------

% [MH 2006-03-06] This should still be changed to mention sbml-standard or
% whatever we use in the end, if we can decide in time for this spec.

SBML has been, and continues to be, developed in collaboration
with an international community of researchers and software
developers.  As in many projects, the primary mode of interaction
between members is electronic mail.  Discussions about SBML take
place on the mailing list
\link{http://sbml.org/forums}{sbml-discuss@caltech.edu}.  The
mailing list archives and a web browser-based interface to the
list are available at \url{http://sbml.org/forums/}.

Beginning with the release of \sbmltwotwo, a new low-volume,
broadcast-only mailing list is available where notifications of
revisions to the SBML specification, notices of votes on SBML
technical issues, and other critical matters are announced.  This
list is \link{http://sbml.org/forums}{sbml-announce@caltech.edu}
and anyone may subscribe to it freely.  This list will never be
used for advertising and its membership list will never be
disclosed.  \emph{It is vitally important that all users of SBML
  stay informed about revisions and other developments by
  subscribing to this list}, even if they do not wish to
participate in discussions on the
\link{http://sbml.org/forums}{sbml-discuss@caltech.edu} list.
Please visit the SBML project web site, \url{http://sbml.org/},
for information about how to subscribe to
\link{http://sbml.org/forums}{sbml-announce@caltech.edu} as well
as for access to the list archives.

In Section~\ref{sec:acknowledgements}, we attempt to acknowledge
as many contributors to SBML's development as we can, but as SBML
evolves, it becomes increasingly difficult to detail the
individual contributions on a project that has truly become an
international community effort.


%-----------------------------------------------------------------------------
\subsection{SBML Levels, Versions, and Revisions}
\label{sec:levels-versions-revisions}
%-----------------------------------------------------------------------------

Major releases of SBML are termed \emph{levels} and represent
substantial changes to the composition and structure of the
language.  The release of SBML defined in this document, \sbmltwo,
represents an incremental evolution of the language resulting from
the practical experiences of many users and developers working
with \sbmlone since since its introduction in the year
2001~\citep{hucka:2001,hucka:2003}.  All of the constructs of
Level~1 can be mapped to Level~2.  In addition, a subset of the
constructs in Level~2 can be mapped to Level~1.  However, the
levels remain distinct; a valid SBML Level~1 document is not a
valid SBML Level~2 document, and likewise, a valid SBML Level~2
document is not a valid SBML Level~1 document.

Minor releases of SBML are termed \emph{versions} and constitute
changes within an SBML Level to correct, adjust, and refine
language features.  The present document defines \changed{SBML \thisLV}.
Differences in the definitions of data structures between
Version~3 and Version~2 are highlighted in red in the diagrams of
this document; a separate document provides a detailed listing of
the changes between these versions of \sbmltwo as well as between
\changed{\sbmltwothree} and \sbmlonetwo.

Specification documents inevitably require minor editorial changes
as its users discover errors and ambiguities.  As a practical
reality, these discoveries occur over time.  In the context of
SBML, such problems are formally announced publicly as
\emph{errata} in a given specification document.  Borrowing
concepts from the World Wide Web Consortium~\citep{jacobs:2004},
we define SBML errata as changes of the following types: (a)
formatting changes that do not result in changes to textual
content; (b) corrections that do not affect conformance of
software implementing support for a given combination of SBML
Level and Version; and (c) corrections that \emph{may} affect such
software conformance, but add no new language features.  A change
that affects conformance is one that either turns conforming data,
processors, or other conforming software into non-conforming
software, or turns non-conforming software into conforming
software, or clears up an ambiguity or insufficiently documented
part of the specification in such a way that software whose
conformance was once unclear now becomes clearly conforming or
non-conforming~\citep{jacobs:2004}.  In short, errata do not
change the fundamental semantics or syntax of SBML; they clarify
and disambiguate the specification and correct errors.  (New
syntax and semantics are only introduced in SBML Versions and
Levels.)

SBML errata result in new \emph{Revisions} of the SBML
specification.  Each revision is numbered with an integer, with
the first release of the specification being given the revision
number~1.  Subsequent revisions of an SBML specification document
contain a section listing the accumulated errata issued since the
first revision.  A complete list of the errata for \changed{\sbmltwothree}
since the publication of Revision~1 is also made publicly
available at
\url{http://sbml.org/specifications/sbml-level-2/version-\sbmlversion/errata/}.
Announcements of errata and revisions to the SBML specification,
as well as releases of new SBML Levels, are made on the
\link{http://sbml.org/forums}{sbml-announce@caltech.edu} mailing
list.


%-----------------------------------------------------------------------------
\subsection{Language features and backward compatibility}
\label{sec:deprecated-features}
%-----------------------------------------------------------------------------

Some language features of previous SBML Levels and Versions have
been either deprecated or removed entirely in \changed{\sbmltwothree}.  For
the purposes of SBML specifications, the following are the
definitions of \emph{deprecated feature} and \emph{removed
  feature}:
\begin{description}
  
\item \emph{Removed language feature}: A syntactic construct that
  was present in previous SBML Levels and/or Versions within a
  Level, and has been removed beginning with a specific SBML Level
  and Version.  Models containing such constructs do not conform
  to the specification of that SBML Level and Version.
  
\item \emph{Deprecated language feature}: A syntactic construct
  that was present in previous SBML Levels and/or Versions within
  a Level, and while still present in the language definition, has
  been identified as non-essential and planned for future removal.
  Beginning with the Level and Version in which a given feature is
  deprecated, software tools should not generate SBML models
  containing the deprecated feature; however, for backward
  compatibility, software tools reading SBML should support the
  feature until it is actually removed.

\end{description}

As a matter of SBML design philosophy, the preferred approach to
removing features is by deprecating them if possible.  Immediate
removal of SBML features is not done unless serious problems have
been discovered involving those features, and keeping them would
create logical inconsistencies or extremely difficult-to-resolve
problems.  The deprecation or outright removal of features in a
language, whether SBML or other, can have significant impact on
backwards compatibility.  Such changes are also inevitable over
the course of a language's evolution.  SBML must by necessity
continue evolving through the experiences of its users and
implementors.  Eventually, some features will be deemed unhelpful
despite the best intentions of the language editors to design a
timeless language.

Throughout the SBML specification, removed features are discussed
in the text of the sections where the features previously appeared.
The features removed in \changed{\sbmltwothree} are as follows:
\begin{itemize}
  
\item The \token{offset} field on \UnitDefinition.  (See
  Section~\ref{sec:unit-structure}.)  The definition of offsets in
  \sbmltwoone was in fact incorrect; moreover, a proper
  implementation would have required a complete change in the SBML
  unit scheme.  Few models appeared to use offsets on unit
  definitions, so the impact of this change on models is expected
  to be small.
  
\item The \val{Celsius} predefined unit.  (See
  Section~\ref{sec:unit-structure}.)  The removal of offsets on
  unit definitions meant an inconsistency existed if the Celsius
  predefined unit was left in the system.  Removing Celsius
  removes the inconsistency.  Alternative ways of using Celsius
  are discussed in Section~\ref{sec:unit-structure}.
  
\item The \token{substanceUnit} and \token{timeUnits} fields on
  \KineticLaw.  (See Section~\ref{subsec:kinetic-law}.)  The
  ability to redefine the substance units on each reaction
  separately, coupled with other features in \sbmltwotwo, created
  the opportunity for defining a valid system of reactions which
  potentially could not be combined into a consistent system of
  equations without external knowledge.

\item The \token{timeUnits} field \Event.  (See
  Section~\ref{sec:events}.)  This field was judged to add
  needless complexity and inconsistency.  For instance, the
  ability to change the time units of the delay of an \Event to be
  different from the units of time for the whole model meant that
  computing an \Event's time of triggering and its delay might
  have to be done using two different sets of units.  The ability
  to redefine the units of time for the delay of an \Event was
  also inconsistent with the lack of such a field on other SBML
  components involving an element of time; for example, \RateRule,
  and now \KineticLaw, have no such fields.

\end{itemize}
Throughout the SBML specification, deprecated features are
explicitly indicated in the definitions of the constructs
affected.  The features deprecated in \sbmltwotwo are as follows:
\begin{itemize}
  
\item The \token{charge} field on \Species.  (See
  Section~\ref{sec:charge}.)  This field does not appear to be
  supported by any existing software\changed{, and moreover, since
    its value cannot be accessed from mathematical formulas in }
  \changed{SBML anyway}, the impact of this change is expected to
  be small.

\begin{blockChanged}

\item The \token{spatialSizeUnits} field on \Species.
    (See Section~\ref{sec:species-units}.)  This field introduces
    an implicit unit conversion between the spatial units used in
    defining the quantity of a species and the size of the
    compartment in which the species is located.  Although the
    unit conversions can be worked out unambiguously, the feature
    adds confusion and makes software support more complex, yet
    offers few if any benefits.

\end{blockChanged}

\end{itemize}

Despite these changes, \sbmltwotwo is designed to be maximally
backward compatible with \sbmltwoone.  An XML document defining a
valid model in \sbmltwoone, after changing the XML namespace and
\token{version} attribute values on the \token{sbml} container
element (see Section~\ref{sec:sbml}), can become a valid
\sbmltwotwo document, subject to the following provisions:
\begin{enumerate}
  
\item Any uses of the field \token{offset} on \UnitDefinition must
  be removed and the unit definitions or the model changed
  appropriately to account for this data structure difference.
  See Section~\ref{sec:unitdefinitions}.
  
\item Any references to the previously predefined unit
  \val{Celsius} must be removed and unit definitions or the model
  changed as needed to incorporate conversion between Celsius
  and kelvin degrees.  (The latter is predefined in SBML.)  See
  Section~\ref{sec:unitdefinitions}.
  
\item Any references to the previously defined fields
  \token{substanceUnits} and \token{timeUnits} on \KineticLaw
  must be removed and the model rewritten to incorporate the
  necessary unit conversions in some other fashion.  See
  Section~\ref{subsec:kinetic-law}.
  
\item \sbmltwoone did not clearly specify the value space of
  integer and floating-point numbers permitted in the MathML
  expressions in SBML; moreover, it used the XML Schema type
  \val{integer} instead of \sbmltwotwo's \val{int}.  Although
  extremely unlikely, some previously valid \sbmltwoone documents
  \emph{may} not be valid in Version~2 as a result of these
  changes.  See Sections~\ref{sec:integer}, \ref{sec:double}
  and~\ref{sec:cn-token} for more information.

\item \sbmltwoone did not define a default value for the field
  \token{fast} on \Reaction.  In \sbmltwotwo, a default value
  \emph{is} defined, and the value is \val{false}.  Further,
  software tools \emph{must} respect the value or indicate to the
  user that they do not have the capacity to do so.  See
  Section~\ref{sec:fast}.
  
\item \sbmltwotwo is somewhat stricter about how the content of
  \token{annotation} elements must be organized and written..
  Previously valid \sbmltwoone documents \emph{may} need changes
  to their \token{annotation} elements to comply with the new
  specification.  See Section~\ref{sec:annotation-use} for more
  details.
  
\item \sbmltwotwo is slightly stricter about how the content of
  \token{notes} elements must be organized.  Previously valid
  \sbmltwoone documents \emph{may} need changes to their
  \token{notes} elements to comply with the new specification.  See
  Section~\ref{sec:notes} for more details. 
  
\item \sbmltwotwo corrects numerous errata and ambiguities
  discovered in \sbmltwoone.  These errata are listed on the
  project web site at \url{http://sbml.org}.  As a result of
  changes to \sbmltwo implied by these errata, some existing
  \sbmltwoone models, even when modified as explained above, may
  still not be compliant with Version~2.  The ultimate impact of
  the changes depends on the specific features used by a given
  model and the assumptions under which the model was created.

% Mike - really we ought to identify these cases.  With more time
% I could go through the errata list...

\end{enumerate}


%-----------------------------------------------------------------------------
\subsection{Notational conventions}
\label{sec:notation}
%-----------------------------------------------------------------------------

We define SBML data objects using a graphical notation based upon
UML, the Unified Modeling
Language~\citep{eriksson:1998,oestereich:1999}.  This UML-based
definition in turn is used to define an XML
Schema~\citep{biron:2000,fallside:2000,thompson:2000} for SBML.
The XML Schema defines the encoding of SBML documents in XML.  In
this section, we briefly summarize this UML-based approach and
notation and its mapping to XML Schema~1.0.  More details are
available in a separate document~\citep{hucka:2000b}.

There are three main advantages to using UML as a basis for
defining SBML data structures.  First, compared to using other
notations or a programming language, the UML visual
representations are generally easier to grasp by readers who are
not computer scientists.  Second, the notation is
implementation-neutral: the defined structures can be encoded in
any concrete implementation language---not just XML, but C, Java
and other languages as well.  Third, UML is a de facto industry
standard that is documented in many resources.  Readers are
therefore more likely to be familiar with it than other notations.


\subsubsection{Typographical conventions for names}
\label{sec:typographical}

The following typographical notations are used in this document to
distinguish object classes from other kinds of entities:

\begin{description}
  
\item \abstractclass{AbstractClass}: Abstract classes are classes
  that are never instantiated directly, but rather serve as
  parents of other classes.  Their names begin with a capital
  letter and they are printed in a slanted sans-serif typeface.
  In electronic document formats, the class names are also
  hyperlinked to their definitions in the specification.  For
  example, in the PDF and HTML versions of this document, clicking
  on the word \SBase will send the reader to the section
  containing the definition of this class.
  
\item \class{Class}: Names of ordinary (concrete) classes begin
  with a capital letter and are printed in an upright sans-serif
  typeface.  In electronic document formats, the class names are
  also hyperlinked to their definitions in the specification.  For
  example, in the PDF and HTML versions of this document, clicking
  on the word \Species will send the reader to the section
  containing the definition of this class.

\item \token{SomeThing}, \token{otherThing}: Fields within
  classes, primitive data type names, literal XML strings, and
  generally all tokens \emph{other} than SBML UML class names, are
  printed in an upright typewriter typeface.  Primitive types
  defined in SBML begin with a capital letter, but unfortunately,
  XML Schema~1.0 does not follow any convention and primitive XML
  types may either start with a capital letter (e.g,.
  \primtype{ID}) or not (e.g., \primtype{double}).

\end{description}


\subsubsection{Notational conventions for object fields}
\label{sec:notation-fields}

The basis of this UML-to-XML Schema approach is to translate
object classes such as \SBase into XML Schema~1.0 complex types.  When
instances of these classes are expressed in XML, they are
implemented as XML \emph{elements} and their fields are
implemented either as \emph{attributes} on the elements, or as
subelements.  The following example class definition illustrates
the notation used for different types of fields in this
specification document:

\begin{figure}[h]
  \centering
  \begin{classbox}{ExampleClass}
    field1: int                                                                \\
    field2: Species[0..*]                                                      \\
    field3: double \{ use="optional" default="0.0" \}                          \\
    math: Math \{ namespace="http://www.w3.org/1998/Math/MathML" \}            \\
    field4: (math : Math \{ namespace="http://www.w3.org/1998/Math/MathML" \}) \\
  \end{classbox}
\end{figure}

The symbols \token{field1}, \token{field2}, etc., represent
fields in an object class.  The colon immediately after the name
separates the name of the field (on the left) from the type of
data that it stores (on the right).

The order of fields implemented as subelements in the XML encoding
\emph{is} significant and \emph{must} follow the order given in
the corresponding UML diagram.  Fields are implemented as
subelements when they are a complex object comprised of
fields, or if they are a list of objects.  Fields implemented as
subelements are \token{field2}, \token{math} and \token{field4} in
the example. This ordering constraint also holds true when a
subclass inherits fields from a base class: the base class field
elements must occur before those introduced by the subclass.  This
ordering constraint is introduced by aspects of XML Schema beyond
SBML's control. (Software developers should beware that the
ordering requirement is a frequent cause of compatibility
problems; validating XML parsers will generate errors if the field
ordering of an XML element does not correspond to the SBML object
class definition.)

Expressions in curly braces (\token{\{\}}) shown after a field
type indicate additional constraints placed on the field.  We
express constraints using the XML Schema language.  In the
examples above, the text \token{\{use="optional" default="0.0"\}}
indicates that the field \token{field3} is optional and that it
has a default value of $0.0$.  A constraint of the form
\token{\{namespace="}\emph{X}\token{"\}} indicates that the
field is not in the SBML Level 2 XML namespace but resides in the
given XML namespace \emph{X}.  If a field is in a different
namespace, then the type of the field will not be defined by the
SBML UML but rather by another source.  In the examples above, the
\token{math} field and its content is defined in the MathML
namespace.

\paragraph{Simple attribute fields}

% FIXME
% MH version had this -- not sure yet why AF changed it:
%
%In the example above, \token{field1} is a field that would be
%translated into an XML attribute.  Its value can be a simple
%scalar type such as \primtype{string}, \primtype{SId} and
%\primtype{double}, as well as enumeration types.  All of the other
%fields shown in the example above are implemented as XML
%subelements---elements contained within the element that
%represents an instance of the class.  They can represents lists
%and substructures.

A field whose value can be a simple scalar type such as
\primtype{string}, \primtype{SId} and \primtype{double}, as well
as enumeration types is implemented as an XML attribute.  In the
example above \token{field1} and \token{field3} are fields that
would be translated into XML attributes.

\paragraph{Lists}

Square brackets (\token{[]}) just after a type name indicate that
the field contains a list of elements each having the same type.
Specifically, the notation \token{[0..*]} signifies a list
containing zero or more elements, the notation \token{[1..*]}
signifies a list containing at least one element, and so on, with
the asterisk character indicating an unbounded upper limit.

The approach used here to translate from a list form into XML is,
first, to create a subelement named
\token{listOf}\rule{0.5in}{0.5pt}\token{s}, where the blank
indicates the capitalized name of the field.  (For example,
\token{listOfField2s}.)  Within this subelement are placed elements
each of which has the name of the type, beginning with a lowercase
letter.  Here is an example:

\begin{example}
<listOfCompartments>
    <compartment id="cytosol" size="2.5"/>
    <compartment id="mitochondria" size="0.3"/>
</listOfCompartments>
\end{example}

When list fields can have zero elements (i.e., the type name is
followed by \token{[0..*]}), the
\token{listOf}\rule{0.5in}{0.5pt}\token{s} element is optional.
That is, a missing \token{listOf}\rule{0.5in}{0.5pt} element in an
SBML XML instance document indicates that the list is empty.  The
\token{listOf}\rule{0.5in}{0.5pt} elements, when present, should
always have content.


\paragraph{Substructures}

As we have seen a field definition of the form \token{X : B}
defines a field \emph{X} of type \emph{B}.  If \emph{B} is a
complex type consisting of multiple fields then \emph{X} is
implemented as an element.  A field definition of the form
\token{X : (A : B)} defines an element \emph{X} that contains a
field \emph{A} with type \emph{B}.  If \emph{A} is the string
\token{any} then the element \emph{X} contains an arbitrary
sequence of elements. A field definition of the form \token{X : (
A : B ) \{ C \}} is similar except that the field \emph{X} and its
content is constrained by constraint \emph{C}. A field definition
of the form \token{X : ( A : B \{ C \} ) } is similar except that
the field \emph{A} and its content is constrained by constraint
\emph{C}. In the examples above the field \token{field4} is an
element which contains a \token{math} field.  The \token{math}
field is in the MathML namespace but \token{field4} is in the SBML
namespace.


\paragraph{Additional notes about the translation to XML Schema}

The class definitions are mapped to \xmlschemaone
\token{complexType} elements.  A class inheriting fields from a
base class is constructed in XML Schema using a \token{extension}
element.  The fields that are implemented as XML attributes are
represented in XML Schema as \token{attribute} elements. The
fields that are implemented as XML elements are represented in XML
Schema as \token{element} elements within a \token{sequence}
element.  See Appendix~\ref{apdx:schema} for a mapping of this SBML
specification to XML schema.  Not all of the constraints on SBML
documents described in this document can be practically expressed
in \xmlschemaone. Appendix~\ref{apdx:validation-rules} defines
additional rules, beyond what is encoded in the XML Schema for
SBML, that must followed to produce a valid SBML document.
See~\cite{walmsley:2002} for more information on XML Schema.
