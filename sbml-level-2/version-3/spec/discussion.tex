% -*- TeX-master: "sbml-level-2-version-3"; fill-column: 66 -*-
% $Id$
% $Source$
% ----------------------------------------------------------------

\section{Discussion}
\label{sec:discussion}

The volume of data now emerging from molecular biotechnology leave
little doubt that extensive computer-based modeling, simulation
and analysis will be critical to understanding and interpreting
the data~\citep{abbott:1999,gilman:2000,popel:1998,smaglik:2000}.
This has lead to an explosion in the development of computer tools
by many research groups across the world.  The explosive rate of
progress is exciting, but the rapid growth of the field is
accompanied by problems and pressing needs.

One problem is that simulation models and results often cannot be
directly compared, shared or re-used, because the tools developed
by different groups often are not compatible with each other.  As
the field of systems biology matures, researchers increasingly
need to communicate their results as computational models rather
than box-and-arrow diagrams.  They also need to reuse published
and curated models as library components in order to succeed with
large-scale efforts~\citep[e.g., the Alliance for Cellular
Signaling;][]{gilman:2000,smaglik:2000}.  These needs require that
models implemented in one software package be portable to other
software packages, to maximize public understanding and to allow
building up libraries of curated computational models.

We offer SBML to the systems biology community as a suggested
format for exchanging models between simulation/analysis tools.
SBML is an open model representation language oriented
specifically towards representing systems of biochemical
reactions.

Our vision for SBML is to create an open standard that will enable
different software tools to exchange computational models.  SBML
is not static; we continue to develop and experiment with it, and
we interact with other groups who seek to develop similar markup
languages.  We plan on continuing to evolve SBML with the help of
the systems biology community to make SBML increasingly more
powerful, flexible and useful.


%=============================================================================
\subsection{Future enhancements: SBML Level 3 and beyond}
\label{sec:level-3}
%=============================================================================

Many people have expressed a desire to see additional capabilities
added to SBML.  The following summarizes additional features that
are under consideration to be included in SBML Level~3:
\begin{itemize}
  
\item \emph{Arrays}.  This will enable the creation of arrays of
  components (species, reactions, compartments and submodels).
  
\item \emph{Connections}.  This will be a mechanism for describing
  the connections between items in an array.
  
\item \emph{Geometry}.  This will enable the encoding of the
  spatial characteristics of models including the geometry of
  compartments, the diffusion properties of species and the
  specification of different species concentrations across
  different regions of a cell.
  
\item \emph{Model Composition}.  This will enable a large model to
  be built up out of instances of other models.  It will also
  allow the reuse of model components and the creation of several
  instances of the same model.
  
\item \emph{Multi-state and Complex Species}.  This will allow the
  straight-forward construction of models involving species with a
  large number of states or species composed of subcomponents.
  The representation scheme would be designed to contain the
  combinatorial explosion of objects that often results from these
  types of models.
  
\item \emph{Diagrams}.  This feature will allow components to be
  annotated with data to enable the display of the model in a
  diagram.
  
\item \emph{Dynamic Structure}.  This will enable model structure
  to vary during simulation.  One aspect of this allowing rules
  and reactions to have their effect conditional on the state of
  the model system.  For example in SBML Level 2 it is possible to
  create a rule with the effect:
\begin{linenomath}
\begin{equation*}
\frac{d s}{d t} =
\left\{
\begin{array}{ll}
     0 & \mbox{if $s>0$}\\
     y & \mbox{otherwise}
\end{array}
\right.
\end{equation*}
\end{linenomath}
Dynamic restructuring would enable the expression of the following example:
\begin{linenomath}
\begin{equation*}
\begin{array}{ll}
\mbox{if $s>0$} & \dfrac{d s}{d t} = y
\end{array}
\end{equation*}
\end{linenomath}
where $s$ is not determined by the rule when $s \leq 0$.

\item \emph{Tie-breaking algorithm}.  This will include a
  controlled vocabulary and associated fields on models to
  indicate the simultaneous event tie-breaking algorithm required
  to correctly simulate the model.
  
\item \emph{Distributions}.  This will provide a means of
  specifying random variables and statistical distribution of
  values.

\end{itemize}


%%=============================================================================
%\subsection{Relationships to Other Efforts}
%\label{sec:other-efforts}
%%=============================================================================

%There are a number of ongoing efforts with similar goals as those of SBML.
%Many of them are oriented more specifically toward describing protein
%sequences, genes and related entities for database storage and search.
%These are generally not intended to be computational models, in the sense
%that they do not describe entities and behavioral rules in such a way that
%a simulation package could ``run'' the models.

%The effort perhaps closest in spirit to SBML is
%CellML\tm~\citep{hedley:2001b}.  CellML is an XML-based markup language
%designed for storing and exchanging computer-based biological models.  It
%includes facilities for representing model structure, mathematics and
%additional information for database storage and search.  Models are
%described in terms of networks of connections between discrete components,
%where a component is a functional unit that may correspond to a physical
%compartment or simply a convenient modeling abstraction.  Components
%contain variables and connections contain mappings between the variables of
%connected components.  CellML provides facilities for grouping components
%and specifying the kinds of relationships that may exist between
%components.  It also uses MathML~\citep{w3c:2000b} for expressing
%mathematical relationships between components and provides the ability to
%use ECMAScript (formerly known as JavaScript) to define functions.

%The constructs in CellML tend to be at a more abstract and general level
%than those in SBML Level~2, and describe the structure and underlying
%mathematics of cellular models in a very general way.  By contrast, SBML is
%closer to the internal object model used in model analysis software.
%Because SBML Level~2 is being developed in the context of interacting with
%a number of existing simulation packages, it is a more concrete language
%than CellML and may be better suited to its purpose of enabling
%interoperability with existing simulation tools.

%The development of SBML Level 2 has benefited from discussions with the
%developers of CellML.  The developers of SBML and CellML are actively
%engaged in ensuring that the two representations can be translated between
%each other.


%%=============================================================================
%\subsection{Tracking the XML Schema Standard}
%\label{sec:tracking-xml}
%%=============================================================================

%One of the problems in attempting to define an XML Schema for SBML is that,
%at the time of this writing, the XML Schema
%specification~\citep{biron:2000,thompson:2000} has not actually been
%finalized.  This has been another motivation for defining SBML in terms of
%abstract data structures in a UML-based notation rather than directly as an
%XML Schema.

%The moving-target status of the XML Schema standard definition requires
%that we plan to update the Schema corresponding to SBML.  The following
%is our planned approach for handling changes in the Schema standard:
%\begin{enumerate}

%\item The definition of SBML Level~2 in this document is
%independent of XML
%  Schema.  Therefore, the definition of SBML Level~2 expressed here can
%  remain the same regardless of what happens to the exact form of XML
%  Schema.  Among other benefits, this allows developers to leave their
%  programs' internal data structures unchanged in the face of possible
%  revisions in the Schema standard.

%%\item In Appendix~\ref{apdx:schemas}, we provide an XML Schema
%%  corresponding to SBML Level~2 that has been created using the current
%%  definition of XML Schema from the W3C
%%  Organization~\citep{biron:2000,thompson:2000}.
%%
%\item Whenever the definition of XML Schema is updated by the W3C in the
%  future, we will issue a revised version of the XML Schema for SBML
%  Level~2 that conform to the updated standard.  We will leave the previous
%  versions still available for reference.  The updated XML Schemas for SBML
%  Level~2 will be identical to the previous versions except where changes
%  in XML Schema force a change in the definition of the Schema for SBML
%  Level~2.

%\end{enumerate}
