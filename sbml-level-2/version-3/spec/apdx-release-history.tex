% -*- TeX-master: "sbml-level-2-version-3"; fill-column: 66 -*-
% $Id$
% $Source$
% ----------------------------------------------------------------

\begin{blockChanged}

% Contortion to make the section number be colored for the rest of
% this subsection.
\renewcommand{\thesection}{\changed{\Alph{section}}}

\section{\changed{Release history of SBML Level 2 Version 3 up through
  \thisRelease}}
\label{apdx:release-history}

This section provides a list of the changes made within SBML Level 2
Version 3, for all releases of the specification.

\subsection{\changed{Changes in Release 2 relative to Release 1}}

The following changes are marked throughout this document using red-colored
text.  The \emph{ID \#} of an issue refers to the reference number
automatically assigned to the issue in the online tracking system at
\url{http://sbml.org/issue-tracker}.  The gaps in this numbering are not
consequential for SBML and do not necessarily signify deleted issues.

\newcolumntype{P}[1]{>{\raggedright\hspace{0pt}\arraybackslash}p{#1}}

\begin{table}[h]
  \begin{blockChanged}
  \small
  \centering
  \begin{tabular}{lP{2.7in}@{\hspace*{15pt}}P{1.25in}P{1.25in}}
    \toprule
    & & \multicolumn{2}{c}{\textbf{Page(s) and line number(s) in documents}}\\
    \cmidrule(r){3-4}
    \textbf{ID \#}
    & \multicolumn{1}{c}{\textbf{Description}}
    & \multicolumn{1}{c}{\textbf{Release 2}} 
    & \multicolumn{1}{c}{\textbf{Release 1}}\\
    \midrule

    1742495
    & \emph{Incorrect MathML in the example of Sec. 7.10}. 
    The example contained the MathML expression
    \token{<apply><cn>0</cn></apply>}, which is not valid MathML.
    The \token{<cn>0</cn>} portion should not be inside \token{<apply>}.
    & p.~119 lines~2 and~20
    & p.~118 lines~66--69; p.~119 lines~17--19\\
    \\[-3pt]

    1745160
    & \emph{Validation rule 10204 is incorrect}.  Validation rule
    10204 incorrectly stated that the \token{definitionURL}
    is only allowed on MathML \token{csymbol} elements, whereas
    in fact it must be allowed on \token{semantics} as well.
    In the section on SBO, the examples already use the attribute
    on \token{semantics}.  The validation rule needs to be
    modified, as does the text in Sec.~3.4.1.
    & p.~140 line~31; p.~21~line~31
    & p.~140 lines~20-21\\
    \\[-3pt]

    1760157
    & \emph{Incorrect statement about \token{id} on \Event.}
    The text stated that attribute \token{id} was required on an \Event
    instance, but this was not true and disagreed with the UML diagram, in
    which the field is actually optional.
    & p.~77, line 19
    & p.~78, lines 2--3\\
    \\[-3pt]

    1761663
    & \emph{Incorrect description of event action}.  The text
    stated that ``When the event fires, the value of the model component
    identified by \token{variable} is changed by the \EventAssignment to
    the value computed by the \token{math} element''.  This was incorrect:
    the change takes place when the event assignment is \emph{executed}.
    & p.~80 line 21
    & p.~80 lines 19--20\\
    \\[-3pt]

    1772814
    & \emph{Bad typo in \token{<annotation>} example}.  The example on
    page 17 had \token{xmlns:"URL"}.  The ``\token{:}'' character
    should actually be ``\token{=}''.
    & p.~17 line 31
    & p.~17 line 31\\
    \\[-3pt]
    
    1785279
    & \emph{Inconsistency in example of Sec.~7.7}.  The text description
    said the volume of the compartment was equal to 1, but the SBML
    disagreed.  Moreover, the text description was so confusing that people
    misinterpreted it.
    & p.~114 line~58 to~p.~115 line~2
    & p.~114 line~61\\
    \\[-3pt]

    1785712
    & \emph{Error in rate equation example}.  The rate expressions
    examples on p.72 incorrectly had negative signs.  The negatives
    are added later and should not have appeared where they did.
    & p.~72 line~44; p.~73~line~2
    & p.~72, lines 37, 39\\

    \bottomrule
  \end{tabular}
  \end{blockChanged}
\end{table}

\begin{table}[t]
  \begin{blockChanged}
  \small
  \centering
  \begin{tabular}{lP{2.7in}@{\hspace*{15pt}}P{1.25in}P{1.25in}}
    \toprule
    & & \multicolumn{2}{c}{\textbf{Page(s) and line number(s) in documents}}\\
    \cmidrule(r){3-4}
    \textbf{ID \#}
    & \multicolumn{1}{c}{\textbf{Description}}
    & \multicolumn{1}{c}{\textbf{Release 2}} 
    & \multicolumn{1}{c}{\textbf{Release 1}}\\
    \midrule

    1787185
    & \emph{Out of order words}.  The text of validation rule 21103
    contained words out of order.
    & p.~150 line~33
    & p.~150 line~33\\
    \\[-3pt]

    1792674
    & \emph{Attribute \token{encoding} must be allowed on MathML elements
      \token{annotation} and \token{annotation-xml}}.  The \token{encoding}
    element was incorrectly disallowed from these MathML elements, both in
    validation rule 10203 and in Sec.~3.4.1.  It must be allowed for
    \token{annotation} and \token{annotation-xml} as well.  The
    SBML MathML subset XML Schema allowed \token{encoding}, so the
    text and validation rule were at odds with the schema definition.
    & p.~21 line~30; p.~140~line~28
    & p.~140 line~27\\ 
    \\[-3pt]

    1796300
    & \emph{Error in stoichiometry of example formulas in Sec.~4.13.6}. The
    last example in that section, involving grams as units, had 
    expressions such as $1000 \cdot m_A$, when in fact they should have
    divided by molecular mass.
    & p.~76 lines 25--26
    & p.~75 lines 19--20\\
    \\[-3pt]

    1797728
    & \emph{\Species' \token{constant} attribute is \val{false} by
      default}.  A sentence in the text description incorrectly stated that
    the default of \Species' \token{constant} attribute is \val{true}.
    & p.~49 line 11
    & p.~49 line 11\\
    \\[-3pt]

    1797729
    & \emph{Confusing language about species size}.  The passage on p.~49
    had language that referred to a species' ``size'', but it would be less
    confusing to talk about the species' quantity rather than its size.
    Compartments have size, but species don't.
    & p.~49 line 11
    & p.~49 line 11\\
    \\[-3pt]

    1800685
    & \emph{Clarification needed about shadowing of parameters}.  The
    implications of local parameters shadowing global identifiers of
    any kind was not stated explicitly enough and could too easily
    be missed by readers.
    & p.~19 line 24; p.~70~lines~19--24
    & Sec.~3.3.1 and p.~70 lines~17--19\\

    \bottomrule
  \end{tabular}
  \end{blockChanged}
\end{table}


\end{blockChanged}


% Restore the section command definition done at the beginning of
% this file.
\renewcommand{\thesection}{\Alph{section}}
