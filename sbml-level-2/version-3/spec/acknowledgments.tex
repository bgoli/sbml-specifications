% -*- TeX-master: "sbml-level-2-version-3"; fill-column: 66 -*-
% $Id$
% $Source$
% ----------------------------------------------------------------

\section{Acknowledgments}
\label{sec:acknowledgements}
\label{sec:acknowledgments}

This specification document benefited from repeated reviews and
feedback by members of the SBML Team, especially Sarah Keating,
Bruce Shapiro, Ben Bornstein, and Maria Schilstra.  We thank them
for their efforts.  We also give special thanks to Stefan Hoops,
James Schaff, Martin Ginkel, Rainer Machn\'{e}, Sven Sahle, Robert
Phair, and Herbert Sauro for critical discussions of the
mathematical theory underlying simulations of SBML models in the
final stages of developing this specification.

The development of SBML was originally funded entirely by the
Japan Science and Technology Agency (JST) under the ERATO Kitano
Symbiotic Systems Project during the years 2000--2003.  The
principal investigators were Hiroaki Kitano and John~C.\ Doyle.
The original SBML Team was lead by Hamid Bolouri and consisted of
Hamid Bolouri, Andrew Finney, Herbert Sauro, and Michael Hucka.

We gratefully acknowledge sponsorship from many funding agencies.
Support for the continued development of SBML and associated
software, meetings and activities today comes from the following
sources: the National Institute of General Medical Sciences (USA)
via grant number GM070923; the National Human Genome Research
Institute (USA); the International Joint Research Program of NEDO
(Japan); the JST ERATO-SORST Program (Japan); the Japanese
Ministry of Agriculture; the Japanese Ministry of Education,
Culture, Sports, Science and Technology; the BBSRC e-Science
Initiative (UK); the DARPA IPTO Bio-Computation Program (USA); and
the Air Force Office of Scientific Research (USA).  Additional
support has been or continues to be provided by the California
Institute of Technology (USA), the University of Hertfordshire
(UK), the Molecular Sciences Institute (USA), and the Systems
Biology Institute (Japan).

SBML was first conceived at the JST/ERATO-sponsored \emph{First
  Workshop on Software Platforms for Systems Biology}, held in
April, 2000, at the California Institute of Technology in
Pasadena, California, USA.  The participants collectively decided
to begin developing a common XML-based declarative language for
representing models.  A draft version of the Systems Biology
Markup Language was developed by the Caltech ERATO team and
delivered to all collaborators in August, 2000.  This draft
version underwent extensive discussion over mailing lists and then
again during the \emph{Second Workshop on Software Platforms for
  Systems Biology} held in Tokyo, Japan, November 2000.  A revised
version of SBML was issued by the Caltech ERATO team in December,
2000, and after further discussions over mailing lists and in
meetings, we produced a specification for SBML
Level~1~\citep{hucka:2001}.

\sbmltwo was conceived at the \emph{5th Workshop on Software
  Platforms for Systems Biology}, held in July 2002, at the
University of Hertfordshire, UK.  The participants collectively
decided to revise the form of SBML in \sbmltwo.  The first draft
of the Level~2 Version~1 document was released in August 2002. The
final set of features in \sbmltwoone was finalized in May 2003 at
the \emph{7th Workshop on Software Platforms for Systems Biology}
in Ft.\ Lauderdale, Florida.  \sbmltwotwo was largely finalized
after the 2005 SBML Forum meeting in Boston and a final document
was issued in September 2006.

\sbmltwothree was developed with contributions from so many people
constituting the worldwide \emph{SBML Forum} that we regret it has
become infeasible to list individuals by name.  We are grateful to
everyone on the
\link{http://sbml.org/forums}{sbml-discuss@caltech.edu} and
\link{http://sbml.org/forums}{libsbml-discuss@caltech.edu} mailing
lists, the creators of CellML~\citep{hedley:2001b}, the members of
the DARPA Bio-SPICE project, and the authors of the following
software SBML-aware systems:
\href{http://depts.washington.edu/ventures/UW_Technology/Emerging_Technologies/CSI.php}{BALSA},
\href{http://www.basis.ncl.ac.uk}{BASIS},
\href{http://contraintes.inria.fr/BIOCHAM/}{BIOCHAM},
\href{http://www.cis.upenn.edu/biocomp}{BioCharon},
\href{http://diana.imim.es/ByoDyn}{ByoDyn},
\href{http://www.biocyc.org}{BioCyc},
\href{http://biocomp.ece.utk.edu/tools.html}{BioGrid},
\href{http://www.biomodels.net}{BioModels},
\href{http://cellsignaling.lanl.gov/bionetgen}{BioNetGen},
\href{http://www.bioanalyticsgroup.com/}{BioPathway~Explorer},
\href{http://www.cis.upenn.edu/biocomp/new_html/biosketch.php3}{Bio
  Sketch Pad},
\href{http://www.chemengr.ucsb.edu/~ceweb/faculty/doyle/biosens/BioSens.htm}{BioSens},
\href{http://www.biospice.org}{BioSPICE~Dashboard},
\href{http://biocomp.ece.utk.edu/tools.html}{\mbox{BioSpreadsheet}},
\href{http://labs.systemsbiology.net/bolouri/software/BioTapestry/}{BioTapestry},
\href{http://www.biouml.org/}{BioUML},
\href{https://bioinformatics.musc.edu/bstlab/}{BSTLab},
\href{http://kurata21.bse.kyutech.ac.jp/cadlive/}{CADLIVE},
\href{http://celldesigner.org}{CellDesigner},
\href{http://www-aig.jpl.nasa.gov/public/mls/cellerator/}{Cellerator},
\href{http://sbml.org/software/cellml2sbml/}{CellML2SBML},\\
\href{http://www.bii.a-star.edu.sg/sbg/cellware}{\mbox{Cellware}},
\href{http://common-lisp.net/project/cl-sbml/}{CL-SBML},
\href{http://sg.ustc.edu.cn/MFA/cleml}{CLEML},
\href{http://www.copasi.org}{COPASI},
\href{http://www.msr-unitn.unitn.it/Rpty_Soft_Sim.php}{Cyto-Sim},
\href{http://www.cytoscape.org/}{Cytoscape},
\href{http://biosim.genebee.msu.su/dbsdownload_en.html}{DBsolve},
\href{http://magnet.systemsbiology.net/software/Dizzy}{Dizzy},
\href{http://ecell.sourceforge.net/}{E-CELL},
\href{http://www.jweimar.de/ecellJ}{ecellJ},
\href{http://biocomp.ece.utk.edu/}{ESS},\\
\href{http://www.mpi-magdeburg.mpg.de/projects/fluxanalyzer}{FluxAnalyzer},
\href{http://arep.med.harvard.edu/moma/FluxorPipeline.tar.gz}{Fluxor},
\href{http://www.gepasi.org/}{Gepasi},
\href{http://www.basis.ncl.ac.uk/software.html}{Gillespie2},
\href{http://www.cis.upenn.edu/biocomp/new_html/software.php3}{HSMB},
\href{http://www.cis.upenn.edu/biocomp/new_html/software.php3}{HybridSBML},
\href{http://www.insilico-biotechnology.com}{INSILICO discovery},
\href{http://numericatech.com/jacobian.htm}{JACOBIAN},
\href{http://www.sys-bio.org/}{Jarnac},
\href{http://www.sys-bio.org/}{JDesigner},
\href{http://jigcell.biol.vt.edu}{JigCell},
\href{http://nsr.bioeng.washington.edu/PLN/Members/butterw/JSIMDOC1.6/JSim_Home.stx}{JSim},
\href{http://jjj.biochem.sun.ac.za/index.html}{JWS Online},
\href{http://www.sbml.org/kegg2sbml.html}{KEGG2SBML},
\href{http://kinetikon.molgen.mpg.de/main}{Kineticon},
\href{http://www.sbml.org/software/libsbml}{libSBML},
\href{http://www.sbml.org/mathsbml.html}{MathSBML},
\href{http://mesord.sourceforge.net/}{MesoRD},\\
\href{http://www.metabologica.com}{MetaboLogica},
\href{http://mbel.kaist.ac.kr/mfn/}{MetaFluxNet},
\href{http://www.simtec.mb.uni-siegen.de/software_mmt2.0.html}{MMT2},
\href{http://bioinformatics.oxfordjournals.org/cgi/content/abstract/20/3/316?maxtoshow=&HITS=10&hits=10&RESULTFORMAT=&fulltext=kiehl&searchid=1130434154165_1865&stored_search=&FIRSTINDEX=0&journalcode=bioinfo}{Modesto},
\href{http://www.molsci.org/~lok/moleculizer/}{Moleculizer},
\href{http://monod.molsci.org}{Monod},
\href{http://narrator-tool.org}{Narrator},
\href{http://strc.herts.ac.uk/bio/maria/NetBuilder/}{NetBuilder},
\href{http://oscill8.sf.net}{Oscill8},\\
\href{https://panther.appliedbiosystems.com/pathway/}{PANTHER
  Pathway}, \href{http://jubilantbiosys.com/pd.htm}{PathArt},
\href{http://eminch.gmxhome.de/pathscout11}{PathScout},
\href{http://innetics.com/}{PathwayLab},
\href{http://bioinformatics.ai.sri.com/ptools/}{Pathway~Tools},
\href{http://biospice.lbl.gov/PathwayBuilder/}{PathwayBuilder},
\href{http://patika.org}{PATIKAweb},
\href{http://pavesy.mpimp-golm.mpg.de}{PaVESy},
\href{http://mpf.biol.vt.edu/software/homegrown/pet/}{PET},
\href{http://page.mi.fu-berlin.de/~trieglaf/PNK2e/index.html}{PNK},
\href{http://www.reactome.org/}{Reactome},
\href{http://www.integrativebioinformatics.com/processdb.html}{ProcessDB},
\href{http://tunicata.techfak.uni-bielefeld.de/proton}{PROTON},
\href{http://www.basis.ncl.ac.uk/software.html}{pysbml},
\href{http://pysces.sourceforge.net}{PySCeS},
\href{http://ariadnegenomics.com/technology/simulation.html}{runSBML},
\href{http://sabio.villa-bosch.de/SABIORK/}{SABIO-RK},\\
\href{http://www.tbi.univie.ac.at/~raim/odeSolver/}{SBML ODE
  Solver}, \href{http://sysbio.molgen.mpg.de/SBML-PET/}{SBML-PET},
\href{http://www.ebi.ac.uk/compneur-srv/SBMLeditor.html}{SBMLeditor},
\href{http://sysbio.molgen.mpg.de/sbmlmerge/}{SBMLmerge},
\href{http://epbi-radivot.cwru.edu/}{SBMLR},
\href{http://www.dim.uchile.cl/~dremenik/SBMLSim/}{SBMLSim},
\href{http://sbml.org/software/sbmltoolbox}{SBMLToolbox},
\href{http://www-timc.imag.fr/timb/SBliD/}{SBliD},
\href{http://www.sbtoolbox.org/}{SBToolbox},
\href{http://sbw.kgi.edu/}{SBW},
\href{http://www.ucl.ac.uk/oncology/MicroCore/microcore.htm}{SCIpath},
\href{http://www.sigpath.org}{SigPath},
\href{http://depts.washington.edu/ventures/UW_Technology/Emerging_Technologies/CSI.php}{SigTran},
\href{http://www.ifak-system.com/swt/simulation/index.php?level=swtSIMBA4}{SIMBA},
\href{http://www.mathworks.com/products/simbiology/}{SimBiology},
\href{http://bioinformatics.nyu.edu/Projects/Simpathica/}{Simpathica},
\href{http://projects.villa-bosch.de/bcb/software/software/Ulla/SimWiz/}{SimWiz},
\href{http://sloppycell.sourceforge.net/}{SloppyCell},\\
\href{http://smartcell.embl.de}{SmartCell},
\href{http://www.lionbioscience.com/solutions/e20472/srspathwayeditor}{SRS
  Pathway Editor},
\href{http://www.pdn.cam.ac.uk/comp-cell/StochSim.html}{StochSim},
\href{http://www.engineering.ucsb.edu/~cse/StochKit/}{StochKit},
\href{http://www.sysbio.pl/stocks/}{STOCKS},
\href{http://teranode.com/products/index.php}{TERANODE~Suite},
\href{http://www.sourceforge.net/projects/trelis}{Trelis},
\href{http://www.vcell.org}{Virtual Cell},
\href{http://webcell.kaist.ac.kr/}{WebCell},
\href{http://www.sys-bio.org/}{WinSCAMP}, and
\href{http://www.math.pitt.edu/~bard/xpp/xpp.html}{XPPAUT}.

